Cambio a reglas
\begin{BussinesRule}{BR8}{Fecha de Nacimiento correcta.}
	\BRitem[Tipo:] Regla de integridad referencial o estructural. 
				% Otras opciones para tipo: 
				% - Regla de integridad referencial o estructural. 
				% - Regla de operación, (calcular o determinar un valor.).
				% - Regla de inferencia de un hecho.
	\BRitem[Clase:] Habilitadora. 
				% Otras opciones para clase: Habilitadora, Cronometrada, Ejecutive.
	\BRitem[Nivel:] Control. % Otras opciones para nivel: Control, Influencia.
	\BRitem[Descripción:]	Las Fechas de Nacimiento que se registran en el SINACEM para cualquier Persona debe ser mayores al día Primero de Enero del año 1900 y menor a la Fecha Actual.
	\BRitem[Motivación:] Evitar fraudes al PRONIM por el registro de personas que no han nacido al momento de su registro.
	\BRitem[Sentencia:] $\forall p \in Persona \Rightarrow 01-Enero-1900~<~p.fechaDeNacimiento~<~fechaActual$.
	\BRitem[Ejemplo positivo:] Para el día 12 de Octubre del 2013, cumplen la regla: 		
        \begin{itemize}
        	\item 11 de Octubre del 2013
			\item 20 de Diciembre del 2010
			\item 2 de Enero del 1900
        \end{itemize}
	
	\BRitem[Ejemplo negativo:] Para el día 12 de Octubre del 2013, no cumplen la 
		\begin{itemize}
        	\item 12 de Octubre del 2013
			\item 20 de Diciembre del 2014
			\item 1 de Enero del 1900
			\item 31 de Diciembre del 1899
        \end{itemize}
	
	\BRitem[Referenciado por:] \hyperlink{CUCE3.2}{CUCE3.2}, \hyperlink{CUCE3.3}{CUCE3.3}.
\end{BussinesRule}

\begin{BussinesRule}{BR129}{Determinar si un Estudiante puede inscribir Seminario.} 
	\BRitem[Tipo:] Regla de integridad referencial o estructural. 
				% Otras opciones para tipo: 
				% - Regla de integridad referencial o estructural. 
				% - Regla de operación, (calcular o determinar un valor.).
				% - Regla de inferencia de un hecho.
	\BRitem[Clase:] Habilitadora. 
				% Otras opciones para clase: Habilitadora, Cronometrada, Ejecutive.
	\BRitem[Nivel:] Control. % Otras opciones para nivel: Control, Influencia.
	\BRitem[Descripción:] Un Estudiante requere del 80\% de créditos para inscribirse a un Seminario y no haber cursado y reprobado otro seminario.
	\BRitem[Ejemplo positivo:] 
	
	\BRitem[Ejemplo negativo:] 
	
	\BRitem[Referenciado por:] 
\end{BussinesRule}

\begin{BussinesRule}{BR130}{Determinar si un Estudiante puede inscribirse en un Seminario}
	\BRitem[Tipo:] Regla de inferencia de un hecho.
				% Otras opciones para tipo: 
				% - Regla de integridad referencial o estructural. 
				% - Regla de operación, (calcular o determinar un valor.).
				% - Regla de inferencia de un hecho.
	\BRitem[Clase:] Habilitadora. 
				% Otras opciones para clase: Habilitadora, Cronometrada, Ejecutive.
	\BRitem[Nivel:] Control. % Otras opciones para nivel: Control, Influencia.
	\BRitem[Descripción:] El Estudiante debe pertenecer a la Carrera del Seminario y debe haber Cupo en el grupo del Seminario.
	\BRitem[Ejemplo positivo:] 
	
	\BRitem[Ejemplo negativo:] 
	
	\BRitem[Referenciado por:] 
\end{BussinesRule}

\begin{BussinesRule}{BR143}{Validar el horario del estudiante}
	\BRitem[Tipo:] Regla de operación, (calcular o determinar un valor.).
				% Otras opciones para tipo: 
				% - Regla de integridad referencial o estructural. 
				% - Regla de operación, (calcular o determinar un valor.).
				% - Regla de inferencia de un hecho.
	\BRitem[Clase:] Habilitadora. 
				% Otras opciones para clase: Habilitadora, Cronometrada, Ejecutive.
	\BRitem[Nivel:] Control. % Otras opciones para nivel: Control, Influencia.
	\BRitem[Descripción:] Las Materias y Seminarios inscritos por el alumno, en un periodo específico, no pueden impartirse en el mismo día de la semana en horas traslapadas.
	\BRitem[Ejemplo positivo:] 
	
	\BRitem[Ejemplo negativo:] 
	
	\BRitem[Referenciado por:] 
\end{BussinesRule}

\begin{BussinesRule}{BR180}{Calcular costos del Estudiante}
	\BRitem[Tipo:] Regla de operación, (calcular o determinar un valor.).
				% Otras opciones para tipo: 
				% - Regla de integridad referencial o estructural. 
				% - Regla de operación, (calcular o determinar un valor.).
				% - Regla de inferencia de un hecho.
	\BRitem[Clase:] Habilitadora. 
				% Otras opciones para clase: Habilitadora, Cronometrada, Ejecutive.
	\BRitem[Nivel:] Control. % Otras opciones para nivel: Control, Influencia.
	\BRitem[Descripción:] Los servicios se cobran de la siguiente forma:
		\begin{Citemize}
			\item {\em Estudiantes Regulares:} Se les Cobran todos los servicios al 100\% de su costo.
			\item {\em Estudiantes becados:} Se les otorga un 80\% de descuento en el costo de todos los servicios (antes del IVA).
			\item {\em Estudiantes extranjeros:} Se les cobran los servicios al 200\% del costo registrado.
		\end{Citemize}
	\BRitem[Sentencia:] $\forall~e~\in~\mathbb{E}\textrm{studiantes}~\land~\forall~s~\in \mathbb{S}\textrm{eminario}~\Rightarrow$
		\begin{displaymath}
			Costo(e,s) = \left\{ \begin{array}{ll}
			s.costo & , si~e.tipo = \textrm{Estudiante regular}\\
			{s.costo}\over{5} & , si~e.tipo = \textrm{Estudiante becado}\\
			s.costo \cdot 2 & , si~e.tipo = \textrm{Estudiante extranjero}
			\end{array} \right.
		\end{displaymath}
	\BRitem[Ejemplo positivo:] 
	
	\BRitem[Ejemplo negativo:] 
	
	\BRitem[Referenciado por:] 
\end{BussinesRule}

\begin{BussinesRule}{BR45}{Calcular impuestos por seminario}
	\BRitem[Tipo:] Regla de operación, (calcular o determinar un valor.).
				% Otras opciones para tipo: 
				% - Regla de integridad referencial o estructural. 
				% - Regla de operación, (calcular o determinar un valor.).
				% - Regla de inferencia de un hecho.
	\BRitem[Clase:] Habilitadora. 
				% Otras opciones para clase: Habilitadora, Cronometrada, Ejecutive.
	\BRitem[Nivel:] Control. % Otras opciones para nivel: Control, Influencia.
	\BRitem[Descripción:] Los impuestos corresponden al 16\% correspondientes al IVA.
	\BRitem[Sentencia:] $Impuesto(e, s) = Costo(e, s)\cdot0.16$.
	\BRitem[Ejemplo positivo:] 
	
	\BRitem[Ejemplo negativo:] 
	
	\BRitem[Referenciado por:] 
\end{BussinesRule}

\begin{BussinesRule}{BR100}{Recibo del Estudiante por inscripción a Seminario.}
	\BRitem[Tipo:] Regla de operación, (calcular o determinar un valor.).
				% Otras opciones para tipo: 
				% - Regla de integridad referencial o estructural. 
				% - Regla de operación, (calcular o determinar un valor.).
				% - Regla de inferencia de un hecho.
	\BRitem[Clase:] Habilitadora. 
				% Otras opciones para clase: Habilitadora, Cronometrada, Ejecutive.
	\BRitem[Nivel:] Control. % Otras opciones para nivel: Control, Influencia.
	\BRitem[Descripción:] El  Recibo del Estudiante debe mostrar el total del costo con el siguiente desglose:
		\begin{displaymath}\begin{array}{lr}
			Costo: & \$ XXX.XX\\
			Descuento~aplicado~(YY\%): & \$ XXX.XX\\
			Subtotal: & \$ XXX.XX\\
			IVA~(16\%): & \$ XXX.XX\\\hline
			Total: & \$ XXX.XX
		\end{array}\end{displaymath}
	\BRitem[Sentencia:] $CostoTotal = Costo(e, s) + Impuesto(e, s)$.
	\BRitem[Ejemplo positivo:] 
	
	\BRitem[Ejemplo negativo:] 
	
	\BRitem[Referenciado por:] 
\end{BussinesRule}

