\subsection{Prueba de sistema}
Esta prueba se realiza una vez se integraron los actores y controladores a los
niveles.
\subsubsection{Objetivo de la prueba}
Verificar el flujo de la navegación del juego.
\subsubsection{Herramientas utilizadas durante la prueba}
\textit{Unity.}
\subsubsection{Aplicación de la prueba}
Esta prueba inicia desde la escena de menú principal en donde se verifica que
el controlador del menú realiza las validaciones correspondientes antes de
crear o cargar una partida; de igual forma se verifica que aparezcan los
mensajes de confirmación a cada caso, sea el de confirmación de la nueva partida
o el que notifica que no hay datos previamente guardados.
\\
\par
La siguiente escena a probar es el menú de selección de nivel. En este se verifica
que la información mostrada por la interfaz corresponda al nivel que se desea
acceder. Después, se verifica que en efecto el juego no permite acceder a niveles
que aún no se desbloquean.
\\
\par
Para finalizar la prueba se verifica que se realicen las transiciones entre
niveles y cinemáticas. De igual forma se prueba la funcionalidad de botones de
navegación de los niveles referentes al panel de pausa, fin de partida y nivel
completado.
\subsubsection{Conclusiones de la prueba}
Al finalizar esta prueba se pudo confirmar que las transiciones entre escenas se
realiza de manera correcta; salvo en algunos casos pero fue debido a que el nombre
de la escena a la que se debería redirigir no estaba escrito correctamente o no
coincidía con el nombre de la escena a la que debía ir. Con esto se puede concluir
que se cumple el mapa de navegación que se propuso en el documento de diseño
realizado en trabajo terminal 1.

