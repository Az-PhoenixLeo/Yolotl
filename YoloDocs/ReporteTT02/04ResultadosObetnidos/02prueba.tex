\subsection{Prueba de integración}
Esta prueba se realiza una vez se integraron los actores y controladores a los
niveles.
\subsubsection{Objetivo de la prueba}
Verificar el funcionamiento lógico de los componentes del nivel al ser integrados
para formar un nivel entero.
\subsubsection{Herramientas utilizadas durante la prueba}
\textit{Unity.}
\subsubsection{Aplicación de la prueba}
Para realizar esta prueba es necesario jugar los niveles para observar que el
comportamiento de los controladores y los actores se ejecute correctamente al
integrarse con otros actores. En esta prueba también se ajustan las áreas activas de las
plataformas a fin de que su funcionamiento no se detenga si se alejan mucho del
jugador al realizar su recorrido.
                
\subsubsection{Conclusiones de la prueba}
Al finalizar esta prueba se puede verificar que los controladores funcionan de
manera correcta; sin embargo, es necesario realizar ajustes referentes a los
tiempos de transiciones entre escenas y los valores de las áreas activas de
varias plataformas y obstáculos ya que con sus valores iniciales algunas
plataformas se detenían al realizar su recorrido dado que el jugador se salía
de su área activa y se volvía inalcanzable. En cuanto al obstáculo de
\textit{WindCreator} se ajusto el tamaño del área activa garantizando que el
obstáculo se encuentre activo cuando el jugador llegue a donde se encuentra éste.  