\chapter{Introducción}

% Debe ser tener una longitud de al menos una página y media.

En esté trabajo se presentará la realización del proyecto final para titulación de la carrera de Ingeniería
en Sistemas Computacionales, realizada en el Instituto Politécnico Nacional en la Escuela Superior de Cómputo, 
ubicada en la Ciudad de México, México, en fecha al mes de Mayo del año 2018. Dicho trabajo consistirá 
en el desarrollo de un videojuego con temática de la cultura prehispánica en México, 
en específico la cultura Mexica, aplicando los conocimientos adquiridos durante la estancia en la carrera. 

Este trabajo se realiza con el propósito de encontrar una nueva alternativa de medio de transmisión para este 
tipo de contenido cultural e historico del país. También se realiza con el fin de entender la nuevas herramientas 
del mundo laboral en este campo de trabajo, pues se pretende entender y utilizar el motor de juego Unity 
que hoy en día es utilizado para creación de muchos de los videojuegos que existen actualmente. 
También se usa como delimitante de desarrollo, que el videojuego será realizado para un dispositivo móvil, 
pues se verá en el siguiente documento el incremento de uso de dispositivos móviles en la población 
tanto mundial como nacional a lo largo de los años de forma bastante acelerada.	 

Pues a primera instancia se puede observar el tipo de personas que juegan, 
los ingresos que se generan en esta industria, los tipos de industria que existen, 
las consecuencias positivas, las consecuencias negativas que generarían, 
las ganancias como profesionista en esta rama y como realizar un proyecto de esta naturaleza.

Los videojuegos utilizan diferentes medios que estimulan varios sentidos a la vez, creando una experiencia propia
enfocada a la inmersión. El INJUVE \cite{injuveespana2005} ha dado a conocer información donde se puede observar que los
jóvenes y los videojuegos llevan una interacción diaria y sobre cualquier tema, desde educativo hasta de ocio.

Aquí los jóvenes mayores de trece años representarán el público al que se desea presentar el proyecto, es por eso que se necesitará de realizar 
pruebas para conocer tanto la efectividad del medio que se ha escogido para transmitir la información, 
como aquellas características que pueden ser útiles posteriormente, que indiquen cuales son los factores 
que interesan al público y puedan ser utilizados para un producto con mayor impacto.

Por otra parte, la sociedad mexicana ignora aspectos históricos culturales y existe desinterés por conocer su legado
según INEGI \cite{inegi2017}. Algunas investigaciones \cite{guillermobonfilbatalla1987}, hablan sobre los aspectos negativos que 
ocasiona esta situación, pues de manera psicológica e inconsciente, refleja por parte de la gente un nivel de desprecio
a su propio país el hecho de no conocerlo por completo. Es aquí donde se puede ver la importancia del proyecto, pues no
sólo se hablará de comunicar contenido e ideas ó de encontrar una manera diferente de ocio para los usuarios, 
si no de utilizar herramientas digitales para %% Idea importante
ofrecer una alternativa cultural bien fundamentada en los hechos históricos que se presentan, intentando inspirar y motivar al patriotismo, 
a un espíritu de conciencia y que cada persona pueda colaborar en impulsar logros y la realización 
de un país con sentido de orgullo.

Existen nuevos métodos de interacción entre las personas y los eventos, situaciones o contexto en específico de forma 
que sea atractiva para gente. La efectividad depende mucho de aspectos psicológicos y de personalidad de cada persona.

Tomando en cuenta los aspectos anteriores, el proyecto consistirá en %% IDEA IMPORTANTE
analizar, diseñar, implementar y probar un sistema de entretenimiento, en la categoría de videojuego, 
que %% idea importante 
difunda aspectos de la cultura mexicana.

Se investigará más a fondo la historia de los videojuegos para elegir el público objetivo potencial o las personas alcanzables, los procesos y metodología para poder realizar un videojuego, aspectos a considerar para el proyecto, la cultura y ramas que abarca dentro de la sociedad, la cultura y la tecnología como se relacionan entre sí, las herramientas con las que se cuentan para la realización del videojuego, las teorías que pueden usarse, los conflictos que pueden ocurrir, como solucionar los problemas que se nos presenten y por su puesto el cambio o impacto que se tenga al presentarlo al público.

El videojuego se presentará de manerá adecuada y sobre todo con fidelidad a los acontecimientos a investigar para el tema de México prehispánico, en específico el inframundo del pueblo mexica. De manera breve la historía del juego consistirá en tomar como personaje principal a Malinalli (mejor conocida como la Malinche), cuya travesía será a través del Mictlán (o inframundo mexica), que son nueve escenarios o pruebas que plantea la mitología para completar el camino de purificación de un alma, pues Malinalli busca la manera de revivir a su padre con la ayuda de el dios Xolotl, que a cambio de recuperar su padre le ayudará a obtener el dominio del Mictlán derrotando a los guardianes de cada nivel y el dios del inframundo Mictlantecutli.

Al final se presentará un videojuego en plataforma móvil en dispositivos android como producto principal, como productos secundarios se tendrá modificaciones dentro de plantillas de la metodología usada, Huddle y pruebas del mismo juego para identificar el funcionamiento que tiene y opiniones de usuarios.