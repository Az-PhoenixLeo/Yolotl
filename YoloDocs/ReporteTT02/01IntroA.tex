\chapter{Introducción}

% Debe ser tener una longitud de al menos una página y media.

En este trabajo se presentará la realización del proyecto final para titulación de la carrera de Ingeniería
en Sistemas Computacionales que ofrece la Escuela Superior de Cómputo del Instituto Politécnico Nacional (E.S.COM. - I.P.N), 
ubicada en la Ciudad de México, México. Se describirá el desarrollo de un videojuego con temática de la cultura
prehispánica en México, en específico la cultura Mexica, aplicando los conocimientos adquiridos durante la carrera. 

%% Revisar que concuerde con la presentación y pasar a revisión con Fabiola.
El propósito del proyecto será encontrar una nueva alternativa de medio de transmisión para
contenido cultural e histórico mexicano. Se propondrá al videojuego como dicha alternativa, 
particularmente para dispositivos móviles. Se utilizará el motor de juego Unity(R), 
%% https://unity3d.com/es/public-relations/brand
herrmienta ampliamente utilizada por las empresas desarrolladoras de videojuegos.

% Esta parte es la justificación 
Los videojuegos utilizan diferentes medios que estimulan varios sentidos a la vez, creando una experiencia propia
enfocada a la inmersión. El INJUVE ha dado a conocer información\cite{injuveespana2005} donde se puede observar que los
jóvenes y los videojuegos llevan una interacción diaria, tratando temas desde educativos hasta de ocio.

Se evaluará la efectividad del medio propuesto, el videojuego, revisando características útiles que indiquen
factores de interes. 

Por otra parte, la sociedad mexicana ignora aspectos históricos culturales y existe desinterés por conocer su legado
según INEGI\cite{inegi2017}. Algunas investigaciones\cite{guillermobonfilbatalla1987} hablan sobre los aspectos negativos que 
ocasiona esta situación, y menciona que, de manera psicológica e inconsciente, refleja un nivel de desprecio a su país.
Es aquí donde se puede ver la importancia del proyecto, pues no sólo se hablará de comunicar contenido 
e ideas ó de encontrar una manera diferente de ocio para los usuarios, 
si no de utilizar herramientas digitales para %% Idea importante
ofrecer una alternativa cultural bien fundamentada en los hechos históricos que se presentan, intentando inspirar 
y motivar al patriotismo, a un espíritu de conciencia y que cada persona pueda colaborar e impulsar logros comunes,
fomentando un país con sentido de orgullo.


Se revisarán los métodos de interacción entre las personas y los eventos,situaciones o contexto en específico. 
Esta revisión tendrá en cuenta que sea atractiva para gente, consideranto que la efectividad depende mucho 
de aspectos psicológicos y de personalidad de cada persona.

Tomando en cuenta los aspectos anteriores, el proyecto consistirá en %% IDEA IMPORTANTE
analizar, diseñar, implementar y probar un sistema de entretenimiento, en la categoría de videojuego, 
que %% idea importante 
difunda aspectos de la cultura mexicana.

Se investigará la historia de los videojuegos y se seleccionará a los jóvenes mayores de trece años como público 
objetivo del proyecto. También se revisarán los procesos y metodologías para poder realizar un videojuego, 
las herramientas de desarrollo, las teorías, los conflictos y soluciones a los problemas que se presentarán y
y se considerará el cambio o impacto que se tenga al presentarlo al público. Se seleccionará a Huddle[Referencia a Huddle]
como la metodología para el desarrollo por estar enfocada a los estudios de categória {\it indie}, siendo
estos los más parecido en conformación al equipo de desarollo del proyecto.

El videojuego se presentará concentrándose en la fidelidad sobre el tema de México prehispánico, 
en específico el inframundo del pueblo mexica. De manera breve, la historía del juego tiene como personaje principal a 
Malinalli (mejor conocida como la Malinche) y su aventura en el Mictlán (o inframundo mexica). 
El Mictlán estará representado por los nueve escenarios o pruebas que plantea la mitología mexica para completar 
el camino de purificación de un alma. Malinalli buscará la manera de revivir a su padre con la ayuda del dios Xolotl 
a cambio de obtener el dominio del Mictlán, para lo que necesitarán derrotar a los guardianes de cada nivel, incluyendo 
a Mictlantecutli, el dios del inframundo.

Al final se presentará un videojuego en plataforma móvil en dispositivos Android(R) como producto principal, 
además de que se propondrán modificaciones a las plantillas de la metodología Huddle, y se mostrará el 
resultado de las pruebas aplicadas al videojuego para evaluar su funcionamiento y conocer la opinión de los 
usuarios finales.
