\subsection{Vicio en el videojuego}\label{vicioVJ}
Un videojuego se está convirtiendo en una adicción cuando la ansiedad se sobrepone al placer del juego y como síntoma se tiene la urgencia de jugar sin pensar en las consecuencias. El vicio se establece con cierta facilidad, ya que el juego ofrece un entorno de inmersión al combinar los aspectos interactivos e identificarse con el personaje o situación.

En la siguiente tabla se muestra \ref{tab:tablaOpinión} las opiniones más comunes a favor y en contra de los videojuegos sin incluir mitos o falsedades, todas estás afirmaciones son demostrables o no han sido posible desmentirlas con hechos o pruebas concretas.
\begin{table}[]
	\centering
\caption{Opiniones comunes a favor y en contra de los videojuegos presentadas en el artículo "Características del perfil multifásico de la personalidad de los jugadores de videojuegos" \cite{quinteroscaracteristicas}}
\label{tab:tablaOpinión}
	\begin{tabular}{|l|l|}
		\hline
		\multicolumn{1}{|c|}{\textbf{A favor}}                             & \multicolumn{1}{c|}{\textbf{En contra}} \\ \hline
		Entretienen                                                        & Provocan adicción                       \\ \hline
		Ejercitan la coordinación óculo-manual                             & Promueven conductas violentas           \\ \hline
		Estimulan la capacidad de lógica y reflexión                       & Aíslan socialmente                      \\ \hline
		Ayudan a concentrar la atención                                    & Limitan la imaginación                  \\ \hline
		Son un potencial muy adecuado para distintas aplicaciones sociales & Restan tiempo de otras actividades      \\ \hline
	\end{tabular}
\end{table}

	
\subsubsection{Síntomas de vicio}
A continuación se muestran síntomas identificables visualmente que ya representan una adicción al juego dada por el centro de psicología Bilbao \cite{centrodepsicologíabilbaos.l.p.}:
\begin{itemize}
	\item El jugador parece estar absorto al juego, sin atender cuando lo llaman.
	\item Siente demasiada tensión, incluso aprieta las mandíbulas cuando juega.
	\item No aparta la vista de la pantalla.
	\item Empieza a perder interés por otras actividades que practicaba.
	\item Trastornos del sueño.
	\item Distanciamiento de familia y amigos.
	\item No respeta los horarios estipulados.
\end{itemize}


\subsubsection{Carcaterísticas de un videojuego con potencial al vicio}
Los videojuegos (en especial los free-to-play) contienen características de vicio como: la necesidad de concluir “tareas incompletas”, síntomas de abstinencia y la posibilidad de jugarlo en todo momento. En este apartado solo mencionaremos algunas de ellas y almenos las más visibles en muchos videojuegos presentadas en la tesis psicológica por Monteros \cite{montero2014ocios}.

 \begin{itemize}
 	\item El efecto Zeigarnik: Se tiene incomodidad de las personas por tener “tareas incompletas”, en el caso de un videojuego se genera la necesidad por terminar el juego. El juego a su vez puede contemplar el nivel de porcentaje completado de un juego, provocando al jugador dicho síntoma haciéndolo jugar hasta su completado.
 	
 	\item Síntoma de abstinencia: En donde se priva o dificulta la posibilidad de realizar una actividad a una persona. En un juego tenemos como se establece turnos de juego, para recuperarlos se establece un límite de tiempo de espera. Usualmente estos tiempos de espera son de aproximadamente media hora o múltiplos de ella, pues psicológicamente este tiempo es lo que soporta una persona con una “tarea incompleta” en mente.
 	
 	\item La competencia: Que consiste en una disputa entre personas que aspiran a un mismo objetivo o a la superioridad en algo. Así, en un juego, ayudado en la mayoría de las veces por las redes sociales se puede compartir y comparar el avance entre los jugadores.
 	 \\[1pt]
 	
 \end{itemize}

\subsubsection{Causas del vicio ajenas al videojuego }
Entre los jóvenes entre 13 y 18 años especialmente existen muchas más causas para caer en el vicio de un videojuego, pueden ser factores psicológicos, emocionales, del entorno en el que se desarrollan y más. Incluso muchos de los factores siguientes encajan en otros tipos de adicciones.
\begin{itemize}
	\item Atención inexistente de los padres.
	\item No hay límites establecidos por la familia.
	\item Los valores no están asentados.
	\item Utilización de los videojuegos como "niñera".
	\item El joven no tiene sentido de pertenencia o no es aceptado en los grupos sociales que interactúa.
	\item Necesidad de escape de la realidad a un medio virtual.
	\item Establece mayor libertad expresión (o intenciones verdaderas) dentro del juego, ya sean positivas o negativas.
	\item Que la persona padezca alguna enfermedad que imposibilite realizar otras actividades .
	\item Trastornos psicológicos como depresión, impulsividad o ansiedad.
	\item Situación ante la solución de problemas y toma de decisiones.
	\item Falta de control emocional.
\end{itemize}

\subsubsection{Prevención del vicio}
Como toda adicción existen formas de prevenir llegar a ella.
Dadas las situaciones, causas, factores dentro del vicio del videojuego y con lectura de técnicas pedagógicas, las propuestas de solución se dan como sigue:

\begin{itemize}
	\item Establecer un horario de juego:
	\begin{itemize}
		\item Se puede poner una alarma que avise al jugador que ha estado jugando demasiado tiempo.
		\item Se puede programar a cierto tiempo jugado un bloqueo.
	\end{itemize}
	\item Complementariamente a la programación de un horario de juego, deberán establecerse qué actividades se llevarán a cabo en los momentos en los que no se va a jugar:
	\begin{itemize}
		\item Poder agregar al juego notas recordatorias de las actividades por hacer.
		\item Alarmas personalizadas dentro del juego.
	\end{itemize}
	\item Evitar los juegos online, al menos hasta que se tenga una organización del tiempo libre que impida dedicar mucho tiempo a dichos videojuegos.
	\begin{itemize}
		\item Evitar ser un juego online.
	\end{itemize}
	\item  No instalar la consola ni el ordenador en la habitación.
	\begin{itemize}
		\item El juego va a ser inaccesible a horas determinadas como la madrugada y noche.
	\end{itemize}
	\item Los padres deben conocer los videojuegos.
	\begin{itemize}
		\item Pedir forzosamente los correos de los tutores.
		\item Mandar mensajes a los tutores con un registro en resumen de lo que se está jugando.
		\item Habilitar una opción de bloqueo para los padres.
		\item Bloqueo automático e informe con estadística dependiendo de frecuencia de uso y tiempo de uso.
	\end{itemize}
\end{itemize}