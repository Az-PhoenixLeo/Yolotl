\section{Propuesta}
En esta sección se presenta a manera de resumen las propuestas y los conceptos 
definidos durante el trabajo terminal 1, tales como el planteamiento del problema, 
conceptos y definiciones referentes al videojuego y su desarrollo, la definición 
y delimitación de la cultura y el planteamiento de la solución que se desarrolla 
durante el trabajo terminal.  

%====== Planteamiento del problema ======%
\subsection{Planteamiento del problema}
En México existe un fuerte desinteres y desconocimiento hacia su cultura e historia 
nacional. De acuerdo con la Tercera Encuesta Nacional de Cultura Constitucional, 
el 52.7\% de los encuestados desconoce el año en que se aprobó la constitución 
nacional y no la relaciona con la Revolución Mexicana \cite{RefConsti}. Con base 
en la encuesta realizada por Parametría, empresa dedicada a la investigación 
estratégica de la opinión y análisis de resultados, solo el 32\% de su encuestados 
supó que México se independizó de España, el 51\% desconoce el país del que se 
independizó México, mientras que el resto del porcentaje de los encuestados 
piensa que México se independizó de otro país que no es España; la misma 
encuesta realizada por Parametría señala que el 25\% de los encuestados mencionaron 
personajes historicos ajenos a la independencia de México como participes de 
ésta y el 12\% respondió no saber que personajes historicos participaron en la 
independencia\cite{RefParametria}. 
 
%======Marco Teorico ======%
\subsection{Marco Teorico}
En esta sección se presentan los conceptos básicos para comprender el trabajo 
realizado durante el desarrollo del trabajo terminal, tales como la definición 
del videojuego, sus características, su clasificación, las metodologías de 
desarrollo, las herramientas para el desarrollo y la cultura. 

\subsubsection{Videojuego}
El grupo de periodista especializado en tecnología y desarrollo de software 
Carricay define al videojuego como: "una aplicación interactiva orientada 
al entretenimiento que, a través de ciertos mandos o controles, permite simular 
experiencias en la pantalla de un televisor, una computadora u otro dispositivo 
electrónico"\cite{Ref_DefVideo}.
\\
\par
Al igual que con otros productos tecnológicos, la evolución de los videojuegos 
ha sido vertiginosa, resultando complicado mencionar características comunes 
para todos los videojuegos. Sin embargo,en el libro “\textit{Marketing} y videojuegos: 
\textit{Product pacement, in-game, adevertising y 
advergaming}” se menciona que existen seis características comunes en los 
videojuegos: Interactividad, entretenimiento, jugabilidad, simulación \textbackslash 
virtualidad, inmersión y multiplataformidad\cite{RefCarac}; a continuación se 
menciona en que consisten cinco de las seis características, esto debido a que 
la última no se encuentra presente en todos los juegos y el mismo autor de la 
obra la menciona como una caracteristica opcional a tomar en cuenta:

	\begin{itemize}
		\item \textbf{Interactividad:} En el articulo "\textit{Defining Virtual Reality:
		 Dimensions Determining Telepresence}" se define la interactividad como la 
		 capacidad de los usuarios para participar y modificar la forma y el contenido 
		 de un entorno mediado en tiempo real\cite{RefInteractividad}.  
		
		\item \textbf{Entretenimiento:} en el articulo "Las Tecnologías del
		 Entretenimiento: Pasado, Presente y Futuro", el entretenimiento "se asocia, 
		 usualmente, de hacer algo que nos divierte, algo que podemos hacer solos o con 
		 otros, para entretenernos o divertirnos, en nuestro tiempo libre, o tal vez, 
		 algo que nos relaje o que nos haga reír"\cite{RefEntretenimiento}. 
		
		\item \textbf{Jugabilidad:} en el libro “\textit{Marketing} y videojuegos: 
	\textit{Product pacement, in-game, adevertising y advergaming}” se define la 
	jugabilidad como "la relación que existe entre todas las acciones reacciones e 
	interacciones tanto del videojugador como el videojuego como entre los propios 
	sistemas y subsistemas programados en el videojuego"\cite{RefCarac}.		
	
		\item \textbf{Simulación \textbackslash Virtualidad:} La simulación "se trata 
		de una representación a medida cuyo objetivo nos permite interactuar y 
		relacionarnos con lo representado según nuestros intereses"\cite{RefCarac}.
		
		\item \textbf{Inmersión:} Con base en el libro "La vida en la pantalla: La
		 construcción de la identidad en la era de internet", la inmersión es un 
		 proceso psicológico que se produce cuando la persona deja de percibir de 
		 forma clara su medio natural al concentrar toda su atención en un objeto,
		  narración, imagen o idea que le sumerge en un medio artificial 
		  \cite{RefInmersion}. Por su parte en la tesis "Libertad dirigida: Análisis 
		  formal del videojuego como sistema, su estructura y su avataridad", la 
		  inmersión se entiende como la coherencia de la ficción del juego y su 
		  aceptación por el jugador.\cite{refInmersionNavarro}  
	\end{itemize} 

Los videojuegos pueden se clasificados con base a su jugabilidad, en el libro 
"Juego. Historia, Teoría y Práctica del Diseño Conceptual de 
Videojuegos"\cite{Ref_JuegoDisenio} se propone la siguiente clasificación.
	\begin{itemize}
		\item \textbf{Juegos de acción:} Son juegos usualmente de temática 
				violenta. El jugador lucha por su supervivencia, para ello se vale 
				de armas o habilidades de  combate. 
			%==== Juegos de estrategia ====%
				\item \textbf{Juegos de estrategia:} Para que el jugador logre sus 
				objetivos en este tipo de juegos, éste debe de planear una estrategia, 
				normalmente a lago plazo. 
			%==== Juegos de Rol ====%
				\item \textbf{Juegos de Rol:} La mecánica de los juegos de rol gira 
				en torno a un grupo de héroes, con habilidades y progresión definidos; 
				el grupo de héroes debe de trabajar coordinadamente para cumplir un 
				objetivo; estos héroes pueden ser controlados por un solo jugador o 
				por varios. El jugador deberá explorar un mundo de gran tamaño 
				haciendo evolucionar a	sus personajes y sus habilidades. 
				\item \textbf{Videojuego de aventura:} Son parecidos a los juegos de 
				Rol; con la peculiaridad de que tienen una progresión más lineal y no 
				se hace tanto énfasis en los combates, siendo su eje principal la 
				narrativa.
				\item \textbf{Videojuegos de deportes:} Son todos aquellos videojuegos 
				que tratan sobre deportes que no involucren la conducción de un 
				vehículo. Pueden ser juegos sobre fútbol, fútbol americano, tenis, etc.
				\item \textbf{Videojuegos de carreras de vehículos:} Son todos aquellos 
				se centran en las carreras con todo tipo de vehículos, mayoritariamente 
				automóviles.
				\item \textbf{Videojuegos {\it puzzle:}} Este tipo de juego involucra 
				la resolución de un problema a partir de la utilización de una serie 
				limitada de recursos, por lo que si los recursos no se utilizan de la 
				manera correcta el problema no podrá ser solucionado.  
	\end{itemize}
	
Dentro de la clasificación de los juegos de acción entren los juegos de plataforma, 
definidos por una jugabilidad donde el jugador debe de controlar a un personaje 
con el que se dezplazará saltando entre plataformas y esquivando todo tipo de 
obstáculos y enemigos\cite{Ref_JuegoDisenio}. 
Es importante que se entienda el concepto del videojuego, sus características, 
su clasificación y la jugabilidad básica de un juego de plataforma ya que el 
presente Trabajo Terminal gira entorno al desarrollo de un videojuego de plataforma.

\subsubsection{Metodología de desarrollo}
Las metodologías de desarrollo de software son un conjunto de procedimientos, 
técnicas y ayudas a la documentación para el desarrollo de productos software
\cite{Ref_metodologia}. Para el presente trabajo terminal se consideran las 
siguientes metodologías como candidatas a implementar para guiar el desarrollo:

\begin{itemize}
	\item \textbf{Metodología en cascada:} Sigue una progresión lineal por lo que 
	cualquier error que no se haya detectado con antelación afectara todas las 
	fases que le sigan provocando una redefinición en el proyecto y por ende un 
	aumento en los costos de producción del sistema \cite{Ref:CarCascada}.Esta 
	metodología se divide en las siguientes etapas:
		\begin{itemize}
			\item \textbf{Análisis de los requisitos del software.}
			\item \textbf{Diseño.}
			\item \textbf{Codificación.}
			\item \textbf{Pruebas.}
			\item \textbf{Mantenimiento.}
		\end{itemize}
	\item \textbf{Metodología en \textit{Scrum}:} \textit{Scrum} parte de la visión 
	general que se desea que el producto alcance; a partir de esta visión se inicia la 
	división del proyecto en diferentes módulos. \textit{Scrum} implementa una 
	jerarquía entre los módulos en donde los módulos de mayor jerarquía son los 
	que se desarrollaran al inicio del proyecto o durante las primeras iteraciones 
	o \textit{sprints} \cite{Ref_ScrumRef}.Cada sprint se compone de las siguientes 
	fases:
	\begin{itemize}
		\item \textbf{Concepto}.
		\item \textbf{Especulación}.
		\item \textbf{Exploración}.
		\item \textbf{Revisión}.
		\item \textbf{Cierre}\cite{Ref_ScrumGuia}. 
	\end{itemize}
	\item \textbf{Metodología de Programación extrema:} Es una metodología de 
	desarrollo ágil y adaptable, soporta cambios de requerimientos sobre la marcha. 
	Su principal objetivo es aumentar la productividad y minimizar los procesos 
	burocráticos, por lo que el software funcional tiene mayor importancia que la 
	documentación\cite{Ref_XP}.
	\item \textbf{Metodología \textit{Huddle}:} Es una metodología cuya funcionalidad 
 se basa en la metodología \textit{Scrum}, con la diferencia de que está orientada al
  desarrollo de videojuegos.  De naturaleza ágil, resulta óptimo para equipos 
  multidisciplinarios de 5 a 10 personas; es iterativa, incremental y evolutiva 
  \cite{Ref_Huddle}. \textit{Huddle} se divide en las siguientes etapas:
  	\begin{itemize}
  		\item \textbf{Preproducción}.
		\item \textbf{Producción}.
		\item \textbf{Postmorten}.
  	\end{itemize}
\end{itemize}

Tras un riguroso análisis comparativo entre metodologías, se elige a 
\textit{Huddle}  como la metodología a guiar el desarrollo del Trabajo Terminal; 
esta elección se basa principalmente en que dicha metodología esta enfocada a 
videojuegos y no requiere ser adaptada por lo que se puede llevar a cabo el 
proyecto de manera directa sin tener que invertir tiempo en adaptar la metodología 
a las necesidades del desarrollo de un videojuego.

\subsubsection{Herramientas de desarrollo}
Como cualquier desarrollo de software, el desarrollo de un videojuego requiere 
se software especializado tal como un motor de juego, editores de imágenes, software 
de diseño, de edición de audio, etc. En este apartado se van a definir algunas de 
las herramientas utilizadas durante la elaboración del trabajo terminal.
\\
\par
La primera herramienta a definir es el del motor de juego. El motor de juego, 
también conocido como \textit{Game Engine}, parte del concepto de reutilización; 
es decir, es posible generar juegos a partir de un código base y común mediante una 
separación adecuada de los componentes fundamentales, tal como visualización de 
gráficos, control de colisiones, físicas, entrada de datos etc \cite{Ref:MutorGraf}; 
esto permite a quienes trabajen en un juego puedan centrarse en todos aquellos 
detalles que hacen al juego único. Dentro del mercado existen diferentes 
opciones de motores de juego tales como \textit{Unity3D}, \textit{UnrealEngine} 
y \textit{CryEngine}, por citar algunos. Para el presente trabajo terminal se 
decide por utilizar \textit{Unity3D} ya que ofrece: 

	\begin{itemize}
		\item Desarrollo muliplataforma, lo que permite aumentar la escalabilidad 
		del proyecto. 
		\item Curva de aprendizaje rápido.
		\item Comunidad de desarrolladores activa.
		\item Tres opciones de lenguajes de programacion para utilizar: 
		\textit{$\sharp C$,javaScript y Boo}.
		\item No requiere de muchos recursos para su instalación.
		\item Uso de diferentes tipos de licencia lo que permite contar con una 
		licencia gratuita, de pago y una de negocios. No existiendo mucha diferencia 
		de funcionalidad entre la licencia libre y la de pago.
	\end{itemize}
 En lo que refiere a la creación del entorno gráfico del videojuego, es decir de 
 sus sprites, se decide utilizar los \textit{software} de diseño 
 \textit{Adobe Phtoshop} y \textit{Corel Draw}. Ya que al momento de elegir dichos 
 softwares ya se contaba con experiencia previa sobre su funcionamiento y no 
 requiere ningún tipo de periodo de prueba para familiarizarse con su funcionamiento. 
 Ambos \textit{softwares} son de pago y para el desarrollo del presente trabajo 
 terminal se utiliza una licencia personal por lo que si se desea comercializar 
 el juego va a ser necesario adquirir otro tipo de licencia para le generación 
 de \textit{sprites}.


\subsubsection{Cultura}
Una vez explicado lo que es el videojuego, su metodologia de desarrollo y las 
herramientas a usar para desarrollarlo, es preciso definir lo que es la cultura; 
para tal objetivo el presnete trabajo se vale de la definición propuesta por la 
Organización de las Naciones Unidas para la Educación, la Ciencia y la Cultura 
(UNESCO, por sus siglas en inglés). La UNESCO define la cultura como “el conjunto 
de los rasgos distintivos, espirituales y materiales, intelectuales y afectivos 
que caracterizan a una sociedad o un grupo social. La cultura engloba, además 
de las artes y las letras, los modos de vida, los derechos fundamentales al ser 
humano, los sistemas de valores, las tradiciones y las creencias; de igual forma 
la cultura da al hombre la capacidad de reflexionar sobre sí mismo\cite{RefCultura}”. 
Bajo su misma definición la UNESCO, se plantea que la importancia de la cultura 
radica en su capacidad de hacer a los seres humanos racionales, críticos y 
éticamente comprometidos; ya que, través de ella se disciernen los valores y se 
efectúan opciones. Siendo por medio de ella que el hombre se expresa, toma conciencia 
de sí mismo, se reconoce como un proyecto inacabado, pone en cuestión sus propias 
realizaciones, busca incansablemente nuevas significaciones, y crea obras que lo 
trascienden \cite{RefCultura}.
\\
\par
Para efectos del presente trabajo terminal, este únicamente va a abordar la cultura 
de carácter historica, es decir la cultura que hace referencia a la herencia social, 
es decir aquella que relaciona a la sociedad con su pasado
\cite{RefculturaClasificacionEl}.

\subsection{Planteamiento de la solución}
Con el fin de fomentar la cultura y la historia se desarrolla Yolotl, un videojuego 
de plataforma y aventuras en dos dimensiones para dispositivos móviles android 5.1 
de gama media alta.
\\
\par
Las razones por las que se aborda la solución del problema con un videojuego se 
debe principalmente a diferentes factores tales como:

\begin{itemize}
	\item \textbf{El estado de la industria mexicana de los videojuegos:} En el 
	2017 México ocupó el 12 puesto en cuanto a consumo de videjuegos percibiendo 
	un ingreso de 1.4 mil millones de dolares en esta industria. A su vez México 
	cuenta con 49.2 millones de jugadores\cite{Ref_JuegosGanancia}.

	\item \textbf{El auge de los juegos para teléfonos móviles:} En el 2017 la 
	industria del videojuego tuvo ganancias de 108.9 mil millones de dolares de 
	los cuales el 32\% de las ganancias fueron generadas por los teléfonos 
	inteligentes y un 10\% por las tablets; con este porcentaje los teléfonos 
	superaron a las consolas de mesa en ingresos\cite{Ref_JuegosGanancia}. 

	\item \textbf{El consumo de teléfonos móviles en México}: En el 2017 México 
	contaba con 52 millones de usuarios de teléfonos móviles, lo que lo ubicó en 
	el 9 puesto a nivel mundial en el consumo de teléfonos inteligentes
	\cite{Ref_TelefonosGanancia}.

	\item \textbf{La interactividad de un videojuego:} Como se menciona en el 
	artículo \textit{Identification with the Player Character as Determinant 
	of VideoGames Enjoyment}: en los videojuegos, la interactividad juega un 
	papel importante para identificar y adoptar un determinado concepto, ya que 
	dentro del videojuego el jugador no es un espectador, pues participa 
	directamente en la historia e interactúa con el mundo del personaje; esto 
	genera una relación íntima entre el jugador y el personaje puesto que es 
	gracias al jugador que el personaje puede avanzar en la historia y a su vez 
	es gracias al personaje que el jugador puede interactuar con la historia
	\cite{PlayerIdentification}. 
\end{itemize}