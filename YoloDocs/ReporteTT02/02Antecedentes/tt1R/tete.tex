\subsection{Presentado en TT1}\label{tt1}
A continuación se muestra lo presentado en TT1, que representa la investigación, análisis y prototipos del proyecto.

\subsubsection{Contexto}\label{contexto}
Primero nos encontramos a determinar que es la cultura; aquella que define la identidad de un individuo por sus creencias religiosas, de pensamiento, sentimentales y sociales. Mientras que la historia son todos los sucesos pasados dentro de un espacio específico. Entonces la cultura histórica es aquellos aspectos arquitectónicos, de pensamiento, éticas y morales, grupos de pertenencia y convivencia. Dentro de la hultura histórica encontramos que el 41\% de los mexicanos no asisten a eventos culturales y presentan una gran desinformación de ella.

Luego nos encontramos que existen formas de presentar cualquier tema o actividad dentro de los juegos. A partir de eso investigamos sobre los videojuegos, juegos que se presentan a través de un medio visiual o auditivo en el que existe interacción por diferentes dispositivos de entrada. Dentro de los videojuegos hay clasificaciones por contenido, que son las utilizadas para determinar el tipo de público al que va dirigido, además de que es una clasificación conveniente para limitar compra y venta. Sin embargo existen muchas más clasificaciones, en donde el tipo de contenido, dispositivo u objetivo es su pertenencia.

\subsubsection{Viabilidad}\label{viabilidad}
Con esta información ya en mano, buscamos sobre la situación actual de comercio de los videojuegos. Pasando primero por los ingresos a nivel mundial y en que dispositvos en el año 2017; a nivel mundial se tiene un ingreso de \$108.9 mil millones de dolares, el 42\% está en los dispositvos móviles, tanto teléfonos inteligentes como tabletas, el resto queda en la computadora y consolas. Así México queda en el doceavo lugar de consumo de videojuegos a nivel mundial, con un ingreso percibido de \$1.4 mil millones de dolares en una población de 130 millones de mexicanos. De la población en México el 20\% juega videojuegos, de ese porcentaje 45\% juega en el celular y 40\% juega de una a dos veces por semana.

\subsubsection{Análisis}\label{analisis}
Una vez contemplada la información, se decide que el proyecto se haga en un dispositvo móvil android, dado que el 86\% de los usuarios de dispositivos móviles tienen android y como versión mínima 5.0 lolipop con su uso de 32\% de las personas.

La metodología a usar será Hundle, que consiste en un parecido a la metología scrum solo que está adaptada a la creación y desarrollo de videojuegos, donde se establece un proceso general de preproducción,  producción y postmortem. Destacando aquí que se hicieron algunos ajustes dentro de la parte de preproducción agregando y modificando apartados daddo que eran necesarios aclarar antes de empezar con el proyecto.

Luego se pasó a definir las herramientas de desarrollo, donde se escogió debido a su flexibilidad multiplataforma el motor de juego Unity y como herramientas de dibujo corel draw x5 y photoshop.

Como arquitectura a usar se eligió modelo vista controlador, donde se dividirá el código en controladores, actores y auxiliares, estos últimos ayudarán a los controladores y actores en situaciones específicas donde no se encuentre en ninguna de estas características.

\subsubsection{Progresión y prototipos}\label{proypro}
Ya realizadas las investigaciones y análisis previo para el proyecto, se estableció la progresión que iba a tener el juego, junto con definir las interfaces que contendría y su interacción entre ellas.
Se estableció la mecánica del juego donde se definía los botones de acción del personaje, el espacio de características del personaje y un apartado para determinar otros factores como items o vida del enemigo.

Ya en los prototipos se pasó a el maquetado de los niveles uno y dos, determinando personajes y eventos, así como prototipos de uso de la herramienta Unity.Al final dando como resultado dos prototipos conteniendo el nivel uno y dos y un prototipo de uso de la herramienta.