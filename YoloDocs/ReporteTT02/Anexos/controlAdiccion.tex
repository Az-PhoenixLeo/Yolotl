\chapter{Control de adicción en el jugador} \label{Anexo:AdiccJuga}
En este apartado se presentan las diferentes estrategias que se pueden 
implementar dentro del presente trabajo terminal para evitar el desorden de 
juego en sus jugadores. Cada estrategia se describe y en seguida se mencionan 
las implicaciones de su implementación.

\section{Notificación de confirmación para continuar la partida}
Esta mecánica consiste en que después de que el juego detecta que el jugador ha 
pasado mucho tiempo jugando (aproximadamente más de una hora); el juego se pausa 
a sí mismo y le comunica al jugador que ha pasado mucho tiempo jugando, dándole 
la opción de terminar la partida o continuar. Esta mecánica es utilizada por 
algunos juegos de consola como \textit{Just Dance} o aquellos que involucran 
alguna actividad física; en el caso de \textit{Just Dance}, el juego se pausa a 
sí mismo y le pregunta al jugador si no se siente fatigado por el largo tiempo 
que éste ha estado activo.
\subsection{Implicaciones}
Esta mecánica exige que se manejen módulos que contabilicen el tiempo que el 
jugador este dentro de la aplicación y lance mensajes después de que detecte que 
ha pasado una cantidad significativa de tiempo. Si bien la implementación de 
esta mecánica no parece involucrar modificaciones significativas dentro de los 
módulos ya existentes del juego; no es muy recomendable de ser llevada a la 
práctica ya que se vale totalmente de la buena voluntad del jugador para 
controlarse y no representa un limitador real del tiempo de partida.

\section{Control paterno}
En esta mecánica de control se vincula la partida del jugador con una cuenta de 
correo perteneciente a algún padre o tutor. La aplicación envía a esta cuenta un 
registro de número de veces que el jugador abre el juego y cuánto tiempo 
permanece en éste. Este sistema de control también se le puede implementar el 
envió de un formulario al correo del tutor en el cual éste pueda indicarle al 
juego cuantas veces al día y cuánto tiempo puede ser abierta diariamente o 
semanalmente. 

\subsection{Implicaciones}
Esta mecánica es una de las que requiere mayor modificaciones y agregaciones al 
sistema pues su implementación exige un módulo de monitoreo de tiempo, envió de 
formularios, verificación de permisos y que el juego tenga la capacidad de 
bloquearse o desbloquearse con base en la información enviada por el padre o 
tutor. Si bien esta mecánica de control puede resultar eficiente con los 
jugadores más jóvenes dentro del público objetivo; no se debe de dejar a un lado 
que la misma puede ser considerada intrusiva para los jugadores de mayor edad, 
pudiendo repercutir de manera negativa en la experiencia de juego. 

\section{Sistema de vidas}
Esta mecánica plantea la existencia de una cantidad de vidas que el jugador 
deberá de ocupar para poder avanzar dentro del juego, ya sea cuando ingrese a un 
nivel o para revivir dentro del mismo. La cantidad de vidas usualmente se 
restauran después de una determinada cantidad de tiempo, por ejemplo, en el 
juego de \textit{Pokemon Shuffle} desarrollado por \textit{Nintendo} para 
dispositivos móviles, las vidas con las que cuenta el jugador se van restaurando 
de una en una cada quince minutos.  Por otro lado, existen juegos como\textit{ 
Kingdom Hearts Union X} en donde la mecánica que limita el tiempo de juego 
depende de las misiones que el jugador decida emprender; ya que para iniciar una 
misión el jugador debe de pagar con energía (\textit{AP}), existiendo misiones 
que requieren mayor cantidad de \textit{AP} que otras. Es importante recalcar 
que estas mecánicas son implementadas por las empresas no solo para limitar el 
tiempo que el jugador invierte dentro de la aplicación, sino que este sistema 
permite a las empresas implementar microtransacciones dentro de sus juegos al 
ofrecerle alternativas a sus jugadores para seguir jugando sin la necesidad de 
esperar por nuevas vidas o \textit{AP.}  
\subsection{Implicaciones}
El desarrollo de un sistema de vidas dentro del juego demanda la implementación 
de nuevas variables globales dentro del sistema, tal como la cantidad de vida y 
un contador de tiempo que restaure las mismas. De igual forma, es necesario 
implementar un módulo robusto que soporte las alteraciones del reloj del 
dispositivo móvil en caso de que el jugador decida adelantar el reloj de su 
dispositivo tratando de engañar al sistema y que este le restaure vidas sin que 
haya pasado el tiempo de espera.  El uso de estas nuevas variables y la creación 
de estos nuevos módulos repercutiría directamente en el funcionamiento de las 
clases controladoras ya existentes del juego, las modificaciones a las clases 
controladoras repercuten directamente en el tiempo de desarrollo del juego por 
lo que esta estrategia no se considera viable para el cronograma que ya se tenía 
previsto al inicio del proyecto.
\\
\par
Por otra parte, el uso de un sistema de vidas no es recomendable para evitar el 
desorden de juego, esto debido a que existen diversos estudios en los que se ha 
logrado determinar que este tipo de estrategias favorecen al desorden de juego 
al generar ansiedad en los jugadores con los tiempos de espera entre partidas.

\section{Observaciones}
El desorden de juego es un problema que aqueja a muchos jugadores a nivel 
mundial sin embargo es un problema complejo y que depende de muchos factores. El 
principal reto que existe para desarrolladores para hacerle frente a este 
problema es que muchas de las estrategias que se podrían implementar, como las 
mecánicas de control anteriormente descritas, llegan a incrementar la 
complejidad del juego como sistema o repercuten en la inmersión y experiencia 
del usuario. De igual forma, es importante mencionar que el desorden de juego es 
una patología que se está estudiando recientemente y no existe, a la fecha del 
presente trabajo terminal, información concluyente de ningún centro de 
investigación sobre medidas para evitarlo o diagnosticarlo. Con lo anterior este 
trabajo terminal no implementara ninguna de las estrategias presentadas en este 
anexo; pues las implicaciones para su implementación no son posibles de llevar a 
cabo durante la duración del Trabajo Terminal por lo que se dejan como mera 
mención para quienes deseen retomar el trabajo a futuro.
