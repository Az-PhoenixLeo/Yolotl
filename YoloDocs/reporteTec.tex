\documentclass[11pt,letterpaper]{article}
\usepackage[utf8]{inputenc}
\usepackage[spanish]{babel}
\usepackage{graphicx}
\usepackage{subfigure}
\usepackage[top=1in, bottom=1in, right=1in, left=1in]{geometry}

\begin{document}
	\author{Hernández Bautista Yasmine Pilar, Márquez 		Hernández Karla Rocío}
	\title{Reporte Técnico}
	\maketitle
	\tableofcontents
	\section{Introducción}
	\section{Planteamiento del problema}
	
	\section{Marco teórico conceptual}
		\subsection{Videojuego}
			\subsubsection{¿Qué es un videojuego?}
			\subsubsection{Clasificacion de los videojuegos}
			\subsubsection{Industria del videojuego en México}	
		\subsection{Videojuegos lúdicos}		
		\subsection{Desarrollo de videojuegos}
			\subsubsection{Metodologias de desarrollo}
			\subsubsection{Pipeline}
			\subsubsection{Motores gráficos}
			\subsubsection{Software auxiliar}
		\subsection{Cultura}
		\subsection{Cultura Digital}
				
	\section{Estado del arte}
	\section{Trabajo realizado}
		\subsection{Documento de diseño}
		\subsection{Investigación para el manejo de Unity}
		\subsection{Diagramas}
		\subsection{Programación de niveles / Creación de niveles}
	\section{Resultados obtenidos}
	\section{Observaciones}
	\section{Bibliografia}	
	\section{Anexos}
\end{document}