\section{Análisis del juego}
En esta sección se presenta el analisis que se realizo del juego con base en el 
documento de diseño descrito en el apartado \ref{docDisenio}.Primeramente se 
describe al usuario que se identificó y las acciones que éste puede realizar 
dentro del juego; en segundo lugar se habla de las clases que componen el juego, 
describiéndose lo tres tipos de clases en el que se clasificaron, para 
posteriormente listar que clases pertenecen a cada un de los tipos. Para finalizar 
esta sección se describe la comunicación entre clases para tres procesos que 
realiza el juego. 

	\subsection{Usuario del sistema}
	Tomando como referencia el documento de diseño se identificó el siguiente actor:
	\begin{itemize}
		\item \textbf{Jugador:} Es el usuario del sistema y quien interactua con
		el mismo. Dentro del sistema el jugador puede: 
			\begin{itemize}
				\item Empezar una partida nueva.	
				\item Cargar una partida ya existente.
				\item Elegir un nivel para jugar.  
				\item Jugar.
				\item Pausar Nivel.
				\item Reanudar nivel pausado.
				\item Ver cinemática.
			\end{itemize}			   
	\end{itemize}	
	
	\subsection{Clases del juegos} \label{ClasesJuego}
	A partir del documento de diseño se identificaron al rededor de (); estas 
	clases se dividen en tres grupos:
		\begin{itemize}
			\item \textbf{Controladores:} Clases encargadas de controlar la gestión de 
			la partida y la navegación dentro del juego. Estas clases regulan las acciones 
			de las clases actoras y desencadenan determinados eventos dentro del juego
			dependiendo de las acciones del Jugador y de las reglas de los niveles.
			
			\item \textbf{Actores:} Son las clases que modelan a los enemigos, obstáculos,
			\textit{checkpoints} y el jugador.
			 
			 \item \textbf{Auxiliares:} Son todas aquellas clases que ayudan a los controladores
			 a cumplir con su funcionalidad al permitirle a los controladores obtener datos 
			 para inicializar valores o garantizar las transiciones entre interfaces.
		\end{itemize}
		
	\subsubsection{Clases controladoras} \label{ClaseCtrl}
		\begin{itemize}
			\item \textbf{PrincipalMenuCtrl:} Esta clase se encarga de la funcionalidad del menú 
			principal. Esta clase esta a cargo de:
			\begin{itemize}
				\item Empezar nueva partida.
				\item Cargar nueva partida.
				\item Mostrar mensajes de confirmación antes de proceder con cambios 
				irreversibles a los datos de partida.
				\item Mostrar mensaje de aviso en caso de no encontrar exista una partida 
				guardada. 
			\end{itemize}
			%=======================================
			\item \textbf{SelectLevelMenu:} Esta clase controla la funcionalidad del menú de 
			selección nivel. Esta clase realiza:
			\begin{itemize}
				\item Habilitar solo los niveles y cinemáticas que el jugador haya 
				desbloqueado.
				\item No permitir que el jugador pueda acceder a niveles o cinemáticas 
				que el jugador no haya desbloqueado.
				\item Direccionar al jugador al nivel o a la cinemática que seleccionó.
			\end{itemize}
			%=======================================
			\item \textbf{GameDataCtrl:} Esta clase controla el archivo de los datos de 
			partida, este archivo sera de formato binario, este formato es un tipo de 
			archivo que propio de Unity y permite proteger los datos de las partidas evitando
			que estos puedan ser modificados por el jugador, garantizando así la integridad 
			de la información. Está ligada a la mayoría de los controladores pues de ella depende 
			 guardar y cargar el progreso del jugador para inicializar valores como: la 
			 vida del jugador, su cantidad de \textit{Tonalli}, los niveles disponibles, etc. 
			 Dentro de su funcionalidad está: 
			\begin{itemize}
				\item Verificar la existencia del archivo de datos de partida.
				\item Crear un archivo de datos de partida.
				\item Leer los datos del archivo de datos de partida.
				\item Escribir datos en el archivo de partida.
			\end{itemize}
			%=======================================
			\item \textbf{DialogueCtrl:} Esta clase se encarga del despliegue de diálogos
			en las cinemáticas y dentro de los niveles. Esta clase realiza las siguientes 
			actividades:
			\begin{itemize}
				\item Iniciar el despliegue de diálogos.
				\item Mostrar el dialogo siguiente.
				\item Finalizar el despliegue diálogos. 
			\end{itemize}	
			%=======================================
			\item \textbf{TalkedCharactersCtrl:} Esta clase asigna una instancia de la 
			clase \textit{Dialogue} a cada una de las instancias de la clase 
			\textit{TalkedCharacter}.
			%=======================================
			\item \textbf{CutsceneCtrl:} Esta clase se encarga de vincular el despliegue 
			de diálogos con las animaciones de las cinemáticas. Esta clase tiene diversas 
			clases hijas que heredan su funcionalidad de vinculación de diálogos y animación 
			incorporando las consideraciones necesarias para el control de cada cinemática.
			%=======================================
			\item \textbf{AudioCtrl:} Esta clase esta a a cargo de generar los sonidos de \textit{SFX} 
			dentro del juego utilizando la posición del jugador o de los enemigos. 
			%=======================================
			\item \textbf{LevelCtrl:} Esta clase controla el nivel que el jugador esta 
			jugando. Esta clase realiza las siguientes acciones:
			\begin{itemize}
				\item Inicializar los atributos de la clase \textit{Player}.
				\item Verificar que el jugador este vivo.
				\item Actualizar la barra de vida del jugador.
				\item Actualizar la barra de cantidad \textit{Tonalli}.
				\item Pausar el juego.
				\item Reanudar juego pausado. 
			\end{itemize}
			Esta clase tiene clases hijas que se encargan de:
				\begin{itemize}
					\item Verificar que se cumplan los objetivos específicos del nivel.
					\item Actualizar los objetivos del nivel.
					\item Actualizar los contadores de los objetivos.
					\item Guardar el progreso obtenido en el nivel.
					\item Inicializar los valores del jugador con base al \textit{checkpoint} activo. 
				\end{itemize}
			%=======================================
			\item \textbf{CameraCtrl:} Esta clase controla el desplazamiento de la cámara.
			%=======================================
			\item \textbf{MobileUICtrl:} Esta clase se encarga de comunicar al jugador 
			con la clase \textit{Player}. Es a través de esta clase que el jugador puede controlar 
			al personaje jugable. Esta clase le permite al jugador:
			\begin{itemize}
				\item Mover al personaje jugable a la derecha.
				\item Mover al personaje a la izquierda.
				\item Detener el movimiento del jugador.
				\item Actualizar la barra de cantidad \textit{Tonalli}.
				\item Pausar el juego.
				\item Reanudar juego pausado. 
			\end{itemize}
			%=======================================
			\item \textbf{ArrowCreator:} Esta clase crea objetos que instancían al prefab "Arrow".								
		\end{itemize}				  
	\subsubsection{Clases Actores}
A continuación se listan las clases actores: 
		\begin{itemize}
			\item \textbf{Player:} Esta clase se encarga de las acciones del personaje 
			jugable, actualizar sus estados y gestionar el valor de sus atributos. 
			Esta clase esta a cargo de:
			\begin{itemize}
				\item Mover al personaje jugable de manera horizontal.
				\item Controlar la maquina de estados de las animaciones del personaje 
				jugable.
				\item Detectar las colisiones del personaje jugable.
				\item Actualizar la cantidad de vida al recibir daño.
				\item Realizar disparo de \textit{Tonalli}.
				\item Actualizar la cantidad de \textit{Tonalli} al efectuar un disparo.  
				\item Saltar. 
			\end{itemize}
			%=======================================
			\item \textbf{TalkedCharacter:} Esta clase modela el funcionamiento de un
			ciudadano con el que el jugador debe interactuar en el nivel 1.
			Esta clase esta a cargo de:
			\begin{itemize}
				\item Mostrar un ícono de diálogo para indicarle al jugador que debe de 
				interactuar con éste.
				\item Ocultar el ícono de diálogo.
				\item Indicarle a la clase DialogueCtrl que se inicia un diálogo.  
			\end{itemize}
			%=======================================
			\item \textbf{DroppingPlatform:} Esta clase modela el funcionamiento 
			del obstáculo Plataforma que cae, por lo que al hacer contacto con un objeto de 
			la clase \textit{GroundCollisionCtrl} el objeto que instancie esta clase 
			empezará a caer despues de \textit{n} segundos. 
			%=======================================
			\item \textbf{MovingPlatform:} Esta clase modela el funcionamiento 
			del obstáculo Plataforma que móvil. Haciendo uso de dos posiciones: A y B, el 
			objeto que instancia esta clase se mueve de manera cíclica de la posición A a 
			la B y de la B a la A. 
			%=======================================
			\item \textbf{DisappearingPlatform:} Esta clase modela el funcionamiento 
			del obstáculo Plataforma que desaparece, esta clase hace que el objeto que la 
			instancie aparezca y desaparezca de manera cíclica, activando y desactivando 
			los colisionadores del objeto. 
			%=======================================
			\item \textbf{Stalagmite:} Esta clase modela el comportamiento del obstáculo 
			estalagmita. Esta clase hace que el objeto que la instancie caiga cuando 
			detecte que el jugador se posiciona por debajo de este objeto.    
			%=======================================
			\item \textbf{PushingObstacle:} Esta clase modela el comportamiento de dos 
			obstáculos: viento temporal y bolas de nieve. Haciendo uso de una posición, la
			clase determina hacia que dirección debe de incrementar su dimensión y el tamaño 
			de su colisionador.      
			%=======================================	
			\item \textbf{Arrow:} Clase que produce un movimiento vertical descendente al 
			objeto que la instancia.   
			%=======================================
			\item \textbf{\textit{Enemy}:} Esta clase modela el comportamiento común que 
			tienen los enemigos de tipo normal y los de tipo jefe. De esta clase heredan 
			su funcionamiento las clase \textit{NormalEnemy} y \textit{BossEnemy}.Esta 
			clase se encarga de:  
			\begin{itemize}
				\item Controlar las transiciones de la maquina de estados que controla las 
				animaciones del enemigo.
				\item Gestiona la detección de colisiones del enemigo.
				\item Actualizar la cantidad de vida del Enemigo.
			\end{itemize}
			%=======================================
			\item \textbf{\textit{NormalEnemy}:} Esta clase modela el comportamiento común que 
			tienen los enemigos de tipo normal. Esta clase hereda su funcionamiento de la clase
			 \textit{Enemy}. La clase NormalEnemy hereda su funcionalidad a las clases 
			\textit{ Jaguar, Bird, Armadillo, PurpleGost} y \textit{RedGost}.Esta 
			clase se encarga de:  
			\begin{itemize}
				\item Controlar el patrón de movimiento de los enemigos de tipo 
				normal.
				\item  Verificar la cercanía que tiene el enemigo normal con otros objetos, 
				obstáculos y enemigos para ajustar su rango de acción y evitar que interfiera con el 
				funcionamiento de otro objeto.
			\end{itemize}					
			%=======================================
			\item \textbf{\textit{Jaguar}:} Esta clase modela el comportamiento del enemigo jaguar 
			(Consultar la ficha de personaje en la Sección de personajes en el documento de diseño). 
			Esta clase hereda su funcionamiento de la clase \textit{NormalEnemy}. Esta 
			clase se encarga de:  
			\begin{itemize}
				\item Sincronizar el desplazamiento del jaguar con su maquina de estados para que modele
				el patrón de ataque del jaguar. 
			\end{itemize}
			%=======================================
			\item \textbf{\textit{Bird}:} Esta clase modela el comportamiento de dos personajes de tipo
			normal: Chara enana y Zopilote (Consultar la ficha de personaje en la Sección de personajes en 
			el documento de diseño). Esta clase hereda su funcionamiento de la clase 
			\textit{NormalEnemy}. Esta clase se encarga de:  
			\begin{itemize}
				\item Sincronizar el desplazamiento del ave con su maquina de estados para que modele
				el patrón de ataque de Chara enana y del zopilote. 
			\end{itemize}
			%=======================================
			\item \textbf{\textit{Armadillo}:} Esta clase modela el comportamiento del personaje 
			armadillo (Consultar la ficha de personaje en la Sección de personajes en 
			el documento de diseño). Esta clase hereda su funcionamiento de la clase 
			\textit{NormalEnemy}. Esta clase se encarga de:  
			\begin{itemize}
				\item Sincronizar el desplazamiento del jaguar con su maquina de estados para que modele
				el patrón de ataque del armadillo. 
			\end{itemize}
			%=======================================
			\item \textbf{\textit{PurpleGost}:} Esta clase modela el comportamiento del personaje 
			fantasma purpura (Consultar la ficha de personaje en la Sección de personajes en 
			el documento de diseño). Esta clase hereda su funcionamiento de la clase 
			\textit{NormalEnemy}. Esta clase se encarga de:  
			\begin{itemize}
				\item Sincronizar el desplazamiento del jaguar con su maquina de estados para que modele
				el patrón de ataque del fantasma purpura. 
			\end{itemize}
			
			%=======================================
			\item \textbf{\textit{RedGost}:} Esta clase modela el comportamiento del personaje 
			fantasma rojo (Consultar la ficha de personaje en la Sección de personajes en 
			el documento de diseño). Esta clase hereda su funcionamiento de la clase 
			\textit{NormalEnemy}. Esta clase se encarga de:  
			\begin{itemize}
				\item Sincronizar el desplazamiento del jaguar con su maquina de estados para que modele
				el patrón de ataque del fantasma rojo.
				\item Instanciar el objeto ShootEnemy. 
			\end{itemize}
			%=======================================
			\item \textbf{\textit{BossEnemy}:} Esta clase modela el comportamiento común que 
			tienen los enemigos de tipo ¿jefe. Esta clase hereda su funcionamiento de 
			la clase \textit{Enemy}. La clase BosslEnemy hereda su funcionalidad a las clases 
			\textit{ Xochitonal, Tepeyollotl,Itzpapalotl, Mictlecayotl, Tlazolteotl, 
			Itztlacoliuhqui, Nexoxcho, MictlantecutliPhase01, MictlantecutliPhase02} 
			y \textit{MictlantecutliPhase03}.Esta clase se encarga de:  
			\begin{itemize}
				\item Controlar el patrón de movimiento de los enemigos de tipo 
				jefe.
				\item  Verificar la cercanía que tiene el enemigo tipo jefe con el jugador 
				y con los limites del escenario.
			\end{itemize}	
			%=======================================
			\item \textbf{\textit{Xochitonal}:} Esta clase modela el comportamiento del personaje 
			\textit{Xochitónal} (Consultar la ficha de personaje en la Sección de personajes en 
			el documento de diseño). Esta clase hereda su funcionamiento de la clase 
			\textit{BossEnemy}. Esta clase se encarga de:  
			\begin{itemize}
				\item Sincronizar el desplazamiento del \textit{Xochitónal} con su maquina 
				de estados para que modele el patrón de \textit{Xochitónal}.
				\item Toma decisiones en cuanto a los ataques a realizar basándose en su 
				nivel de vida.
			\end{itemize}
			%=======================================
			\item \textbf{\textit{Tepeyollotl}:} Esta clase modela el comportamiento del personaje 
			\textit{Tepeyóllotl} (Consultar la ficha de personaje en la Sección de personajes en 
			el documento de diseño). Esta clase hereda su funcionamiento de la clase 
			\textit{BossEnemy}. Esta clase se encarga de:  
			\begin{itemize}
				\item Sincronizar el desplazamiento del \textit{Tepeyóllotl} con su maquina 
				de estados para que modele el patrón de \textit{Tepeyóllotl}.
				\item Toma decisiones en cuanto a los ataques a realizar basándose en su 
				nivel de vida.
			\end{itemize}
			%=======================================
			\item \textbf{\textit{Itzpapalotl}:} Esta clase modela el comportamiento del personaje 
			\textit{Itzpápalotl} (Consultar la ficha de personaje en la Sección de personajes en 
			el documento de diseño). Esta clase hereda su funcionamiento de la clase 
			\textit{BossEnemy}. Esta clase se encarga de:  
			\begin{itemize}
				\item Sincronizar el desplazamiento del \textit{Itzpápalotl} con su maquina 
				de estados para que modele el patrón de \textit{Itzpápalotl}.
				\item Toma decisiones en cuanto a los ataques a realizar basándose en su 
				nivel de vida.
			\end{itemize}
			%=======================================
			\item \textbf{\textit{Mictlecayotl}:} Esta clase modela el comportamiento del personaje 
			\textit{Mictlecayotl} (Consultar la ficha de personaje en la Sección de personajes en 
			el documento de diseño). Esta clase hereda su funcionamiento de la clase 
			\textit{BossEnemy}. Esta clase se encarga de:  
			\begin{itemize}
				\item Sincronizar el desplazamiento del \textit{Mictlecayotl} con su máquina 
				de estados para que modele el patrón de \textit{Mictlecayotl}.
				\item Toma decisiones en cuanto a los ataques a realizar basándose en su 
				nivel de vida.
			\end{itemize}
			%=======================================
			\item \textbf{\textit{Tlazolteotl}:} Esta clase modela el comportamiento del personaje 
			\textit{Tlazoltéotl} (Consultar la ficha de personaje en la Sección de personajes en 
			el documento de diseño). Esta clase hereda su funcionamiento de la clase 
			\textit{BossEnemy}. Esta clase se encarga de:  
			\begin{itemize}
				\item Sincronizar el desplazamiento del \textit{Tlazoltéotl} con su maquina 
				de estados para que modele el patrón de \textit{Tlazoltéotl}.
				\item Toma decisiones en cuanto a los ataques a realizar basándose en su 
				nivel de vida.
			\end{itemize}
			%=======================================
			\item \textbf{\textit{Itztlacoliuhqui}:} Esta clase modela el comportamiento del personaje 
			\textit{Itztlacoliuhqui} (Consultar la ficha de personaje en la Sección de personajes en 
			el documento de diseño). Esta clase hereda su funcionamiento de la clase 
			\textit{BossEnemy}. Esta clase se encarga de:  
			\begin{itemize}
				\item Sincronizar el desplazamiento del \textit{Itztlacoliuhqui} con su máquina 
				de estados para que modele el patrón de \textit{Itztlacoliuhqui}.
				\item Toma decisiones en cuanto a los ataques a realizar basándose en su 
				nivel de vida.
			\end{itemize}
			%=======================================
			\item \textbf{\textit{Nexoxcho}:} Esta clase modela el comportamiento del personaje 
			\textit{Nexoxcho} (Consultar la ficha de personaje en la Sección de personajes en 
			el documento de diseño). Esta clase hereda su funcionamiento de la clase 
			\textit{BossEnemy}. Esta clase se encarga de:  
			\begin{itemize}
				\item Sincronizar el desplazamiento del \textit{Nexoxcho} con su maquina 
				de estados para que modele el patrón de \textit{Nexoxcho}.
				\item Toma decisiones en cuanto a los ataques a realizar basándose en su 
				nivel de vida.
			\end{itemize}
			%=======================================
			\item \textbf{\textit{MictlantecutliPhase01}:} Esta clase modela el 
			comportamiento del personaje \textit{Mictlantecutli} (Consultar la ficha 
			del personaje en la Sección de personajes en 
			el documento de diseño). Esta clase hereda su funcionamiento de la clase 
			\textit{BossEnemy}. Esta clase se encarga de:  
			\begin{itemize}
				\item Sincronizar el desplazamiento del \textit{Mictlantecutli} con su máquina 
				de estados para que modele el patrón de \textit{Mictlantecutli}.
				\item Toma decisiones en cuanto a los ataques a realizar basándose en su 
				nivel de vida.
			\end{itemize}
			%=======================================
			\item \textbf{\textit{MictlantecutliPhase02}:} Esta clase modela el 
			comportamiento del personaje \textit{Mictlantecutli} (Consultar la ficha 
			del personaje en la Sección de personajes en 
			el documento de diseño). Esta clase hereda su funcionamiento de la clase 
			\textit{BossEnemy}. Esta clase se encarga de:  
			\begin{itemize}
				\item Sincronizar el desplazamiento del \textit{Mictlantecutli} con su maquina 
				de estados para que modele el patrón de \textit{Mictlantecutli}.
				\item Toma decisiones en cuanto a los ataques a realizar basándose en su 
				nivel de vida.
			\end{itemize}
			%=======================================
			\item \textbf{\textit{MictlantecutliPhase03}:} Esta clase modela el 
			comportamiento del personaje \textit{Mictlantecutli} (Consultar la ficha 
			del personaje en la Sección de personajes en 
			el documento de diseño). Esta clase hereda su funcionamiento de la clase 
			\textit{BossEnemy}. Esta clase se encarga de:  
			\begin{itemize}
				\item Sincronizar el desplazamiento del \textit{Mictlantecutli} con su maquina 
				de estados para que modele el patrón de \textit{Mictlantecutli}.
				\item Toma decisiones en cuanto a los ataques a realizar basándose en su 
				nivel de vida.
			\end{itemize}
			%=======================================
			\item \textbf{Checkpoint:} Esta clase permite guardar el progreso con en 
			cuanto objetivos del juego para inicializar al jugador en esa posición
			en caso de que el jugador muera. Los datos que contienen las instancias de
			esta clase solo perduraran mientras el jugador se mantenga dentro del nivel,
			una vez que el jugador abandone el nivel los datos se destruirán y el jugador
			deberá iniciar el nivel desde el inicio.  
			%========================================
			\item \textbf{FollowedCharacter:} Esta clase controla a un personaje que se 
			desplaza siguiendo un patron de movimiento dependiente de un conjunto de objetos 
			que sirven como nodos a sus desplazamiento. El objeto que instancie esta clase 
			siempre se va a desplazar manteniendo una distancia constante de personaje jugable,
			cuando esta distancia aumente, el personaje se detendrá hasta que el personaje 
			jugable vuelva a mantenerse a la distancia aceptable.    
			%========================================
			\item \textbf{GostBulletCtrl:} Esta clase controla el desplazamiento del 
			disparo generado por la clase RedGost.
			%========================================
			\item \textbf{ BulletCtrl:} Esta clase controla el desplazamiento del 
			disparo generado por la clase Player; el valor del atributo de velocidad dependerá 
			del atributo del player que indica hacia donde esta mirando (izquierda o derecha).
		\end{itemize}		
	\subsubsection{Clases Actores}
A continuación se listan las clases auxiliares: 
		\begin{itemize}
			\item \textbf{LoaderScene:} Esta clase permite la transición 
			entre escenas. Esta clase es auxiliar de las clases:
			\begin{itemize}
				\item Controladores de nivel.
				\item Controladores de cinemáticas.
				\item PrincipalMenuCtrl.
				\item SelectLevelMenu.
			\end{itemize}
			%===============================			
			\item \textbf{DestroyWithDelay:} Esta clase destruye al objeto que la 
			instancia después de \textit{n} segundos. Esta clase es instanciada por 
			los GameObjects:
			\begin{itemize}
				\item TonalliBullet.
				\item GostBullet.
			\end{itemize}
			%===============================
			\item \textbf{GroundCollisionCtrl:} Esta clase detecta y gestiona todas las 
			colisiones de suelo del personaje jugable. Esta clase es auxiliar de:
			\begin{itemize}
				\item Player.
			\end{itemize}
			%===============================
			\item \textbf{Dialogue:} Esta clase tiene por atributos el nombre de un 
			personaje y un dialogo con la capacidad de ser serializados para que estos 
			puedan ser almacenados y leídos desde un archivo de tipo JASON. Esta clase 
			es auxiliar de:
			\begin{itemize}
				\item Dialogues.
				\item TalkedCharacter
				\item TalkedCharactersCtrl
			\end{itemize}
			%===============================
			\item \textbf{Dialogues:} Esta clase tiene por atributos una lista de instancias 
			de la clase Dialogue con la capacidad de ser serializados para que estos 
			puedan ser almacenados y leídos desde un archivo de tipo JASON. Esta clase 
			es auxiliar de:
			\begin{itemize}
				\item Dialogues.
				\item TalkedCharacter.
				\item TalkedCharactersCtrl.
			\end{itemize}
			%===============================
			\item \textbf{FileDialogues:} Esta clase lee datos desde un archivo JASON y 
			serializa su contenido para instanciarlo en la clase Dialogues. Esta clase 
			es auxiliar de:
			\begin{itemize}
				\item FileDialogue.
				\item TalkedCharactersCtrl.
				\item CutsceneCtrl.
			\end{itemize}
			%===============================
			\item \textbf{PlayerAudio:} Esta clase contiene todos los archivos de audio que
			corresponden a las acciones del jugador como correr, disparar, morir, etc. El 
			objetivo de esta clase es facilitar la vinculación entre los audios y la clase
			AudioCtrl. 
			\begin{itemize}
				\item AudioCtrl
			\end{itemize}
			
			%===============================
			\item \textbf{AudioFX:} Esta clase contiene todos los archivos de audio que
			corresponden a los efectos de sonido del juego ajenos al jugador. El 
			objetivo de esta clase es facilitar la vinculación entre los audios y la clase
			AudioCtrl. 
			\begin{itemize}
				\item AudioCtrl
			\end{itemize}
			
			%===============================
			\item \textbf{CutsceneData:} Esta clase tiene por atributos datos que permiten 
			llevar el control de las cinemáticas disponibles para ver y de las que faltan 
			por desbloquear. Esta clase contiene atributos que tienen la capacidad de ser 
			serializables con el fin de que puedan ser almacenados en un archivo de tipo
			binario y a su vez instanciados cuando sean leidos desde el archivo.
			\begin{itemize}
				\item GameDataCtrl.
				\item GameData.
			\end{itemize}
			
			%===============================
			\item \textbf{LevelData:} Esta clase tiene por atributos datos que permiten 
			llevar el control de los niveles disponibles para jugar y de los que faltan 
			por desbloquear. Esta clase contiene atributos que tienen la capacidad de ser 
			serializables con el fin de que puedan ser almacenados en un archivo de tipo
			binario y a su vez instanciados cuando sean leídos desde el archivo.
			\begin{itemize}
				\item GameDataCtrl.
				\item GameData.
			\end{itemize}
			
			%===============================
			\item \textbf{GameData:} Esta clase tiene por atributos datos que permiten 
			llevar el control de todo el progreso del jugador. Esta clase contendra 
			atributos de la clase \textit{Player} que serán utilizados por las clases hijas de 
			\textit{LevelCtrl} para incializar la partida. Esta clase contiene atributos que tienen
			 la capacidad de ser serializables con el fin de que puedan ser almacenados 
			 en un archivo de tipo binario y a su vez instanciados cuando sean leidos 
			 desde el archivo.
			\begin{itemize}
				\item GameDataCtrl.
			\end{itemize}
			
\end{itemize}		
		
	\subsection{Comunicación entre clases}
	El videojuego, por su característica de retroalimentación constante con el 
	juagdor, involucra una comunicación en tiempo real entre las clases involucradas. 
	En este apartado se describen la comunicación tres procesos que sea por su 
	complejidad o por su impacto en la funcionalidad del juego son considerados como
	los más relevantes:
		\begin{itemize}
			\item Empezar nueva partida.
			\item Personaje jugable salta.
			\item Batalla contra jefe.
		\end{itemize}	
		

		\subsubsection{Empezar nueva partida}
	Este caso consiste en que el jugador desde la intefaz 01 (ver aparatado
	  \ref{TraReaInterfaces}) empieza una nueva partida.  	
	\begin{itemize}
		\item Clases involucradas
			\begin{itemize}
				\item PrincipalMenuCtrl.
				\item GameDataCtrl.
				\item GameData.
				\item LevelData.
				\item CutsceneData.
				\item LoaderScene.
			\end{itemize}
			
		

\item Trayectoria de comunicación principal.
		\begin{enumerate}
				\item $\lbrack$PrincipalMenuCtrl$\rbrack$ Detectar que el jugador pulsó el 
				botón de Empezar partida.
				\item $\lbrack$PrincipalMenuCtrl$\rbrack$ Mostrar el mensaje donde pide la 
				la confirmación del jugador para realizar los cambios en el archivo de 
				datos de partida.
				\item $\lbrack$PrincipalMenuCtrl$\rbrack$ Detectar que el jugador pulsó el 
				botón de aceptar (Trayectoria A).
				\item $\lbrack$PrincipalMenuCtrl$\rbrack$ Comunicar la confirmación a GameCtrl.
				\item $\lbrack$GameDataCtrl$\rbrack$ Crear una instancia de la clase GameData 
				con los valores predeterminados de inicio.
				\item $\lbrack$GameDataCtrl$\rbrack$ Crear una instancia de la clase LevelData 
				con los valores predeterminados de inicio.
				\item $\lbrack$GameDataCtrl$\rbrack$ Crear una instancia de la clase CutsceneData 
				con los valores predeterminados de inicio.
				\item $\lbrack$GameDataCtrl$\rbrack$ Crear nuevo archivo de partida las 
				intancias de las clases GameData, LevelData, CutsceneData.
				\item $\lbrack$GameDataCtrl$\rbrack$ Comunicar a LoaderScene que el archivo 
				de la partida han sido creado.
				\item $\lbrack$LoaderScene$\rbrack$ Cargar cinemática 1.
		\end{enumerate}
		
	\item Trayectoria A (El jugador pulsa cancelar).
			\begin{enumerate}
				\item[{A.}1] $\lbrack$PrincipalMenuCtrl$\rbrack$ Detectar que el jugador pulsó el 
				botón de cancelar.
				\item[{A.}2] $\lbrack$PrincipalMenuCtrl$\rbrack$ Cerrar mensaje de confirmación.
			\end{enumerate}
	\end{itemize}	
		\subsubsection{Personaje jugable salta}
	Este caso consiste en que el jugador desde un nivel oprime el botón de saltar.  	
	\begin{itemize}
		\item Clases involucradas
			\begin{itemize}
				\item Player.
				\item MobileUICtrl.
			\end{itemize}
			
		\item Trayectoria de comunicación principal.
		\begin{enumerate}
				\item $\lbrack$MobileUICtrl$\rbrack$ Detectar que el jugador pulsó el 
				botón de saltar.
				\item $\lbrack$MobileUICtrl$\rbrack$ Avisa a la clase Player que el botón 
				saltar fue oprimido.
				\item $\lbrack$Player$\rbrack$ Confirma que el atributo isJumping se 
				falso (Trayectoria A).
				\item $\lbrack$Player$\rbrack$ Ejecutar método saltar.
				\item $\lbrack$Player$\rbrack$ Actualizar el valor de atributo isJumping
				en verdadero. 
				\item $\lbrack$Player$\rbrack$ Actualizar el valor de atributo CanDoubleJump 
				en verdadero.
		\end{enumerate}
		
		\item Trayectoria A (Atributo isJumping es verdadero).
			\begin{enumerate}
				\item[{A.}1] $\lbrack$Player$\rbrack$ Confirma que el atributo isJumping es 
				verdadero.
				\item[{A.}2] $\lbrack$Player$\rbrack$ Confirma que el atributo CanDoubleJump 
				es verdadero (Trayectoria B).
				\item[{A.}3] $\lbrack$Player$\rbrack$ Ejecutar método saltar.
				\item $\lbrack$Player$\rbrack$ Actualizar el valor de atributo CanDoubleJump 
				en falso.
			\end{enumerate}
			
		\item Trayectoria B (Atributo CanDoubleJump es falso).
			\begin{enumerate}
				\item[{B.}1] $\lbrack$Player$\rbrack$ No ejecuta método saltar.
			\end{enumerate}
	\end{itemize}	
	\subsubsection{Batalla contra jefe}
	El jugador se enfrenta contra un enemigo de tipo jefe. Para garantizar la 
	generalidad de la comunicación entre clases, se ejemplificara la comunicación 
	con la clase BossEnemy y con la clase LevelCtrl en lugar que con sus 
	respectivas clases hijas.
		
	\begin{itemize}
		\item Clases involucradas
			\begin{itemize}
				\item Player.
				\item MobileUICtrl.
				\item GameDataCtrl.
				\item GameData.
				\item BossEnemy
				\item LevelCtrl.
				\item LoaderScene.
			\end{itemize}
			
		\item Trayectoria de comunicación principal.
		\begin{enumerate}
				\item $\lbrack$LevelCtrl$\rbrack$ Solicitar valores a la clase GameDataCtrl 
				para inicializar a la clase Player. 
				\item $\lbrack$GameDataCtrl$\rbrack$ Recibe la solicitud de la clase 
				LevelCtrl.
				\item $\lbrack$GameDataCtrl$\rbrack$ Leer datos del archivo binario.
				\item $\lbrack$GameDataCtrl$\rbrack$ Serializar datos del archivo binario 
				en una instancia de la clase GameData.
				\item $\lbrack$GameDataCtrl$\rbrack$ Enviar instancia de la clase GameData 
				a la clase LevelCtrl.
				\item $\lbrack$LevelCtrl$\rbrack$ Recibe instancia de la clase GameData.
				\item $\lbrack$LevelCtrl$\rbrack$ Inicializar valores de la clase Player 
				usando los valores de la instancia de GameData.
				\item $\lbrack$LevelCtrl$\rbrack$ Habilitar MobileUICtrl.
				\item Las trayectorias A, B, D y F se ejecutaran de manera paralela.
				\item Repetir el punto anterior mientras atributo isAlive de Player o 
				BossEnemy sea false.
				\item $\lbrack$LevelCtrl$\rbrack$ Canfirmar isAlive de jugador es veradero (Ruta H).
 				\item $\lbrack$LevelCtrl$\rbrack$ Desabilita MobileUICtrl.
 				\item $\lbrack$LevelCtrl$\rbrack$ Enviar nueva instancia de GameData con el 
 				progreso del juego a GameDataCtrl.
 				\item $\lbrack$GameDataCtrl$\rbrack$ Escribir los datos de GameData en el archivo
 				Binario de partida.
 				\item $\lbrack$GameDataCtrl$\rbrack$ Confirmar a LevelCtrl que se ha salvado 
 				el progreso de la partida.
 				\item $\lbrack$GameDataCtrl$\rbrack$ Comunicar a LoaderScene que el archivo 
				de la partida han sido salvado.
				\item $\lbrack$LoaderScene$\rbrack$ Cargar interfaz de Menú de selección de 
				nivel.
		\end{enumerate}
		
		\item Trayectoria A (MobileUICtrl detecta boton oprimido por el usuario).
			\begin{enumerate}
				\item[{B.}2] $\lbrack$MobileUICtrl$\rbrack$ Detectar botones oprimidos por el 
				jugador. 
				\item[{B.}2] $\lbrack$MobileUICtrl$\rbrack$ Confirmar a clase Player sobre que 
				botón se oprimió.
				\item[{B.}3] $\lbrack$Player$\rbrack$ Ejecuta la acción con base al boton oprimido.
			\end{enumerate}
			
		\item Trayectoria B (Player detecta colisión con BossEnemy).
			\begin{enumerate}
				\item[{B.}1] $\lbrack$Player$\rbrack$ Solicitar contidad de daño a BossEnemy.
				\item[{B.}2] $\lbrack$BossEnemy$\rbrack$ Enviar contidad de daño a Player.
				\item[{B.}3] $\lbrack$Player$\rbrack$ Restar cantidad de daño de BossEnemy a
				cantidad de vida de Player.
				\item[{B.}4] $\lbrack$Player$\rbrack$ Confirmar cantidad de vida de Player es 
				mayor a cero (Volver a trayectoria principal en 9; Trayectoria C).
			\end{enumerate}
		
		\item Trayectoria C (Cantidad de vida de Player menor o igual a cero).
			\begin{enumerate}
				\item[{C.}1] $\lbrack$Player$\rbrack$ Volver IsAlive igual a false.
			\end{enumerate}
			
		\item Trayectoria D (BossEnemy detecta colisión con el objeto TonalliBullet).
			\begin{enumerate}
				\item[{D.}1] $\lbrack$BossEnemy$\rbrack$ Solicitar contidad de daño a TonalliBullet.
				\item[{D.}2] $\lbrack$TonalliBullet$\rbrack$ Enviar contidad de daño a BossEnemy.
				\item[{D.}3] $\lbrack$BossEnemy$\rbrack$ Restar cantidad de daño de TonalliBullet a
				cantidad de vida de BossEnemy.
				\item[{D.}4] $\lbrack$BossEnemy$\rbrack$ Confirmar cantidad de vida de BossEnemy es 
				mayor a cero (Volver a trayectoria principal en 9; Trayectoria C).
			\end{enumerate}
			
		\item Trayectoria E (Cantidad de vida de BossEnemy menor o igual a cero).
			\begin{enumerate}
				\item[{E.}1] $\lbrack$BossEnemy$\rbrack$ Volver IsAlive igual a false.
			\end{enumerate}
			
		\item Trayectoria F.
			\begin{enumerate}
				\item[{F.}1] $\lbrack$LevelCtrl$\rbrack$ Actualizar Barra de cantidad de 
				vida jugador. 
				\item[{F.}1] $\lbrack$LevelCtrl$\rbrack$ Actualizar Barra de cantidad de 
				tonalli jugador. 
			\end{enumerate}
		
		\item Trayectoria H (Atributo isAlive de Player es falso).
			\begin{enumerate}
				\item[{F.}1] regresar al punto 1 de ruta principal. 
			\end{enumerate}
	\end{itemize}			
