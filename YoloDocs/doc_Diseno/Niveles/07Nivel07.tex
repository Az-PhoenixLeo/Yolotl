\section{Nivel 7}  \label{Nivel:Niv07}
	\subsection{Título del nivel}
	Castigo.
	\subsection{Encuentro}
	Este nivel estará disponible después de vencer al jefe del noveno sexto (ver apartado \ref{Nivel:Niv06}).
	\subsection{Descripción}
	Malinalli y Xólotl se adentran en los dominios de Itztlacoliuhqui. Xolotl sabe que deben de avanzar con cautela pues Itztlacoliuhqui los espera con deseos de vengar la muerte de Itzpápalotl. Malinalli y Xolotl deberán cruzar Temiminalóyan y los peligros que trae consigo.
	\subsection{Objetivos}
	\begin{itemize}
		\item Atravesar la zona de plataformas. En este nivel lloverán flechas de manera periódica, por lo que el jugador deberá de avanzar de forma cautelosa y refugiarse en las zonas seguras durante la lluvia de flechas. En la parte superior derecha de la pantalla habrá una barra que se ira llenando, cuando la barra este llena iniciara la lluvia de flechas, el jugador podrá atrasar el llenado de la barra destruyendo las estatuas del sol que están distribuidas por el nivel.  
		\item Derrotar a Itztlacoliuhqui. En la batalla contra Itztlacoliuhqui, éste incrementara su tamaño transformándose en una versión gigantesca de sí mismo.  Por lo que el jugador deberá de subir una serie de plataformas para atacar la cabeza de Itztlacoliuhqui, siendo este punto el único en donde Itztlacoliuhqui disminuirá su barra de vida. Las plataformas que hay en esta batalla serán del tipo de las que aparecen y desaparecen con el tiempo. La barra que indica cuando llueven flechas se mantendrá, pero se llenará más rápido que en la zona de plataformas.
	\end{itemize}
	\subsection{Progreso}
	Al terminar el nivel el jugador:
\begin{itemize}
        \item Habrá incrementado la cantidad de tonalli de Malinalli. 
        \item Desbloqueara las siguientes cinemáticas:
\begin{itemize}
        \item Cinemática 33 (ver apartado \ref{Cin:Cinematica33}). 
        \item Cinemática 34 (ver apartado \ref{Cin:Cinematica34}).
        \item Cinemática 35 (ver apartado \ref{Cin:Cinematica35}).
\end{itemize}
        \item Desbloqueará El nivel 8 (ver apartado  \ref{Nivel:Niv08}) del juego en el menú seleccionable (ver apartado \ref{inter:interfaz03}).
\end{itemize} 
	\subsection{Enemigos}
	\begin{itemize}
		\item Fantasmas morados (ver apartado \ref{per:fantasmaM}).
		\item Fantasmas rojos (ver apartado \ref{per:fantasmaR}).
		\item Itztlacoliuhqui (ver apartado \ref{per:itztlacoliuhqui}).
	\end{itemize}
	\subsection{Items}
\begin{itemize}
        \item   Cacao (ver apartado \ref{item:cacao}).
        \item Flor de Vainilla (ver apartado \ref{item:vainilla}).
\end{itemize}
	\subsection{Personajes}
	\begin{itemize}
		\item Malinalli. Ver en \ref{per:malinalli}.
		
		\item Xolotl. Ver en \ref{per:xolotl}.
		
		\item Itztlacoliuhqui. Ver en \ref{per:itztlacoliuhqui}.
	\end{itemize}
	\subsection{Escenario}
\begin{itemize} 
	\item Fondo: A lo lejos se verá un volcán en erupción. El cielo es un perpetuo atardecer. Se observan montañas sin vegetación y una amplia meseta seca, cuyo suelo se muestra cuarteado denotando que la zona está pasando por un periodo de sequía. 
	\item Suelo: Piso rocoso y rocoso agrietado.
	\item Obstáculos:
	\begin{itemize}
	\item Plataforma móvil (ver apartado \ref{obs.plataformaM}).
			\item Plataforma que cae (ver apartado \ref{obs.plataformaD}).
			\item Plataforma que desaparece (ver apartado \ref{obs.PlatDes}).
			\item Lluvia de flechas (ver apartado \ref{obs.lluviaF}).
\end{itemize}	 
	
	\item Objetos de fondo: Sin objetos de fondo.
\end{itemize}	
	\subsection{Referencia a BGM y SFX}
	\begin{itemize}
		\item BGM
			\begin{itemize}
				\item Música plataforma séptimo nivel (ver apartado \ref{Musica:N07_ZN01}).
				\item Música jefe séptimo nivel (ver apartado \ref{Musica:N07_ZN02}).
			\end{itemize}
		\item SFX
			\begin{itemize}
				\item Pasos (ver aparatado \ref{SFX:Pasos})
				\item Viento (ver apartado \ref{SFX:Viento}).
				\item Golpe (ver apartado \ref{SFX:golpe}).
			\end{itemize}
			\item Silbido (ver apartado \ref{SFX:silbido}).
	\end{itemize}
	\subsection{Referencia a FX}
	\begin{itemize}
		\item Explosión de energía tonalli rojo (ver apartado \ref{FX:ExpTonR})
	\item Explosión de energía tonalli verde (ver apartado \ref{FX:ExpTonV})
	\item Temblor (ver apartado \ref{FX:temblor}).
	\item Cámara obscura (ver apartado \ref{FX:CamObs}).
	\end{itemize}
	