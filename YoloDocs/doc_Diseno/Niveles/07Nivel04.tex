\section{Nivel 4} \label{Nivel:Niv04}
        \subsection{Título del nivel}
        Itztépetl: Montaña de obsidiana.
        \subsection{Encuentro}
Tercer nivel del inframundo, se desbloquea derrotando a Tepeyollótl en el tercer nivel (ver apartado \ref{Nivel:Niv03}).
        \subsection{Descripción}
        Luego de la huida de Tepeyollótl, Malinalli y Xólotl iniciarán una carrera contra el tiempo para poder llegar a su segunda guarida y evitar que contacte a Tezcatlipoca pero primero deberán superar el Itztépetl, hogar de Itzpapálotl. Así deberán adentrarse en las profundidades de una cueva para encontrar templo subterráneo donde se esconde Itzpapálotl. 
        \subsection{Objetivos}
En este nivel el jugador deberá:        
\begin{itemize}
        \item Descubrir el camino correcto hacia el templo de Itzpapálotl, el jugador podrá explorar de manera horizontal y vertical el nivel.
        \item Superar las diferentes plataformas y obstáculos que hay en el laberinto. 
        \item Derrotar a los enemigos que hay en el laberinto.
        \item Derrotar a Itzpapálotl.
\end{itemize}
        \subsection{Progreso}
Al final del nivel el jugador:
\begin{itemize}
        \item Aumentará el valor de la salud de Malinalli.
        \item Desbloquear las siguientes cinemáticas:
		\begin{itemize}
			\item Cinemática 16 (ver apartado \ref{Cin:Cinematica16}).
			\item Cinemática 17 (ver apartado \ref{Cin:Cinematica17}).
			\item Cinemática 18 (ver apartado \ref{Cin:Cinematica18}).
			\item Cinemática 19 (ver apartado \ref{Cin:Cinematica19}).
			\item Cinemática 20 (ver apartado \ref{Cin:Cinematica20}).
			\item Cinemática 21 (ver apartado \ref{Cin:Cinematica21}).
			\item Cinemática 22 (ver apartado \ref{Cin:Cinematica22}).
		\end{itemize}
        \item Desbloquear El nivel 5 (ver apartado \ref{Nivel:Niv03}) del juego en el menú seleccionable (ver apartado \ref{inter:interfaz03}).
        \subsection{Enemigos}
                \begin{itemize}
                        \item Fantasma rojo. Ver en \ref{per.fantasmaR}. 
            \begin{itemize}
				\item Animación fuego.
				\item Animación disparo.
			\end{itemize}
                        \item Picos de obsidiana. Ver en \ref{obs.piedrasF}.
                        \item Itzpapalotl. Ver en \ref{per.itzpapalotl}.
                \end{itemize}
        \subsection{Items}
                \begin{itemize}
                        \item   Cacao.
                        \item Flor de Vainilla.
                \end{itemize}
        \subsection{Personajes}
        \begin{itemize}
                \item Malinalli. Ver en \ref{per.malinalli}.
                \begin{itemize}
                        \item Animación correr.
                        \item Animación saltar.
                        \item Animación correr caracola.
                        \item Animación saltar caracola.
                        \item Animación normal.
                        \item Animación recibir daño.
						\item Animación morir.
                \end{itemize}
                \item Xolotl.Ver en \ref{per.xolotl}.
                	\begin{itemize}
						\item Animación salto.
						\item Animación normal.
					\end{itemize}
                \item Itzpapalotl. Ver en \ref{per.itzpapalotl}.
                \begin{itemize}
                        \item Animación disparar fuego.
                        \item Animación embestida.
                        \item Animación caminar.
                        \item Animación desvanecerse.
                        \item Animación aparecer.
                \end{itemize}
        \end{itemize}
\subsection{Escenario}
\begin{itemize} 
        \item Fondo:
                \begin{itemize}
                        \item Zona de plataformas:
\\
\par
El nivel esta ubicado en el subterráneo, el fondo deberá parecer a una mina con cristales de luz verde que salen del suelo y de la pared.
                        \item Zona del jefe:
Es el interior del templo,el templos solo es iluminado por los cristales verdes. La batalla se desarrollara en un cuarto de entrenamiento por lo que habrán diferentes armas en las paredes.
                \end{itemize}
        \item Suelo:
                \begin{itemize}
                        \item Suelo rocoso: Para la zona de las plataformas.
                        \item Suelo pavimentado: Zona jefe.
                \end{itemize}
	  \item Obstáculos:
                \begin{itemize}
                        \item Viento.
                \end{itemize}
        \item Objetos de fondo:
                \begin{itemize}
                        \item Mariposa: Esta se encontrara si se esta siguiendo el camino correcto para llegar al templo.
                \end{itemize}
\end{itemize}   

        \subsection{Música y efectos de sonido}
                \begin{itemize}
                        \item Temas del juego.
                \begin{itemize}
                        \item Música plataforma cuarto nivel (ver apartado \ref{Musica:N04_ZN01}).
				\item Música jefe cuarto nivel (ver apartado \ref{Musica:N04_ZN02})
                \end{itemize}
				\item Efectos de sonido.
                \begin{itemize}
                        \item 
                        \item Viento (ver apartado\ref{SFX:Viento}).
                        \item Sonido de fuego.				
                \end{itemize}
                \end{itemize}

        \subsection{Referencia a BGM y SFX}
\begin{itemize}    
    \item Explosión de burbuja (Ver apartado \ref{FX:ExpAgua}).
	\item Explosión de energía tonalli rojo (Ver apartado \ref{FX:ExpTonR})
        \item Haz de luz \ref{FX:HazLuz}
\end{itemize}