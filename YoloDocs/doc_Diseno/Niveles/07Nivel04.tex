\section{Nivel 4} \label{Nivel:Niv04}
        \subsection{Título del nivel}
        Alas de obsidiana.
        \subsection{Encuentro}
Tercer nivel del inframundo, se desbloquea derrotando a Tepeyollótl en el tercer nivel (ver apartado \ref{Nivel:Niv03}).
        \subsection{Descripción}
        Luego de la huida de Tepeyollótl, Malinalli y Xólotl iniciarán una carrera contra el tiempo para poder llegar a su segunda guarida y evitar que contacte a Tezcatlipoca pero primero deberán superar el Itztépetl, hogar de Itzpapálotl. Así deberán adentrarse en las profundidades de una cueva para encontrar templo subterráneo donde se esconde Itzpapálotl. 
        \subsection{Objetivos}
En este nivel el jugador deberá:        
\begin{itemize}
        \item Descubrir el camino correcto hacia el templo de Itzpapálotl, el jugador podrá explorar de manera horizontal y vertical el nivel. A manera de ayudarle al jugador a encontrar el camino, habrá pequeñas mariposas azules en las zonas del mapa que correspondan al camino hacia la guarida de Itzpapálotl. 
        \item Encontrar tres llaves que permiten abrir la entrada a la guarida de Itzpapálotl. Cada llave estará oculta en una zona del mapa custodiada por un enemigo. El jugador deberá de derrotar al enemigo que protege la llave, una vez derrotado el enemigo, éste explotará dejando caer la llave en la posición en la que se encontraba al momento de ser derrotado. Una vez que el jugador derrote al enemigo, deberá de colisionar con la llave para obtenerla. Cuando el jugador colisione con la llave el contador que lleva el control de cuantas llaves ha recogido el jugador se actualizara en uno. El contador de las llaves se ubicará en la parte superior derecha de la pantalla precedido por el icono de la llave. 
        \item Superar las diferentes plataformas y obstáculos que hay en el laberinto. 
        \item Derrotar a los enemigos que hay en el laberinto.
        \item Derrotar a Itzpapálotl.
\end{itemize}
        \subsection{Progreso}
Al final del nivel el jugador:
\begin{itemize}
        \item Aumentará la cantidad de vida de Malinalli.
        \item Desbloquear las siguientes cinemáticas:
		\begin{itemize}
			\item Cinemática 16 (ver apartado \ref{Cin:Cinematica16}).
			\item Cinemática 17 (ver apartado \ref{Cin:Cinematica17}).
			\item Cinemática 18 (ver apartado \ref{Cin:Cinematica18}).
			\item Cinemática 19 (ver apartado \ref{Cin:Cinematica19}).
			\item Cinemática 20 (ver apartado \ref{Cin:Cinematica20}).
			\item Cinemática 21 (ver apartado \ref{Cin:Cinematica21}).
			\item Cinemática 22 (ver apartado \ref{Cin:Cinematica22}).
		\end{itemize}
        \item Desbloquear El nivel 5 (ver apartado \ref{Nivel:Niv05}) del juego en el menú seleccionable (ver apartado \ref{inter:interfaz03}).
\end{itemize}        
        \subsection{Enemigos}
                \begin{itemize}
                        \item Fantasma rojo. Ver en \ref{per:fantasmaR}. 
            \begin{itemize}
				\item Animación fuego.
				\item Animación disparo.
			\end{itemize}
			\item Fantasma morado (ver apartado \ref{per:fantasmaM}).
			\begin{itemize}
				\item Animación fuego.
			\end{itemize}
            \item Picos de obsidiana (ver apartado \ref{obs:piedrasF}).
             \item Itzpapalotl (ver apartado \ref{per:itzpapalotl}).
                \end{itemize}
        \subsection{Items}
                \begin{itemize}
                        \item   Cacao (ver apartado \ref{item:cacao}).
                        \item Flor de Vainilla (ver apartado \ref{item:vainilla}).
                        \item Llave a la guarida de Itzpapálotl.
                        \item Mariposa azul (ver apartado \ref{item:Mariposa})
                \end{itemize}
        \subsection{Personajes}
        \begin{itemize}
                \item Malinalli (ver apartado \ref{per:malinalli}).
                \begin{itemize}
                        \item Animación correr.
                        \item Animación saltar.
                        \item Animación correr caracola.
                        \item Animación saltar caracola.
                        \item Animación normal.
                        \item Animación recibir daño.
						\item Animación morir.
                \end{itemize}
                \item Xolotl (ver apartado \ref{per:xolotl}).
                	\begin{itemize}
						\item Animación salto.
						\item Animación normal.
					\end{itemize}
                \item Itzpapalotl (ver apartado \ref{per:itzpapalotl}).
                \begin{itemize}
                        \item Animación disparar fuego.
                        \item Animación embestida.
                        \item Animación caminar.
                        \item Animación desvanecerse.
                        \item Animación aparecer.
                \end{itemize}
        \end{itemize}
\subsection{Escenario}
\begin{itemize} 
        \item Fondo:
                \begin{itemize}
                        \item Zona de plataformas:
\\
\par
El nivel esta ubicado en el subterráneo, el fondo deberá parecer a una mina con cristales de luz verde que salen del suelo y de la pared.
                        \item Zona del jefe:
Es el interior del templo,el templos solo es iluminado por los cristales verdes. La batalla se desarrollara en un cuarto de entrenamiento por lo que habrán diferentes armas en las paredes.
                \end{itemize}
        \item Suelo:
                \begin{itemize}
                        \item Suelo rocoso: Para la zona de las plataformas.
                        \item Suelo pavimentado: Zona jefe.
                \end{itemize}
	  \item Obstáculos:
                \begin{itemize}
                        \item Viento (ver apartado \ref{obs.vientoT}).
                \end{itemize}
        \item Objetos de fondo:
                \begin{itemize}
                        \item Cristal verde: Cristal de luz verde que ilumina algunas zonas de la cueva.
                \end{itemize}
\end{itemize}   

        \subsection{Referencia a BGM y SFX}
                \begin{itemize}
                        \item BGM.
                \begin{itemize}
                        \item Música plataforma cuarto nivel (ver apartado \ref{Musica:N04_ZN01}).
				\item Música jefe cuarto nivel (ver apartado \ref{Musica:N04_ZN02})
                \end{itemize}
				\item SFX.
                \begin{itemize}
                        \item Aleteo de alas (ver apartado \ref{SFX:Aleteo}).
                        \item Viento (ver apartado\ref{SFX:Viento}).
                        \item Sonido de fuego (ver apartado \ref{SFX:Fuego}).			
                \end{itemize}
                \end{itemize}

        \subsection{Referencia a FX}
\begin{itemize}    
	\item Explosión de energía tonalli rojo (Ver apartado \ref{FX:ExpTonR})
	\item Explosión de energía tonalli verde (Ver apartado \ref{FX:ExpTonV})
    \item Haz de luz \ref{FX:HazLuz}
    
\end{itemize}