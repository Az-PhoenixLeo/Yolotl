\section{Nivel 3}\label{Nivel:Niv03}
	\subsection{Título del nivel}
	Tépetl Monamicyan: Lugar en que se juntan las montañas	
	\subsection{Encuentro}
Segundo nivel del inframundo, solo se puede acceder a él si se ha derrotado al jefe del segundo nivel.
	\subsection{Descripción}
	Tras haber superado el primer nivel del Mictlán, Xólotl y Malinalli han llegado al Monamicyan, nivel gobernado por Tepeyóllotl. En este nivel Malinalli debera superar diferentes plataformas para llegar a la guarida de Tepeyóllotl y así obtener el poder del segundo guardián del infreamundo.
	\subsection{Objetivos}
	El jugador deberá:
\begin{itemize}
	\item Superar diferentes plataformas. En este nivel la exploración se hará de manera vertical, por lo que el espacio horizontal del nivel se vera reducido. A lo largo del nivel el jugador deberá coordinar los saltos de Malinalli para superar los obstáculos del nivel.
	\item Derrotar enemigos que protegen diferentes zonas del mapa, sin morir; cada enemigo infringirá una cantidad diferente de daño.
	\item Derrotar al jefe del nivel: Tepeyóllotl. Tepeyóllotl se encontrará protegido por su armadura de piedra por lo que sus ataques serán lentos, pero infringirán gran daño.
\end{itemize}
	\subsection{Progreso}
Al final del nivel el jugador:
\begin{itemize}
	\item Mejorará la cantidad de energía espiritual de Malinalli lo que permitirá usar más ataques con la caracola.
	\item Podrá ver las siguienteDesbloqueara una cinemática sobre el pasado de Malinallis cinemáticas: 
		\begin{itemize}
			\item Cinemática 12 (ver apartado \ref{Cin:Cinematica12}).
			\item Cinemática 13 (ver apartado \ref{Cin:Cinematica13}).
			\item Cinemática 14 (ver apartado \ref{Cin:Cinematica14}).
		\end{itemize}
	\item Desbloqueara el nivel cuatro (ver apartado \ref{Nivel:Niv04}) en el menú de selección de nivel (ver apartado \ref{inter:interfaz03}).
\end{itemize} 
	\subsection{Enemigos}
\begin{itemize}
	\item 	Fantasma rojo (ver apartado \ref{per:fantasmaR}).
		\begin{itemize}
				\item Animación fuego.
				\item Animación disparo.
			\end{itemize}
	\item Armadillo (ver apartado \ref{per:armadillo}).
		\begin{itemize}
			\item Animación de sacar picos.
			\item Animación rodar. 
		\end{itemize}
	\item Roca (ver apartado \ref{obs.rocas}).
		\begin{itemize}
			\item Animación romperse.
			\item Animación rodar.
		\end{itemize}
	\item Tepeyóllotl (ver apartado \ref{per:tepeyollotl}). 

\end{itemize}
	\subsection{Items}
\begin{itemize}
	\item 	Cacao.
	\item	Flor de vainilla.
\end{itemize}
	\subsection{Personajes}
\begin{itemize}
	\item Malinalli (ver apartado \ref{per:malinalli}).
		\begin{itemize}
			\item Animación correr.
			\item Animación saltar.
			\item Animación correr con caracola.
			\item Animación saltar con caracola.
			\item Animación normal.
			\item Animación recibir daño.
			\item Animación morir.
		\end{itemize}
	\item Xolotl (ver apartado \ref{per:xolotl}).
		\begin{itemize}
				\item Animación salto.
				\item Animación normal/nado.
		\end{itemize}
	\item Tepeyóllotl (ver apartado\ref{per:tepeyollotl}).
		\begin{itemize}
			\item Animación correr.
			\item Animación saltar.
			\item Animación rugir.
			\item Animación recibir daño.
			\item Animación morir.
		\end{itemize}
\end{itemize}
\subsection{Escenario}
\begin{itemize} 
	\item Fondo:
\begin{itemize}
	\item Bosque frondoso desde una perspectiva de tres puntos de fuga visto desde arriba. Tendrá un cielo azul despejado. A lo lejos se verá una montaña muy alta, con un templo en la cima.
\end{itemize}
	\item Suelo:
		\begin{itemize}
			\item Suelo con pasto.
		\end{itemize}
	\item Obstáculos:
		\begin{itemize}
			\item Plataforma móvil (ver apartado \ref{obs.plataformaM}).
			\item Plataforma débil que cae (ver apartado \ref{obs.plataformaD}).
		\end{itemize}
	\item Objetos de fondo:
		\begin{itemize}
			\item Arbusto.
			\item Árbol frágil.
		\end{itemize}
\end{itemize}	
	\subsection{Referencia a BGM y SFX}
\begin{itemize}
	\item BGM:
		\begin{itemize}
			\item Música plataforma tercer nivel (ver apartado\ref{Musica:N03_ZN01}).
			\item Nombre: Música jefe tercer nivel (ver apartado \ref{Musica:N03_ZN02}).
		\end{itemize}
	\item SFX:
		\begin{itemize}
			\item Viento (ver apartado \ref{SFX:Viento}).
			\item Rugido (ver apartado \ref{SFX:Rugido}).
			\item Roca estrellándose (ver apartado\ref{SFX:RocaEs}).
			\item Pasos (ver apartado \ref{SFX:Pasos})
		\end{itemize}
\end{itemize} 
	\subsection{Referencia FX}
\begin{itemize}
	\item Explosión de burbuja (Ver apartado \ref{FX:ExpAgua}).
	\item Explosión de energía tonalli rojo (Ver apartado \ref{FX:ExpTonR}).
\end{itemize}