\section{Nivel 10} \label{Nivel:Niv10}
	\subsection{Título del nivel}
	El rey del Mictlán.
	\subsection{Encuentro}
	Este nivel estará disponible después de vencer al jefe del noveno nivel (ver apartado \ref{Nivel:Niv09}). 
	\subsection{Descripción}
	El jugador caminará por un pasillo dentro del palacio de Mictlantecutli hasta llegar a la sala del trono en donde lo enfrentará.
	\subsection{Objetivos}
	El jugador deberá:
	\begin{itemize}
		\item Llegar a la sala del trono de Mictlantecutli. El jugador atravesará un pasillo recto sin enemigos ni plataformas. 
		\item Derrotar a Mictlantecutli. La batalla contra Mictlantecutlin estará dividida en tres etapas, en cada una Mictlantecutli cambiará de forma y su fuerza se incrementará. Cada vez que el jugador le baje un cuarto de la cantidad de vida total de Mictlantecutlin aparecerán los items  de grano de cacao (ver apartado \ref{item:cacao}) y flor de vainilla (ver apartado \ref{item:vainilla}).
		\begin{itemize}
			\item En la primera etapa Mictlantecutli se enfrentara al jugador en su forma normal (ver figura ), sus ataques serán rápidos. 
			\item En su segunda etapa Mictlantecutli toma una forma gigantesca y utiliza un patrón de ataque parecido al de Itztlacoliuhqui en el séptimo nivel (ver apartado \ref {Nivel:Niv07}). 
			\item En su tercera y última etapa, Mictlantecutlin recupera su tamaño normal, su forma vuelve a ser la de antes pero ahora se muestra de color verde como si fuera un espectro, el suelo desaparece y aparecen diferentes plataformas. Esta batalla seguirá la mecánica de que las plataformas aparezcan y desaparezcan o se muevan como en la batalla contra Mictlecayotl en el quinto nivel (ver apartado \ref {Nivel:Niv05}). En esta etapa las habilidades de Mictlantecutlin serán una combinación de las habilidades más fuertes de los guardianes de los niveles anteriores.
		\end{itemize}
	\end{itemize}
	\subsection{Progreso}
	Desbloqueara las siguientes cinemáticas:
\begin{itemize}
        \item Cinemática 47 (ver apartado \ref{Cin:Cinematica47}).
        \item Habrá terminado el juego.
\end{itemize}
	\subsection{Enemigos}
	\begin{itemize}
		\item Mictlantecutli (ver apartado \ref{per.mictlantecutli}).
	\end{itemize}
	\subsection{Items}
\begin{itemize}
        \item   Cacao (ver apartado \ref{item:cacao}).
        \item Flor de Vainilla (ver apartado \ref{item:vainilla}).
\end{itemize}
	\subsection{Personajes} 
	\begin{itemize}
		\item Malinalli (ver apartado \ref{per.malinalli}).
		\item Xolotl (ver apartado \ref{per.xolotl}).
		
		\item Mictlantecutli (ver apartado \ref{per.mictlantecutli}).
		
	\end{itemize}
	\subsection{Escenario}
\begin{itemize} 
	\item Fondo: 
	\begin{itemize}
		\item Interior del palacio del Rey del Mictlán: pasillo iluminado por antorchas de oro con cráneos incrustados. Por las ventanas se puede apreciar un cielo nocturno despejado.
		\item Sala del trono: En el centro se encuentra su trono. Hay nueve antorchas iluminando la sala, cada antorcha esta decorada con algún elemento que haga remembranza a los guardianes del Mictlán. 
	\end{itemize}
	\item Suelo: Suelo pavimentado.
	\item Obstáculos:
	\begin{itemize}
			\item Plataforma móvil (ver apartado \ref{obs.plataformaM}).
			\item Plataforma que cae (ver apartado \ref{obs.plataformaD}).
			\item Piso resbaladizo. (ver apartado \ref{obs.pisoC}).
			\item Plataforma que desaparece (ver apartado \ref{obs.PlatDes}).
		\end{itemize}
	\item Objetos de fondo: 
		\begin{itemize}
			\item Antorchas.
		\end{itemize}
\end{itemize}	
	\subsection{Referencia a BGM y SFX}
	\begin{itemize}
		\item BGM.
			\begin{itemize}
				\item Música pasillo del palacio decimo nivel (ver apartado\ref{Musica:N10_ZN01}).
				\item Música jefe noveno nivel ver apartado\ref{Musica:N09_ZN02}).
			\end{itemize}
		\item SFX.
			\begin{itemize}
				\item Pasos (ver apartado\ref{SFX:Pasos}).
				\item Explosión de agua (ver apartado\ref{SFX:ExpAgua}).
				\item Roca estrellándose (ver apartado\ref{SFX:RocaEs}).
				\item Sonido de fuego (ver apartado\ref{SFX:Fuego}).
				\item Explosión de huesos (ver apartado\ref{SFX:exHuesos}).
			\end{itemize}
	\end{itemize}
	\subsection{Referencia a FX}
	\begin{itemize}
		\item Circulo de luz  (ver apartado\ref{FX:CirLuz}).
		\item Explosión de burbuja (ver apartado\ref{FX:ExpAgua}).
		\item Explosión de energía tonalli rojo (ver apartado \ref{FX:ExpTonR}).
		\item Explosión de energía tonalli verde (ver apartado \ref{FX:ExpTonV}).
		\item Explosion rocas  (ver apartado\ref{FX:ExpRoc}).
		\item Cámara obscura (ver apartado \ref{FX:CamObs}).
		\item Explosión de hueso (ver apartado \ref{FX:exHuesos}).
	\end{itemize}
