\section{Nivel 5}
        \subsection{Título del nivel}
        Cehuelóyan: Lugar donde hay mucha nieve
        \subsection{Encuentro}
        Este nivel se desbloqueara después de haber terminado el Nivel 4.
        \subsection{Descripción}
Malinalli y Xólotl han llegado al Cehuelóyan, nivel del Mictlán custodiad por Mictlecayotl. Lo primero en recibirlos es el gélido viento de la zona pero no hay tiempo para lamentar el clima. Malinalli deberá mantener ambos ojos bien abiertos y coordinar cada movimiento ya que cada centímetro de nieve puede ser un lugar seguro o sitio frágil que caerá al vació.      
        \subsection{Objetivos}
En este nivel el jugador deberá:
\begin{itemize}
        \item Superar las diferentes plataformas. El nivel estará más centrado en las plataformas que en derrotar enemigos. Nuevamente el juego vuelve a implementar una progresión horizontal para la exploración. 
        \item  Derrotar a Mictlecayotl. La batalla contra el jefe de zona se desarrollará en una serie de plataformas que modificaran su posición cada determinado tiempo o desaparecerán . 
\end{itemize}


        \subsection{Progreso}
        Al terminar el nivel el jugador:
\begin{itemize}
        \item Habrá incrementado la energía espiritual de Malinalli. 
        \item Desbloqueara las siguientes cinemáticas:
\begin{itemize}
        \item Cinemática 24. 
        \item Cinemática 25.
        \item Cinemática 26.
        \item Cinemática 27.
\end{itemize}
        \item Desbloqueará El nivel 5 del juego en el menú seleccionable.
\end{itemize}

        \subsection{Enemigos}
\begin{itemize}
        \item Fantasma morado. Ver en \ref{per.fantasmaM}.
        \begin{itemize}
				\item Animación fuego.
		\end{itemize}
        \item Chara enana. Ver en ??.
        \begin{itemize}
				\item Animación vuelo.
				\item Animación caída en picada.
		\end{itemize}
		\item Mictlecayotl. Ver en \ref{per.mictlecayotl}.
\begin{itemize}
        \item   Animación vuelo.
\end{itemize}			
\end{itemize}
        \subsection{Items}
\begin{itemize}
        \item Cacao.
        \item Flor de vainilla.
\end{itemize}
        \subsection{Personajes}
        \begin{itemize}
                \item Malinalli. Ver en \ref{per.malinalli}.
                \begin{itemize}
                        \item Animación correr.
                        \item Animación saltar.
                        \item Animación correr caracola.
                        \item Animación saltar caracola.
                        \item Animación normal.
                        \item Animación recibir daño.
						\item Animación morir.
                \end{itemize} 
                \item Xolotl. Ver en \ref{per.xolotl}.
                \begin{itemize}
					\item Animación salto.
					\item Animación normal.
				\end{itemize}
                \item Mictlecayotl. Ver en \ref{per.mictlecayotl}.
                \begin{itemize}
                        \item Animación disparar viento.
                        \item Animación invocar tornado.
                        \item Animación correr caracola.
                        \item Animación saltar caracola.
                        \item Animación normal.
                        \item Animación recibir daño.
						\item Animación morir.
                \end{itemize} 
        \end{itemize}
        \subsection{Escenario}
\begin{itemize} 
        \item Fondo:
\begin{itemize} 
        \item El fondo son montañas cubiertas de nieve. El cielo es nublado sin la posibilidad de estar seguro en que momento del día es, las nubes de este nivel son de color azul claro, como la nieve.
\end{itemize} 
        \item Suelo:
\begin{itemize} 
        \item Suelo cubierto de nieve.
\end{itemize} 
        \item Obstáculos:
			\item Plataforma móvil. Ver en \ref{obs.plataformaM}.
			\item Plataforma que cae. Ver en \ref{obs.plataformaD}.
			\item Piso resbaladizo. Ver en \ref{obs.pisoC}.
			\item Picos de hielo. Ver en \ref{obs.piedrasF}.
			\item Bola de nieve. Ver en \ref{obs.bolasN}.
		\end{itemize}
        \item Objetos de fondo: Sin objetos de fondo.
\end{itemize}   
        \subsection{Música y efectos de sonido}
\begin{itemize} 
        \item Temas del juego:
\begin{itemize} 
        \item Zona de plataformas: Este tema se caracterizara por tener una fuerte carga de nostalgia y renuncia hacia las personas que se aman, ya que representa la nostalgia que siente Mictlecayotl por todo lo que perdió cuando fue condenada a estar en el Mictlán por tratar de asesinar a Tezcatlipoca. El tema puede contener algunas notas como de caja de música y coros.
        \item Batalla contra jefe: Este tema debe de contener un poco de la nostalgia pero a su vez debe de denotar fuerza y violencia. Su ritmo debe de ser acelerado pues refleja la personalidad de fuerte e innegociable de Mictlecayotl.
\end{itemize}
        \item Efectos de sonido:
\begin{itemize} 
        \item Viento soplando.
        \item Hielo resquebrajándose.
        \item Pasos sobre el hielo o la nieve.	
\end{itemize}

        \subsection{Referencia a BGM y SFX}
\begin{itemize} 
        \item Ventisca de nieve: sera una ventisca suave pero constante en todo el nivel.
\item Explosión de energía tonalli: Este efecto se presentara en dos colores:
        \begin{itemize}
                \item Explosión de energía tonalli rojo: efecto cuando los enemigos normales son derrotados.
                \item Explosión de energía tonalli rojo: Efecto cuando la cantidad de salud de Malinalli llega a cero.
        \end{itemize}
\end{itemize} 
\end{itemize} 