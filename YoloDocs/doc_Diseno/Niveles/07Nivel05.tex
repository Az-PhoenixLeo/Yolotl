\section{Nivel 5} \label{Nivel:Niv05}
        \subsection{Título del nivel}
        El viento del norte.
        \subsection{Encuentro}
       Este nivel estará disponible después de vencer al jefe del cuarto nivel (ver apartado \ref{Nivel:Niv04}).
        \subsection{Descripción}
Malinalli y Xólotl han llegado al Cehuelóyan, nivel del Mictlán custodiad por Mictlecayotl. Lo primero en recibirlos es el gélido viento de la zona pero no hay tiempo para lamentar el clima. Malinalli deberá mantener ambos ojos bien abiertos y coordinar cada movimiento ya que cada centímetro de nieve puede ser un lugar seguro o sitio frágil que caerá al vació.      
        \subsection{Objetivos}
En este nivel el jugador deberá:
\begin{itemize}
        \item Superar las diferentes plataformas. El nivel estará más centrado en las plataformas que en derrotar enemigos. Nuevamente el juego vuelve a implementar una progresión horizontal para la exploración. 
        \item  Derrotar a Mictlecayotl. La batalla contra el jefe de zona se desarrollará en una serie de plataformas que modificaran su posición cada determinado tiempo o desaparecerán. El jugador deberá saltar constantemente para no caer de la plataformas mientras ataca a Mictlecayotl y a su vez deberá evitar los ataques de la diosa.
\end{itemize}


        \subsection{Progreso}
        Al terminar el nivel el jugador:
\begin{itemize}
        \item Habrá incrementado la cantidad de tonalli de Malinalli. 
        \item Desbloqueara las siguientes cinemáticas:
\begin{itemize}
        \item Cinemática 24 (ver apartado \ref{Cin:Cinematica24}). 
        \item Cinemática 25 (ver apartado \ref{Cin:Cinematica25}).
        \item Cinemática 26 (ver apartado \ref{Cin:Cinematica26}).
        \item Cinemática 27 (ver apartado \ref{Cin:Cinematica27}).
\end{itemize}
        \item Desbloqueará El nivel 6 (ver apartado  \ref{Nivel:Niv06}) del juego en el menú seleccionable (ver apartado \ref{inter:interfaz03}).
\end{itemize}

        \subsection{Enemigos}
\begin{itemize}
        \item Fantasma morado (ver apartado \ref{per:fantasmaM}).
        
        \item Chara enana  (ver apartado \ref{per:chara}).
        
		\item Mictlecayotl (ver apartado \ref{per:mictlecayotl}).
			
\end{itemize}
        \subsection{Items}
\begin{itemize}
        \item   Cacao (ver apartado \ref{item:cacao}).
        \item Flor de Vainilla (ver apartado \ref{item:vainilla}).
\end{itemize}
        \subsection{Personajes}
        \begin{itemize}
                \item Malinalli. Ver en \ref{per:malinalli}.
                
                \item Xolotl. Ver en \ref{per:xolotl}.
              
                \item Mictlecayotl. Ver en \ref{per:mictlecayotl}.
                
        \end{itemize}
        \subsection{Escenario}
\begin{itemize} 
        \item Fondo:
\begin{itemize} 
        \item El fondo son montañas cubiertas de nieve. El cielo es nublado sin la posibilidad de estar seguro en que momento del día es, las nubes de este nivel son de color azul claro, como la nieve.
\end{itemize} 
        \item Suelo:
\begin{itemize} 
        \item Suelo cubierto de nieve.
\end{itemize} 
        \item Obstáculos:
        \begin{itemize} 
			\item Plataforma móvil (ver apartado \ref{obs.plataformaM}).
			\item Plataforma que cae (ver apartado \ref{obs.plataformaD}).
			\item Piso resbaladizo. (ver apartado \ref{obs.pisoC}).
			\item Plataforma que desaparece (ver apartado \ref{obs.PlatDes}).
			\item Bola de nieve (ver apartado \ref{obs.bolasN}).
		\end{itemize}
        \item Objetos de fondo: Sin objetos de fondo.
\end{itemize}   
        \subsection{Referencia a BGM y SFX}
\begin{itemize} 
        \item BGM:
\begin{itemize} 
        \item Música plataforma quinto nivel (ver apartado \ref{Musica:N05_ZN01}).
        \item Música jefe quinto nivel (ver apartado \ref{Musica:N05_ZN02}).
\end{itemize}
        \item SFX:
\begin{itemize} 
        \item Viento (ver apartado \ref{SFX:Viento}).
        \item Hielo resquebrajándose (ver apartado \ref{SFX:HieloRes}).
        \item Pasos sobre el hielo o la nieve(ver apartado \ref{SFX:PasHiel}).	
\end{itemize}
\end{itemize}

        \subsection{Referencia a FX}
\begin{itemize} 
        \item Ventisca de nieve: (ver apartado \ref{FX:VenNieve}).
        \item Explosión de energía tonalli rojo (Ver apartado \ref{FX:ExpTonR})
	\item Explosión de energía tonalli verde (Ver apartado \ref{FX:ExpTonV})
	\item Respiración (\ref{Ver apartado FX:Respiracion}) 
\end{itemize} 