	\section{Nivel 9} \label{Nivel:Niv09}
	\subsection{Título del nivel}
	El último caballero del rey.
	\subsection{Encuentro}
	Este nivel estará disponible después de vencer al jefe del octavo nivel (ver apartado \ref{Nivel:Niv08}).  
	\subsection{Descripción}
	Luego de derrotar a Tepeyóllotl, Malinalli y Xólotl se dirigen hacia el penúltimo nivel del Mictlán. Éste nivel promete ser el más difícil para ambos, pues esta vez el enemigo a vencer no serán seres mágicos sino sus propios miedos y demonios internos.
	\\
	\par
Este es el único nivel que está dividido en tres etapas, dos de plataformas y una de batalla contra el jefe. Siendo también el único nivel en el que se tendrá dos personajes jugables: Xólotl para la primera etapa de plataformas y Malinalli para la segunda etapa de plataformas y la batalla contra el jefe.

	\subsection{Objetivos}
	El jugador deberá:	
	\begin{itemize}
		\item Superar la zona de Tula. Zona en la que el jugador deberá de controlar a Xólotl. El jugador explorará la ciudad de Tula, en donde deberá seguir a la figura de Quetzalcóatl por varias zonas de la ciudad. Durante la persecución de Quetzalcóatl le dirá mostrará diferentes diálogos en donde contará la relación que tenía con Xólotl.
		\item Superar la zona de Oluta. En esta zona el jugador controlara a Malinalli. El jugador explorará el palacio de Oluta, en donde seguirá a Malinalli versión niña. En esta zona el jugador tendrá que dialogar con algunos de los  habitantes del palacio para poder avanzar entre las habitaciones del palacio.
		\item Derrotar a Nexoxcho. Nexoxcho tomará la forma del padre de Malinalli para torturarla psicológicamente.  El jugador deberá de enfrentar a Nexoxcho, sin embargo, por el estado emocional de Malinalli, la velocidad a la que se mueve el jugador serán más lenta, la capacidad de salto se reducirá y el gasto de tonalli por disparo se infrementará.
	\end{itemize}
	\subsection{Progreso}
		Al terminar el nivel el jugador:
\begin{itemize}
        \item Habrá incrementado la cantidad de vida de Malinalli. 
        \item Desbloqueara las siguientes cinemáticas:
\begin{itemize}
        \item Cinemática 44 (ver apartado \ref{Cin:Cinematica44}). 
        \item Cinemática 45 (ver apartado \ref{Cin:Cinematica45}).
        \item Cinemática 46 (ver apartado \ref{Cin:Cinematica46}).
\end{itemize}
        \item Desbloqueará El nivel 10 (ver apartado  \ref{Nivel:Niv10}) del juego en el menú seleccionable (ver apartado \ref{inter:interfaz03}). 
 \end{itemize}
	\subsection{Enemigos}
	\begin{itemize}
		\item Fantasmas rojos (ver apartado \ref{per:fantasmaR}).
		\begin{itemize}
				\item Animación fuego.
				\item Animación disparo.
			\end{itemize}
		\item Fantasmas morados (ver apartado \ref{per:fantasmaM}).
		\begin{itemize}
				\item Animación fuego.
			\end{itemize}
		\item Deidades. Ver en ??.
		\item Nexchocho (ver apartado \ref{per:nexoxcho}).
	\end{itemize}
	\subsection{Items}
\begin{itemize}
        \item   Cacao (ver apartado \ref{item:cacao}).
        \item Flor de Vainilla (ver apartado \ref{item:vainilla}).
\end{itemize}
	\subsection{Personajes}
	\begin{itemize}
		\item Malinalli (ver apartado \ref{per:malinalli}).
		\begin{itemize}
			\item Animación correr.
			\item Animación saltar.
			\item Animación correr con caracola.
			\item Animación saltar con caracola.
			\item Animación normal.
			\item Animación recibir daño.
			\item Animación morir.
		\end{itemize}
		\item Xólotl (ver apartado \ref{per:xolotl}).
		\begin{itemize}
				\item Animación salto.
				\item Animación normal.
		\end{itemize}
		\item Nexchocho (ver apartado \ref{per:nexoxcho}).
		\begin{itemize}
			\item Animación apuñalar.
			\item Animación caminar.
			\item Animación recibir daño.
			\item Animación morir.
		\end{itemize}
	\end{itemize}
	\subsection{Escenario}
\begin{itemize} 
	\item Fondo: 
		\begin{itemize}
			\item Ciudad de Tula: Se mostraran casa, calles y personas de fondo en plano más cercano al jugador. A lo lejos se verán grandes templos y más edificaciones que muestren el esplendor de la ciudad. Se puede ver un cielo nocturno despejado.
			\item Ciudad de Tula destruida: Las edificaciones de la ciudad veían ahora están parcialmente destruidas o destruidas. Hay fuego y personas en el suelo o corriendo. Se puede ver un cielo nocturno apocado por el humo de la ciudad.
			\item Interior del palacio de Oluta: El cielo que se ve por las ventanas es nocturno. Hay antorchas iluminando el interior del palacio, de las paredes cuelgan finas telas de colores.
		\end{itemize}
	\item Suelo: Roca negra, agua negra.
	\item Obstáculos: Sin obstáculos.
	\item Objetos de fondo: Sin objetos.
\end{itemize}	
	\subsection{Referencia a BGM y SFX}
	\begin{itemize}
		\item BGM.
			\begin{itemize}
				\item Música plataforma noveno nivel, Tula (ver apartado \ref{Musica:N09_ZN01T}).
				\item Música plataforma noveno nivel, Oluta primera etapa (ver apartado \ref{Musica:N09_ZN01C}).
				\item Música plataforma noveno nivel, Oluta segunda etapa (ver apartado \ref{Musica:N09_ZN01C02}).
				\item Música jefe noveno nivel (ver apartado \ref{Musica:N09_ZN02}).
			\end{itemize}
		\item SFX.
			\begin{itemize}
				\item Pasos (ver apartado \ref{SFX:Pasos}).
				\item Viento (ver apartado \ref{SFX:Viento}).
				\item Sonido de fuego (ver apartado \ref{SFX:Fuego}).
				\item Gritos de personas (ver apartado \ref{SFX:griPersonas}).
			\end{itemize}
	\end{itemize}
	\subsection{Referencia a FX}
	\begin{itemize}
		\item Relámpagos (ver apartado \ref{FX:Relam}).
		item Explosión de energía tonalli rojo (ver apartado \ref{FX:ExpTonR}).
	\item Explosión de energía tonalli verde (ver apartado \ref{FX:ExpTonV}).
	\item Haz de luz (Ver apartado{FX:HazLuz}).
	\end{itemize}