\section{BGM}
\subsection{Música  mercado}\label{Musica:N01_ZN01}
La música de fondo que suena en el mercado debe de evocar un sentimiento de cotidianidad, seguridad y alegría. En el tema predominaran instrumentos de viento. 		
\subsection{Música  selva}\label{Musica:N01_ZN02}
La música en la selva debe de ser sobre inicio de un viaje. Debe de evocar en el jugador el deseo de aventura y a la vez hacerlo sentir que está en un lugar conocido.
\subsection{Música  Xolotl} \label{Musica:Xolotl}
La música cuando Xólotl le habla a Malinalli debe de ser un tema que se ajuste con la personalidad del Dios ya que será el tema que sonara cuando él tenga diálogos importantes que decir, así que debe ser un tema que muestre determinación y misterio.
\subsection{Música plataforma segundo nivel}\label{Musica:N02_ZN01}
La musicalización que corresponde a la zona de enemigos normales de este nivel debe de reflejar que uno se encuentra en la entrada de un lugar misterios y peligroso, pero de gran carga espiritual (Por lo que se recomienda alguna canción que contenga coros, los coros preferiblemente deben de estar en Náhualt y deben de hablar sobre el sufrimiento de las almas que no pudieron cruzar).
\subsection{Música jefe segundo nivel}\label{Musica:N02_ZN02}
La música que acompaña a la batalla con Xochitonal debe de hacer sentir peligro que se está luchando por sobrevivir pero sin perder la carga espiritual que conlleva este nivel. 
\subsection{Música plataforma tercer nivel} \label{Musica:N03_ZN01}
La Música  de este nivel debe reflejar el sentimiento de adrenalina y vértigo que lleva escalar una superficie frágil y peligrosa.
\subsection{Nombre: Música jefe tercer nivel} \label{Musica:N03_ZN02}
 La batalla contra el jefe debe de mostrar un tema que evoque la misma adrenalina de peligro con el agregado de que la sensación de peligro se intensifica.

\subsection{Música plataforma cuarto nivel}\label{Musica:N04_ZN01}
La música del nivel evoca a que se está explorando un lugar misterioso y peligroso pero lleno de maravillas.
\subsection{Nombre: Música jefe cuarto nivel} \label{Musica:N04_ZN02}
La música de batalla contra el jefe debe de reflejar el ritmo desenfrenado de batalla que tiene Itzpapálotl, se propone una canción parecida a Bipolar Nightmare de la banda sonora de Nier Automata.

\subsection{Música plataforma quinto nivel} \label{Musica:N05_ZN01}
Este tema se caracterizara por tener una fuerte carga de nostalgia y renuncia hacia las personas que se aman, ya que representa la nostalgia que siente Mictlecayotl por todo lo que perdió cuando fue condenada a estar en el Mictlán por tratar de asesinar a Tezcatlipoca. El tema puede contener algunas notas como de caja de música y coros.

\subsection{Música jefe quinto nivel} \label{Musica:N05_ZN02}
Este tema debe de contener un poco de la nostalgia pero a su vez debe de denotar fuerza y violencia. Su ritmo debe de ser acelerado pues refleja la personalidad de fuerte e innegociable de Mictlecayotl. 


\subsection{Música plataforma sexto nivel} \label{Musica:N06_ZN01}
El tema de este nivel debe de reflejar el sentimiento de libertad que viene con poder volar, debe de reflejar la versatilidad del viento, pero a su vez debe de mostrar el creciente peligro que hay en un lugar donde impera el caos por la falta de orden representado como la falta de gravedad. 

\subsection{Música jefe sexto nivel} \label{Musica:N06_ZN02}
El tema de la batalla contra Tlazoltéotl debe reflejar la personalidad liberal y caótica de ésta. El tema debe de hacer remembranza a los ritmos de danza prehispánica combinado con algunos coros en nahual para reflejar el carácter peligroso de la Diosa. 

\subsection{Música plataforma séptimo nivel} \label{Musica:N07_ZN01}
El tema de la zona de plataformas debe de reflejar la tristeza y el vacío que siente Itztlacoliuhqui tras la muerte de Itzpápalotl. La canción debe de reflejar el sentimiento que deja tras de sí el perder a la persona que más se ama.

\subsection{Música jefe séptimo nivel} \label{Musica:N07_ZN02}
Este tema debe reflejar la ira y el deseo de castigo de Itztlacoliuhqui a Xólotl. Un  ejemplo de cómo debe de sonar este tema Simone de Nier automata.

\subsection{Música plataforma octavo nivel} \label{Musica:N08_ZN01}
La música de esta etapa debe de refleja que se está ante un punto de quiebre en donde no hay marcha atrás. La música debe de dar la sensación de que se está atravesando un camino lleno de dificultades y misterios.

\subsection{Música jefe octavo nivel} \label{Musica:N08_ZN02}
El tema de jefe de este nivel debe de reflejar la resolución de Tepeyóllotl, es decir es un tema que refleja la determinación del deseo de poder frenar a un enemigo poderoso, pero a su vez debe de mostrar la aceptación de Tepeyoóllotl sobre su eminente final, final que Tepeyólotl ha aceptado con orgullo.

\subsection{Música plataforma noveno nivel, Tula} \label{Musica:N09_ZN01T}
El tema debe demostrar que se está en una ciudad que tuvo un gran esplendor pero se encuentra destruida.

\subsection{Música plataforma noveno nivel, Oluta primera etapa} \label{Musica:N09_ZN01C}
Este tema debe de evocar la sensación de que se regresa a casa después de un lago viaje.

\subsection{Música plataforma noveno nivel, Oluta segunda etapa} \label{Musica:N09_ZN01C02}
Este tema debe de evocar la desesperación por encontrar a un ser amado ante la incertidumbre no saber si se encuentra bien.

\subsection{Música jefe noveno nivel} \label{Musica:N09_ZN02}
Este tema debe de reflejar la tristeza que siente Malinalli ante la muerte de su padre y culpabilidad de no haber emprendido su viaje antes para salvar el alma de éste. 

\subsection{Música pasillo del palacio decimo nivel} \label{Musica:N10_ZN01}
Este tema debe de reflejar que se está entrando a un lugar de gran importancia en  donde se ostenta mucho poder político; sin embrago el tema también debe de mostrar el sentimiento de impotencia de u rey que se ve abandonado por sus súbditos. 

\subsection{Música jefe noveno nivel} \label{Musica:N09_ZN02}
El tema de batalla contra el jefe final debe de reflejar que se está luchando contra un enemigo de gran poder y debe de emocionar al jugador, pues la canción debe de ser un impulso épico como retribución por haber llegado tan lejos. Ejemplo hellfire de Final fantasy XV. 