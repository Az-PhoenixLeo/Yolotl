\section{Cinemática 39. A orillas del río del Apanohualoyan. ext/noche.} \label{Cin:Cinematica39}
 \textsc{Personajes}:
 \begin{itemize}
 \item Xólotl.
 \item Malinalli.
 \end{itemize}
\textit{Ambos personajes están de pie sobre las orillas del río. El agua del río es negra y calmada, no muestra ninguna perturbación, aunque la agites con la mano. El cielo esta estrellado y despejado, a lo lejos se observa un bosque.}
\begin{center}
\textsc{\underline{Malinalli}}
\\
\par
Nuevamente tendremos que nadar.
\\
\par
\textsc{\underline{Xólotl}}
\\
\par
Esa es la idea. Al menos en gran medida. Esta vez no será como el río Apanohuacalhuia. Este tiene su pequeño truco. No esperes encontrar almas errantes aquí. Todo aquel que no lo logra cruzar ayuda a teñir el río de negro. Apanohualoyan, jamás debe de ser cruzado por la superficie o terminaras varado sin la posibilidad de volver a la orilla. Es en sus profundidades que se encuentra la guarida de Nexoxcho. Espero que estés lista para nadar, no te preocupes por la respiración; el tonalli de Mictlecayotl lo hara por ti. Mantén los ojos bien abiertos.
(Xólotl salta al agua. Malinalli lo sigue)
\end{center}