\section{Cinemática 10. Sobre la cima de una montaña en el Tepeme Monamictlan. exterior /día.}
\label{Cin:Cinematica10}
 \textsc{Personajes}:
 \begin{itemize}
 \item Xólotl (Forma xoloitzcuintle).
	\item Malinalli Tenelpan.
 \end{itemize}
\textit{El viento sopla, se observa un día brillante y despejado.  Las montañas se ven cubiertas de pasto.}

\begin{center}
\textsc{\underline{Xólotl}}
\\
\par
Hermosa vista ¿No crees? (Malinalli lo mira desconcertada) ¿No iras a decirme que te da miedo la altura o sí? Porque si es así lamento decirte que la vas a pasar muy mal. Este lugar es el hogar de un pequeño y problemático gato. Cuando se instauro la nueva jerarquía divina luego de la creación del quinto sol, se asignaron diferentes dioses para salvaguardar el Mictlán. Aunque todos cumplen con su deber, pocos son los que están aquí de manera voluntaria y por gusto.
\\
\par
\textsc{\underline{Malinalli}}
\\
\par
No lo entiendo. Si el Mictlan crea su propia energía para evitar que alguien haga trampa en el viaje de purificación ¿Para que tener guardianes que vigilen?
\\
\par
\textsc{\underline{Xólotl}}
\\
\par
Es bastante sencillo. ¿Te haz preguntado porque no fui directo a los trece cielos, conquistarlo y con mi nueva posición obligar a los habitantes del Mictlán a jurarme lealtad? La respuesta es sencilla, todos los habitantes de los trece cielos son leales a su líder y lucharían por él. Esa lealtad no siempre fue así ni es gratuita. Hace mucho tiempo, hubo diferentes rebeliones. Muchas deidades trataron de tomar el control de todo. Entonces los cuatro hermanos tomaron una medida para proteger su reinado: Sellaron la entrada del mundo de los mortales a los trece cielos. Así que solo se puede entrar por el Mictlán. Y para asegurarse de ninguna amenaza les llegara sin aviso, nombraron a diferentes divinidades los guardianes de Mictlán. Cada uno más poderoso que el anterior. 
\\
\par
\textsc{\underline{Malinalli}}
\\
\par
Eso significaría que cada uno de ellos sería un sacrificio para proteger el orden. ¿Cómo se garantiza de que todos sean leales sin son conscientes de que son desechables?
\\
\par
\textsc{\underline{Xólotl}}
\\
\par
Haciendo que uno de los guardianes los vigile todo el tiempo con la promesa de que si los problemas tocan la puerta el podrá huir para informarle a Tezcatlipoca lo ocurrido. 
\\
\par
\textsc{\underline{Malinalli}}
\\
\par
¿Y si ese guardían fuera el traidor?
\\
\par
\textsc{\underline{Xólotl}}
\\
\par
No lo sería. Si hay algo peor que un traidor es un traidor que destroza la confianza de Tezcatlipoca. El dios negro no es benevolente con nadie que le traicione. Eso no lleva a nuestro pequeño gato. Pase lo que pase, debemos de acabar con el antes de que pueda advertirle a Tezcatlipoca que he vuelto.
\end{center}