\section{Cinemática 1. Afueras de Tenochtitlan. exterior / tarde.} \label{Cin:Cinematica01}
	
\textsc{Personajes}:
\begin{itemize}
	\item Quetzalcóatl (Forma human).
	\item Xólotl (Forma human).
\end{itemize}
\textit{El viento sopla, Quetzalcóatl (en su forma humana) está sentado, viendo en dirección de la ciudad. La ciudad Tenochtitlan se ve parcialmente destruida, de ella sale humo. Entra Xólotl (En su forma humana)}.
\begin{center}
\textsc{\underline{Xólotl}}
\par
De todos los lugares y personas posibles, jamás creí que te encontraría precisamente a ti en este lugar.
\par
\textit{(Quetzalcóatl sigue observando la ciudad en silencio)}
\par
Así es como serán las cosas.
\par 
\textit{(Quetzalcóatl continua en silencio, Xólotl se molesta)}
\par
¿Guardaras silencio para siempre? ¿Ese es el modo en que me castigaras por lo que he hecho?
\\
\par
\textsc{\underline{Quetzalcóatl}}
\par
La caída del quinto sol. ¿Quién lo hubiera creído? Todo cuanto conocimos está por cambiar o desaparecer.
\par
\par
\textsc{\underline{Xólotl}}
\\
\par
Jamás debiste irte. Las cosas habrían sido diferentes si nos hubieras guiado.
\\
\par
\textsc{\underline{Quetzalcóatl}}
\par
Te equivocas. Las cosas son como tenían que ser. Dime… ¿Valió la pena? El duro viaje y las batallas. ¿Todo eso te dio lo que tanto anhelabas?
\\
\par
\textsc{\underline{Xólotl}}
\\
\par
\textit{(Guarda silencio por un momento)}
\\
\par
No …
\end{center}