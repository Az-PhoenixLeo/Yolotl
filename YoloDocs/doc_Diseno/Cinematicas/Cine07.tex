\section{Cinemática 7. palacio de Olula. interior/ día.}
\label{Cin:Cinematica07}
\textsc{Personajes}:
\begin{itemize}
 \item Malinalli Tenelpan (de cinco años aproximadamente).
	\item Tenépal
\end{itemize}

\textit{Malinalli está durmiendo en el suelo al lado de unos códices. Su padre entra.} 

\begin{center}
\textsc{\underline{Tenépal }}
\\
\par
Malinalli. Malinalli. Despierta. 
\\
\par
\textsc{\underline{Malinalli}}
\\
\par
Padre. ¡Has vuelto! (Abraza a su padre, está la carga) ¿Cómo te ha ido? ¿Tenochtitlan es tan grande como dicen?
\\
\par 
\textsc{\underline{Tenépal }}
\\
\par
Paciencia, mi pequeña. Una pregunta a la vez. Más que Tenochtitlan, lo que me ha llamado la atención han sido todas las ciudades que vi en mi camino. (Camina) El imperio es extenso, pero tanto territorio dividido y sin unificar puede ser algo peligroso.
\\
\par
\textsc{\underline{Malinalli}}
\\
\par
(Alarmada) ¿Peligroso? ¿Significa que algo malo te pasará?
\\
\par
\textsc{\underline{Tenépal }}
\\
\par
Nada me va a pasar, pequeña. El imperio es extenso y algún día será unificado. Tienes que prepárate para ello. Tendrás un papel importante cuando ese momento llegue.
\\
\par
\textsc{\underline{Malinalli}}
\\
\par
¿Cómo podre jugar un papel importante si soy solo una mujer? Madre dice que mi único papel será casarme y darle más hijos al imperio.
\\
\par
\textsc{\underline{Tenépal }}
\\
\par
Mi amada niña. Tu papel será más que eso. (Baja a Malinalli)
\\
\par
\textsc{\underline{Malinalli}}
\\
\par
¿Cómo puedes estar tan seguro de eso?
\\
\par
\textsc{\underline{Tenépal }}
\\
\par
Un padre siempre lo sabe. Malinalli, eres diferente de las demás chicas. (Se arrodilla ante su hija y la toma de las manos) Eres más inteligente y brillante de lo que crees. Prométeme que jamás permitirás que los demás subestimen los que estas destinada a ser. (Le pone el broche de flor en la cabeza a Malinalli) En tus manos está el poder de crear el ombligo de la Luna.
\\
\par
\textsc{\underline{Malinalli}}
\\
\par
Te lo prometo padre. 
\end{center}