\section{Cinemática 9. Castillo de Mictlantecuhtli. Interior /noche.}
 \label{Cin:Cinematica09}
  \textsc{Personajes}:
  \begin{itemize}
  \item Mictlantecuhtli.
	\item Mictecacíhuatl.
	\item Tepeyóllotl.
	\item Mictlecayotl.
	\item Itzpápalotl.
	\item Itztlacoliuhqui.
	\item Nexoxcho.
	\item Tlazoltéolt.
  \end{itemize}
  
  \textit{La sala del trono en el castillo de Mictlantecuhtli tiene una iluminación de pocas velas. Al fondo y centrados a la pared se encuentran los tronos de los dioses Mictlantecuhtli y Mictecacíhuatl. Ambos están sentados. Al lado de los tronos hay unos candelabros hechos de huesos. Frente a ellos se encuentras las figuras proyectadas de los demás guardianes del Mictlán. Las proyecciones de los dioses son de color verde y aproximan la figura de los mismos sin dar un detalle exacto de como son.}
  
\begin{center}
	MICTLANTECUHTLI
	\\
\par
Hay un intruso en el Mictlán. Uno peligroso al parecer ya que Xochitónal ha caído. Hace unos instantes sentí como su tonalli desparecía.  
\\
\par
\textsc{\underline{Tepeyóllotl}}
\\
\par
No me sorprende su caída. Xochitónal siempre fue más palabras que acción.
\\
\par
\textsc{\underline{Mictecacíhuatl}}
\\
\par
Puede que no fuera el más eficiente de los guardianes, pero esta situación demanda una atención prioritaria. Es un ataque directo al orden y no vamos a tolerar ninguna otra rebelión a la jerarquía divina.
\\
\par
\textsc{\underline{Itzpápalotl}}
\\
\par
Majestad, ¿Cómo esta tan segura que el invasor pertenece a nuestra jerarquía? A lo que a mí respecta podría ser un Dios de otro pueblo, no sería la primera vez que estaríamos en guerra con Dioses extranjeros.
\\
\par
\textsc{\underline{Mictlantecuhtli}}
\\
\par
Más motivo aun para actuar con cautela y terminar este asunto de golpe. Tepeyóllotl, los invasores se dirigen a tu morada. Termina el asunto rápido y eficientemente. Pueden retirarse.
\end{center}