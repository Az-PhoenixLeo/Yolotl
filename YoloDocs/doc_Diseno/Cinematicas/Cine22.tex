\section{Cinemática 22. entrada al Cehuelóyan. ext/día.}
 \label{Cin:Cinematica20}
\textsc{Personajes}:
\begin{itemize}
\item Xólotl
\item Malinalli Tenelpan.
\end{itemize}
\textit{El lugar está cubierto de nieve. El cielo está despejado, pero sopla el viento helado. De fondo se pueden ver montañas cubiertas totalmente por hielo.}
\begin{center}
\textsc{\underline{Xólotl}}
\\
\par
Finalmente llegamos al lugar donde nacen los vientos del norte, casi es la mitad de nuestro viaje.
\\
\par
\textsc{\underline{Malinalli}}
\\
\par
Lo que paso con Itzpápalotl
\\
\par
\textsc{\underline{Xólotl}}
\\
\par
Será mejor que no pienses en ello. Me gustaría pensar que la guardiana del Cehuelóyan podría ser una aliada. A diferencia de mí, ella fue directamente a tratar de matar a Tezcatlipoca.
\\
\par
\textsc{\underline{Malinalli}}
\\
\par
Suena como una divinidad muy temeraria.
\\
\par
\textsc{\underline{Xólotl}}
\\
\par
Una poderosa aliada, de no ser que ella es muy devota a Quetzalcóatl. Difícilmente podría con fiar en mí.
\\
\par
\textsc{\underline{Malinalli}}
\\
\par
Ahora que lo mencionas. Quetzalcóatl es un dios benévolo. ¿Porqué no interfirió a tu favor cuando te rehusaste a ser sacrificado? 
\\
\par
\textsc{\underline{Xólotl}}
\\
\par
Malinalli, no siempre los dioses son tan amables y buenos con otros dioses que como lo serían con los humanos. Quetzalcóatl no es ese ente benevolente y virtuoso que muchos creen. 
\end{center}