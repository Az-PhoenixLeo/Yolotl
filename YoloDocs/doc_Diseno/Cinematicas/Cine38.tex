\section{Cinemática 38. Palacio de Mictlantecuhtli. int/día.} \label{Cin:Cinematica38}
 \textsc{Personajes}:
 \begin{itemize}
 \item Mictlantecuhtli.
 \item Mictecacíhuatl.

 \end{itemize}
\textit{Mictlantecuhtli está sentado en su trono. Mictecacíhuatl entra a toda velocidad, se acerca a Mictlantecuhtli. Él la observa, sabe lo que ella está a punto de decir.}
\begin{center}
\textsc{\underline{Mictecacíhuatl}}
\\
\par
Tepeyolotl ha caído. Lo único que nos separa de los invasores e Nexoxcho. Tepeyolotl pensaba que podía negociar con la niña mortal.
\\
\par
\textsc{\underline{Mictlantecuhtli}}
\\
\par
Por lo que desaparecio sin avisarle a Tezcatlipoca sobre la situación. 
\\
\par
\textsc{\underline{Mictecacíhuatl}}
\\
\par
Exactamente.
\\
\par
\textsc{\underline{Mictlantecuhtli}}
\\
\par
Se sobrepreocupa, mi señora. Desde que Nexoxcho fue asignado como guardián del octavo nivel, nadie ha sido capaz de pasar su guardia. Se requiere mucha valentía para tratar de enfrentarnos. Sin embargo, se requiere mayor valentía para enfrentar los conflictos del alma. Xólotl y su niña mortal no pasaran los dominios de Nexoxcho.
\\
\par
\textsc{\underline{Mictecacíhuatl}}
\\
\par
He escuchado a siete de los nueve guardianes decir lo mismo respecto a si mismos. Todos han caído. ¿Qué garantiza que Nexoxcho será diferente?
\\
\par
\textsc{\underline{Mictlantecuhtli}}
\\
\par
Si tanto le preocupa su seguridad. La invito a irse a los trece cielos a informarle a Tezcatlipoca lo que ha ocurrido. No voy a entregar mi reino sin dar batalla.
\\
\par
\textsc{\underline{Mictecacíhuatl}}
\\
\par
Le tomare la palabra, mi señor. Que la próxima vez que nos veamos sean con noticias de victoria.
\end{center}