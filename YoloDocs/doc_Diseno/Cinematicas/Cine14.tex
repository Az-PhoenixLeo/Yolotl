\section{Cinemática 14. Entrada al Itztépetl. Interior /noche.} \label{Cin:Cinematica14}
 \textsc{Personajes}:
 \begin{itemize}
 \item \textsc{\underline{Xólotl}} (Forma xoloitzcuintle).
 \item Malinalli Tenelpan.

 \end{itemize}
 \textit{Ambos personajes están dentro de una cueva.  La cueva esta iluminada por la luz de cristales. Se puede escuchar de vez en cuando el sonido del agua cayendo. }

\begin{center}
\textsc{\underline{Malinalli}}
\\
\par
Me gustaría preguntarle algo. ¿Porqué los otros guardianes del inframundo lo llaman cobarde?
\\
\par
\textsc{\underline{Xólotl}}
\\
\par
Digamos que cuando Quetzalcóatl creo el quinto Sol, exigió a los demás dioses sacrificarse para dar forma al nuevo mundo e instaurar un nuevo orden. Algunos se sacrificaron sin dudarlo, otros nos opusimos a morir. No me mires así. Morir en el fuego porque Quetzalcóatl y Tezcatlipoca destruyeron los cuatro soles anteriores por jugar a tener la razón no sonaba como algo justo para mí. Nadie quiso escuchar mis razones, solo me llamaron cobarde y traidor. Desde entonces estoy exiliado.
\\
\par
\textsc{\underline{Malinalli}}
\\
\par
Lamento oír eso.
\\
\par
\textsc{\underline{Xólotl}}
\\
\par
No sientas pena por mí. Encontré mi verdadero destino en el exilio. Que no te engañen los cristales. Estamos en la entrada de un lugar peligroso. Si se puede, me gustaría tener a la guardiana de ese sitio como aliada. 
\end{center}