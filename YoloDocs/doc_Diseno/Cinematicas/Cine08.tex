\section{Cinematica 8. a las orillas del rio Apanohuacalhuia en el Itzcuintlan. Exterior /noche.}
\label{Cin:Cinematica08}
 \textsc{Personajes}:
 \begin{itemize}
 	\item Malinalli Tenelpan (edad actual).
	\item Xólotl.
\textit{Malinalli está en el suelo a la orilla del lago, Xólotl está a su lado tratando de reanimarla con sus patas delanteras.}
 \end{itemize}
 
\begin{center}
\textsc{\underline{Malinalli}}
\\
\par
(Toce)
\\
\par
\textsc{\underline{Xólotl}}
\\
\par
Despiertas. Por un momento creí que morirías. La parte positiva de si morías es que tu alma ya no tendría que cruzar el río y te habrías ahorrado un paso del viaje. Aunque morir por ahogamiento te habría llevado al cielo dominado por Tláloc.
\\
\par
\textsc{\underline{Malinalli}}
\\
\par
¿Qué ocurrió? (trata de levantarse)
\\
\par
\textsc{\underline{Xólotl}}
\\
\par
Tómatelo con calma, casi te ahogas. Xochitónal te derribó. Por suerte evite que murieras. Antes de que te lo preguntes, lograste acabar con él. Nunca había visto a ningún humano tener tanta afinidad al tonalli de un Dios. 
\\
\par
\textsc{\underline{Malinalli}}
\\
\par
Si puedes transformarte en lo que sea ¿Por qué no te transformaste en algo que volara y evitábamos el río? fácilmente pudimos haber ido y atacar por aire a…
\\
\par
\textsc{\underline{Xólotl}}
\\
\par
No es tan simple, la energía del Mictlán me impide transformarme en algo que no sea propio de cada nivel del Mictlán. ¿Haz visto alguna ave en esta zona? No, es por eso que no puedo hacerlo. Si tuviera más poder la energía del Mictlán no me afectaría. Suficiente charla. Tenemos que irnos. Destruir a Xochitónal alertará a cualquiera de los guardianes y Mictlantecuhtli enviará a alguien a investigar en cualquier momento.

\end{center}