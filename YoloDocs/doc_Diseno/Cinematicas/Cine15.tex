\section{Cinemática 15. guarida de itzpápalotl. interior/noche.} \label{Cin:Cinematica15}
 \textsc{Personajes}:
 \begin{itemize}
 \item Xólotl (Forma xoloitzcuintle).
 \item Malinalli Tenelpan.
 \item Itzpápalotl
 \end{itemize}
 \textit{La guarida de Itzpápalotl ya ha sido descrita en la cinemática 11. Malinalli entra a la guarida. Se ve maravillada por la cantidad de mariposas que hay. Intenta tocar una y entonces todas vuelan hacia el centro de la guarida, ahí vuelan en círculos formando un torbellino, después vuelan en todas direcciones dejando ver a Itzpápalotl.}
 \begin{center}
 \textsc{\underline{Xólotl}}
 \\
\par
\textsc{\underline{Itzpápalotl}}. Luces diferente desde la última vez que te vi. Me alegra verte.
\\
\par
\textsc{\underline{Itzpápalotl}}
\\
\par
No puedo decir lo mismo que tú. Esperaba no volver a verte. Debiste haberte mantenido lejos de todo esto. Será rápido. No te preocupes por la mortal, la devolveré a donde pertenece, hare que Nexoxcho altere su memoria y piense que todo esto fue un sueño.
\\
\par
\textsc{\underline{Xólotl}}
\\
\par
Muy amable de tu parte. No deseo enfrentarme a ti. Comprendo tus motivos por los que te refugiaste en el Mictlán. Por eso deseo una alianza. El orden que creare no necesitara más peleas.
\\
\par
\textsc{\underline{Itzpápalotl}}
\\
\par
(Las mariposas vuelven a volar alrededor de Itzpápalotl) ¿Refugiarme? No comprendes nada en lo absoluto. Despídete de ella, en unos instantes serás la sombra de un mal sueño.  (Itzpápalotl extiende sus alas, invoca su arma y adopta una pose de combate).
 \end{center}