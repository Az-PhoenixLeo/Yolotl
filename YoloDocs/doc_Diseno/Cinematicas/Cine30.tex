\section{Cinematica 30. Palacio de Olula. int/día. }  \label{Cin:Cinematica30}
 \textsc{Personajes}:
 \begin{itemize}
 \item Malinalli Tenelpan.
\item Cimatl.
\item Huenupan.
\item Guardias.
 \end{itemize}
\textit{La Cimatl está sentada en el centro de la sala. Carga en sus brazos un bebé. El Huenupan entra a la sala. Besa a su esposa y mira con amor al bebé.}
\begin{center}
\textsc{\underline{Huenupan}}
\\
\par
Mi amada esposa en compañía de mi primogénito. He oído rumores de lo más bajos en la ciudad. Un gobernante de las provincias lejanas desea casar a su hijo con Malinalli.
\\
\par
\textsc{\underline{Cimatl}}
\\
\par
¿No serían esas buenas noticias?
\\
\par
\textsc{\underline{Huenupan}}
\\
\par
Tal parece que muchos desean usar el buen nombre de tu antiguo esposo para armar ejércitos que se alcen contra el imperio. Mientras tu hija exista, existirá peligro para el imperio y para el futuro de nuestro hijo.
\\
\par
\textsc{\underline{Cimatl}}
\\
\par
Entonces está claro lo que debes de hacer. No la asesines, o los partidarios de su padre la convertirán en una mártir. A ellos les interesa esposar una niña de cuna noble, no una esclava. 
(La escena se desvanece. Malinalli se encuentra en el suelo leyendo varios códices. Dos guardias entran y la toman de los brazos. Malinalli sin comprender que ocurre trata de resistirse, pero un guardia la abofetea. Malinalli grita. Su madre entra a la habitación. Malinalli rompe el agarre de los guardias y corre a refugiarse tras su madre. Los guardias la toma nuevamente)
\\
\par
\textsc{\underline{Malinalli}}
\\
\par
Suéltenme. Madre, diles que se detengan.
\\
\par
\textsc{\underline{Cimatl}}
\\
\par
Yo no parí esclavos. Llévensela. El Señor de Centla recibirá gustoso el pago por las provisiones.
\\
\par
\textsc{\underline{Malinalli}}
\\
\par
¡Madre! ¡Diles que se detengan! ¡Diles que ha sido un error! ¡Madre! (Los guardias se llevan a Malinalli).
\end{center}