\section{Cinemática 4. A las orillas del rio Apanohuacalhuia en el Itzcuintlan. Exterior /noche.}
\label{Cin:Cinematica04}
 \textsc{Personajes}:
\begin{itemize}
	\item Malinalli Tenelpan.
	\item Xólotl (Forma xoloitzcuintle).
\end{itemize}
\textit{Malinalli y Xólotl entran caminando. Malinalli está sorprendida por todo lo que está viendo.}

\begin{center}
XÓLOTL
\\
\par
Bienvenida al Mictlán, el lugar donde los muertos van, como debes de saber el Mictlán se divide en nueve niveles. Itzcuintlan es el primero de ellos, que lugar más acogedor ¿No crees? Todas estas almas están aquí para cruzar el lago. Si tu fueras como cualquiera de ellos lo siguiente que tendrías que hacer sería conseguiré un xoloitzcuintle para nadar a su lado, aunque no se hayas sido lo suficientemente digna como para que uno te ayude.
\\
\par
\textsc{\underline{Malinalli}}
\\
\par
Entonces ¿Nadaras junto a mí? ¿O debería empezar a buscar un xoloitzcuintle?
\\
\par
XÓLOTL
\\
\par
No estás muerta, por mucho que lo intentaras los xoloitzcuintles no te harían caso porque no pueden verte. Tú puedes verlos a ellos y tocarlos, pero ellos no pueden verte a ti, trata de no perturbarlos a la bestia que custodia el lago se molestara de verdad.
\\
\par
\textsc{\underline{Malinalli}}
\\
\par
Si no puedo tocarlos ¿Cómo voy a cruzar el río?
\\
\par
XÓLOTL
\\
\par
Para eso estoy aquí (se transforma en ajolote). Sube, tenemos que cruzar lo más rápido que se pueda antes de que alguien nos descubra. (Malinalli sube) Recuerda, nada de tocar a los xoloitzcuintles.
\end{center}
