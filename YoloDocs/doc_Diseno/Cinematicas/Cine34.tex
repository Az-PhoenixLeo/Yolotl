\section{Cinemática 34. Palacio de Oluta. int/día. }
\label{Cin:Cinematica34}
 \textsc{Personajes}:
 \begin{itemize}
 \item Malinalli Tenelpan (cinco años).
\item Tenépal.
\item Guardias.
 \end{itemize}
\textit{Malinalli esta con su padre. Él le muestra diversos códices. De entre los códices saca un mapa.}
\begin{center}
\textsc{\underline{Malinalli}}
\\
\par
Mi padre era un gobernante honorable gentil que protegía a su gente del abuso del imperio. Él veía las cosas con otros ojos. Donde la mayoría veía problemas, él veía una posibilidad de hacer las cosas mejores para su gente. Él era consciente de una realidad que el imperio prefería ignorar: El imperio era extenso, pero fuera de su capital no era amado. (Se muestra al Tenépal hablando con diferentes hombres de alta cuna) Utilizando su poder político, Padre comenzó a unificar el sur del imperio: su misión era formar el legendario ombligo de la luna, aquella nación contada en canciones por los adivinos. La nación que heredaría toda la gloria de las culturas anteriores a ella. (El Tenépal camina por los pasillos cuando lo abordan varios guardias y lo apuñalan) Sin embrago, un ideal grande genera una gran sombra. Traicionado por uno de sus hombres, Padre fue mandado a asesinar por el Tlatoani. Con padre muerto, sus seguidores abandonaron su causa por temor a compartir el mismo destino que él. Padre murió muy pronto, sin que el destino le permitiera cumplir su sueño. El deber y el honor no pudieron protegerlo ni a mí. Voy a cambiar eso, le daré una segunda oportunidad.
\end{center}