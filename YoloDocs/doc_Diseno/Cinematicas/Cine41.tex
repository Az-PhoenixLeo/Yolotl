\section{Cinemática 41. Templo mayor de Tula. int/noche.}
\label{Cin:Cinematica41}
 \textsc{Personajes}:
 \begin{itemize}
 \item Xólotl.
 \item Quetzalcóatl.
 \item Malinalli (adulta).
 \end{itemize}
\textit{El templo mayor esta bellamente decorado con pinturas de los dioses y pedestales de oro. Quetzalcóatl está orando rodeado de diferentes sacerdotes. Xólotl entra al templo. Los sacerdotes se congelas y Quetzalcóatl se pone de pie.}
\begin{center}
\textsc{\underline{Quetzalcóatl}}
\\
\par
¿Te has cansado ya de correr y mentir? Cuando todos nos sacrificamos lo único en lo que pensabas era en ti mismo. Siempre buscas como justificarte. Te da tanto miedo el miedo que siempre buscaras alguien a quien responsabilizar por tus acciones.
\\
\par
\textsc{\underline{Xólotl}}
\\
\par
Nexoxcho. Se que estás ahí. Deja de esconderte. Tus ilusiones no me afectan.
(El escenario se desmorona, se muestra la ciudad en llamas, Quetzalcóatl se desvanece).
¿Qué es lo que pretendes que te diga? ¿Qué no pude salvarlos? ¿Qué hice mal en tratar de negociar con Tezcatlipoca y Quetzalcóatl? El único miedo que siento es el tuyo.
(Aparece Malinalli)
\\
\par
\textsc{\underline{Malinalli (Adulta)}}
\\
\par
Hola, Xólotl.
\\
\par
\textsc{\underline{Xólotl}}
\\
\par
¿Malinalli? Eres otra ilusión.
\\
\par
\textsc{\underline{Malinalli (Adulta)}}
\\
\par
¿En verdad lo soy o soy solo el reflejo de lo que estoy destinada a ser? (Malinalli extiende sus brazos el escenario cambia a la ciudad de Tenochtitlan, la ciudad es incendiada por el asalto de barcos) Hermoso ¿No lo crees? Estoy segura de que los viste cuando viajaste por todo el mundo buscando un compañero para tu travesía.
\\
\par
 \textsc{\underline{Xólotl}}
\\
\par
¡Nexoxcho! ¿Qué pretendes? Muéstrate ya.
\\
\par
\textsc{\underline{Malinalli (Adulta)}}
\\
\par
¿Cuándo muere un dios verdaderamente? Cuando hieres su cuerpo de gravedad, el cuerpo desaparece y deja el tonalli. Si nadie reclama ese tonalli el Dios puede regresar al cabo de un tiempo. Por el contrario, si un dios reclama ese tonalli pero otro más derrota a ese dios y restaura el tonalli del primer dios caído, inevitablemente vuelve. ¿Cuándo desaparece un dios? Dímelo, Xólotl. Estoy segura de que tú lo sabes, viste varios desaparecer durante tus viajes.
\\
\par
\textsc{\underline{Xólotl}}
\\
\par
Un dios desaparece cuando el pueblo que cree en él muere.
\\
\par
\textsc{\underline{Malinalli (Adulta)}}
\\
\par
(Se acerca a Xólotl y lo abraza) ¿No es exactamente lo que vez aquí? Pasaste tanto tiempo luchando contra tus hermanos que no pudiste prever el peor de los escenarios. Al final del camino, tú no ganas, Quetzalcóatl no gana ni Tezcatlipoca. Puede que Qutezalcóatl y Tezcatlipoca terminaran con ciudades. No importa, solo tú eres el que destrozara civilizaciones enteras. Felicidades.
\\
\par
\textsc{\underline{Xólotl}}
\\
\par
Ha sido suficiente. ¡Nexoxcho! Tienes razón. Temo a morir y solo me importa lo que me pase a mí. Me importa un bledo la jerarquía divina y lo que le pase al resto de los dioses. Estoy cansado de vivir a la sombra de los demás como muchos otros. Le temo a la muerte tanto como ustedes le temen al cambio. Viven sus vidas gozando de su posición privilegiada que se han olvidado de algo muy importante: todo ciclo tiene un fin. (Xólotl usa un ataque de energía y rompe la ilusión, vuelve a estar en las profundidades del río). Debo de encontrar a Malinalli.  
\end{center}