
%%FORMATO PARA INGRESAR DEFINICION
%%%  \newglossaryentry{nombre} %como será escrito en el texto
%%%   {
%%%	   name={Nombre}, %como será escrito en el %%%   glosario
%%%	   description={aquí defines que es nombre}
%%%   }

\newglossaryentry{IPS}
{
	name={In-Plane Switching},
	description={Es un panel LCD clásico que tiene una disposición interna de los cristales líquidos que evita las fugas y pérdidas de luz, fenómeno que degradaría la calidad de imagen y la definición de colores, especialmente los oscuros. Cabe destacar que los colores dispuestos en dicho cristal líquido son iluminados por detrás con una segunda capa de focos LED.}
}

\newglossaryentry{CC BY-NC-SA}
{
	name={Atribución-NoComercial-CompartirIgual },
	description={Esta licencia permite a otros distribuir, remezclar, retocar, y crear a partir de tu obra de modo no comercial, siempre y cuando te den crédito y licencien sus nuevas creaciones bajo las mismas condiciones.}
}

\newglossaryentry{NPC}
{
	name={Non-playable character},
	description={Personaje no jugable en un juego.}
}

\newglossaryentry{Checkpoint}
{
	name={Punto de control},
	description={Punto de progreso temporal en el que el juego guardará esa posición y funcionara como punto de inicio mientras estés jugando.}
}

\newglossaryentry{Centla}
{
	name={Centla},
	description={Centla es un municipio del estado mexicano de Tabasco, localizado en la región del río Usumacinta y en la subregión de los Pantanos.}
}

\newglossaryentry{Tianguis}
{
	name={Tianguis},
	description={Tianguis (del náhuatl tiānquiz(tli) 'mercado') es el mercado tradicional que ha existido en Mesoamérica desde la época prehispánica y que ha ido evolucionando en forma y contexto social a lo largo de los siglos.}
}


