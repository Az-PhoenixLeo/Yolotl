\chapter{BGM y SFX.}
\section{BGM}
\subsection{Música  mercado}\label{Musica:N01_ZN01}
La música de fondo que suena en el mercado debe de evocar un sentimiento de cotidianidad, seguridad y alegría. En el tema predominaran instrumentos de viento. 		
\subsection{Música  selva}\label{Musica:N01_ZN02}
La música en la selva debe de ser sobre inicio de un viaje. Debe de evocar en el jugador el deseo de aventura y a la vez hacerlo sentir que está en un lugar conocido.
\subsection{Música  Xolotl} \label{Musica:Xolotl}
La música cuando Xólotl le habla a Malinalli debe de ser un tema que se ajuste con la personalidad del Dios ya que será el tema que sonara cuando él tenga diálogos importantes que decir, así que debe ser un tema que muestre determinación y misterio.
\subsection{Música plataforma segundo nivel}\label{Musica:N02_ZN01}
La musicalización que corresponde a la zona de enemigos normales de este nivel debe de reflejar que uno se encuentra en la entrada de un lugar misterios y peligroso, pero de gran carga espiritual (Por lo que se recomienda alguna canción que contenga coros, los coros preferiblemente deben de estar en Náhualt y deben de hablar sobre el sufrimiento de las almas que no pudieron cruzar).
\subsection{Música jefe segundo nivel}\label{Musica:N02_ZN02}
La música que acompaña a la batalla con Xochitonal debe de hacer sentir peligro que se está luchando por sobrevivir pero sin perder la carga espiritual que conlleva este nivel. 
\subsection{Música plataforma tercer nivel} \label{Musica:N03_ZN01}
Zona de plataformas: La Música  de este nivel debe reflejar el sentimiento de adrenalina y vértigo que lleva escalar una superficie frágil y peligrosa.
\subsection{Nombre: Música jefe tercer nivel} \label{Musica:N03_ZN02}
Batalla contra el jefe: La batalla contra el jefe debe de mostrar un tema que evoque la misma adrenalina de peligro con el agregado de que la sensación de peligro se intensifica.

\subsection{Música plataforma cuarto nivel}\label{Musica:N04_ZN01}
La música del nivel evoca a que se está explorando un lugar misterioso y peligroso pero lleno de maravillas.
\subsection{Nombre: Música jefe cuarto nivel} \label{Musica:N04_ZN02}
La música de batalla contra el jefe debe de reflejar el ritmo desenfrenado de batalla que tiene Itzpapálotl, se propone una canción parecida a Bipolar Nightmare de la banda sonora de Nier Automata.

\subsection{Música plataforma quinto nivel} \label{Musica:N05_ZN01}
Este tema se caracterizara por tener una fuerte carga de nostalgia y renuncia hacia las personas que se aman, ya que representa la nostalgia que siente Mictlecayotl por todo lo que perdió cuando fue condenada a estar en el Mictlán por tratar de asesinar a Tezcatlipoca. El tema puede contener algunas notas como de caja de música y coros.

\subsection{Música jefe quinto nivel} \label{Musica:N05_ZN02}
Este tema debe de contener un poco de la nostalgia pero a su vez debe de denotar fuerza y violencia. Su ritmo debe de ser acelerado pues refleja la personalidad de fuerte e innegociable de Mictlecayotl.


\subsection{Música plataforma sexto nivel} \label{Musica:N06_ZN01}
\subsection{Música jefe sexto nivel} \label{Musica:N07_ZN01}


\section{Efectos de sonido} 
\subsection{Bullicio.} \label{SFX:Bullicio}
Efecto sonóro de fondo de gente hablando y caminando en el tianguis
\subsection{Pasos.}\label{SFX:Pasos}
Este efecto se activará cuando Malinalli camine.
\subsection{Viento.}\label{SFX:Viento}
Sonido de una briza de viento, se usará en el juego para .
\subsection{Ladrido.}\label{SFX:Ladrido}
Este efecto sonará cuando Xólotl le robe el objeto a Malinalli.
\subsection{Explosión de agua.} \label{SFX:ExpAgua}
 Este sonido se usará cuando de utilice la habilidad burbuja (ver apartado \ref{hab.burbujas}).
\subsection{Salpicadura de agua.} \label{SFX:SalAgua}
Sonido cuando un personaje salta y aterriza en el agua.  
\subsection{Rugido.} \label{SFX:Rugido}
Este sonido sera utilizado cuando se active la habilidad de rugido (ver apartado \ref{hab.RugAtur}) o la habilidad de Lluvia de rocas (ver apartado \ref{hab.LLuviaRocas}).
\subsection{Roca estrellándose.} \label{SFX:RocaEs}
Este sonido acompañará las habilidades impacto (ver apartado \ref{hab.impacto}) y lluvia de rocas (ver apartado \ref{hab.LLuviaRocas}).
\subsection{Aleteo de alas.} \label{SFX:Aleteo}
Sonido del aleteo de una mariposa. Ese sonido se usará cada vez que Itzpápalotl se mueve.