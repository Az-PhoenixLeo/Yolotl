\chapter{Conclusiones}
En esta sección se presentan a manera de conclusiones las adiciones que se harían 
sobre la metodología \textit{huddle} a fin de mejorarla y el trabajo a futuro a 
realizar sobre el juego.

\section{Adiciones a la metodologia \textit{huddle}}
En esta sección se muestran las adiciones a la metodología \textit{huddle} como 
producto de analizar las estrategias desarrolladas durante todo el proyecto en 
la etapa de postproducción. Las modificaciones que se proponen son las siguientes:

\begin{itemize}
	\item \textbf{Dos documentos de diseño}. Para mejorar el diseño del documento 
	de diseño se propone realizar dos documentos de diseño: uno para los programadores 
	y otro para los artistas. La necesidad de separar el documento de diseño en dos 
	nace en la etapa de producción en la que se perdía de tiempo leyendo el documento 
	de diseño la hora de programar los actores ya que este contenía tanto información 
	de diseño de personajes como información para la programación del comportamiento 
	de los actores; un documento especializado para programadores ahorraría tiempo 
	en su consulta ya que incluiría unicamente información que le resulta importante 
	al programador. Dividir el documento de diseño también permitiría una 
	especialización del mismo; bajo este enfoque ambos documentos contendrán 
	información relevante solo para los perfiles que lo leen pudiendo incluir en 
	el documento de diseño del programador los diagramas de clases y parte de la 
	documentación desde la etapa de la preproducción. Por otra parte, el documento 
	del artista haría más énfasis en el desarrollo de la historia, los personajes, 
	los escenarios, los diálogos, el diseño del lore, etc. El reto que vendría con 
	la creación de dos documentos de diseño sería el lograr que ambos documentos 
	se encuentren en sinfonía y ninguno tenga información que contradiga al otro, 
	ya que ambos deben de hablar del mismo proyecto variando unicamente el enfoque 
	desde el que se observa dicho proyecto.

	\item \textbf{Orientar el desarrollo a los componentes no a los niveles}. 
	Orientar el desarrollo a componentes permite paralelizar el desarrollo del 
	proyecto, haciendo que cada integrante del equipo se encargue de una cantidad 
	de componentes determinada sin que se generen conflictos de código. Reuniendo 
	todos los componentes hasta que se construyan los niveles. El principal reto 
	de trabajo de esta reorientación del trabajo es que requiere que todos los 
	integrantes del equipo de desarrollo sigan las pautas de diseño al pie de la 
	letra y en caso de modificaciones, debe existir una buena comunicación para que 
	se realicen los cambios sin que estos impacten significativamente en el trabajo 
	ya realizado.

	\item\textbf{Mantener el modelo actores-controladores para futuros proyectos}. 
	Esto debido a que dicho modelo permite tener escalabilidad dentro del proyecto.
\end{itemize}
\section{Trabajo a futuro}
Al termino del presente trabajo termina y a fin de mejorar la experiencia de 
los jugadores, los siguientes puntos pueden ser mejorados o agregados a fin de 
enriquecer la experiencia del jugador:

\begin{itemize}
	\item Mejorar el comportamiento del salto.
	\item Reducir el tiempo de respuesta del personaje.
	\item Aumentar indicadores numéricos a las barras de vida y de \textit{tonalli} 
	para que el jugador sepa que cantidad exacta le queda de dichos atributos.
	\item Agregar mensajes de confirmación a los botones cuya funcionalidad es 
	abandonar el nivel que se esta jugando.
	\item Implementar nuevos bloques de animación en enemigos y objetos a fin de 
	enriquecer la experiencia visual del juego.
	\item Agregar animación de personajes a las cinemáticas. 
	\item Agregar audio de voces en náhuatl para los diálogos de las cinemáticas.
	\item Agregar una animación a los enemigos para indicar que han recibido daño.
	\item Agregar una barra de vida a los enemigos.
	\item Agregar música propia al juego.
\end{itemize}

