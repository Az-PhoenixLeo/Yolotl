\chapter{Cuestionario de aceptación del juego} \label{Anexo:Cuestionario}
En este apéndice se muestra el cuestionario que se le presentó a las jugadores durante la etapa de pruebas de aceptación.
\\
\par
Háblanos de ti:
\\
\par
Responde según se le pide.
\begin{enumerate}
    \item Género:
        \begin{enumerate}
            \item Mujer.
            \item Hombre.
            \item Prefiero no decirlo.
        \end{enumerate}
    \item Edad:
        \begin{enumerate}
            \item De 13 a 15 años.
            \item De 16 a 18 años.
            \item De 19 a 21 años.
            \item De 22 a 24 años.
            \item Mayor de 24 años
        \end{enumerate}
    \item Marca del dispositivo móvil para la prueba:
        \begin{itemize}
            \item Sony.
            \item Samsung.
            \item Huawei.
            \item Motorola.
            \item ZTE.
            \item Xiaomi.
            \item Otro.
        \end{itemize}
    \item Versión de Android de su dispositivo
        \begin{enumerate}
            \item Alguna distribución inferior a Android 5.
            \item Alguna distribución de Android 5.
            \item Alguna distribución de Android 6.
            \item Alguna distribución de Android 7.
            \item Alguna distribución superior a Android 7.
        \end{enumerate}
    \item Usualmente cuantas horas de juego acostumbra en dispositivos móviles:
        \begin{enumerate}
            \item Menos de una hora.
            \item Más de 1 pero menos de 2 horas.
            \item Más de 2 pero menos de 3 horas.
            \item Más de 3 pero menos de 4 horas.
            \item Más de 4 horas.
        \end{enumerate}
\end{enumerate}
Háblanos del juego:
\\
\par
Responde según se le pide.
\begin{enumerate}
    \item El nivel a probar es
        \begin{enumerate}
            \item La chica y el perro (Ciudad).
            \item La chica y el perro (Selva).
            \item Nadie cruza mis dominios(Plataforma).
            \item Nadie cruza mis dominios(Jefe).
            \item La guarida del jaguar (Plataforma).
            \item La guarida del jaguar(Jefe).
            \item Alas de obsidiana (Plataforma).
            \item Alas de obsidiana (Jefe).
            \item El viento del norte (Plataforma).
            \item El viento del norte (Jefe).
            \item Sin gravedad (Plataforma).
            \item Sin gravedad (Jefe).
            \item Castigo (Plataforma).
            \item Castigo (Jefe).
            \item La última batalla del jaguar (Plataforma).
            \item La última batalla del jaguar (Jefe).
            \item El último caballero del rey (Plataforma A).
            \item El último caballero del rey (Plataforma B).
            \item El último caballero del rey (Jefe).
            \item El rey del Mictlán (Jefe 01).
            \item El rey del Mictlán (Jefe 02).
            \item El rey del Mictlán (Jefe 03).
        \end{enumerate}
    \item Del 1 al 5, considerando el 5 como excelente y el 1 como pésimo,
    ¿Cómo consideras el movimiento del personaje?
        \begin{enumerate}
            \item 1.
            \item 2.
            \item 3.
            \item 4.
            \item 5.
        \end{enumerate}
    \item En caso de que consideres ineficiente o pésimo el movimiento del
    personaje, ¿Qué es lo que hace que lo consideres de esa forma?
        \begin{enumerate}
             \item La velocidad de movimiento es muy lenta.
             \item La velocidad de movimiento es muy rápida.
             \item El salto es inestable.
             \item Otro.
        \end{enumerate}
    \item Del 1 al 5, considerando el 5 como excelente y el 1 como pésimo, ¿Cómo
    consideras la respuesta de la GUI?
        \begin{enumerate}
            \item 1.
            \item 2.
            \item 3.
            \item 4.
            \item 5.
        \end{enumerate}
    \item En caso de que consideres ineficiente o pésimo la respuesta de la GUI,
    ¿Qué es lo que hace que lo consideres de esa forma?
        \begin{enumerate}
            \item El tamaño de los botones es muy pequeño.
            \item La posición del botón del salto no es intuitiva,  debería
            estar en la posición del botón del disparo.
            \item El personaje tarda mucho en responder.
            \item Otro.
        \end{enumerate}
    \item Del 1 al 5, considerando el 5 como excelente y el 1 como pésimo, ¿Cómo
    consideras el uso de tonalli(magia)?
        \begin{enumerate}
            \item 1.
            \item 2.
            \item 3.
            \item 4.
            \item 5.
        \end{enumerate}
    \item En caso de que consideres ineficiente o pésimo el uso de Tonalli, ¿Qué
    es lo que hace que lo consideres de esa forma?
        \begin{enumerate}
            \item No hay suficiente cantidad de Tonalli.
            \item No hay forma de ver la cantidad de disparos que puedo hacer.
            \item La restricción de una cantidad específica de tonalli.
            \item Otro.
        \end{enumerate}
    \item Del 1 al 5, considerando el 5 como excelente y el 1 como pésimo, ¿Cómo     consideras la actualización de la barra de vida?
        \begin{enumerate}
            \item 1.
            \item 2.
            \item 3.
            \item 4.
            \item 5.
        \end{enumerate}
    \item En caso de que consideres ineficiente o pésima la actualización de la
    barra de vida ¿Qué es lo que hace que lo consideres de esa forma?
        \begin{enumerate}
            \item No hay un sonido que indique cuando estoy próximo a quedarme
            sin vida.
            \item No hay una animación más llamativa que indique que he recibido
            daño.
            \item No hay un indicador numérico que me permita ver cuanta vida me
            queda.
            \item Otro.
        \end{enumerate}
    \item Del 1 al 5, considerando el 5 como excelente y el 1 como pésimo, ¿Cómo
    consideras la actualización de la barra de tonalli?
        \begin{enumerate}
            \item 1.
            \item 2.
            \item 3.
            \item 4.
            \item 5.
        \end{enumerate}
    \item En caso de que consideres ineficiente o pésima la actualización de la
    barra de tonalli ¿Qué es lo que hace que lo consideres de esa forma ?
        \begin{enumerate}
            \item No hay sonido que indique cuando me he quedado sin tonalli.
            \item No hay una animación más llamativa que indique que he gastado
            tonalli.
            \item No hay un indicador numérico que me permita ver cuántos
            disparos me quedan.
            \item Otro.
        \end{enumerate}
    \item Del 1 al 5, considerando el 5 como muy fuerte y el 1 como muy débil,
    ¿Qué tan fuerte es (son) el (los) enemigo(s)?
        \begin{enumerate}
            \item 1.
            \item 2.
            \item 3.
            \item 4.
            \item 5.
        \end{enumerate}
    \item En caso de que nivel sea una plataforma, seleccione el enemigo que
    consideres más débil:
        \begin{enumerate}
            \item Fantasma morado.
            \item Fantasma rojo.
            \item Armadillo.
            \item Zopilote.
            \item Jaguar.
        \end{enumerate}
    \item ¿Qué es lo que hace que consideres a ese enemigo como el más fuerte?
        \begin{enumerate}
            \item Su patrón de movimiento.
            \item Su cantidad de daño.
            \item Su cantidad de vida.
            \item Su modo de ataque.
            \item Otro.
        \end{enumerate}
    \item En caso de que nivel sea una plataforma, seleccione el enemigo que
    consideres más fuerte:
        \begin{enumerate}
            \item Fantasma morado.
            \item Fantasma rojo.
            \item Armadillo.
            \item Zopilote.
            \item Jaguar.
        \end{enumerate}
    \item ¿Qué es lo que hace que consideres a ese enemigo como el más débil?
        \begin{enumerate}
            \item Su patrón de movimiento.
            \item Su cantidad de daño.
            \item Su cantidad de vida.
            \item Su modo de ataque.
            \item Otro.
        \end{enumerate}
    \item Cuantas veces moriste antes de completar el nivel.
        \begin{enumerate}
            \item 0 o 1 vez.
            \item 2 o 5 veces.
            \item 5 u 8 veces.
            \item Más de 8 veces.
        \end{enumerate}
    \item ¿A que se debe que haya muerto la mayoría de las veces?
        \begin{enumerate}
            \item El(los) enemigo(s) son demasiado fuertes.
            \item El(los) enemigo(s) son demasiados.
            \item Por un obstáculo (Rocas, viento, estalagmitas).
            \item El personaje tarda mucho en responder a los botones que
            oprimo.
            \item Me quede sin magia en un momento decisivo de la batalla.
            \item Falta de items para curarme.
            \item Falta de item para recuperar tonalli.
        \end{enumerate}
    \item ¿Consideras tus muertes como un factor de estrés o como un factor de reto?:
        \begin{enumerate}
            \item Es un factor de estrés.
            \item Es un factor de reto.
        \end{enumerate}
\end{enumerate}