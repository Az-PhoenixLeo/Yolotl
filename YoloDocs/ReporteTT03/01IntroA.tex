\chapter{Introducción}

En esté trabajo se presentará la realización del proyecto final para titulación de la carrera de Ingeniería en Sistemas Computacionales, realizada en el Instituto Politécnico Nacional en la Escuela Superior de Cómputo, ubicada en la Ciudad de México, México, en fecha al mes de Mayo del año 2018. Dicho trabajo consistirá en el desarrollo de un videojuego con temática de la cultura prehispánica en México, en específico la cultura Mexica, aplicando los conocimientos adquiridos durante la estancia en la carrera. 

Este trabajo se realiza con el propósito de encontrar una nueva alternativa de medio de transmisión para este tipo de contenido cultural e historico del país. También se realiza con el fin de entender la nuevas herramientas del mundo laboral en este campo de trabajo, pues se pretende entender y utilizar el motor de juego Unity que hoy en día es utilizado para creación de muchos de los videojuegos que existen actualmente. También se usa como delimitante de desarrollo, que el videojuego será realizado para un dispositivo móvil, pues se verá en el siguiente documento el incremento de uso de dispositivos móviles en la población tanto mundial como nacional a lo largo de los años de forma bastante acelerada.	 

En la época actual nos vemos rodeados por tecnología por todos lados, esta ya es parte de nuestra vida diaria, de nuestras actividades tanto de trabajo, de nuestro medio de comunicación, como medio de información e incluso de nuestro medio de entretenimiento. Las generaciones recientes han crecido con la evolución de la tecnología a un ritmo acelerado, a tal punto que la tecnología ya es parte de su cultura.

Dentro de la evolución de la tecnología se encuentra los videojuegos, una industria de entretenimiento.  
Nos daremos a la tarea de investigar los puntos de impacto que tiene. Pues a primera instancia podremos observar el tipo de personas que juegan, los ingresos que se generan en esta industria, los tipos de industria que existen, las consecuencias positivas, las consecuencias negativas que generarían, las ganancias como profesionista en esta rama y como realizar un proyecto de esta naturaleza.

Los videojuegos hacen al jugador involucrarse con varios sentidos a la vez en lo que se le presenta, así se crea una experiencia propia como cualquiera de la vida pues provoca la inmersión. El INJUVE ha dado a conocer información donde se puede observar quelos jóvenes y los videojuegos llevan una interacción diaria y sobre cualquier tema, desde educativo hasta de ocio. Aquí los jóvenes representarán el público a alcanzar del proyecto, es por eso que se necesitará de realizar pruebas de estudio para conocer tanto la efectividad del medio que hemos escogido para transmitir la información que queremos, como aquellas características que pueden ser útiles a posterior, que nos indiquen cuales son los factores que interesan al público y puedan ser utilizados para un producto con mayor impacto.

Por otra parte tenemos que la sociedad ignora en su mayoría los aspectos históricos culturales propios de su país, específicamente de México.Se denota un gran desinterés por parte de la gente el siquiera conocer su legado. Esto dado la investigación que llevaremos, nos daremos cuenta sobre los aspectos negativos que ocasiona esta situación, pues de manera psicológica e inconsciente, refleja por parte de la gente un nivel de desprecio a su propio país el hecho no conocerlo por completo. Aunque es un tema complicado de explicar se harán las referencias pertinentes para aquellos que quieran abundar en el tema. Es aquí donde se da la importancia del proyecto, pues no solo hablamos de comunicar contenido e ideas, no solo hablamos de encontrar una manera diferente de ocio para los usuarios, si no de utilizar la tecnología que se tiene al alcance que está en desarrollo, para cambiar incluso el pensamiento de una nación, no solo por el hecho del autoconocimiento nacional, si no para inspirar y motivar al patriotismo, a un espíritu de conciencia y que cada persona pueda colaborar en impulsar logros y la realización de un país con sentido de orgullo.

Las herramientas y metodologías para combinar temas de manera armoniosa tambien han ido evolucionando y desarrollándose junto con la tecnología. Al presente tenemos nuevos métodos, que facilitan la interacción de las personas con eventos, situaciones o contexto en específico de forma que sea atractiva para gente, claro que esto depende mucho de aspectos psicológicos y de personalidad de cada uno, pero se incluyen recomendaciones dentro de los métodos para mejorar el acercamiento del individuo con la actividad que se desea que realice, se verán diversas maneras para lograr esto y todo dependerá del rumbo que tome el proyecto durante su desarrollo.

Tomando en cuenta todos los aspectos anteriores, el proyecto consistirá en juntar ambas ideas para usar el videojuego como difusión del aspecto cultural mexicano. Se investigará más a fondo la historia de los videojuegos para elegir el público objetivo potencial o las personas alcanzables, los procesos y metodología para poder realizar un videojuego, aspectos a considerar para el proyecto, la cultura y ramas que abarca dentro de la sociedad, la cultura y la tecnología como se relacionan entre sí, las herramientas con las que se cuentan para la realización del videojuego, las teorías que pueden usarse, los conflictos que pueden ocurrir, como solucionar los problemas que se nos presenten y por su puesto el cambio o impacto que se tenga al presentarlo al público.

Al final se presentará un videojuego como producto con pruebas de diferentes tipos para demostrar el resultado ante la sociedad y los cambios que se presentaron en las personas.