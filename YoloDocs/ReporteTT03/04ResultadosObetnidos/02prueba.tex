\subsection{Prueba de integración}
Esta prueba se realiza una vez se integraron los actores y controladores a los
niveles.

\subsubsection{Objetivo}
Verificar el funcionamiento lógico de los componentes del nivel al ser integrados
para formar un nivel entero.

\subsubsection{Herramientas}
Para la realización de esta prueba se utiliza el motor gráfico de \textit{Unity.}

\subsubsection{Aplicación}
Para realizar esta prueba es necesario jugar los niveles y observar que el
comportamiento de los controladores y los actores se ejecute correctamente al
integrarse con otros actores. En esta prueba también se ajustan las áreas activas de las
plataformas a fin de que su funcionamiento no se detenga si se alejan mucho del
jugador al realizar su recorrido.
                
\subsubsection{Resultados}
Al finalizar esta prueba se puede verificar que los controladores funcionan de
manera correcta; sin embargo, es necesario realizar ajustes referentes a los
tiempos de transiciones entre escenas y los valores de las áreas activas de
varias plataformas y obstáculos ya que con sus valores iniciales algunas
plataformas se detenían al realizar su recorrido dado que el jugador se salía
de su área activa y se volvía inalcanzable. En cuanto al obstáculo de
\textit{WindCreator} se ajusto el tamaño del área activa garantizando que el
obstáculo se encuentre activo cuando el jugador llegue a donde se encuentra éste. 

\subsubsection{Conclusiones}
Al finalizar esta prueba se puede concluir lo siguiente:
\begin{itemize}
    \item El desarrollo orientado a componentes facilita identificar los errores en
    las clases actoras antes de ser integradas a los niveles.
    \item Existen errores que sólo pueden ser detectados al integrar más de un actor
    y en situaciones muy específicas como el caso de los marcadores.
    \item No solo los errores de funcionamiento en el código de las clases actoras
    impactan negativamente en la experiencia del jugador; en ocasiones la experiencia
    se ve afectada por los valores que se le asignan a las clases que componen los
    actores como es el caso de los jefes y de los obstáculos.
\end{itemize} 