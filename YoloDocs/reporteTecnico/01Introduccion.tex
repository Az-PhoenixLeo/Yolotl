\chapter{Introducción}

%introducción **anteproyecto**
%tema: problema de estudio, teoria asociada con el tema, antecedentes históricos
%objetivo
%alcance
%plan de desarrollo (se divide en x partes)
%valor o importancia del tema (opcional)
%resumir logros (opcional)

En este trabajo el problema de estudio será la ignorancia que existe de la sociedad en México sobre su propia cultura. Para ello definiremos lo que es la cultura y el como toma lugar específicamente en este país. Pues la investigación que se mostrará señala la relación que existe entre el desarrollo social de un país, con el saber que la gente tiene de su propia historia, arte, tradiciones y más conocimiento involucrado que se verá. La información será mostrada en estadísticas y encuestas a lo largo de los años, hasta una comparación reciente. 
\\[1pt]

Para atacar este problema considerarémos como propuesta de solución la creación de un videojuego. Este videojuego no solo deberá cumplir con las características y componentes de un videojuego común, si no también deberá pasar por un proceso llamado gamificación y cumplir con requisitos que ayudarán al nivel de impacto de la solución del problema. Para ello veremos lo que significa un videojuego y los pasos a seguir para el proceso de gamificación.
\\[1pt]

Como antecedentes históricos en el tema de videojuegos, tenemos que desde el año 1952 se tiene registro de elaboración de juegos en medios digitales. como ejemplo en este mismo año se tiene el juego OXO desarrollado por Alexander S. Douglas como proyecto de tesis de su universidad, que consiste en el juego comúnmente conocido aquí como "gato", donde la interacción se daba entre un jugador humano y una máquina. Los primeros juegos tenían como propósito solo el entretenimiento y en cierta medida la investigación de las capacidades de la nueva tecnología que se presentaba en la época, pues por el momento se daban dentro de institutos o centros de estudio. Poco a poco se iban introduciendo nuevos componentes o mejoras a los videojuegos como la interfaz o el hecho de poder involucrar a dos jugadores en un mismo juego y convertirlo en una competición. Después por el año de 1966 los videojuegos empezaron a interactuar con otros dispositivos físicos como el televisor. Así los videojuegos empezaron a adentrarse en el mundo de la industria y comercio, lo que ocasionó una evolución acelerada de los mismos. A partir de los años 1900 es cuando los videojuegos toman el inicio de una fuerte popularidad comercial. Esta popularidad incremento las disciplinas y herramientas para el desarrollo de un videojuego, ocasionando de igual manera la creación de nuevos objetivos de un videojuego, como ejemplo de ello son los videojuegos educativos. Al final, actualmente los videojuegos es una industria que genera más dinero que el cine, toma a cualquier público objetivo y es usado como herramienta, que abarca no solo el entretenimiento. Ahora en vez de ser los videojuegos solo un estudio de la tecnología, los videojuegos se utilizan como apoyo de estudio a otra áreas.    
\\[1pt]

El objetivo consiste en que el videojuego a crear, de una manera divertida insite a los jugadores a un interés mayor al tema de la cultura. No se perderá de vista como elemento principal el entretenimiento del jugador, ya que se quiere de una manera implícita enseñar al jugador historia y mitología de México. El contexto histórico que se tomará será la época prehispánica Méxicana, donde claramente existiran eventos de fantasía para no perder la atención del jugador. Al final el alcance que se busca, es que el jugador se motive a aprender más al respecto y lo mostrado en el juego se haya aprendido en cualquier medida. 
\\[1pt]

Para ello este reporte se divide en una descripción detallada del planteamiento del problema, el marco teórico que cuenta con los temas a entender que son; los videojuegos, su definición, la clasificación que existe, la industria mundial que lo rodea, un estudio de mercado y la industria en México; el desarrollo de videojuegos, las metodologías que existen, que es pipeline, los motores gráficos que se utilizan y el software auxiliar a ocupar; la gamificación, sus características, los tipos de jugadores que existen, el proceso para realizarlo y la finalidad; la cultura; la cultura digital, la división que ha tomado la educación cultural; por último en el marco teórico los videojuegos educativos, después se seguirá con el estado del arte para entender las características del proyecto, el trabajo realizado para ver con lo que se ha encontrado, los resultados obtenidos y al final las observaciones que se tendrán del proyecto.
\\[1pt]

Los videojuegos están dentro de las tecnologías, las cuales están en constante cambio y evolución. Así de igual manera, los medios tecnológicos pueden ayudar a la evolución de otras áreas y problemas actuales. Por ello es importante aprender a usar la tecnología y usarla en beneficio.
\\[1pt]
