\chapter{Bibliografia}	

\begin{thebibliography}{X}
	%%%%%%%%%%%%INTERNET%%%%%%%%%%
	%Iniciales y Apellido del autor (año, mes y día). Título (edición) [Tipo de medio, generalmente Online]. Available: Url
	\bibitem{gameficacion} 
	\textnormal S. Moll (2014, Junio 5). Gamificación: 7 claves para entender qué es y cómo funciona[Online]. Available: http://justificaturespuesta.com/gamificacion-7-claves-para-entender-que-es-y-como-funciona
	
	\bibitem{gameDef} 
	\textnormal S. Deterding, D. Dixon, R. Khaled, L.E. Nacke (2011, Mayo 7). Gamification: Toward a Definition[Online]. Available: http://gamification-research.org/wp-content/uploads/2011/04/02-Deterding-Khaled-Nacke-Dixon.pdf

	\bibitem{vid01} 
	\textnormal P. Antolinos, "La industria del videojuego generará casi 109.000 millones de dólares en 2017", \textit{Forbes} [Online]. Año 2017, Agosto 28. Available: http://www.periodistadigital.com/tecnologia/gadgets/2017/08/28/la-industria-del-videojuego-generara-casi-109-000-millones-de-dolares-en-2017.shtml
	
	\bibitem{vid02} 
	\textnormal Entertainment Software Association (1998-2017). Guía de clasificaciones de la ESRB[Online] Available: http://www.esrb.org/ratings/ratings\_guide\_sp.aspx
		
	\bibitem{vid03} 
	\textnormal E. Zuñiga, "Videojuegos en México: un mercado de más de 22,000 mdp", \textit{Forbes} [Online]. Año 2017, Mayo 19. Available: https://www.forbes.com.mx/videojuegos-mexico-mercado-mas-22000-mdp/
	
	\bibitem{vid04} 
	\textnormal A. Ling (2017, Abril 30). Sobre el desarrollo de videojuegos en México[Online]. Available: https://www.unocero.com/videojuegos/sobre-el-desarrollo-de-videojuegos-en-mexico/
	
	
	\bibitem{vid05} 
	\textnormal R.S. Contreras (2017, Septiembre 18). La industria del videojuego en México[Online]. Available: http://invdes.com.mx/los-investigadores/la-industria-del-videojuego-mexico/
	
	\bibitem{vid06} 
	\textnormal C. Bourne, V. Salgado (2016, Diciembre 22). Los videojuegos pueden transformar el aula[Online]. Available: http://www.aikaeducacion.com/tendencias/los-videojuegos-transforman-aula/
	
	\bibitem{vid07} 
	\textnormal M. Herger, A. Keeler, R. Nemire, A.W. Schwarz, M. Turchinsky, B. Yuhnke (2014, Marzo 5). Games and gamification[Online]. Available: https://www.nmc.org/event-archive/nmc-on-the-horizon-games-and-gamification/
	
	\bibitem{vid08} 
	\textnormal "¿Cuál es la misión de los centros de cultura digital", \textit{El Universal} [Online]. Año 2017, Agosto 11. Available: http://www.eluniversal.com.mx/articulo/cultura/2017/08/11/cual-es-la-mision-de-los-centros-de-cultura-digital
	
	\bibitem{vid09} 
	\textnormal J. Freire, "Cultura digital y prácticas creativas en la educación", \textit{Universidad y sociedad del conocimiento} [Online], vol. 6, no. 1, 2009. Available: http://openaccess.uoc.edu/webapps/o2/bitstream/10609/3231/1/freire.pdf
	
	\bibitem{defVid} 
	\textnormal G.A. Morales, C.E. Nava, L.F. Fernández, M.A. Rey, "Procesos de desarrollo para videojuegos", Tesis [Online], Instituto de Ingeniería y tecnología, UACJ, Juárez, CH, Mex. 2010. Available: erevistas.uacj.mx/ojs/index.php/culcyt/article/download/299/283
	
	\bibitem{vid10}
	\textnormal R. Hunicke, M. LeBlanc, R. Zubek, "MDA: A formal approach to game design and game research" [Online], Game developers conference, San Jose, 2004. Available: http://www.cs.northwestern.edu/~hunicke/MDA.pdf

	\bibitem{vid11}
	\textnormal M.J.P. Wolf, \textit{El medio de los videojuegos}. Primera edición. Texas, USA, 2001.
	
	{El medio de los videojuegos}, en el capítulo sexto, titulado “El género y el videojuego”, del libro por Mark J. P. Wolf
	
	\bibitem{vid12}
	\textnormal L.A. Chong , "Analisis de las artes digitales en la creación de personajes y jugabilidad interactiva", Tesis [Online], 2015. Available: https://www.emaze.com/@AFRICWZL/Tesis-Artes-Digitales
\end{thebibliography}