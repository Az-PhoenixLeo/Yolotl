\subsection{Primer Prototipo}
Con el fin de poder familiarizarse con el motor de Unity
se desarrollo un primer prototipo. Este primer prototipo basaba su funcionamiento 
en algunas de las mecánicas de la sección de plataformas del segundo nivel, con 
algunas diferencias en cuanto al manejo de la cantidad de vida, el personaje 
jugable y la colocación de algunos botones en la GUI del juego.
\\
\par
En el primer prototipo el jugador puede:
\begin{itemize}
	\item Mover al persoanje jugable hacia la derecha.
	\item Mover al personaje jugable hacia la izquierda.
	\item Hacer que el personaje jugable salte.
	\item Hacer que el personaje jugable dispare.
\end{itemize}

Con este primer prototipo el equipo de desarrollo se familiarizó con:
\begin{itemize}
	\item Crear Sprites.
	\item Implementar un personaje jugable.
	\item Implementar Interfaz gráfica.
	\item Manejar un score para el juego.
	\item Manejar efectos especiales para determinadas acciones del personaje jugable.
	\item Manejar archivos para preservar datos de la partida.
	\item Crear una inteligencia artificial sencilla para los enemigos.
\end{itemize}

\subsubsection{Creando sprites.}
Los sprites, dentro de unity, son todos aquellos objetos gráficos de dos 
dimensiones. Para este prototipo en particular se crearon sprites haciendo uso 
de Corel Draw X5 y Photoshop y tambien se descargaron algunos sprites desde 
el sitio web GameArt2D.  %% Revisar el uso de marcas registradas Photoshop y Corel

\subsubsection{Implementación del personaje jugable}
En este primer prototipo, el objeto de juego que corresponde al personaje 
jugable se nombró: Cat. 
\\
\par
Para su implementación se manejó una máquina de estados que permitiera la transición 
entre sus bloques de animación cuando se ejecutaran las diferentes acciones
(Ver figura ). %% falta la referencia a la figura ó incluir la figura.

Para lograr la vinculación entre el personaje jugable y su máquina de estados 
correspondiente, al personaje jugable se le agregó el componente Animator 
(Ver figura ); %% falta la referencia a la figura ó incluir la figura.
el cual permite esa vinculación.   
\\
\par 
También fue necesario agregarle al personaje jugable los componentes de 
RigidBody2D, para el manejo de físicas y movimiento, y el componente 
BoxCollider2D, para la detección de colisiones.
\\
\par 
La programación del personaje jugable se realizó con la creación de una 
clase llamada PlayerManager. Esta clase es equivalente a 
la clase Player descrita en el apartado \ref{ClaseCtrl}, con algunas diferencias.
La principal diferencia es que PlayerManager instancía clases de tipo controlador 
para lograr reaccionar ante eventos como cuando 
el personaje jugable recibe daño (ver figura ). %% referencia?? 
\\
\par 
En cuanto como la clase PlayerManager, ésta gestiona acciones como disparar, saltar, 
moverse hacia la izquierda, moverse hacia la derecha, su funcionamiento es igual
al que poseé la clase Player.
\\
\par 
La acción de disparar se consiguió creando un GameObject llamado Bullet. 
Este objeto se le agrego el componente RigidBody2D, para que tuviera fisicas y 
movimiento, y el componente CircleCollider2D, para la deteccion de colisones. 
Este objeto se le vinculó con dos clases: BulletCtrl, clase que se 
encarga del desplazamiento de Bullet, y DestroyWithDelay, clase que destruye el 
objeto Bullet después de un tiempo para evitar que el nivel se sature de objetos.
\\
\par
Para la respuesta a la detección de colisiones, se manejó un un sistemas de Tag 
o etiquetas con el cual se etiquetaban a los objetos como: 
\begin{itemize}
	\item Player: corresponde al personaje jugable.
	\item GameController: corresponde al objeto que instancia la clase LevelCtrl.
	\item Ground: Aquellos objetos que funjan como suelo.
	\item Rock: Objetos que representan el obstáculo piedras filosas.
	\item Enemy: Para los enemigos.
	\item Dog: Para los xoloitzcuintles.
\end{itemize}   
Cuando el personaje jugable detecta una colisión, pregunta por la etiqueta del 
objeto con el que esté colisionando y dependiendo de la etiqueta es la forma 
en que reacciona. A continuación se mencionan las etiquetas que detecta y su 
reacción:
\begin{itemize}
	\item \item Ground: Actualiza atributos de tipo boleano para indicar que el 
	personaje no se encuentra saltando y por lo tanto habilita esta opción para 
	cuando el jugador oprima el botón de salto.
	\item Rock: Instancia la clase LevelCtrl y le pide que ejecute el método 
	PlayerDiedAnimation para indicar la muerte del jugador.
	\item Enemy: Instancia la clase LevelCtrl y le pide que ejecute el método 
	PlayerDiedAnimation para indicar la muerte del jugador.
	\item Dog: Instancia la clase LevelCtrl y le pide que ejecute el método 
	UpdateXoloCount, despues destruye el objeto con el tag Dog y en su posición 
	muestra un efecto de brillos. 
\end{itemize} 

\subsubsection{Implementar Interfaz gráfica} 
Unity permite utilizar GameObject de tipo UI para la impelementacion de la interfaz.
En este prototipo la Interfaz gráfica se construyó a partir de dos Canvas: uno 
para los botones y otro para el conteo de vidas, la puntuación de jugador. 
Ambos Canvas se vincularon con el objeto cámara para garantizar un tamaño constante 
y se configuraron para que se adaptaran al tamaño de la pantalla (ver figura ).    
 \\
 \par
 El canvas que corresponde al conteo de vidas y la puntuación esta vinculado con 
 la clase LevelCtrl, clase que se encarga de la actualización de las vidas y la 
 puntuación del jugador. Mientras que el canvas referente a los botones se encuentra 
 vinculado con la clase MobileUICtrl para comunicar al jugador con la clase Player 
 y con el personaje jugable (ver figura ).  
 
 \subsubsection{Manejo de un score para el juego y preservación de datos de la partida}
 Como se mencionó en el apartado referente al personaje jugable quien se encarga 
 de controlar el score del juego es la clase LevelCtrl. Esta clase es mandada a 
 llamar cada vez que el personaje jugable colisiona con objeto con el tag Enemy, 
 Rock o Dog, o cuando el objeto Bullet colisiona con un objeto con el tag enemigo. 
 Cada vez que se actualiza el score, los datos son almacenados en un archivo 
 binario esto permite que si el jugador muere su progreso no se pierda. Cuando 
 lo que se actualiza es la cantidad de vidas disponible primero se pregunta con 
 cuanta vidas se cuenta  dependiendo de eso es si se reiniciara el juego o se mostrara 
 los botones de fin de partida en los cuales se le pregunta al jugador reanudar o 
 salir de la aplicación.   
 
 \subsubsection{Creación de una inteligencia artificial sencilla para los enemigos}
 Para la creación de inteligencia artificial de los enemigos fue necesaria 
 la implentación de un conjunto de objetos vacíos (es decir, objetos que solo 
 contienen el atributo de Trasnform). Por medio de estos objetos vacios la clase 
 Encargada del patrón de movimiento marcaría los tiempos de desplazamiento entre 
 punto y punto o las acciones que el enemigo realizaría entre punto y punto (ver figura ).  
