\subsection{Segundo Prototipo}
El segundo prototipo modela el comportamiento del primer nivel del juego (ver 
apartado \ref{Niveles}). 

\subsubsection{Creación de Sprites}
Para este prototipo se crearon todos los sprites a utilizar, salvo por los botones 
de la Interfaz de usuario, ya que estos se conservaron del prototipo anterior. En 
total se crearon:
\begin{itemize}
	\item Tres imágenes de fondo.
	\item Tres iconos.
	\item Diez imágenes para la animación de Malinalli.
	\item Seis imágenes para la animación de Xólotl en su forma xoloitzcuintle.
	\item Seis imágenes para la animación de Xólotl en su forma humano.
	\item Una imagen para Quetzalcóatl.
	\item Cinco imágenes para los ciudadanos
	\item Cuatro imágenes para el suelo.
	\item Cuatro imágenes para los objetos de fondo de la ciudad.
	\item Dos imágenes para los objetos de fondo de la jungla.
	\item Cinco imágenes para la animación del Jaguar.
\end{itemize}  
Consultar el Anexo para ver el resto de los sprites creados.

\subsubsection{Menú principal}
En este apartado se modifico la propuesta de diseño que se tenía en las interfaces 
(ver apartado \ref{TraReaInterfaces}), en lo respecta al acomodo de botones (ver 
figura ). Al igual que en el prototipo uno, se configuraron los botones y el 
canvas para que pudieran adaptar su tamaño si se modificaba el tamaño de la 
pantalla.
\\
\par
En cuanto a la funcionalidad, en este segundo prototipo, el menú principal solamente 
tiene habilitado el botón de Empezar partida. De momento el botón de empezar partida 
direcciona únicamente a la primera parte del primer nivel sin aun mostrar las 
cinemáticas correspondientes.
\\
\par
El tipo de letra empleada en esta interfaz fue descargada del sito web 
Dafont y su creadora es Karla Vazquez. 

\subsubsection{Implementación personaje jugable}
En cuanto a la implementación del personaje jugable, se trato de reutilizar la lógica 
del personaje jugable del primer prototipo, con algunas diferencias. A continuación
se listan los cambios más significativos con respecto a la clase del prototipo anterior:
\begin{itemize}
	\item La diferencia más importante es que en esta versión del personaje jugable ya 
	se vincula la interacción entre la clase \textit{Player} y la clase 
	\textit{GroundCollisionCtrl.}
	\item La clase \textit{Player} deja de instanciar clases de tipo Controlador.
	\item El estado \textit{Idle} de la maquina de estados se renombra como
	 \textit{Normal}.
	 \item Se maneja atributos de tipo boolean para que funjan como banderas de 
	 activación de métodos en la vinculación con la clase \textit{MobileUICtrl}.
\end{itemize}

\subsubsection{Primer mitad del nivel uno}
En esta sección se acomodaron los sprites de los ciudadanos, objetos de fondo de la
 ciudad y el suelo, con base en la maqueta del nivel (ver anexo ). En esta prototipo 
 aun no se pudo implementar la clase hija de LevelCtrl que le corresponde al nivel 1; 
 por lo que para emular la transición entre niveles se creó un objeto vació y se 
 le agregó un Box Collider 2D que al hacer contacto con el personaje jugable, realiza 
 la transición a la segunda mitad del primer nivel.
 
 \subsubsection{Segunda mitad del primer nivel}
 En esta segunda mitad se implementó la clase FollowedCharacter para la persecución 
 de Xólotl en su forma xoloitzcuintle. En Xólotl también fue necesario implementar 
 una maquina de estados para sus animación. Como se puede ver en la figura la 
 maquina de estados de Xólotl no es tan compleja como la del personaje jugable Malinalli.   