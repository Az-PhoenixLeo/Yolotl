\subsection{Requerimientos funcionales.}
	\begin{longtable}[c]{ | m{5cm} | m{10cm}|} 
		\hline
		\rowcolor{cyan}RF-001\label{RF:01} & Mostrar mensajes de confirmación. \\ 
		\hline
		Descripción	& El juego mostrará mensajes de confirmación para poder proceder con acciones que afecten de manera irreversible la partida del jugador. \\ 
		\hline
		Estado &  \\ 
		\hline
		Prioridad &  \\
		\hline
	%--------------------------------------------------
		\rowcolor{cyan}RF-002\label{RF:02} & Iniciar partida. \\ 
		\hline
		Descripción	& El jugador podrá iniciar una partida nueva desde el menú principal del juego. \\ 
		\hline
		Estado &  \\ 
		\hline
		Prioridad &  \\
		\hline
		%--------------------------------------------------	
		\rowcolor{cyan}RF-003\label{RF:03} & Cargar partida.\\ 
		\hline
		Descripción	& El jugador podrá cargar una partida existente desde el menú principal del juego. \\ 
		\hline
		Estado &  \\ 
		\hline
		Prioridad &  \\
		\hline
		%--------------------------------------------------
		\rowcolor{cyan}RF-004\label{RF:04} & Seleccionar nivel a jugar. \\ 
		\hline
		Descripción	& El juego contará con un menú de selección de niveles en donde el jugador podrá seleccionar el nivel que desee jugar. \\ 
		\hline
		Estado &  \\ 
		\hline
		Prioridad &  \\
		\hline
		%--------------------------------------------------
		\rowcolor{cyan}RF-005\label{RF:05} & Desbloquear de Niveles. \\ 
		\hline
		Descripción	& El jugador desbloqueará niveles de manera secuencial, es decir, el nivel dos lo desbloqueara al finalizar el nivel uno. \\ 
		\hline
		Estado &  \\ 
		\hline
		Prioridad &  \\
		\hline
		%--------------------------------------------------	
		\rowcolor{cyan}RF-006\label{RF:06} & Controlar al personaje principal. \\ 
		\hline
		Descripción	& El jugador podrá controlar al personaje principal por medio de una interfaz gráfica de usuario. \\ 
		\hline
		Estado &  \\ 
		\hline
		Prioridad &  \\
		\hline
		%--------------------------------------------------	
		\rowcolor{cyan}RF-007\label{RF:07} & Mostrar cantidad de vida del jugador.\\ 
		\hline
		Descripción	& Cuando el jugador juegue un nivel, el juego le mostrará en tiempo real la cantidad de vida con la que dispone.  \\ 
		\hline
		Estado &  \\ 
		\hline
		Prioridad &  \\
		\hline
		%--------------------------------------------------	
		\rowcolor{cyan}RF-008\label{RF:08} & Mostrar cantidad de tonalli del jugador.\\ 
		\hline
		Descripción	& Cuando el jugador juegue un nivel, el juego le mostrará en tiempo real la cantidad de tonalli con la que dispone.  \\ 
		\hline
		Estado &  \\ 
		\hline
		Prioridad &  \\
		\hline
		%--------------------------------------------------	
		\rowcolor{cyan}RF-009\label{RF:09} & Mostrar el progreso de los objetivos del nivel.\\ 
		\hline
		Descripción	& En aquellos niveles con objetivos específicos como encontrar objetos, hablar con otros personajes, etc.; El juego le mostrará al jugador su progreso en tiempo real. \\ 
		\hline
		Estado &  \\ 
		\hline
		Prioridad &  \\
		\hline
		%--------------------------------------------------	
		\rowcolor{cyan}RF-010\label{RF:10} & Guardar progreso general del juego.\\ 
		\hline
		Descripción	& Al terminar un nivel el juego guardara de manera automática el progreso del jugador. \\ 
		\hline
		Estado &  \\ 
		\hline
		Prioridad &  \\
		\hline
		%--------------------------------------------------	
		\rowcolor{cyan}RF-011\label{RF:11} & Guardar progreso del juego dentro del nivel.\\ 
		\hline
		Descripción	& Dentro de los niveles existirán puntos de guardado llamados checkpoints. Estos puntos permitirán al jugador almacenar su progreso y volver a esa posición en caso de morir dentro de un nivel.\\ 
		\hline
		Estado &  \\ 
		\hline
		Prioridad &  \\
		\hline
	\end{longtable}