\subsubsection{CU-003 Seleccionar nivel.} \label{CU:03}
\begin{longtable}[c]{ | m{5cm} | m{10cm}|} 
		\hline
		\rowcolor{cyan}CU-003. & Seleccionar nivel. \\ 
		\hline
		Descripción. & A partir de los niveles desbloqueados por el jugador, el jugador podrá elegir el nivel que desea jugar.\\ 
		\hline
		Reglas de negocio. & RN-002, RN-003.\\ 
		\hline
		Precondición. & Existirá un archivo con los datos del juego.\par
El juego lee los valores del archivo de datos de partida para inicializar el menú de selección de nivel.
 \\
		\hline
		Postcondición. & El juego iniciara el nivel seleccionado con los datos leídos desde el archivo.\\
		\hline
		Errores. & ER-002.\\
		\hline
\end{longtable}
\begin{itemize}
	\item[•] Trayectoria Principal.
		\begin{enumerate}
			\item El jugador se encuentra en la interfaz del menú de selección de nivel. 
			\item El jugador toca los botones de control del carrusel de niveles para ver la opción que tiene para elegir.
			\item El juego despliega la información del nivel activo en la selección del carrusel.
			\item Las acciones 2 y 3 se repiten hasta que el jugador elije un nivel.
			\item El jugador selecciona el botón de iniciar partida (Ruta A).
			\item El juego inicia los valores necesarios para que el jugador pueda jugar el nivel. 
			\item El juego lee los valores del archivo de datos de partida para inicializar el nivel seleccionado.
		\end{enumerate}
	\item[•] Trayectorias Secundarias.
		\begin{itemize}
			\item Ruta A. 
				\begin{enumerate}
					\item En caso de que el jugador haya seleccionado una cinemática. El juego no necesita inicializar valores usando el archivo de partida 
					\item El jugador ver la cinemática (Ver caso de uso )

				\end{enumerate}
		\end{itemize}
\end{itemize}