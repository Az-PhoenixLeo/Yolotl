\subsubsection{CU-002 Cargar partida.} \label{CU:02}
\begin{longtable}[c]{ | m{5cm} | m{10cm}|} 
		\hline
		\rowcolor{cyan}CU-002. & Cargar partida. \\ 
		\hline
		Descripción. & El jugador cargará una partida ya existente. \\ 
		\hline
		Reglas de negocio. & RN-00\ref{RN:01}\\ 
		\hline
		Precondición. & Existirá un archivo con los datos del juego en caso de que no sea la primera vez que se empiece una partida. \\
		\hline
		Postcondición. & El juego iniciara la partida con los datos leídos desde el archivo.\\
		\hline
		Errores. & ER-002.\\
		\hline
\end{longtable}
\begin{itemize}
	\item[•] Trayectoria Principal.
		\begin{enumerate}
			\item El jugador inicia la aplicación.
			\item El jugador toca la pantalla táctil en la pantalla de inicio.
			\item El juego carga la interfaz del menú principal.
			\item El jugador selecciona el botón de Cargar partida (Ruta A).
			\item El juego comprobara que existe un archivo con los datos de la partida (Ruta B).
			\item El juego leerá los datos del archivo de datos de la partida.
			\item El juego iniciara la partida con base en los datos leídos.
			\item El juego redireccionará al jugador a la interfaz de Menú de selección de nivel.

		\end{enumerate}
	\item[•] Trayectorias Secundarias.
		\begin{itemize}
			\item Ruta A. 
				\begin{enumerate}
					\item El jugador selecciona el botón de Empezar partida (Ver caso de uso \ref{CU:01}).
				\end{enumerate}
			\item Ruta B.
				\begin{enumerate}
					\item El juego no encuentra ningún archivo con los datos de la partida (ver error 002).
					\item El juego despliega el mensaje 002.
					\item El jugador cierra el mensaje 002.
					\item El jugador Empieza partida (ver caso de uso ).

				\end{enumerate}
		\end{itemize}
\end{itemize}