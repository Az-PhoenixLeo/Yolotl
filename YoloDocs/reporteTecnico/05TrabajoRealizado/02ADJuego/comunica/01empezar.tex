	\subsubsection{Empezar nueva partida}
	Este caso consiste en que el jugador desde la intefaz 01 (ver aparatado
	  \ref{TraReaInterfaces}) empieza una nueva partida.  	
	\begin{itemize}
		\item Clases involucradas
			\begin{itemize}
				\item PrincipalMenuCtrl.
				\item GameDataCtrl.
				\item GameData.
				\item LevelData.
				\item CutsceneData.
				\item LoaderScene.
			\end{itemize}
			
		

\item Trayectoria de comunicación principal.
		\begin{enumerate}
				\item $\lbrack$PrincipalMenuCtrl$\rbrack$ Detectar que el jugador pulsó el 
				botón de Empezar partida.
				\item $\lbrack$PrincipalMenuCtrl$\rbrack$ Mostrar el mensaje donde pide la 
				la confirmación del jugador para realizar los cambios en el archivo de 
				datos de partida.
				\item $\lbrack$PrincipalMenuCtrl$\rbrack$ Detectar que el jugador pulsó el 
				botón de aceptar (Trayectoria A).
				\item $\lbrack$PrincipalMenuCtrl$\rbrack$ Comunicar la confirmación a GameCtrl.
				\item $\lbrack$GameDataCtrl$\rbrack$ Crear una instancia de la clase GameData 
				con los valores predeterminados de inicio.
				\item $\lbrack$GameDataCtrl$\rbrack$ Crear una instancia de la clase LevelData 
				con los valores predeterminados de inicio.
				\item $\lbrack$GameDataCtrl$\rbrack$ Crear una instancia de la clase CutsceneData 
				con los valores predeterminados de inicio.
				\item $\lbrack$GameDataCtrl$\rbrack$ Crear nuevo archivo de partida las 
				intancias de las clases GameData, LevelData, CutsceneData.
				\item $\lbrack$GameDataCtrl$\rbrack$ Comunicar a LoaderScene que el archivo 
				de la partida han sido creado.
				\item $\lbrack$LoaderScene$\rbrack$ Cargar cinemática 1.
		\end{enumerate}
		
	\item Trayectoria A (El jugador pulsa cancelar).
			\begin{enumerate}
				\item[{A.}1] $\lbrack$PrincipalMenuCtrl$\rbrack$ Detectar que el jugador pulsó el 
				botón de cancelar.
				\item[{A.}2] $\lbrack$PrincipalMenuCtrl$\rbrack$ Cerrar mensaje de confirmación.
			\end{enumerate}
	\end{itemize}	