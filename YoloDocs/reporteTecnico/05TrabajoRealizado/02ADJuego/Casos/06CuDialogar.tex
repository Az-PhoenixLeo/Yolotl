\subsubsection{CU-006 Dialogar.} \label{CU:06}
\begin{longtable}[c]{ | m{5cm} | m{10cm}|} 
		\hline
		\rowcolor{cyan}CU-006. & Dialogar.\\ 
		\hline
		Descripción. & A través del personaje jugable, el jugador puede dialogar con personajes no jugables.\\ 
		\hline
		Reglas de negocio. & RN-010, RN-011, RN-012.\\ 
		\hline
		Precondición. & El jugador eligió un nivel desde el menú de selección de menú.\\
		\hline
		Postcondición. & Sin postcondición.\\
		\hline
		Errores. & \\
		\hline
\end{longtable}
\begin{itemize}
	\item[•] Trayectoria Principal.
		\begin{enumerate}
			\item El jugador se aproxima a un personaje con un icono de dialogo.
			\item El jugador oprime el botón de saltar para dialogar (Ruta A).
			\item El icono de dialogo desaparece.
			\item El juego lee desde un archivo el dialogo.
			\item El juego muestra el dialogo en una ventana de dialogo.
			\item El jugador oprime el botón de saltar para cerrar el cuadro de dialogo del personaje (Ruta B).
			\item El cuadro de dialogo se cierra.
			\item El icono de dialogo vuelve a aparecer.

		\end{enumerate}
	\item[•] Trayectorias Secundarias.
		\begin{itemize}
			\item Ruta A.
				\begin{enumerate}
					\item El jugador no oprime el botón de salto y se sigue de largo.
				\end{enumerate}
			\item Ruta B.
				\begin{enumerate}
					\item En caso de que el dialogo este repartido en más cuadros de diálogos. El jugador oprime el botón de salto repetidamente para leer el resto del dialogo.
				\end{enumerate}
		\end{itemize}
\end{itemize}