\subsubsection{Clases Actores}
A continuación se listan las clases actores: 
		\begin{itemize}
			\item \textbf{Player:} Esta clase se encarga de las acciones del personaje 
			jugable, actualizar sus estados y gestionar el valor de sus atributos. 
			Esta clase esta a cargo de:
			\begin{itemize}
				\item Mover al personaje jugable de manera horizontal.
				\item Controlar la maquina de estados de las animaciones del personaje 
				jugable.
				\item Detectar las colisiones del personaje jugable.
				\item Actualizar la cantidad de vida al recibir daño.
				\item Realizar disparo de \textit{Tonalli}.
				\item Actualizar la cantidad de \textit{Tonalli} al efectuar un disparo.  
				\item Saltar. 
			\end{itemize}
			%=======================================
			\item \textbf{TalkedCharacter:} Esta clase modela el funcionamiento de un
			ciudadano con el que el jugador debe interactuar en el nivel 1.
			Esta clase esta a cargo de:
			\begin{itemize}
				\item Mostrar un icono de dialogo para indicarle al jugador que debe de 
				interactuar con éste.
				\item Ocultar el icono de dialogo.
				\item Indicarle a la clase DialogueCtrl que se inicia un dialogo.  
			\end{itemize}
			%=======================================
			\item \textbf{DroppingPlatform:} Esta clase modela el funcionamiento 
			del obstáculo Plataforma que cae, por lo que al hacer contacto con un objeto de 
			la clase \textit{GroundCollisionCtrl} el objeto que instancie esta clase 
			empezará a caer despues de \textit{n} segundos. 
			%=======================================
			\item \textbf{MovingPlatform:} Esta clase modela el funcionamiento 
			del obstáculo Plataforma que móvil. Haciendo uso de dos posiciones: A y B, el 
			objeto que instancia esta clase se mueve de manera cíclica de la posición A a 
			la B y de la B a la A. 
			%=======================================
			\item \textbf{DisappearingPlatform:} Esta clase modela el funcionamiento 
			del obstáculo Plataforma que desaparece, esta clase hace que el objeto que la 
			instancie aparezca y desaparezca de manera cíclica, activando y desactivando 
			los colisionadores del objeto. 
			%=======================================
			\item \textbf{Stalagmite:} Esta clase modela el comportamiento del obstáculo 
			estalagmita. Esta clase hace que el objeto que la instancie caiga cuando 
			detecte que el jugador se posiciona por debajo de este objeto.    
			%=======================================
			\item \textbf{PushingObstacle:} Esta clase modela el comportamiento de dos 
			obstáculos: viento temporal y bolas de nieve. Haciendo uso de una posición, la
			clase determina hacia que dirección debe de incrementar su dimensión y el tamaño 
			de su colisionador.      
			%=======================================	
			\item \textbf{Arrow:} Clase que produce un movimiento vertical descendente al 
			objeto que la instancia.   
			%=======================================
			\item \textbf{\textit{Enemy}:} Esta clase modela el comportamiento común que 
			tienen los enemigos de tipo normal y los de tipo jefe. De esta clase heredan 
			su funcionamiento las clase \textit{NormalEnemy} y \textit{BossEnemy}.Esta 
			clase se encarga de:  
			\begin{itemize}
				\item Controlar las transiciones de la maquina de estados que controla las 
				animaciones del enemigo.
				\item Gestiona la detección de colisiones del enemigo.
				\item Actualizar la cantidad de vida del Enemigo.
			\end{itemize}
			%=======================================
			\item \textbf{\textit{NormalEnemy}:} Esta clase modela el comportamiento común que 
			tienen los enemigos de tipo normal. Esta clase hereda su funcionamiento de la clase
			 \textit{Enemy}. La clase NormalEnemy hereda su funcionalidad a las clases 
			\textit{ Jaguar, Bird, Armadillo, PurpleGost} y \textit{RedGost}.Esta 
			clase se encarga de:  
			\begin{itemize}
				\item Controlar el patrón de movimiento de los enemigos de tipo 
				normal.
				\item  Verificar la cercanía que tiene el enemigo normal con otros objetos, 
				obstáculos y enemigos para ajustar su rango de acción y evitar que interfiera con el 
				funcionamiento de otro objeto.
			\end{itemize}					
			%=======================================
			\item \textbf{\textit{Jaguar}:} Esta clase modela el comportamiento del enemigo jaguar 
			(Consultar la ficha de personaje en la Sección de personajes en el documento de diseño). 
			Esta clase hereda su funcionamiento de la clase \textit{NormalEnemy}. Esta 
			clase se encarga de:  
			\begin{itemize}
				\item Sincronizar el desplazamiento del jaguar con su maquina de estados para que modele
				el patrón de ataque del jaguar. 
			\end{itemize}
			%=======================================
			\item \textbf{\textit{Bird}:} Esta clase modela el comportamiento de dos personajes de tipo
			normal: Chara enana y Zopilote (Consultar la ficha de personaje en la Sección de personajes en 
			el documento de diseño). Esta clase hereda su funcionamiento de la clase 
			\textit{NormalEnemy}. Esta clase se encarga de:  
			\begin{itemize}
				\item Sincronizar el desplazamiento del ave con su maquina de estados para que modele
				el patrón de ataque de Chara enana y del zopilote. 
			\end{itemize}
			%=======================================
			\item \textbf{\textit{Armadillo}:} Esta clase modela el comportamiento del personaje 
			armadillo (Consultar la ficha de personaje en la Sección de personajes en 
			el documento de diseño). Esta clase hereda su funcionamiento de la clase 
			\textit{NormalEnemy}. Esta clase se encarga de:  
			\begin{itemize}
				\item Sincronizar el desplazamiento del jaguar con su maquina de estados para que modele
				el patrón de ataque del armadillo. 
			\end{itemize}
			%=======================================
			\item \textbf{\textit{PurpleGost}:} Esta clase modela el comportamiento del personaje 
			fantasma purpura (Consultar la ficha de personaje en la Sección de personajes en 
			el documento de diseño). Esta clase hereda su funcionamiento de la clase 
			\textit{NormalEnemy}. Esta clase se encarga de:  
			\begin{itemize}
				\item Sincronizar el desplazamiento del jaguar con su maquina de estados para que modele
				el patrón de ataque del fantasma purpura. 
			\end{itemize}
			
			%=======================================
			\item \textbf{\textit{RedGost}:} Esta clase modela el comportamiento del personaje 
			fantasma rojo (Consultar la ficha de personaje en la Sección de personajes en 
			el documento de diseño). Esta clase hereda su funcionamiento de la clase 
			\textit{NormalEnemy}. Esta clase se encarga de:  
			\begin{itemize}
				\item Sincronizar el desplazamiento del jaguar con su maquina de estados para que modele
				el patrón de ataque del fantasma rojo.
				\item Instanciar el objeto ShootEnemy. 
			\end{itemize}
			%=======================================
			\item \textbf{\textit{BossEnemy}:} Esta clase modela el comportamiento común que 
			tienen los enemigos de tipo ¿jefe. Esta clase hereda su funcionamiento de 
			la clase \textit{Enemy}. La clase BosslEnemy hereda su funcionalidad a las clases 
			\textit{ Xochitonal, Tepeyollotl,Itzpapalotl, Mictlecayotl, Tlazolteotl, 
			Itztlacoliuhqui, Nexoxcho, MictlantecutliPhase01, MictlantecutliPhase02} 
			y \textit{MictlantecutliPhase03}.Esta clase se encarga de:  
			\begin{itemize}
				\item Controlar el patrón de movimiento de los enemigos de tipo 
				jefe.
				\item  Verificar la cercanía que tiene el enemigo tipo jefe con el jugador 
				y con los limites del escenario.
			\end{itemize}	
			%=======================================
			\item \textbf{\textit{Xochitonal}:} Esta clase modela el comportamiento del personaje 
			\textit{Xochitónal} (Consultar la ficha de personaje en la Sección de personajes en 
			el documento de diseño). Esta clase hereda su funcionamiento de la clase 
			\textit{BossEnemy}. Esta clase se encarga de:  
			\begin{itemize}
				\item Sincronizar el desplazamiento del \textit{Xochitónal} con su maquina 
				de estados para que modele el patrón de \textit{Xochitónal}.
				\item Toma decisiones en cuanto a los ataques a realizar basándose en su 
				nivel de vida.
			\end{itemize}
			%=======================================
			\item \textbf{\textit{Tepeyollotl}:} Esta clase modela el comportamiento del personaje 
			\textit{Tepeyóllotl} (Consultar la ficha de personaje en la Sección de personajes en 
			el documento de diseño). Esta clase hereda su funcionamiento de la clase 
			\textit{BossEnemy}. Esta clase se encarga de:  
			\begin{itemize}
				\item Sincronizar el desplazamiento del \textit{Tepeyóllotl} con su maquina 
				de estados para que modele el patrón de \textit{Tepeyóllotl}.
				\item Toma decisiones en cuanto a los ataques a realizar basándose en su 
				nivel de vida.
			\end{itemize}
			%=======================================
			\item \textbf{\textit{Itzpapalotl}:} Esta clase modela el comportamiento del personaje 
			\textit{Itzpápalotl} (Consultar la ficha de personaje en la Sección de personajes en 
			el documento de diseño). Esta clase hereda su funcionamiento de la clase 
			\textit{BossEnemy}. Esta clase se encarga de:  
			\begin{itemize}
				\item Sincronizar el desplazamiento del \textit{Itzpápalotl} con su maquina 
				de estados para que modele el patrón de \textit{Itzpápalotl}.
				\item Toma decisiones en cuanto a los ataques a realizar basándose en su 
				nivel de vida.
			\end{itemize}
			%=======================================
			\item \textbf{\textit{Mictlecayotl}:} Esta clase modela el comportamiento del personaje 
			\textit{Mictlecayotl} (Consultar la ficha de personaje en la Sección de personajes en 
			el documento de diseño). Esta clase hereda su funcionamiento de la clase 
			\textit{BossEnemy}. Esta clase se encarga de:  
			\begin{itemize}
				\item Sincronizar el desplazamiento del \textit{Mictlecayotl} con su maquina 
				de estados para que modele el patrón de \textit{Mictlecayotl}.
				\item Toma decisiones en cuanto a los ataques a realizar basándose en su 
				nivel de vida.
			\end{itemize}
			%=======================================
			\item \textbf{\textit{Tlazolteotl}:} Esta clase modela el comportamiento del personaje 
			\textit{Tlazoltéotl} (Consultar la ficha de personaje en la Sección de personajes en 
			el documento de diseño). Esta clase hereda su funcionamiento de la clase 
			\textit{BossEnemy}. Esta clase se encarga de:  
			\begin{itemize}
				\item Sincronizar el desplazamiento del \textit{Tlazoltéotl} con su maquina 
				de estados para que modele el patrón de \textit{Tlazoltéotl}.
				\item Toma decisiones en cuanto a los ataques a realizar basándose en su 
				nivel de vida.
			\end{itemize}
			%=======================================
			\item \textbf{\textit{Itztlacoliuhqui}:} Esta clase modela el comportamiento del personaje 
			\textit{Itztlacoliuhqui} (Consultar la ficha de personaje en la Sección de personajes en 
			el documento de diseño). Esta clase hereda su funcionamiento de la clase 
			\textit{BossEnemy}. Esta clase se encarga de:  
			\begin{itemize}
				\item Sincronizar el desplazamiento del \textit{Itztlacoliuhqui} con su maquina 
				de estados para que modele el patrón de \textit{Itztlacoliuhqui}.
				\item Toma decisiones en cuanto a los ataques a realizar basándose en su 
				nivel de vida.
			\end{itemize}
			%=======================================
			\item \textbf{\textit{Nexoxcho}:} Esta clase modela el comportamiento del personaje 
			\textit{Nexoxcho} (Consultar la ficha de personaje en la Sección de personajes en 
			el documento de diseño). Esta clase hereda su funcionamiento de la clase 
			\textit{BossEnemy}. Esta clase se encarga de:  
			\begin{itemize}
				\item Sincronizar el desplazamiento del \textit{Nexoxcho} con su maquina 
				de estados para que modele el patrón de \textit{Nexoxcho}.
				\item Toma decisiones en cuanto a los ataques a realizar basándose en su 
				nivel de vida.
			\end{itemize}
			%=======================================
			\item \textbf{\textit{MictlantecutliPhase01}:} Esta clase modela el 
			comportamiento del personaje \textit{Mictlantecutli} (Consultar la ficha 
			del personaje en la Sección de personajes en 
			el documento de diseño). Esta clase hereda su funcionamiento de la clase 
			\textit{BossEnemy}. Esta clase se encarga de:  
			\begin{itemize}
				\item Sincronizar el desplazamiento del \textit{Mictlantecutli} con su maquina 
				de estados para que modele el patrón de \textit{Mictlantecutli}.
				\item Toma decisiones en cuanto a los ataques a realizar basándose en su 
				nivel de vida.
			\end{itemize}
			%=======================================
			\item \textbf{\textit{MictlantecutliPhase02}:} Esta clase modela el 
			comportamiento del personaje \textit{Mictlantecutli} (Consultar la ficha 
			del personaje en la Sección de personajes en 
			el documento de diseño). Esta clase hereda su funcionamiento de la clase 
			\textit{BossEnemy}. Esta clase se encarga de:  
			\begin{itemize}
				\item Sincronizar el desplazamiento del \textit{Mictlantecutli} con su maquina 
				de estados para que modele el patrón de \textit{Mictlantecutli}.
				\item Toma decisiones en cuanto a los ataques a realizar basándose en su 
				nivel de vida.
			\end{itemize}
			%=======================================
			\item \textbf{\textit{MictlantecutliPhase03}:} Esta clase modela el 
			comportamiento del personaje \textit{Mictlantecutli} (Consultar la ficha 
			del personaje en la Sección de personajes en 
			el documento de diseño). Esta clase hereda su funcionamiento de la clase 
			\textit{BossEnemy}. Esta clase se encarga de:  
			\begin{itemize}
				\item Sincronizar el desplazamiento del \textit{Mictlantecutli} con su maquina 
				de estados para que modele el patrón de \textit{Mictlantecutli}.
				\item Toma decisiones en cuanto a los ataques a realizar basándose en su 
				nivel de vida.
			\end{itemize}
			%=======================================
			\item \textbf{Checkpoint:} Esta clase permite guardar el progreso con en 
			cuanto objetivos del juego para inicializar al jugador en esa posición
			en caso de que el jugador muera. Los datos que contienen las instancias de
			esta clase solo perduraran mientras el jugador se mantenga dentro del nivel,
			una vez que el jugador abandone el nivel los datos se destruirán y el jugador
			deberá iniciar el nivel desde el inicio.  
			%========================================
			\item \textbf{FollowedCharacter:} Esta clase controla a un personaje que se 
			desplaza siguiendo un patron de movimiento dependiente de un conjunto de objetos 
			que sirven como nodos a sus desplazamiento. El objeto que instancie esta clase 
			siempre se va a desplazar manteniendo una distancia constante de personaje jugable,
			cuando esta distancia aumente, el personaje se detendrá hasta que el personaje 
			jugable vuelva a mantenerse a la distancia aceptable.    
			%========================================
			\item \textbf{GostBulletCtrl:} Esta clase controla el desplazamiento del 
			disparo generado por la clase RedGost.
			%========================================
			\item \textbf{ BulletCtrl:} Esta clase controla el desplazamiento del 
			disparo generado por la clase Player; el valor del atributo de velocidad dependerá 
			del atributo del player que indica hacia donde esta mirando (izquierda o derecha).
		\end{itemize}	