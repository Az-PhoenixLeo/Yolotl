\subsubsection{Clases Actores}
A continuación se listan las clases auxiliares: 
		\begin{itemize}
			\item \textbf{LoaderScene:} Esta clase permite la transición 
			entre escenas. Esta clase es auxiliar de las clases:
			\begin{itemize}
				\item Controladores de nivel.
				\item Controladores de cinemáticas.
				\item PrincipalMenuCtrl.
				\item SelectLevelMenu.
			\end{itemize}
			%===============================			
			\item \textbf{DestroyWithDelay:} Esta clase destruye al objeto que la 
			instancia después de \textit{n} segundos. Esta clase es instanciada por 
			los GameObjects:
			\begin{itemize}
				\item TonalliBullet.
				\item GostBullet.
			\end{itemize}
			%===============================
			\item \textbf{GroundCollisionCtrl:} Esta clase detecta y gestiona todas las 
			colisiones de suelo del personaje jugable. Esta clase es auxiliar de:
			\begin{itemize}
				\item Player.
			\end{itemize}
			%===============================
			\item \textbf{Dialogue:} Esta clase tiene por atributos el nombre de un 
			personaje y un dialogo con la capacidad de ser serializados para que estos 
			puedan ser almacenados y leídos desde un archivo de tipo JASON. Esta clase 
			es auxiliar de:
			\begin{itemize}
				\item Dialogues.
				\item TalkedCharacter
				\item TalkedCharactersCtrl
			\end{itemize}
			%===============================
			\item \textbf{Dialogues:} Esta clase tiene por atributos una lista de instancias 
			de la clase Dialogue con la capacidad de ser serializados para que estos 
			puedan ser almacenados y leídos desde un archivo de tipo JASON. Esta clase 
			es auxiliar de:
			\begin{itemize}
				\item Dialogues.
				\item TalkedCharacter.
				\item TalkedCharactersCtrl.
			\end{itemize}
			%===============================
			\item \textbf{FileDialogues:} Esta clase lee datos desde un archivo JASON y 
			serializa su contenido para instanciarlo en la clase Dialogues. Esta clase 
			es auxiliar de:
			\begin{itemize}
				\item FileDialogue.
				\item TalkedCharactersCtrl.
				\item CutsceneCtrl.
			\end{itemize}
			%===============================
			\item \textbf{PlayerAudio:} Esta clase contiene todos los archivos de audio que
			corresponden a las acciones del jugador como correr, disparar, morir, etc. El 
			objetivo de esta clase es facilitar la vinculación entre los audios y la clase
			AudioCtrl. 
			\begin{itemize}
				\item AudioCtrl
			\end{itemize}
			
			%===============================
			\item \textbf{AudioFX:} Esta clase contiene todos los archivos de audio que
			corresponden a los efectos de sonido del juego ajenos al jugador. El 
			objetivo de esta clase es facilitar la vinculación entre los audios y la clase
			AudioCtrl. 
			\begin{itemize}
				\item AudioCtrl
			\end{itemize}
			
			%===============================
			\item \textbf{CutsceneData:} Esta clase tiene por atributos datos que permiten 
			llevar el control de las cinemáticas disponibles para ver y de las que faltan 
			por desbloquear. Esta clase contiene atributos que tienen la capacidad de ser 
			serializables con el fin de que puedan ser almacenados en un archivo de tipo
			binario y a su vez instanciados cuando sean leidos desde el archivo.
			\begin{itemize}
				\item GameDataCtrl.
				\item GameData.
			\end{itemize}
			
			%===============================
			\item \textbf{LevelData:} Esta clase tiene por atributos datos que permiten 
			llevar el control de los niveles disponibles para jugar y de los que faltan 
			por desbloquear. Esta clase contiene atributos que tienen la capacidad de ser 
			serializables con el fin de que puedan ser almacenados en un archivo de tipo
			binario y a su vez instanciados cuando sean leídos desde el archivo.
			\begin{itemize}
				\item GameDataCtrl.
				\item GameData.
			\end{itemize}
			
			%===============================
			\item \textbf{GameData:} Esta clase tiene por atributos datos que permiten 
			llevar el control de todo el progreso del jugador. Esta clase contendra 
			atributos de la clase \textit{Player} que serán utilizados por las clases hijas de 
			\textit{LevelCtrl} para incializar la partida. Esta clase contiene atributos que tienen
			 la capacidad de ser serializables con el fin de que puedan ser almacenados 
			 en un archivo de tipo binario y a su vez instanciados cuando sean leidos 
			 desde el archivo.
			\begin{itemize}
				\item GameDataCtrl.
			\end{itemize}
			
\end{itemize}	