\subsection{Ambientación}
	Para garantizar la inmersión del jugador, el videojuego se vale de diferentes
	 elementos multimedia. Estos elementos son la música de fondo (BGM, pos sus 
	 siglas en inglés), los efectos de sonido (SFX, por sus siglas en inglés) y  
	 efectos espaciales (FX, por su siglas en inglés).
\\
\par	
	Huddle maneja un aparatado para incluir este tipo de elementos dentro de la 
	documentación del juego; sin embargo, no incluye ninguna guía sobre como debería 
	de redactarse las descripciones de BGM y SFX; en consecuencia estos elementos 
	fueron documentados escribiendo el nombre de sonido o música, seguido de una 
	breve descripción del mismo.   
\\
\par
Durante la redacción de este apartado se detectó que existían dos secciones para 
documentar BGM y SFX, con la diferencia de que en una los términos se encontraban 
escritos en sus siglas en inglés y en el otro apartado se encontraba en español, 
por lo que se eliminó el apartado en español. Luego de que se detectara esta 
duplicación de apartados, se descubrió que no existía ningún apartado para documentar 
los FX, seguido de esto se creo dicho apartado. En cuanto a la documentación de FX, 
se siguió la misma estructura que con BGM y SFX: escribir el nombre de FX y 
describirlo para dar una idea de como se vería en el nivel y bajo que interacciones 
se activaría. 