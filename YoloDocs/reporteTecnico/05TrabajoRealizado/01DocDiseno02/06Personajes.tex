\subsection{Personajes}
Al ser una metodología orientada a videojuegos, huddle proporciona una plantilla dentro del documento de diseño para documentar los personajes con los que cuenta el juego. La plantilla se compone los siguientes campos:
	\begin{itemize}
		\ietm Nombre del personaje.
		\item Descripción. 
		\item Imagen.
		\item Concepto. 
		\item Encuentro. 
		\item Habilidades. 
		\item Armas.
		\item Items.  
		\item Personaje no jugable. 
	\end{itemize}
	Durante el proceso de la redacción del documento de diseño se detecto que la plantilla no contenía todos los campos necesarios para documentar completamente las características de los personajes del juego Yolotl. Por lo que se procedió a hacer la modificaciones pertinentes agregando un campo llamado Estatus, este campo permite conocer si el personaje es un enemigo, un que aparece solo durante las cinemáticas, si solo se le menciona al personaje o en su defecto el personaje es un personaje jugable.
	\\
	\par
La segunda modificación fue agregar tres sub secciones al campo de Concepto permitiendo  crear apartados especificos para la historia del personaje antes de la historia que se presneta en el juego, 
  