\subsection{Personajes}
Al ser una metodología orientada a videojuegos, huddle proporciona una plantilla 
dentro del documento de diseño para documentar los personajes con los que cuenta 
el juego. La plantilla se compone los siguientes campos:
	\begin{itemize}
		\item Nombre del personaje.
		\item Descripción. 
		\item Imagen.
		\item Concepto. 
		\item Encuentro. 
		\item Habilidades. 
		\item Armas.
		\item Items.  
		\item Personaje no jugable. 
	\end{itemize}
	De igual forma, huddle maneja una plantilla para documentar los enemigos del 
	juego. Los campos de la plantilla enemigos son:
	\begin{itemize}
		\item Nombre.
		\item Descripción.
		\item Encuentro.
		\item Imagen.
		\item Habilidades.
		\item Armas.
		\item Items.
	\end{itemize}
	Durante el proceso de la redacción del documento de diseño se detectó que la 
	plantilla de personajes y la plantilla de enemigos generaban una repetición de 
	redacción en los personajes que eran enemigos y participes del argumento. Otra 
	problemática a que se detectó fue que la plantillas no contenían todos los campos 
	necesarios para documentar completamente las características de los personajes del 
	juego Yolotl. Por lo que se procedió a hacer la modificaciones pertinentes.
\\
\par	
	En primeramente, se fusionaron las plantillas de personaje y enemigo. Para 
	garantizar que se pudiera saber el rol que desempeñaría el personaje se agregó 
	un campo llamado \textbf{Estatus}.  El campo estatus permite conocer si el 
	personaje es un enemigo, un que aparece solo durante las cinemáticas, si solo 
	se le menciona al personaje o en su defecto el personaje es un personaje jugable. 
	Por lo tanto el campo \textbf{estatus} sustituye al campo \textbf{Personaje no 
	jugable}, ya que este campo se llenaba en caso de que personaje que se estuviera 
	documentando no esta jugable.
	\\
	\par
La segunda modificación fue agregar tres subsecciones al campo de Concepto 
permitiendo  crear apartados específicos para:

	\begin{itemize}
		\item La historia del personaje antes de la historia que se presenta en el 
		juego.
		\item La historia del personaje durante el juego. 
		\item Las relaciones que el personaje tiene con otros personajes.
	\end{itemize}

	Al fusionar las plantillas de enemigos y personajes, se decidió que para aquellos 
	personajes que no tengan una participación en el argumento del juego el campo 
	\textbf{Concepto} puede ser omitido es omitido.
	\\
	\par
	La siguiente modificación que se realizó, fue agregar otro nuevo campo: Patrón 
	de ataque. El campo patrón de ataque contendrá toda la información sobre el 
	comportamiento de batalla de un personaje, indicando bajo que condiciones acciona 
	sus habilidades; el objetivo de este campo es proporcionar información que permita 
	modelar la inteligencia artificial del personaje.Este campo	solo se incluirá en la 
	ficha del personaje si en su estatus se definió que es un enemigo. 
	 \\
	 \par
	 El último cambio agregado a la plantilla consistió en adicionar el campo de \textbf{Bloques de animación}. Como el nombre del bloque lo dice, este campo 
	 proporcionará información sobre cuantas animaciones contendrá el personaje. 
	\\	 
	\par
	El videojuego yolotl cuenta con alrededor de 24 personajes. Salvo por los personajes 
	que no participan en el argumento del juego, cada personaje esta basado en algún 
	dios de la mitología azteca o en personaje histórico; por lo que, para su diseño y 
	definición de habilidades, fue necesario realizar una investigación en códices, 
	artículos de investigación y libros especializados. Los principales retos a 
	afrontar durante la investigación fueron la falta de información sobre algunos 
	dioses y la discrepancia histórica que existe alrededor de la cultura mexica. 
	\\
	\par
	A continuación se en listaran los personajes del juego junto con su estatus y sus 
	habilidades.
	\begin{itemize}
		%=======================================
		\item Malinalli Tenépal.
			\begin{itemize}
				\item Estatus: Personaje jugable. Protagonista.
				\item Habilidades: 
					\begin{itemize}
						\item Saltar.
						\item Disparar tonalli.
					\end{itemize}
			\end{itemize}
		%=======================================
		\item Xólotl.
			\begin{itemize}
				\item Estatus: Personaje jugable. Protagonista (Solo nivel 9). Personaje 
				no jugable.
				\item Habilidades: 
					\begin{itemize}
						\item Cambiar de forma.
					\end{itemize}
			\end{itemize}
		%=======================================
		\item Xochitónal.
			\begin{itemize}
				\item Estatus: Enemigo jefe(nivel 2). Personaje no jugable.
				\item Habilidades: 
					\begin{itemize}
						\item Zambullida.
						\item Burbujas.
					\end{itemize}
			\end{itemize}
		%=======================================
		\item Tepeyóllotl.
			\begin{itemize}
				\item Estatus: Enemigo jefe(nivel 3, nivel 8). Personaje no jugable.
				\item Habilidades: 
					\begin{itemize}
						\item Coraza.
						\item Impacto.
						\item Lluvia rocas.
						\item Rugido aturdidor.
					\end{itemize}
			\end{itemize}
		%=======================================
		\item Itzpápalotl.
			\begin{itemize}
				\item Estatus: Enemigo jefe(nivel 4). Personaje no jugable.
				\item Habilidades: 
					\begin{itemize}
						\item Circulo de fuego.
						\item Embestida aérea.
						\item Invisibilidad.
					\end{itemize}
			\end{itemize}
		%=======================================
		\item Mictlecayotl.
			\begin{itemize}
				\item Estatus: Enemigo jefe(nivel 5). Personaje no jugable.
				\item Habilidades: 
					\begin{itemize}
						\item Tornado.
						\item Ventisca.
					\end{itemize}
			\end{itemize}
		%=======================================
		\item Tlazoltéotl.
			\begin{itemize}
				\item Estatus: Enemigo jefe(nivel 6). Personaje no jugable.
				\item Habilidades: 
					\begin{itemize}
						\item Raíz del diablo.
						\item Energía corrupta.
						\item Circulo protector.
					\end{itemize}
			\end{itemize}
		%=======================================
		\item Itztlacoliuhqui.
			\begin{itemize}
				\item Estatus: Enemigo jefe(nivel 7). Personaje no jugable.
				\item Habilidades: 
					\begin{itemize}
						\item Lluvia de lava.
						\item Manotazo.
						\item Lluvia de flechas.
					\end{itemize}
			\end{itemize}
		%=======================================
		\item Nexoxcho.
			\begin{itemize}
				\item Estatus: Enemigo jefe(nivel 8). Personaje no jugable.
				\item Habilidades: 
					\begin{itemize}
						\item Crear ilusiones a partir de recuerdos.
						\item Apuñalar.
					\end{itemize}
			\end{itemize}
		%=======================================
		\item Mictlantecutli.
			\begin{itemize}
				\item Estatus: Enemigo jefe(nivel 10). Personaje no jugable.
				\item Habilidades: 
					\begin{itemize}
						\item Todos los hombres del rey.
						\item Fuego mortífero.
						\item Penitencia.
						\item Lluvia de huesos.
						\item Estocada mortífera.
						\item Burbujas.
						\item Lluvia de rocas.
						\item Lluvia de lava.
						\item Circulo de fuego.					
					\end{itemize}
			\end{itemize}
		%=======================================
		\item Mictecacíhuatl.
			\begin{itemize}
				\item Estatus: Personaje no jugable.		
			\end{itemize}
		%=======================================
		\item Tenépal.
			\begin{itemize}
				\item Estatus: Personaje no jugable.		
			\end{itemize}
		%=======================================
		\item Cimatl.
			\begin{itemize}
				\item Estatus: Personaje no jugable.		
			\end{itemize}
		%=======================================
		\item Huenupan.
			\begin{itemize}
				\item Estatus: Personaje no jugable.		
			\end{itemize}
		%=======================================
		\item Ciudadanos.
			\begin{itemize}
				\item Estatus: Personaje no jugable.		
			\end{itemize}
		%=======================================
		\item Quetzalcóatl.
			\begin{itemize}
				\item Estatus: Personaje no jugable.		
			\end{itemize}
		%=======================================
		\item Tezcatlipoca.
			\begin{itemize}
				\item Estatus: Personaje no jugable.		
			\end{itemize}
		%=======================================
		\item Tecolote.
			\begin{itemize}
				\item Estatus: Personaje no jugable. Checkpoint.		
			\end{itemize}
		%=======================================
		\item Armadillo.
			\begin{itemize}
				\item Estatus: Enemigo normal. 
				\item Habilidades: 
					\begin{itemize}
						\item Cuchillas de obsidiana.			
					\end{itemize}
			\end{itemize}
		%=======================================
		\item Fantasma rojo.
			\begin{itemize}
				\item Estatus: Enemigo normal. 
				\item Habilidades: 
					\begin{itemize}
						\item Disparo rojo.			
					\end{itemize}
			\end{itemize}
		%=======================================
		\item Fantasma morado.
			\begin{itemize}
				\item Estatus: Enemigo normal. 
				\item Habilidades: 
					\begin{itemize}
						\item Embestida.			
					\end{itemize}
			\end{itemize}
		%=======================================
		\item Jaguar.
			\begin{itemize}
				\item Estatus: Enemigo normal. 
				\item Habilidades: 
					\begin{itemize}
						\item Salto de altura.			
					\end{itemize}
			\end{itemize}
		%=======================================
		\item Zopilote.
			\begin{itemize}
				\item Estatus: Enemigo normal. 
				\item Habilidades: 
					\begin{itemize}
						\item Vuelo en picada.			
					\end{itemize}
			\end{itemize}
		%=======================================
		\item Chara enana.
			\begin{itemize}
				\item Estatus: Enemigo normal. 
				\item Habilidades: 
					\begin{itemize}
						\item Vuelo en picada.			
					\end{itemize}
			\end{itemize}
	\end{itemize}