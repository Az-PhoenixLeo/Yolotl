\subsection{Concepto del juego.}
Una vez definido el concepto general del argumento se procedió a definir las especificaciones técnicas y de jugabilidad del juego. En este punto se inició a escribir el documento de diseño en el orden que propone la plantilla de la metodología. 
\\
\par
En el primer apartado del documento de diseño se definió el concepto del juego. La primera decisión que se tomó en este apartado fue el género de videojuego que se desarrollaría, siendo elegida una combinación de dos géneros: plataforma y aventura. El principal motivo por el que se eligieron dichos géneros fue su complejidad, ya que siendo un equipo de dos personas y considerando el tiempo disponible de desarrollo, elegir géneros que requerían una mayor complejidad como RPG o Shooter minimizarían significativamente la factibilidad del juego.  
\\
\par
Posteriormente se redactó una sinopsis del contenido del juego de jugabilidad e historia del juego; más tarde en el mismo apartado la jugabilidad se describió de manera más detallada en la sección de mecánica de juego, en donde se definieron las acciones básicas del personaje principal y algunas de las reglas que rigen el comportamiento del juego a lo largo de todos los niveles. 
\\
\par
En este apartado también se definieron las tecnologías, tanto en hardware como en software, a utilizar para el desarrollo, eligiendo como plataforma dispositivos móviles con un mínimo de requerimientos técnicos que el teléfono Huawei TAG-L13 con sistema Android 5.2, esto debido a que el mercado de los juegos para dispositivos móviles es el que cuenta con mayor demanda[ ]; en cuanto a software se eligió a Unity como motor de desarrollo por la características ya mencionadas en (\ref{}). 
\\
\par
Finalmente se definieron aspectos legales y comerciales como el tipo de licencia de distribución a Atribución-NoComercial-CompartirIgual CC BY-NC-SA y el público objetivo del juego a jóvenes mayores de 13 años. Este último rubro no solo delimito el contenido argumental del juego y sus mecánicas sino que también fungió como un factor determinante para decidir un comportamiento totalmente offline, esto debido a que la ley orgánica de protección de datos de carácter profesional del manejo de información prohíbe que las aplicaciones puedan obtener información de menores de 14 años[ ] lo que imposibilita la opción de microtransacciones ante la posibilidad de que el jugador ingrese información sensible como número de tarjeta de crédito. 
