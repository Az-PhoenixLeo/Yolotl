\subsection{Idea concepto.}
Si bien la metodología propone un orden en el que se debe de llenar el documento de diseño, es importante aclarar que este orden puede o no seguirse. Lo anterior se debe a la naturaleza creativa y multidisciplinaria del videojuego; lo ocasiona que la idea principal del juego (el concepto) pueda venir bien de la idea de una mecánica de juego o de un argumento. En el caso del juego Yolotl, el juego nació primero como un argumento y después el argumento dio origen a la mecánica por lo que los primero rubros en llenarse fueron aquellos relacionados con la diégesis y el argumento del juego. 
\\
\par
Originalmente, Yolotl narraría la travesía de un guerrero en el Mictlán con el fin de traer de vuelta a la vida a su hermano. Con esta primera idea se propusieron cuatro niveles y una mecánica de juego más orientada a la resolución de puzles y al combate con diferentes armas. Desafortunadamente, esta primera idea jamás termino de aterrizarse y fue abandonada parcialmente. Apoyándose del fomento a la cultura se procedió a crear un nuevo argumento, esta vez con bases históricas más sólidas a fin de permitirle al jugador no solo interactuar con la cosmovisión de los Mexicas sino a su vez con el entorno social de los mismo. Conceptos como el viaje al Mictlán y el pacto con un Dios para revivir a un ser querido fueron algunas de las ideas que se mantuvieron con la segunda idea argumental del juego. 

