\subsection{Obstáculos}
De manera particular, para el proyecto Yolotl, se entiende por obstáculos a aquellos objetos dentro de un nivel que dificultan el avance continuo del jugado o que faciliten el fallo del jugador.
\\
\par
A diferencia de los personajes, la metodología huddle no maneja una plantilla para documentar estos objetos, por lo que se propuso una plantilla propia. Los campos de la plantilla son:
	\begin{itemize}
		\item Nombre del obstáculo.
		\item Descripción: Este campo describe tanto físicamente el objeto como su comportamiento e interacción con el jugador. 
		\item Esquema: Imagen de apoyo que facilita la comprensión de la descripción. 
	\end{itemize}
En total se definieron alrededor de 11 obstáculos, mismos que se podrán encontrar repartidos a lo largo de los niveles del juego. Algunos obstáculos serán exclusivos de un nivel mientras que otros se podrán encontrar en todos los niveles.
 \\
 \par
 A continuación se listarán los obstáculos del juego:
 	\begin{itemize}
 		\item Caja.
 		\item Sacos de cacao.
 		\item Plataforma móvil.
 		\item Plataforma que cae.
 		\item Plataforma que desaparece.
 		\item Estalagmitas.
 		\item Viento temporal.
 		\item Piedras filosas.
 		\item Piso congelado.
 		\item Bolas de nieve.
 		\item Lluvia de flechas.
 	\end{itemize}