\section{Investigación histórica}

Antes de proceder a diseñar el juego, fue necesario realizar una investigación
sobre la cultura mexica y sobre la Malinche. En esta etapa, el principal reto 
que se tuvo que afrontar fue la cantidad de la información que existe sobre 
los mexicas; a pesar de que esta cultura es de las más estudiadas del México
prehispánico, si se le compara con culturas del sur y del norte del país, existe 
una discrepancia en cuanto a su historia y cultura entre los historiadores y 
estudiosos del tema, por lo que a la hora de tomar decisiones sobre la información 
a utilizar se tuvo que optar por alguna de las tantas versiones existentes sobre 
una misma historia.
	\\
	\par
	La investigación que se realizó se puede dividir en tres partes, con base al 
	objeto de estudio de la investigación:
	\begin{itemize}
		\item \textbf{Sociedad Mexica:} historia, tradiciones y clases sociales.
		\item \textbf{Mitología Mexica:} Mitos, leyendas, dioses.
		\item \textbf{Historia de la Malinche:} vida antes de la llegada de los 
		españoles.
	\end{itemize}
	
	En los siguientes apartados se hablará a mayor profundidad sobre la información 
	encontrada en cada parte de la investigación y sobre que información se decidió
	agregar como contenido al juego.
	\\
	\par
	Para mayor comprensión sobre algunos de los puntos expuestos en esta 
	sección, se le recomienda al lector consultar los apartados de personajes y 
	guión en el documento de diseño.
	
	\subsection{Sociedad Mexica}
	La información que se decidió poner en el juego referente a la sociedad Mexica 
	fue referente a aquella que ayude al jugador a darse una idea sobre el contexto 
	que se vivía en esa época y las interacciones sociales que existían entre 
	individuos y culturas. 
	\\
	\par		
	Primeramente se encontró que, antes de la llegada de los españoles, México como 
	nación no existía, sino que en el territorio en donde convivían diversos pueblos 
	con rasgos culturales diversos y que en su gran mayoría estaban enemistados
	\cite{RefMexicasMito}. Esto llevó a la decisión de mostrar a los Mexicas como el 
	imperio conquistador que fue, mostrando a su vez el descontento que existía en los 
	pueblos conquistados por el imperio.
	\\
	\par
	 El segundo dato encontrado que se consideró de gran importancia, fue la 
	 organización de las clases sociales de los Mexicas y la gran importancia que 
	 tenia esa división social en la vida cotidiana; determinando nos solo los 
	 derechos de los habitantes de aquella época, sino a su vez determinaba vestimenta 
	 y comportamiento \cite{RefCivilAztea}. Esta información fue considerada relevante 
	 ya que influenciaría en el diseño de personajes para denotar su rango y 
	 personalidad.
	 \\
	 \par
	 Otro dato importante que influenció el diseño del juego fue la importancia que 
	 tenía el comercio en aquella época, al ser un medio de intercambio de mercancías
	 e información \cite{RefMexicasMito}. Esta información resultaría determinante a la 
	 hora de elegir la locación del nivel introductorio y su posterior diseño.
	 
	 \subsection{Mitología Mexica} 
	 La mitología Mexica es basta y muy extensa. Los Mexicas contaban con sus propios 
	 mitos y dioses y a éstos se les fueron integrando a sus creencias los mitos y
	  deidades de los pueblos que conquistaban. Dentro de sus principales deidades 
	  se encontraban \textit{Tezcatlipoca}, \textit{Quetzalcóatl} y 
	  \textit{Huitzilopochtli} \cite{RefMexicasSol}.
	 \\
	 \par
	 Para el diseño del argumento del videojuego se tomaron diferentes mitos y creencias
	 de los Mexicas, a continuación se mencionan, describiendo brevemente en que
	 consisten y como impactaron el videojuego:
	 \begin{itemize}
		\item \textbf{El mito de los cinco soles:} En este mito se narran la diferentes 
		creaciones y destrucciónes del mundo a manos de los dioses\cite{RefMexicasMito}. 
		Este mito permitió la plantear una historia anterior al juego y de darle una 
		identidad al mundo de los dioses al implementar una jerarquia de deidades 
		similar a la de la sociedad Mexica.
		%===============================================
		\item \textbf{El mito de la creación del hombre del maíz:} Mito que habla como 
		\textit{Quetzalcóatl} y el Dios \textit{Xólotl} descienden al \textit{Mictlán}
		 y cumplen una serie de retos para lograr hacerse de los huesos de la anterior 
		 humanidad y así crear a la actual\cite{RefMexicasMito}. Este mito es de gran 
		 importancia ya que la idea conceptual del juego nació de éste.
		 %===============================================
		\item \textbf{El \textit{Mictlán}:} Este lugar era el equivalente al inframundo
		dentro del pensamiento cristiano. El \textit{Mictlán} estaba constituido por 
		nueve niveles, cada uno vigilado por una deidad. Los difuntos iban a este lugar 
		para purificarse y volver a ser \textit{tonalli}(alma), lo que les permitía 
		volver a ser uno con el mundo. Le viaje al \textit{Mictlán} duraba cuatro años, 
		y durante su duración el difunto debía de afrontar una diferentes retos en 
		cada uno de los niveles del \textit{Mictlán}\cite{RefMexicasSol}. Este lugar 
		mitológico influenció fuertemente el diseño del videojuego, al determinar 
		mecánicas de juego, enemigos y la cantidad de niveles que el juego tiene. En 
		cuanto a los enemigos, se decidió incluir algunos Dioses que no eran propios 
		del \textit{Mictlán} pero que por sus habilidades o simbologia se les podía 
		relacionar con éste; el objetivo de proponer deidades fue para ofrecer una 
		variedad de dioses y que el jugador pudiera conocer dioses no tan famosos de la 
		mitología Mexica.    
	 \end{itemize}	 
	 
	 \subsection{Historia de la Malinche}   
	 La \textit{Malinche} es una de las figuras más controversiales de la historia 
	 mexicana. La investigación sobre su historia antes de la llegada de los españoles
	 resultó tan interesante que repercutió en el trabajo, haciendo que el personaje 
	 histórico se transformara el una de las figuras centrales de la narrativa y en la 
	 protagonista del juego. La figura de Malinche será manejada a lo largo del 
	 argumento del juego sin decirle al jugador quien es en realidad la protagonista 
	 es del juego, planeando revelar su identidad como un giro argumental de la trama.



