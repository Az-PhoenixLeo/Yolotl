\section{Desarrollo de videojuegos.}\label{DesVideojuego}
En esta sección se habla sobre el proceso de desarrollo del videojuego, hablando sobre los pasos que lleva el proceso de desarrollo; para despues describir las metodologias de desarrollo que se emplean, algunas de las metodologías descritas en este apartado son originarias del desarrollo de software convencional pero son adaptadas al desarrollo de software por algunos estudios independientes, es importante mencionar que muchas de las metodologias de desarrollo de videojuegos son propiedad de las empresas que las utilizan y por lo tanto no son de caracter público por lo que no pudieron ser incluidas en este trabajo terminal; para finalizar esta sección se habla sobre el software empleado en el desarrollo de videojuegos, entrandose en el motor de juego, el cual se define y se explica a grandez razgos su arquitectura.
	\subsection{Linea de producción de un videojuego.}\label{Pipelinevideojuego}
	Una linea de producción son los pasos o fases logicos y secuenciales requeridos para obtener un producto. Los pasos que componen la linea de producción dependen del producto que se va a fabricar y de la empresa fabricante. 
	\\
	\par	
	Existe una discrepacia en cuanto a que elementos tiene la linea de producción de un videojuego, siendo la representación más común la planteada por la revista ING, esta linea consiste en tres etapas\cite{Ref_Desarrollo}:
	\begin{itemize}
		\item \textbf{Concepto:} Es la idea que de origen a todo el juego que se va a realizar. Esta puede ser una simple oración en la que se mencione el contexto del juego y su tematica o también puede ser un acuerdo de la compañia desarrolladora para hacer una secuela o una precuela de un videojuego ya existente\cite{Ref_Desarrollo}.
	%===============================		
		\item \textbf{Preproducción:} En esta etapa el equipo de producción redacta el documento de diseño del videojuego, define el argumento, los persoanjes y la jugabilidad. En esta etapa se determinan todas las limitantes tecnicas y creativas que va a tener el proyecto\cite{Ref_Desarrollo}.
	%===============================
		\item \textbf{Producción:} En esta etapa se desarrolla el juego: los artistas desarrollan todos los elementos visuales que se van a emplear, los programadores se encargan de implementar la logica del juego y la jugabilidad establecida en la estapa de preproducción, el equipo de audio se encarga de generar todos los elementos de audio que conlleva el videojuego\cite{Ref_Desarrollo}.
	%===============================
		\item \textbf{Postproducción:} En esta etapa el juego se considera casi terminado y es sometido a diferentes pruebas para medir su rendimiento y encontrar y solucionar todo tipo de errores. Tambien en esta etapa se intensifican las campañas de promosión para el juego\cite{Ref_Desarrollo}.
\end{itemize}	 

	\subsection{Metodologias de desarrollo de videojuegos.}\label{MetodoVideojuego}
	En esta sección se define lo que es una metodologia de desarrollo de software, se mencionan tres metodologias de desarrollo de software que emplea la industria de los videojuegos y al final se menciona una metodologia de desarrollo propia del desarroll de videojuegos. 	
		 \subsubsection{¿Qué es una metodología de desarrollo de software?}
	Las metodologías de desarrollo de software son un conjunto de procedimientos, técnicas y ayudas a la documentación para el desarrollo de productos software\cite{Ref_metodologia}.	En palabras de Gacitúa: "Una Metodología impone un proceso de forma disciplinada sobre el desarrollo de software con el objetivo de hacerlo más predecible y eficiente. Una metodología define una representación que permite facilitar la manipulación de modelos, y la comunicación e intercambio de información entre todas las partes involucradas en la construcción de un sistema"\cite{Ref_Metod}. 

%======================================================		
			\subsubsection{Metodología en cascada}
La metodología de desarrollo en cascada o también conocida como modelo de vida lineal o básico,  fue propuesta por Royce en 1970 y a partir de entonces ha tenido diferentes modificaciones. Sigue una progresión lineal por lo que cualquier error que no se haya detectado con antelación afectara todas las fases que le sigan provocando una redefinición en el proyecto y por ende un aumento en los costos de producción del sistema \cite{Ref:CarCascada}.
Esta metodología se divide en las siguientes etapas:
\begin{itemize}
	\item \textbf{Análisis de los requisitos del software}: En esta etapa se recopilan los requisitos del sistema, se centra especialmente en toda aquella información que pueda resultar de utilidad en la etapa de diseño, tales como tipos de usuarios del sistema, reglas de negocio de la empresa, procesos, etc. En esta etapa se responde la pregunta de ¿Qué se hará? 
	\item \textbf{Diseño}: Esta etapa se caracteriza por definir todas aquellas características que le darán identidad al sistema, tales como la interfaz gráfica, la base de datos, etc. Las características anteriormente definidas se obtendrán de la etapa de análisis. En esta etapa se respondería la pregunta de ¿Cómo se hará? 
	\item \textbf{Codificación}: Terminada la etapa de diseño, lo siguiente es programar y crear todos los elementos necesarios para el funcionamiento del sistema. 
	\item \textbf{Prueba}: Finalizada la decodificación se debe de probar la calidad del sistema. En este punto es importante resaltar que la pruebas no solo abarcan que se confirme que el sistema funcione, sino que también verifica que los usuarios puedan aprender a utilizarlo con facilidad, entre otros aspectos como la seguridad de la información y los tiempos de respuesta del sistema.
	\item \textbf{Mantenimiento}: En esta última etapa se realizarán modificaciones al sistema, sin que esto necesariamente signifique que estos cambios se deban a errores de programación, puesto que esta etapa también abarca agregar nueva funcionalidad al sistema o, en caso de que trabaje con protocolos de estándar internacional, actualizar sus protocolos \cite{Ref:CarCascada}. 
\end{itemize}
Algunos de los inconvenientes que presenta son:
\begin{itemize}
	\item No refleja el proceso de desarrollo real.
	\item Tiempos largos de desarrollo.
	\item Poca comunicación con el cliente.
	\item Revisiones de proyecto de gran complejidad.
\end{itemize}

%================================================
	\subsubsection{Metodología en Scrum}
Desarrollada por Ikujiro Nonaka e Hirotaka Takeuchi a principios de los 80’s, Esta metodología le debe su nombre a la formación scrum de los jugadores de ruby. Scrum es una metodología eficaz para proyectos con requisitos inestables que demandan flexibilidad y rapidez, esto principalmente a su naturaleza iterativa e incremental \cite{Ref_DefScrum}.  
\\
\par
Scrum parte de la visión general que se desea que producto alcance; a partir de esta visión se inicia la división del proyecto en diferentes módulos Scrum implementa una jerarquía entre los módulos en donde los módulos de mayor jerarquía son los que se desarrollaran al inicio del proyecto o durante las primeras iteraciones (sprint). Cada sprint tendrá una duración de hasta seis semanas a lo máximo \cite{Ref_ScrumRef}. 
\\
\par
Durante el proceso de desarrollo del sprint, el equipo tendrá reuniones diarias en donde se definirán metas diarias para lograr completar el objetivo del sprint. Estas reuniones deberán de ser de corta duración (no más de quince minutos) y recibirán el nombre de scrum diario. Al final de cada sprint, el equipo contará con un módulo funcional que el cliente podrá utilizar sin que el sistema este completado.
\\
\par
Cada sprint se compone de las siguientes fases:
\begin{itemize}
	\item Concepto: se define a grandes rasgos las características del producto y se asigna a un equipo para desarrollarlo.
	\item Especulación: Con la información del concepto se delimita el producto, siendo las principales limitantes los tiempos y los costes. Esta es la fase más larga del sprint. En esta etapa se desarrolla basándose en la funcionalidad esperada por el concepto.
	\item Exploración: El producto desarrollado se integra al proyecto.
	\item Revisión: Se revisa lo construido y se contrasta con los objetivos deseados.
	\item Cierre: Se entrega el producto en la fecha programada, esta etapa no siempre significa el fin del proyecto; en ocasiones marca el inicio de la etapa de mantenimiento \cite{Ref_ScrumGuia}. 
\end{itemize}
Uno de los principales componentes de la metodología scrum son los roles, es decir el papel que cada integrante del equipo desempeñara durante el proceso de desarrollo. Los roles se dividen en dos grupos:
\begin{itemize}
	\item Cerdos : Son los que están comprometidos con el proyecto y el proceso de Scrum.
		\begin{itemize}
			\item Product owner: Es el jefe del proyecto y por lo tanto es quien toma las decisiones. Esta persona es quien conoce más del proyecto y las necesidades del cliente. Es el puente de comunicación entre el cliente y el resto del equipo. 
			\item Scrum Master: Se encarga de monitorear que la metodología y el modelo funcionen. Es quien toma las decisiones necesarias para eliminar cualquier inconveniente que pueda surgir durante el proceso de desarrollo. 
			\item Equipo de desarrollo: Estas personas reciben el objetivo a cumplir del Product owner y cuentan con la capacidad de tomar las decisiones necesarias para alcanzar dicho objetivo.
		\end{itemize}
	\item Gallinas: Personas que no participan de manera directa en el desarrollo, sin embargo, su retroalimentación da pie a la planeación de los sprints.
		\begin{itemize}
			\item Usuarios: Son quienes utilizaran el producto.
			\item Stakeholders: Son quienes el proyecto les aportara algún beneficio. Participan en las revisiones del sprint.
			\item Manager: Toma las decisiones finales. Participa en la selección de objetivos y en la toma de requerimientos\cite{Ref_ScrumRef}.
		\end{itemize}
\end{itemize}

%=============================
\subsubsection{Metodología de Programación extrema}
La metodología de programación extrema o metodología XP(por sus siglas en inglés) fue desarrollada por Kent Beck en 1999 basándose en la simplicidad, la comunicación y le retroalimentación de código. Es una metodología de desarrollo ágil y adaptativa, soporta cambios de requerimientos sobre la marcha. Su principal objetivo es aumentar la productividad y minimizar los procesos burocráticos, por lo que el software funcional tiene mayor importancia que la documentación\cite{Ref_XP}.
\\
\par
  XP se fundamenta en doce principios que se agrupan en cuatro categorías. A continuación, se hará mención de estos principios:
\begin{itemize}
	\item Retroalimentación:
		\begin{itemize}
			\item Principio de pruebas: Se define la el periodo de pruebas de funcionalidad del software a partir de sus entradas y salidas como si se tratara de una caja negra.
Planificación: El cliente o su representante definirá sus necesidades y sobre ellas se redactará un documento, el cual servirá para establecer los tiempos de entregas y de pruebas del producto.
			\item Cliente in-situ: El cliente o su representante se integrarán al equipo de trabajo con la finalidad de que participen en la planeación de tareas y en la definición de la funcionalidad del sistema. Esta estrategia se implementa para minimizar los tiempos de inactividad entre reuniones y disminuye la documentación a redactar.
			\item Pair-programming: Se asignan parejas de programadores para desarrollar el producto. Esto generará mejores resultados en menores costos.
		\end{itemize}
	\item Proceso continuo en lugar de por bloques
		\begin{itemize}
			\item Integración continua: Se implementan progresivamente las nuevas características del software. Esta integración no se hace de manera modular ni planeada.
			\item Refactorización: La eliminación de código duplicado o ineficiente les permite a los programadores mejorar sus propuestas en cada entregable.
			\item Entregas pequeñas: Los tiempos de entregas son cortos y permiten la evaluación del sistema bajo escenarios reales.
		\end{itemize}
	\item Entendimiento compartido
		\begin{itemize}
			\item Diseño simple: El programa que se utiliza en los entregables es aquel que tenga la mayor simplicidad y cubra las necesidades del cliente.
			\item Metáfora: expresa la visión evolutiva del proyecto y define los objetivos del sistema mediante una historia.
			\item Propiedad colectiva del código: Todos los programadores son dueños del programa y de las responsabilidades del programa. Un programa con muchos programadores trabajando en él es menos propenso a errores. 
			\item Estándar de programación: Se define la estructura que tendrá el programa a la hora de ser escrito, esto para dar la impresión de que una sola persona trabajo en él.
		\end{itemize}
	\item Bienestar del programador
		\begin{itemize}
			\item Semana de 40 horas: Se minimizan las jornadas de trabajo excesivas para grantizar el mejor desempeño del equipo\cite{Ref_XPPrincipios}.
		\end{itemize}
\end{itemize}
Tal como se puede observar XP, es una metodología fuertemente orientada hacia los miembros del equipo, su bienestar, la interacción entre ellos y en su aprendizaje.

%==========================================
\subsubsection{Metodología Huddle}
Huddle es una metodología creada por el Instituto de Ingeniería y Tecnología de Universidad Autónoma de Ciudad Juárez. Huddle recibe su nombre por las reuniones que se realizan en el futbol americano antes de cada jugada. Su funcionalidad se basa en la metodología Scrum, con la diferencia de que está orientada en el desarrollo de videojuegos.  De naturaleza ágil, resulta óptimo para equipos multidisciplinarios de 5 a 10 personas; es iterativa, incremental y evolutiva \cite{Ref_Huddle}.
\\
\par
Huddle se divide en tres etapas: 
	\begin{itemize}
		\item Preproducción: Consiste en la planeación del juego. En esta etapa se redactará el documento de diseño; este documento contendrá la idea general del juego, su escritura deberá de ser tal que todos los miembros del equipo pueden entenderlo y darse una idea de cómo será el juego una vez que se haya terminado. En esta etapa se definirá el argumento del juego, sus personajes, el género del juego, sus mecánicas, la música, los efectos de sonido, los efectos especiales y su funcionalidad. Huddle proporciona plantilla para realizar este documento, dejando la posibilidad de modificarlo según el equipo considere oportuno.
		\item Producción: Es la etapa más larga y de mayor importancia. Su organización se basa totalmente en la organización iterativa e incremental de Scrum; es decir se harán reuniones diarias en donde se discutirán los objetivos de la iteración. Antes de finalizar cada Sprint, el módulo se someterá a diferentes pruebas para garantizar su funcionalidad. Cuando un Sprint finaliza, se realiza una reunión en la que los elementos del quipo discuten las decisiones tomadas y analizan cuales fueron las decisiones y acciones más eficientes para retomarlas y desechar aquellas que atrasen al proyecto. Al finalizar esta etapa el equipo contará con las versiones alfa y beta del juego. 
		\item Postmorten: En esta etapa se discuten todos los puntos positivos y negativos del proyecto. En esta evaluación se redactará un documento que permita a futuros proyectos efectuar planes de acción más efectivos\cite{Ref_Huddle}.
	\end{itemize}
	\subsection{Software para el desarrollo de videojuegos.}\label{SoftVideojue}
En este apartado se habla del software que comúnmente se emplea para el desarrollo de videojuegos, empezando por el motor de juego, la definición del motor de juego, la arquitectura del motor de juego, los motores de juego más usados en el mercados; para finalizar con una lista del software auxiliar que se usa para generar lo elementos visuales y auditivos que componen al juego.

%==========================================
 
\subsubsection{Motor de juego.}
El motor de juego, también conocido como Game Engine, parte del concepto de reutilización; es decir, es posible generar juegos a partir de un código base y común mediante una separación adecuada de los componentes fundamentales, tal como visualización de gráficos, control de colisiones, físicas, entrada de datos etc \cite{Ref:MutorGraf}; esto permite a quienes trabajen en un juego puedan centrarse en todos aquellos detalles que hacen al juego único.

%==========================================

\subsubsection{Arquitectura del motor}
Los motores de juego se basan en una arquitectura estructurada a capas. Por lo que las capas de nivel superior dependen directamente de las de nivel inferior \cite{Ref:ArquMotor} 
 A continuacion se mencionaran las capas que componen al motor de juego junto a una breve descripción de la capas.
 
 \begin{itemize}
 	\item \textbf{Hardware:} esta capas e relaciona con la plataforma sobre la que se ejecutará el juego. Existen motores gráficos orientados hacia una sola plataforma (dispositivos móviles, consolas caseras, computadoras o consolas portátiles, etc.) y existe motores multiplataforma que permiten el desarrollo simultaneo de un juego para diferentes plataformas (cross-platform) \cite{Ref:ArquMotor}.
 	\item \textbf{Drivers:} Esta capa garantiza la correcta gestión de determinados dispositivos (tarjeta grafica, tarjeta de sonido, etc.) haciendo uso de software de bajo nivel \cite{Ref:ArquMotor}. 
 	\item \textbf{Sistema Operativo:} Esta capa garantiza la comunicación de los procesos que se ejecutan en el sistema operativos y los recursos de la plataforma asociada con el juego \cite{Ref:ArquMotor}.
 	\item {Kits de de desarrollo de software y middleware:} Un Kit de de desarrollo de software(SDK, por sus siglas en inglés) son todas aquella herramientas que le permiten al programador desarrollar aplicaciones informaticas para una plataforma determinada \cite{ref:SDK}. Mientras que un middleware es software que se sitúa entre un sistema operativo y las aplicaciones que se ejecutan en él. Básicamente, funciona como una capa de traducción oculta para permitir la comunicación y la administración de datos en aplicaciones distribuidas \cite{Ref:middleware}. 
 	\item \textbf{Capa independiente de la plataforma:} Esta capa aísla las capas dependientes de la plataforma para la que se va a desarrollar el juego, de las capas superiores que son estándares e independientes de la plataforma \cite{Ref:ArquMotor}. 
 	\item \textbf{Subsistemas principales:} Esta capa esta compuesta sub sistemas que vinculan a todas aquellas utilidades o bibliotecas de utilidades que dan soporte al motor de juegos. Tal como:
 	\begin{itemize}
 		\item Biblioteca matemática.
 		\item Estructuras de datos y algoritmos.
 		\item Gestión de memoria.
 		\item Depuración y logging \cite{Ref:ArquMotor}.
 	\end{itemize}
 	\item \textbf{Gestor de recursos:} Esta capa es responsable de generar una interfaz de comunicación unificada para acceder a las distintas entidades de software que componen el motor de juego, como por ejemplo las escenas, los sonidos o los objetos de juego \cite{Ref:ArquMotor}.
 	\item \textbf{Motor de rendering:} Renderizado (render en inglés) es un término usado en computacion para referirse al proceso de generar una imagen foto realista desde un modelo 3D \cite{Ref:Render}. Esta capa tiene una gran importancia, debido a la naturaleza gráfica del videojuego. El enfoque más utilizado para implementar esta capa es utilizando una arquitectura multi-capa\cite{Ref:ArquMotor}.
 	\item \textbf{Herramientas de depuración:} Esta capa se encarga de depurar y optimizar el motor de juego para obtener un mejor rendimiento\cite{Ref:ArquMotor}.
 	\item \textbf{Motor de Física:} Esta capa se encarga de gestionar la detección de colisiones, su determinación y la posterior respuesta que tendrá el juego ante dicha colisión.
 	\item \textbf{Interfaces de usuario:} Esta capa tiene como objetivo ofrecer una abstracción de las interacciones del usuario con el juego y de tratar todos los eventos de salida, es decir la retroalimentación que el juego le da al usuario\cite{Ref:ArquMotor}.
 	\item \textbf{Networking y multijugador:} Esta capa permite que el juego sea capaz de soportar diferentes jugadores de manera simultanea, ya sea que se encuentren de manera local (es decir en una misma plataforma sin conexión a internet) o de manera online (haciendo uso del internet)\cite{Ref:ArquMotor}.
 	\item \textbf{Subsistema de juego:} Esta capa permite la creación de las mecánicas de juegos; es decir es capa soporta la implementación de un lenguaje de programación, comúnmente de alto nivel, para definir el comportamiento de todos aquellos elementos que componen el juego, como enemigos, cámaras, obstáculos, etc \cite{Ref:ArquMotor}.
 	\item Audio: Esta capa proporciona al moto la capacidad de utilizar archivos de audio para garantizar una mejor experiencia al usuario\cite{Ref:ArquMotor}.
 	\item \textbf{Subsistemas específicos de juego:} En esta capa se implementan todos aquellos módulos que proporcionen una identidad al sistema y por lo tanto son únicos\cite{Ref:ArquMotor}.
 \end{itemize}	
 
%========================================== 
 
 \subsubsection{Motores gráficos existentes en el mercado.}
En este apartado se mencionaran los principales motores de juego que existen en la industria, de igual manera se hará mención de sus principales características.

	\begin{itemize}
			%=====================================
		\item \textbf{Unity3D:} Actualmente Unity es el motor grafico más utilizado en la industria. 
			\begin{itemize}
				\item \textbf{Sistema operativo:} Microsoft ver 10,8, 7(solo 64 bits); MacOs ver X 10.9 en adelante.
				\item \textbf{CPU:} Soporte para el conjunto de instrucciones SSE2.
				\item \textbf{GPU:} Tarjeta gráfica con DX9 (modelo de shader 3.0) o DX11 con capacidades de funciones de nivel 9.3.
				\item \textbf{Memoria RAM:} Depende de la complejidad del proyecto.
				\item \textbf{Desarrollo para plataforma:} Cross-platform.
				\item \textbf{Orientado a 2D/3D:} 2D y 3D.
				\item \textbf{Lenguaje de programación que soporta:} $\sharp C$, javaScript, Boo.
				\item \textbf{Tipo de Licencia:} Maneja tres tipos de licencia, dos de pago y uno gratuito. \cite{Ref:Unity} 
			\end{itemize}		
		%=====================================
		\item \textbf{UnrealEngine:} Considerado por algunas revistas especialistas en videojuegos como el motor de juego más potente. 
			\begin{itemize}
				\item \textbf{Sistema operativo:} Microsoft ver 10,8, 7(solo 64 bits); macOS 10.13 High Sierra y Ubuntu 15.04.
				\item \textbf{CPU:} SQuad-core Intel or AMD, 2.5 GHz or faster (Para Windows), Quad-core Intel, 2.5 GHz or faster(Para Mac y linux).
				\item \textbf{Tarjeta de vídeo:} DirectX 11 compatible graphics card (Para Windows), Metal 1.2 Compatible Graphics Card(Para Mac) y NVIDIA GeForce 470 GTX or higher with latest NVIDIA binary drivers(Linux). 
				\item \textbf{Memoria RAM:} 8GB (Microsoft y Mac) y 16GB (Linux).
				\item \textbf{Desarrollo para plataforma:} Cross-platform.
				\item \textbf{Orientado a 2D/3D:} 2D y 3D.
				\item \textbf{Lenguaje de programación que soporta:} C++.
				\item \textbf{Tipo de Licencia:} licencia de pago pero se debe de pagar el 5 por ciento de las regalias cuando el juego sea publicado. \cite{Ref:Unreal}
			\end{itemize}
		%=====================================
		\item \textbf{CryEngine:} Considerado por algunas revistas especialistas en videojuegos como el motor de juego más potente. 
			\begin{itemize}
				\item \textbf{Sistema operativo:} Microsoft ver 10,8, 7(solo 64 bits y 32 bits).
				\item \textbf{CPU:} Intel Dual-Core min 2GHz (Core 2 Duo and above) o AMD Dual-Core min 2GHz (Phenom II X2 and above).
				\item \textbf{Tarjeta de vídeo:} NVIDIA GeForce 450 series o AMD Radeon HD 5750 series or higher (minimum 1 GB dedicated VRAM GDDR5). 
				\item \textbf{Memoria RAM:} 4GB.
				\item \textbf{Desarrollo para plataforma:} Cross-platform.
				\item \textbf{Orientado a 2D/3D:} 2D y 3D.
				\item \textbf{Lenguaje de programación que soporta:} C++, $\sharp C$ y Lua.
				\item \textbf{Tipo de Licencia:} Licencia gratuita pero ofrece planes de pago para capacitación. \cite{Ref:CryEngine}
			\end{itemize}						
		
	\end{itemize}
	
	%===================================
	\subsubsection{Software auxiliar}
	Además de los motores gráficos el proceso de desarrollo de videojuegos necesita diferentes herramientas auxiliares para la creación de todos aquellos elementos que se necesiten poner dentro del juego, sea personajes, música, fondos, efectos de sonido, etc. A continuación, se mostrará una lista de aplicaciones y páginas web que fungen como herramientas auxiliares en el desarrollo de videojuegos:
	
	\begin{itemize}
	%===================================================
		\item Creación de Sprites (Solo juegos 2D) o texturas.
			\begin{itemize}
	%===================================================
				\item Adobe Photoshop.
					\begin{itemize}
						\item Descripción: Aplicación de diseño y tratamiento de imágenes. Con esta aplicación se pueden crear ilustraciones e imágenes 3d. Su capacidad de manejo de imágenes secuenciales la hacen de gran ayuda en la generación de imágenes de bloques de animación para los sprites de juegos 2D, así como su compatibilidad con Adobe Ilustrator facilitan la vectorización de sprites.
						\item Requerimientos mínimos en Windows:
						\begin{itemize}
							\item Procesador Intel Core 2 o AMD Athlon 64 processor de 2 GHz.
							\item Sistema operativo Microsoft Windows 7, Windows 8.1, o Windows 10.
							\item 2 GB de RAM.
							\item Espacio de 2.6 GB en el disco duro para instalcion en 32 bits; o 3.1 GB para sistemas de 64 bits.
							\item Pantalla de 1024 x 768 con 16-bit de color y 512 MB de VRAM [].
						\end{itemize}
					\end{itemize}
%===================================================					
				\item Adobe Ilustrator.
					\begin{itemize}
						\item Descripción: Esta aplicación de gráficos vectoriales permite crear logotipos, iconos, dibujos, tipografías e ilustraciones para ediciones impresas, la web, vídeos y dispositivos móviles. Su sistema de vectorización de imágenes permite crear sprites de mejor calidad.  Es una buena herramienta para la creación de botones o iconos para la GUI de juegos.
						\item Requerimientos mínimos en Windows:
							\begin{itemize}
								\item Procesador Intel Pentium 4 or AMD Athlon 64 processor
								\item Sistema operativo Microsoft Windows 7, Windows 8.1, o Windows 10
								\item 1 GB de RAM para 32 bits; 2 GB de RAM para 64 bit
								\item 2 GB libres en el disco duro.
								\item Pantalla de 1024 x 768, 1GB de VRAM.
							\end{itemize}

					\end{itemize}
%===================================================
				\item AutoDesk SketchBook.
					\begin{itemize}
						\item Descripción: Herramienta de diseño, más orientada hacia artistas que hacía diseñadores. Es una herramienta de gran utilidad en la creación de arte conceptual para el juego y el diseño de personajes. También posee una herramienta que permite la creación de imágenes secuenciales para bloques de animación. Tiene una total compatibilidad con Adobe Photoshop, por lo que se pueden exportar proyectos desde AutoDesk SketchBook sin el temor de perder detalles de diseño. Su principal ventaja es que se encuentra disponible para dispositivos móviles (Android e IOS) y computadoras (Windows y  MAC), cuenta con tres tipos de licencias: la gratuita (tiene funcionalidad limitada), la de pago (por un único pago se cuenta con varias herramientas de diseño) y la pro (Suscrición mensual que ofrece la total funcionalidad de la aplicación y permite utilizar toda funcionalidad  tanto en dispositivos móviles como en computadoras ).
						\item Requerimientos mínimos en Windows:
							\begin{itemize}
								\item Sistema operativo Windows 7 SP1 (32 bit, 64 bit), Windows 8/8.1 (32 bit, 64 bit), o Windows 10.
								\item Procesador de 1 GHz Intel o AMD CPU.
								\item 1GB de Memoria.
								\item 256 MB de tarjeta gráfica con soporte de OpenGL 2.0.

							\end{itemize}
					\end{itemize}
			\end{itemize}
%===================================================
		\item Modelos 3D y animación 3D.
			\begin{itemize}
				\item Blender.
					\begin{itemize}
						\item Descripción: Aplicación de modelado y animación 3D de licencia libre. Se encuentra disponible para Windows, Linux y macOS. Blender permite la exportación de modelos, paquetes de animación y escenarios enteros a motores gráficos como Unity3D.
						\item Requerimientos mínimos:
							\begin{itemize}
								\item CPU de 32-bit dual core 								\item 2Ghz  con soporte a SSE2.
								\item 2 GB de memoria RAM.
								\item Pantalla de 24 bits 1280 x 768.
								\item OpenGL 2.1 Compatible con gráficos y con 512 MB RAM.
							\end{itemize}
					\end{itemize}
%===================================================
				\item Maya.
					\begin{itemize}
						\item Descripción: Es un software de renderización, simulación, modelado y animación 3D. Maya ofrece un conjunto de herramientas integrado y potente, que puede usar para crear animaciones, entornos, gráficos de movimiento, realidad virtual y personajes. Se encuentra disponible para Windows, Linux y macOS.
						\item Requerimientos mínimos:
							\begin{itemize}
								\item Procesador de varios núcleos de 64 bits Intel o AMD con el conjunto de instrucciones SSE4.2.
								\item 8 GB de RAM.
								\item 4 GB de espacio libre en disco para la instalación.
							\end{itemize}
					\end{itemize}
			\end{itemize}
%===================================================
		\item Edición y creación de sonido.
			\begin{itemize}
				\item Ardour
					\begin{itemize}
						\item Descripción: Software que permite grabar, editar y mezclar audio. Su público objetivo son ingenieros de audio, compositores, músicos y editores se soundtracks. Se encuentra disponible para Mac, Windows y Linux. Posee soporte para pluings.
						\item Requerimientos mínimos para Linux:
						\begin{itemize}
							\item Cualquier procesador de 32 o 64 bits Intel.
							\item Cualquier distribucion de linux con un kernel más actual al 2.3 y libc version 2.25 
							\item 2GB de RAM.
							\item Espacio minimo de 350MB en el disco duro.

						\end{itemize}
					\end{itemize}
			\end{itemize}
\end{itemize}
	
	