\subsubsection{Arquitectura del motor}
Los motores de juego se basan en una arquitectura estructurada a capas. Por lo que las capas de nivel superior dependen directamente de las de nivel inferior \cite{Ref:ArquMotor} 
 A continuacion se mencionaran las capas que componen al motor de juego junto a una breve descripción de la capas.
 
 \begin{itemize}
 	\item \textbf{Hardware:} esta capas e relaciona con la plataforma sobre la que se ejecutará el juego. Existen motores gráficos orientados hacia una sola plataforma (dispositivos móviles, consolas caseras, computadoras o consolas portátiles, etc.) y existe motores multiplataforma que permiten el desarrollo simultaneo de un juego para diferentes plataformas (cross-platform) \cite{Ref:ArquMotor}.
 	\item \textbf{Drivers:} Esta capa garantiza la correcta gestión de determinados dispositivos (tarjeta grafica, tarjeta de sonido, etc.) haciendo uso de software de bajo nivel \cite{Ref:ArquMotor}. 
 	\item \textbf{Sistema Operativo:} Esta capa garantiza la comunicación de los procesos que se ejecutan en el sistema operativos y los recursos de la plataforma asociada con el juego \cite{Ref:ArquMotor}.
 	\item {Kits de de desarrollo de software y middleware:} Un Kit de de desarrollo de software(SDK, por sus siglas en inglés) son todas aquella herramientas que le permiten al programador desarrollar aplicaciones informaticas para una plataforma determinada \cite{ref:SDK}. Mientras que un middleware es software que se sitúa entre un sistema operativo y las aplicaciones que se ejecutan en él. Básicamente, funciona como una capa de traducción oculta para permitir la comunicación y la administración de datos en aplicaciones distribuidas \cite{Ref:middleware}. 
 	\item \textbf{Capa independiente de la plataforma:} Esta capa aísla las capas dependientes de la plataforma para la que se va a desarrollar el juego, de las capas superiores que son estándares e independientes de la plataforma \cite{Ref:ArquMotor}. 
 	\item \textbf{Subsistemas principales:} Esta capa esta compuesta sub sistemas que vinculan a todas aquellas utilidades o bibliotecas de utilidades que dan soporte al motor de juegos. Tal como:
 	\begin{itemize}
 		\item Biblioteca matemática.
 		\item Estructuras de datos y algoritmos.
 		\item Gestión de memoria.
 		\item Depuración y logging \cite{Ref:ArquMotor}.
 	\end{itemize}
 	\item \textbf{Gestor de recursos:} Esta capa es responsable de generar una interfaz de comunicación unificada para acceder a las distintas entidades de software que componen el motor de juego, como por ejemplo las escenas, los sonidos o los objetos de juego \cite{Ref:ArquMotor}.
 	\item \textbf{Motor de rendering:} Renderizado (render en inglés) es un término usado en computacion para referirse al proceso de generar una imagen foto realista desde un modelo 3D \cite{Ref:Render}. Esta capa tiene una gran importancia, debido a la naturaleza gráfica del videojuego. El enfoque más utilizado para implementar esta capa es utilizando una arquitectura multi-capa\cite{Ref:ArquMotor}.
 	\item \textbf{Herramientas de depuración:} Esta capa se encarga de depurar y optimizar el motor de juego para obtener un mejor rendimiento\cite{Ref:ArquMotor}.
 	\item \textbf{Motor de Física:} Esta capa se encarga de gestionar la detección de colisiones, su determinación y la posterior respuesta que tendrá el juego ante dicha colisión.
 	\item \textbf{Interfaces de usuario:} Esta capa tiene como objetivo ofrecer una abstracción de las interacciones del usuario con el juego y de tratar todos los eventos de salida, es decir la retroalimentación que el juego le da al usuario\cite{Ref:ArquMotor}.
 	\item \textbf{Networking y multijugador:} Esta capa permite que el juego sea capaz de soportar diferentes jugadores de manera simultanea, ya sea que se encuentren de manera local (es decir en una misma plataforma sin conexión a internet) o de manera online (haciendo uso del internet)\cite{Ref:ArquMotor}.
 	\item \textbf{Subsistema de juego:} Esta capa permite la creación de las mecánicas de juegos; es decir es capa soporta la implementación de un lenguaje de programación, comúnmente de alto nivel, para definir el comportamiento de todos aquellos elementos que componen el juego, como enemigos, cámaras, obstáculos, etc \cite{Ref:ArquMotor}.
 	\item Audio: Esta capa proporciona al moto la capacidad de utilizar archivos de audio para garantizar una mejor experiencia al usuario\cite{Ref:ArquMotor}.
 	\item \textbf{Subsistemas específicos de juego:} En esta capa se implementan todos aquellos módulos que proporcionen una identidad al sistema y por lo tanto son únicos\cite{Ref:ArquMotor}.
 \end{itemize}