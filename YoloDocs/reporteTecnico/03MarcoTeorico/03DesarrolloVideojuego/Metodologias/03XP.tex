\subsubsection{Metodología de Programación extrema}
La metodología de programación extrema o metodología XP(por sus siglas en inglés) fue desarrollada por Kent Beck en 1999 basándose en la simplicidad, la comunicación y le retroalimentación de código. Es una metodología de desarrollo ágil y adaptativa, soporta cambios de requerimientos sobre la marcha. Su principal objetivo es aumentar la productividad y minimizar los procesos burocráticos, por lo que el software funcional tiene mayor importancia que la documentación [ ].
\\
\par
  XP se fundamenta en doce principios que se agrupan en cuatro categorías. A continuación, se hará mención de estos principios:
\begin{itemize}
	\item Retroalimentación:
		\begin{itemize}
			\item Principio de pruebas: Se define la el periodo de pruebas de funcionalidad del software a partir de sus entradas y salidas como si se tratara de una caja negra.
Planificación: El cliente o su representante definirá sus necesidades y sobre ellas se redactará un documento, el cual servirá para establecer los tiempos de entregas y de pruebas del producto.
			\item Cliente in-situ: El cliente o su representante se integrarán al equipo de trabajo con la finalidad de que participen en la planeación de tareas y en la definición de la funcionalidad del sistema. Esta estrategia se implementa para minimizar los tiempos de inactividad entre reuniones y disminuye la documentación a redactar.
			\item Pair-programming: Se asignan parejas de programadores para desarrollar el producto. Esto generará mejores resultados en menores costos.
		\end{itemize}
	\item Proceso continuo en lugar de por bloques
		\begin{itemize}
			\item Integración continua: Se implementan progresivamente las nuevas características del software. Esta integración no se hace de manera modular ni planeada.
			\item Refactorización: La eliminación de código duplicado o ineficiente les permite a los programadores mejorar sus propuestas en cada entregable.
			\item Entregas pequeñas: Los tiempos de entregas son cortos y permiten la evaluación del sistema bajo escenarios reales.
		\end{itemize}
	\item Entendimiento compartido
		\begin{itemize}
			\item Diseño simple: El programa que se utiliza en los entregables es aquel que tenga la mayor simplicidad y cubra las necesidades del cliente.
			\item Metáfora: expresa la visión evolutiva del proyecto y define los objetivos del sistema mediante una historia.
			\item Propiedad colectiva del código: Todos los programadores son dueños del programa y de las responsabilidades del programa. Un programa con muchos programadores trabajando en él es menos propenso a errores. 
			\item Estándar de programación: Se define la estructura que tendrá el programa a la hora de ser escrito, esto para dar la impresión de que una sola persona trabajo en él.
		\end{itemize}
	\item Bienestar del programador
		\begin{itemize}
			\item Semana de 40 horas: Se minimizan las jornadas de trabajo excesivas para grantizar el mejor desempeño del equipo.[]
		\end{itemize}
\end{itemize}
Tal como se puede observar XP, es una metodología fuertemente orientada hacia los miembros del equipo, su bienestar, la interacción entre ellos y en su aprendizaje.