	\subsection{Software auxiliar}
	Además de los motores gráficos el proceso de desarrollo de videojuegos necesita diferentes herramientas auxiliares para la creación de todos aquellos elementos que se necesiten poner dentro del juego, sea personajes, música, fondos, efectos de sonido, etc. A continuación, se mostrará una lista de aplicaciones y páginas web que fungen como herramientas auxiliares en el desarrollo de videojuegos:
	
	\begin{itemize}
	%===================================================
		\item Creación de Sprites (Solo juegos 2D) o texturas.
			\begin{itemize}
	%===================================================
				\item Adobe Photoshop.
					\begin{itemize}
						\item Descripción: Aplicación de diseño y tratamiento de imágenes. Con esta aplicación se pueden crear ilustraciones e imágenes 3d. Su capacidad de manejo de imágenes secuenciales la hacen de gran ayuda en la generación de imágenes de bloques de animación para los sprites de juegos 2D, así como su compatibilidad con Adobe Ilustrator facilitan la vectorización de sprites.
						\item Requerimientos mínimos en Windows:
						\begin{itemize}
							\item Procesador Intel Core 2 o AMD Athlon 64 processor de 2 GHz.
							\item Sistema operativo Microsoft Windows 7, Windows 8.1, o Windows 10.
							\item 2 GB de RAM.
							\item Espacio de 2.6 GB en el disco duro para instalcion en 32 bits; o 3.1 GB para sistemas de 64 bits.
							\item Pantalla de 1024 x 768 con 16-bit de color y 512 MB de VRAM [].
						\end{itemize}
					\end{itemize}
%===================================================					
				\item Adobe Ilustrator.
					\begin{itemize}
						\item Descripción: Esta aplicación de gráficos vectoriales permite crear logotipos, iconos, dibujos, tipografías e ilustraciones para ediciones impresas, la web, vídeos y dispositivos móviles. Su sistema de vectorización de imágenes permite crear sprites de mejor calidad.  Es una buena herramienta para la creación de botones o iconos para la GUI de juegos.
						\item Requerimientos mínimos en Windows:
							\begin{itemize}
								\item Procesador Intel Pentium 4 or AMD Athlon 64 processor
								\item Sistema operativo Microsoft Windows 7, Windows 8.1, o Windows 10
								\item 1 GB de RAM para 32 bits; 2 GB de RAM para 64 bit
								\item 2 GB libres en el disco duro.
								\item Pantalla de 1024 x 768, 1GB de VRAM.
							\end{itemize}

					\end{itemize}
%===================================================
				\item AutoDesk SketchBook.
					\begin{itemize}
						\item Descripción: Herramienta de diseño, más orientada hacia artistas que hacía diseñadores. Es una herramienta de gran utilidad en la creación de arte conceptual para el juego y el diseño de personajes. También posee una herramienta que permite la creación de imágenes secuenciales para bloques de animación. Tiene una total compatibilidad con Adobe Photoshop, por lo que se pueden exportar proyectos desde AutoDesk SketchBook sin el temor de perder detalles de diseño. Su principal ventaja es que se encuentra disponible para dispositivos móviles (Android e IOS) y computadoras (Windows y  MAC), cuenta con tres tipos de licencias: la gratuita (tiene funcionalidad limitada), la de pago (por un único pago se cuenta con varias herramientas de diseño) y la pro (Suscrición mensual que ofrece la total funcionalidad de la aplicación y permite utilizar toda funcionalidad  tanto en dispositivos móviles como en computadoras ).
						\item Requerimientos mínimos en Windows:
							\begin{itemize}
								\item Sistema operativo Windows 7 SP1 (32 bit, 64 bit), Windows 8/8.1 (32 bit, 64 bit), o Windows 10.
								\item Procesador de 1 GHz Intel o AMD CPU.
								\item 1GB de Memoria.
								\item 256 MB de tarjeta gráfica con soporte de OpenGL 2.0.

							\end{itemize}
					\end{itemize}
			\end{itemize}
%===================================================
		\item Modelos 3D y animación 3D.
			\begin{itemize}
				\item Blender.
					\begin{itemize}
						\item Descripción: Aplicación de modelado y animación 3D de licencia libre. Se encuentra disponible para Windows, Linux y macOS. Blender permite la exportación de modelos, paquetes de animación y escenarios enteros a motores gráficos como Unity3D.
						\item Requerimientos mínimos:
							\begin{itemize}
								\item CPU de 32-bit dual core 								\item 2Ghz  con soporte a SSE2.
								\item 2 GB de memoria RAM.
								\item Pantalla de 24 bits 1280 x 768.
								\item OpenGL 2.1 Compatible con gráficos y con 512 MB RAM.
							\end{itemize}
					\end{itemize}
%===================================================
				\item Maya.
					\begin{itemize}
						\item Descripción: Es un software de renderización, simulación, modelado y animación 3D. Maya ofrece un conjunto de herramientas integrado y potente, que puede usar para crear animaciones, entornos, gráficos de movimiento, realidad virtual y personajes. Se encuentra disponible para Windows, Linux y macOS.
						\item Requerimientos mínimos:
							\begin{itemize}
								\item Procesador de varios núcleos de 64 bits Intel o AMD con el conjunto de instrucciones SSE4.2.
								\item 8 GB de RAM.
								\item 4 GB de espacio libre en disco para la instalación.
							\end{itemize}
					\end{itemize}
			\end{itemize}
%===================================================
		\item Edición y creación de sonido.
			\begin{itemize}
				\item Ardour
					\begin{itemize}
						\item Descripción: Software que permite grabar, editar y mezclar audio. Su público objetivo son ingenieros de audio, compositores, músicos y editores se soundtracks. Se encuentra disponible para Mac, Windows y Linux. Posee soporte para pluings.
						\item Requerimientos mínimos para Linux:
						\begin{itemize}
							\item Cualquier procesador de 32 o 64 bits Intel.
							\item Cualquier distribucion de linux con un kernel más actual al 2.3 y libc version 2.25 
							\item 2GB de RAM.
							\item Espacio minimo de 350MB en el disco duro.

						\end{itemize}
					\end{itemize}
			\end{itemize}
%===================================================
		\item Páginas de descargas.
		\\
		\par
		Todo motor gráfico tiene una opción de tienda en la que los usuarios pueden comprar o utilizar recursos creados por la comunidad de desarrolladores. De igual manera existen diferentes páginas en la web que funcionan como tiendas virtuales de recursos para desarrollo de videojuegos tales como: 
		\begin{itemize}
			\item https://free3d.com/
			\\
			\par
Esta página web contiene modelos, texturas, curvas de animación, materiales y escenarios para entornos 3d tanto gratuitos como de pago. Lo modelos son principalmente compatibles con Maya, Blender y Autodesk.
			\item http://www.gameart2d.com/
			\\
			\par
Esta página contiene sprites en 2D e iconos para GUI gratuitos y de pago. Los sprites se pueden descargar tanto en formato png, como en formato ai y psd.
			\item http://www.sonidosmp3gratis.com/
			\\
			\par
Esta página permite la descarga de efectos de sonido. 
			\item https://soundcloud.com/freebmusic
			\\
			\par
Permite la descarga de música de fondo totalmente gratis.
		\end{itemize}
	\end{itemize}