\section{Videojuego}
			\subsection{Definición}
			Un videojuego es un medio de entretenimiento
			que involucra a un usuario, denominado jugador, en
			una interacción constante entre una interfaz y un
			dispositivo de video. Los videojuegos recrean
			entornos y situaciones virtuales en los que el jugador
			puede controlar la situación para
			alcanzar objetivos por medio de determinadas reglas.
			La interacción se lleva a cabo mediante
			dispositivos de salida y de entrada.
		
			Los videojuegos arte, ciencia y tecnología;
			involucran una plétora de habilidades y
			conocimientos en distintas disciplinas, desde ciencias
			formales hasta ciencias sociales que van más allá del
			típico proyecto de software e implican al mismo
			tiempo la creatividad y la imaginación.
			
			\subsection{Clasificación}
			\subsection{Industria en México}
			En la era que (Dille \& Zuur, 2007) definieron
			como primitiva, cualquier persona podía aprender a
			programar un juego de computadora si se disponía de
			una Computadora Personal (PC) y conocimientos del
			lenguaje Ensamblador. Producir un juego no
			necesitaba más de dos personas y/o más de 3 meses.	
		
