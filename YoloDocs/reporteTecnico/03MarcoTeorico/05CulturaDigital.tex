\documentclass{article}
\begin{document}

\section{Cultura Digital}
Ante el cambio en la producción y distribución de conocimientos y contenidos, las instituciones culturales han tenido que intensificar el desarrollo de competencias digitales en la población, para así distribuir, circular y beneficiarse con las producciones culturales en formatos digitales.

La misión de la cultura digital es generar a través del espacio físico y de plataformas virtuales, programas enfocados en la apropiación y uso creativo y crítico de la tecnologías digitales como herramientas de producción y transformación cultural.

La generación educada en este inicio de siglo XXI
también es audiovisual, pero lo que la caracteriza, principalmente, es que emergen ya en el
interior de una cultura digital, es una generación que llegará a la mayoría de edad "bañada en bits".Se subraya la importancia de estudiar la cultura de esta generación, las maneras en que se relacionan, ya que es en estos procesos donde se pueden adivinar los cambios en la sociedad, las nuevas concepciones del trabajo y las ideologías del futuro.

Nuestro objetivo ha ido precisamente en la línea de cubrir el déficit de investigación que hay
sobre cultura del ocio juvenil vinculado a las nuevas tecnologías, justamente cuando estas
prácticas alcanzan una importancia cada vez mayor, no sólo por el perfeccionamiento de las
tecnologías, ni por el incremento de su uso, sino precisamente por su papel en las relaciones
de consumo. Las definiciones de consumo cultural que no tengan en cuenta el uso de las
nuevas tecnologías perderán rápidamente la capacidad de definir lo que podrían
ser aceptables y asumibles por parte de los mismos adolescentes. 

En el proceso de consumo es crear identidad de nuevas tecnologías de información y comunicación, que efectúan los y las adolescentes en los espacios de ocio, así es posible reconocer la creación de una nueva cultura digital. Ésta se puede observar a través de las prácticas específicas que se producen y que van mucho más allá del simple uso de la conexión.

\subsubsection{En la educación}
 La transición de sistemas cerrados a abiertos y de arquitecturas centralizadas a
 distribuidas facilita el fortalecimiento del aprendizaje en las que se prima la iniciativa del estudiante y sus capacidades
 creativas e innovadoras. Así, los estudiantes en estos nuevos modelos deben actuar cada vez más como socios y pares del profesor en la construcción de conocimiento como una estrategia de aprendizaje. Los estudiantes han de participar activamente en el proceso de aprendizaje, y colaborar tanto entre ellos como con los profesores trabajando tanto individualmente como en equipo. 
 
  Las redes de interés, de alcance global y donde se relacionan con otras personas de intereses similares, independientemente de su localización geográfica, es donde se desarrollan especialmente las capacidades creativas y proporcionan un canal para ganar visibilidad y reputación entre sus pares. 
 En las redes de interés, surgen formas de participación que conforman un aprendizaje informal, al margen de las instituciones educativas, basado en la colaboración con otros usuarios y en el ensayo y error, la exploración y el bricolaje.
 Por tanto, los jóvenes adquieren, principalmente, sus competencias digitales y habilidades tecnológicas en estos espacios digitales informales y su actividad es eminentemente social y apasionada. A diferencia del aula, los jóvenes prefieren los espacios digitales por la autonomía y libertad que les proporciona, y porque el estatus y la autoridad vienen determinados por sus habilidades y no por una jerarquía preestablecida.

\end{document}