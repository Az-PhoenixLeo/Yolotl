\section{Cultura Digital}

La cultura digital son todas aquellas actividades y expresiones humanas en medios digitales. La misión de la cultura digital es generar a través del espacio físico y de plataformas virtuales, programas enfocados al uso creativo y crítico de la tecnologías digitales como herramientas de producción y transformación cultural \cite{vid08}. El objetivo ha ido precisamente cubrir el déficit de investigación que hay sobre cultura del ocio juvenil vinculado a las nuevas tecnologías.
\\[1pt] 

Como ventaja de la cultura digital, si un producto de consumo que tiene en cuenta el uso de las nuevas tecnologías adquirirá la capacidad ser más aceptable en la sociedad. En el proceso de consumir, una persona crea identidad en las nuevas tecnologías de información y comunicación. Como ejemplo, los adolescentes en los espacios de ocio digitales constantemente aportan diversas actividades a la cultura digital. 
\\[1pt]

Y estas actividades digitales se pueden observar que van mucho más allá del simple uso de los medios y la conexión. La generación educada en este inicio de siglo XXI es audiovisual. Por tanto se considera la importancia de estudiar la cultura de esta generación y las maneras en que se relacionan, ya que es en estos procesos donde se pueden adivinar los cambios en la sociedad, las nuevas concepciones del trabajo y las ideologías del futuro.
\\[1pt]

\subsection{Educación digital}

La educación digital por tanto se es toda actividad lúdica presentada en un medio digital. En donde los estudiantes en estos nuevos modelos actúan cada vez más como socios y pares del profesor en la construcción de conocimiento como una estrategia de aprendizaje. 
\\[1pt] 

Como ventajas tenemos que la transición de sistemas de aprendizaje cerrados a abiertos facilita el fortalecimiento del aprendizaje y la iniciativa del estudiante y sus capacidades creativas e innovadoras. Pues las redes de interés, de alcance global y donde se relacionan con otras personas de intereses similares, independientemente de su localización geográfica, es donde se desarrollan especialmente las capacidades creativas y proporcionan un canal para ganar visibilidad y reputación entre sus pares. En las redes de interés, surgen formas de participación que conforman un aprendizaje informal, al margen de las instituciones educativas, basado en la colaboración con otros usuarios, el ensayo, error y la exploración.
\\[1pt]

Pero tambien consideremos que existe la desventaja de que como los jóvenes adquieren, sus competencias y habilidades tecnológicas en estos espacios informales donde su actividad es social y apasionada. No existe un control o establecimiento de reglas. A diferencia del aula, los jóvenes prefieren los espacios digitales por la autonomía y libertad que les proporciona, y porque el estatus y la autoridad vienen determinados por sus habilidades y no por una jerarquía preestablecida\cite{vid09}. Pero aun así existe desventaja el hecho de que no exista una selección de información a la que acceden y que también esta puede ser errónea. 
\\[1pt]


