\section{Cultura Digital}
La misión de la cultura digital es generar a través del espacio físico y de plataformas virtuales, programas enfocados al uso creativo y crítico de la tecnologías digitales como herramientas de producción y transformación cultural \cite{vid08}. El objetivo ha ido precisamente cubrir el déficit de investigación que hay sobre cultura del ocio juvenil vinculado a las nuevas tecnologías, justamente cuando estas prácticas alcanzan una importancia cada vez mayor, no sólo por el perfeccionamiento de las tecnologías, ni por el incremento de su uso, sino precisamente por su papel en las relaciones de consumo.
\\[1pt] 

Las definiciones de consumo cultural que no tengan en cuenta el uso de las nuevas tecnologías perderán rápidamente la capacidad de definir lo que podrían ser aceptables y asumibles por parte de la sociedad. En el proceso de consumo es crear identidad de nuevas tecnologías de información y comunicación, que efectúan los y las adolescentes en los espacios de ocio, así es posible reconocer la creación de una nueva cultura digital. Ésta se puede observar a través de las prácticas específicas que se producen y que van mucho más allá del simple uso de la conexión.
\\[1pt]

La generación educada en este inicio de siglo XXI es audiovisual, lo que la caracteriza es que emergen ya en el interior de una cultura digital, es una generación que llegará a la mayoría de edad "bañada en bits".Se subraya la importancia de estudiar la cultura de esta generación, las maneras en que se relacionan, ya que es en estos procesos donde se pueden adivinar los cambios en la sociedad, las nuevas concepciones del trabajo y las ideologías del futuro.
\\[1pt]

\subsection{Educación digital}
Los estudiantes en estos nuevos modelos actúan cada vez más como socios y pares del profesor en la construcción de conocimiento como una estrategia de aprendizaje. Los estudiantes han de participar activamente en el proceso de aprendizaje, y colaborar tanto entre ellos como con los profesores trabajando tanto individualmente como en equipo. La transición de sistemas cerrados a abiertos y de arquitecturas centralizadas a distribuidas facilita el fortalecimiento del aprendizaje en las que se prima la iniciativa del estudiante y sus capacidades creativas e innovadoras.
\\[1pt] 

Las redes de interés, de alcance global y donde se relacionan con otras personas de intereses similares, independientemente de su localización geográfica, es donde se desarrollan especialmente las capacidades creativas y proporcionan un canal para ganar visibilidad y reputación entre sus pares. En las redes de interés, surgen formas de participación que conforman un aprendizaje informal, al margen de las instituciones educativas, basado en la colaboración con otros usuarios y en el ensayo y error, la exploración y el bricolaje. 
\\[1pt]

Por tanto, los jóvenes adquieren, sus competencias y habilidades tecnológicas en estos espacios informales donde su actividad es social y apasionada. A diferencia del aula, los jóvenes prefieren los espacios digitales por la autonomía y libertad que les proporciona, y porque el estatus y la autoridad vienen determinados por sus habilidades y no por una jerarquía preestablecida\cite{vid09}.
\\[1pt]


