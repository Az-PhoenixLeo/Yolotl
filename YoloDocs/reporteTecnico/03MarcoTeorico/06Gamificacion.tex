\section{Gamificación}
La gamificación es el uso de las mecánicas de juego en entornos ajenos al juego, según el término anglosajon definido por Sebastian Deterding (Diseñador/investigador del diseño de juego para el florecimiento humano) \cite{gameDef}. 
\\[1pt]

\subsection{Mecánicas o reglas}\cite{gameficacion}
Son las normas de funcionamiento que permiten se adquiera un compromiso.
\\[1pt]

\begin{itemize}
	\item Colección: Logros y recompensas.
	\item Puntos: Para motivación y conteo de realizar una tarea.
	\item Ranking: Clasificación o comparación entre participantes.
	\item Nivel: Reflejan el progreso.
	\item Progresión: Consiste en completar el 100\% de la actividad encomendada.	
\end{itemize}

\subsection{Dinámicas de juego}\cite{gameficacion}
Motivan y despiertan el interés de realizar una actividad.
\\[1pt]

\begin{itemize}
	\item Recompensa: Premio por realizar algo.
	\item Competición: Deseo de estar en una determinada posición o grado.
	\item Cooperativismo: Otra forma de competir pero en un grupo con un mismo fin.
	\item Solidaridad: Se fomenta la ayuda entre compañeros y de manera altruista.
\end{itemize}

\subsection{Componentes}\cite{gameficacion}
\begin{itemize}
	\item Logros: Visualizan el alcance de un objetivo.
	\item Avatares: Representación gráfica del usuario.
	\item Medallas: Insignia o distintivo.
	\item Desbloqueo: Permiten avanzar en las actividades.
	\item Regalos: Un presente por la realización correcta de un reto.
\end{itemize}

\subsection{Tipos de jugadores}\cite{gameficacion}
\begin{itemize}
	\item Triunfador: Su finalidad es la consecución de logros y retos.
	\item Social: Le encanta interactuar y socializarse con el resto de compañeros.
	\item Explorador: Tiene tendencia a descubrir aquello desconocido.
	\item Competidor: Su finalidad es demostrar su superioridad frente a los demás.
\end{itemize}

\subsection{Proceso}\cite{gameficacion}
\begin{itemize}
	\item Viabilidad: Determinar si el contenido que se quiere enseñar es jugable.
	\item Objetivos: Definir los objetivos.
	\item Motivación: Valorar la predisposición y el perfil de jugadores.
	\item Implementación: Relación entre el juego y contenido enseñar.
	\item Resultados: Evaluación de la actividad.
\end{itemize}

\subsection{Finalidad}\cite{gameficacion}
\begin{itemize}
	\item Fidelización: Establecer un vínculo del contenido con el jugador.
	\item Motivación: Herramienta contra el aburrimiento del contenido.
	\item Optimización: Recompensar al jugador en aquellas tareas en las que no tiene¿ previsto ningún incentivo.
\end{itemize}