\documentclass[11pt,letterpaper]{article}
\usepackage[utf8]{inputenc}
\usepackage[spanish]{babel}
\usepackage{graphicx}
\usepackage{subfigure}
\usepackage[top=1in, bottom=1in, right=1in, left=1in]{geometry}

\begin{document}
	\author{Hernández Bautista Yasmine Pilar, Márquez 		Hernández Karla Rocío}
	\title{Yolotl}
	\maketitle
	\tableofcontents
	\section{Cinemática 1. Afueras de Tenochtitlan. exterior / tarde.}
	
\textsc{Personajes}:
\begin{itemize}
	\item Quetzalcóatl (Forma human).
	\item Xólotl (Forma human).
\end{itemize}
\textit{El viento sopla, Quetzalcóatl (en su forma humana) está sentado, viendo en dirección de la ciudad. La ciudad Tenochtitlan se ve parcialmente destruida, de ella sale humo. Entra Xólotl (En su forma humana)}.
\begin{center}
\textsc{\underline{Xólotl}}
\par
De todos los lugares y personas posibles, jamás creí que te encontraría precisamente a ti en este lugar.
\par
\textit{(Quetzalcóatl sigue observando la ciudad en silencio)}
\par
Así es como serán las cosas.
\par 
\textit{(Quetzalcóatl continua en silencio, Xólotl se molesta)}
\par
¿Guardaras silencio para siempre? ¿Ese es el modo en que me castigaras por lo que he hecho?
\\
\par
\textsc{\underline{Quetzalcóatll}}
\par
La caída del quinto sol. ¿Quién lo hubiera creído? Todo cuanto conocimos está por cambiar o desaparecer.
\par
\par
\textsc{\underline{Xólotl}}
\\
\par
Jamás debiste haberte ido. Las cosas habrían sido diferentes si nos hubieras guiado.
\\
\par
\textsc{\underline{Quetzalcóatll}}
\par
Te equivocas. Las cosas son como tenían que ser. Eres el más fuerte ahora. Dime… ¿Valió la pena? El duro viaje, las batallas y el sacrificio de los demás ¿Todo eso te dio lo que tanto anhelabas?
\\
\par
\textsc{\underline{Xólotl}}
\\
\par
\textit{(Guarda silencio por un momento)}
\\
\par
No …
\end{center}

\section{Cinemática 2. Mercado de Centla. Exterior /día.}
\textsc{Personajes}:
\begin{itemize}
	\item Malinalli Tenelpan.
	\item Xólotl (Forma xoloitzcuintle).
\end{itemize}

\textit{Malinalli(15 años) está parada cuando Xólotl le roba un objeto.}
\begin{center}
\textsc{\underline{malinalli}}
\\
\par
Mi amo me va a matar. El xoloitzcuintle ha tomado el mensaje que el amo le enviaba a su amada. Tengo que alcanzarlo y recuperar el mensaje. (\textit{Sale corriendo}).
\end{center}

\section{Cinemática 3. Selva a las afueras de Centla. Exterior /día.}
\textsc{Personajes}:
\begin{itemize}
	\item Malinalli Tenelpan.
	\item Xólotl (Forma xoloitzcuintle).
\end{itemize}
\textit{Después de buscar a Xólotl por la selva, Malinalli cae de una colina.}
\begin{center}
	\textsc{\underline{malinalli}}
	\\	
	\par
Eso ha dolido bastante.
(Escucha ruido proveniente de los arbustos, trata de reincorporase. Xólotl sale de los arbustos. Malinalli trata de retroceder, pero se da cuenta de que está atrapada. Xólotl se acerca a ella. Malinalli se queda quieta, cuando el jaguar está muy cerca lo toca en el rostro viéndolo a los ojos tratando de controlar su miedo. Xólotl la mira y se transforma en un xoloitzcuintle)
\\	
	\par
\textsc{\underline{Xólotl}}
\\	
	\par
Hola, Malinalli. Llevo casi una eternidad buscando a alguien como tú.
(Malinalli retrocede, tratando de protegerse)
Tranquila no pienso lastimarte.
(Malinalli sigue tratando de escapar, Xólotl la congela para que lo escuche usando su poder para acercarla a él).
Si quisiera asesinarte ya lo hubiera hecho.
\\	
	\par
\textsc{\underline{malinalli}}
\\	
	\par
Suéltame. Déjame ir. No tengo nada que pueda serte útil.
\\	
	\par
\textsc{\underline{Xólotl}}
\\	
	\par
Todo lo contrario. Tienes exactamente lo que necesito. He oído muchas cosas de ti. Malinalli Tenelpan, la joven de origen noble que ahora es esclava. Al igual que tu pertenecí a lo más alto de mi mundo.
\\	
	\par
\textsc{\underline{malinalli}}
\\	
	\par
¿Qué eres?
\\	
	\par
\textsc{\underline{Xólotl}}
\\	
	\par
Soy Xólotl. Un dios que vino a responder a tus plegarias.
\\	
	\par
\textsc{\underline{malinalli}}
\\	
	\par
¿Porqué un dios atendería las plegarias de una insignificante esclava?
\\	
	\par
\textsc{\underline{Xólotl}}
\\	
	\par
Porque la esclava tiene lo que muchos hombres nobles y de alta cuna carecen: Valentia. 
\\	
	\par
\textsc{\underline{malinalli}}
\\	
	\par
Agradezco sus palabras, pero me temo que se equivoca. Soy solo una esclava, nada más.
\\	
	\par
\textsc{\underline{Xólotl}}
\\	
	\par
El mundo espiritual ha sido tomado por la tiranía de los cuatro hermanos. Toda divinidad que no se acate a sus deseos ha sido destruida o exiliada. El caos gobierna a los dioses. Mi objetivo es liberarlos. Pero no tengo el poder suficiente como para hacerle frente a gobernante supremo de los dioses: Tezcatlipoca.
\\	
	\par
\textsc{\underline{malinalli}}
\\	
	\par
Muy noble de su parte. Sin embrago, no puedo ver cómo es que mi presencia se relaciona con tan gran y noble empresa.
\\	
	\par
\textsc{\underline{Xólotl}}
\\	
	\par
Necesito que alguien baje conmigo al Mictlán, me ayude a reclamar el poder de los guardianes del Mictlán y derrote a Mictlantecuhtli.
\\	
	\par
\textsc{\underline{malinalli}}
\\	
	\par
Una misión digna de un guerrero o un sacerdote no de una esclava.
\\	
	\par
\textsc{\underline{Xólotl}}
\\	
	\par
Ciertamente pero ningún guerrero o sacerdote emprendería una batalla contra los dioses que juraron servir. Los humanos sirven fielmente a sus dioses, hacen guerras por ellos, viven y mueren a su nombre, pero los dioses que gobiernan los trece cielos no miran por los desprotegidos, no les importan los mortales. He visto a los mortales de cerca, he visto sus ciudades nacer y caer. He visto su sufrimiento. Por eso puedo decirte que incluso un esclavo puede cambiar el rumbo de la historia si tiene la motivación adecuada.
(Libera a Malinalli de su poder)
Extrañas a tu padre ¿No es así? Con el Mictlán en mi poder, traerlo de vuelta no sería ningún problema. Con la fuerza de tu padre y sus antiguos ejércitos recuperar tu hogar sería una cuestión de días.  
\\	
	\par
\textsc{\underline{malinalli}}
\\	
	\par
¿Es eso posible? ¿Revivir a alguien es posible?
\\	
	\par
\textsc{\underline{Xólotl}}
\\	
	\par
Con mi poder actual no, pero con el poder de Mictlantecuhtli y los guardianes del inframundo, podre hacer eso y más. ¿Significa que tenemos un trato?
\\	
	\par
\textsc{\underline{malinalli}}
\\	
	\par
¿Cómo destruiré a los Dioses si no soy un guerrero ni un dios?
\textsc{\underline{Xólotl}}
(Invoca la caracola) Con esto. No lo subestimes, el sonido que genera es capaz de destruir cualquier ente que este hecho Tonalli. Una vez lo toque no habrá vuelta atrás. Aquí inicia tu viaje y tu oportunidad de cambiar la historia.
\end{center}

\section{Cinemática 4. A las orillas del rio Apanohuacalhuia en el Itzcuintlan. Exterior /noche.}
 \textsc{Personajes}:
\begin{itemize}
	\item Malinalli Tenelpan.
	\item Xólotl (Forma xoloitzcuintle).
\end{itemize}
\textit{Malinalli y Xólotl entran caminando. Malinalli está sorprendida por todo lo que está viendo.}

\begin{center}
XÓLOTL
\\
\par
Bienvenida al Mictlán, el lugar donde los muertos van, como debes de saber el Mictlán se divide en nueve niveles. Itzcuintlan es el primero de ellos, que lugar más acogedor ¿No crees? Todas estas almas están aquí para cruzar el lago. Si tu fueras como cualquiera de ellos lo siguiente que tendrías que hacer sería conseguiré un xoloitzcuintle para nadar a su lado, aunque no se hayas sido lo suficientemente digna como para que uno te ayude.
\\
\par
MALINALLI
\\
\par
Entonces ¿Nadaras junto a mí? ¿O debería empezar a buscar un xoloitzcuintle?
\\
\par
XÓLOTL
\\
\par
No estás muerta, por mucho que lo intentaras los xoloitzcuintles no te harían caso porque no pueden verte. Tú puedes verlos a ellos y tocarlos, pero ellos no pueden verte a ti, trata de no perturbarlos a la bestia que custodia el lago se molestara de verdad.
\\
\par
MALINALLI
\\
\par
Si no puedo tocarlos ¿Cómo voy a cruzar el río?
\\
\par
XÓLOTL
\\
\par
Para eso estoy aquí (se transforma en ajolote). Sube, tenemos que cruzar lo más rápido que se pueda antes de que alguien nos descubra. (Malinalli sube) Recuerda, nada de tocar a los xoloitzcuintles.
\end{center}

\section{Cinemática 5. a mitad del rio Apanohuacalhuia en el Itzcuintlan. exterior / noche.}
 \textsc{Personajes}:

\begin{itemize}
	\item Malinalli Tenelpan.
	\item Xólotl (Forma ajolote).
	\item Xochitónal.
\end{itemize}

\textit{Malinalli sobre el lomo de Xólotl nadan en el río a toda velocidad. Malinalli ve una silueta gigante que se acerca a ellos a través del agua. La silueta se sitúa cerca de ellos, entonces emerge de las profundidades Xochitónal. Por la corriente que genera al emerger Xochitónal, Malinalli y Xólotl retroceden.}

\begin{center}
\textsc{\underline{xochitónal}}
\\
\par
¿Quién osa perturbar la paz del rio Apanohuacalhuia? (Observa en todas direcciones hasta encontrar a Malinalli y a Xólotl) Este despreciable olor no puede ser más que de uno solo. Xólotl, ingrato cobarde, ¿Te has aburrido de correr y planeas entregarte a Tezcatlipoca? Si es así estas un poco debajo de lo que deberías.
\\
\par
XÓLOTL
\\
\par
Siempre has tenido un sentido del humor único, Xochitónal. Pero, deberías cuidar tus palabras, estás hablándole a un Dios, tú solo eres un capricho que creó Mictlantecuhtli.
\\
\par
\textsc{\underline{xochitónal}}
\\
\par
Ya que haz venido voluntariamente, no puedo dejarte ir. Tezcatlipoca me recompensara por tan buen premio. Solo lo lamento un poco por la mortal que te acompaña, devorare su alma.
\\
\par
XÓLOTL
\\
\par
Sera mejor que te prepares, esta no será una batalla como las que tuvimos en el camino.
\end{center}

\section{Cinemática 6. A mitad del rio Apanohuacalhuia en el Itzcuintlan.  Exterior / noche.}
 \textsc{Personajes}:

\begin{itemize}
	\item Malinalli Tenelpan.
	\item Xólotl (Forma ajolote).
	\item Xochitónal.
\end{itemize}

\textit{Xochitónal es derrotado. Ruge tratando de evitar su final, mueve su cuerpo con ferocidad. }

\begin{center}
\textsc{\underline{xochitónal}}
\\
\par
Derrotado por alguien tan desagradable… Avanza mientras puedas, los siguientes guardianes no serán tan benevolentes. (Trata de sumergirse en el agua. El movimiento de su cola tira a Malinalli de Xólotl, dejándola incosciente).
\\
\par
XÓLOTL
\\
\par
Malinalli… Malinalli
\end{center}

\section{Cinemática 7. palacio de Olula. interior/ día.}
\textsc{Personajes}:
\begin{itemize}
 \item Malinalli Tenelpan (de cinco años aproximadamente).
	\item Tenépal
\end{itemize}

\textit{Malinalli está durmiendo en el suelo al lado de unos códices. Su padre entra.} 

\begin{center}
\textsc{\underline{Tenépal }}
\\
\par
Malinalli. Malinalli. Despierta. 
\\
\par
MALINALLI
\\
\par
Padre. ¡Has vuelto! (Abraza a su padre, está la carga) ¿Cómo te ha ido? ¿Tenochtitlan es tan grande como dicen?
\\
\par 
\textsc{\underline{Tenépal }}
\\
\par
Paciencia, mi pequeña. Una pregunta a la vez. Más que Tenochtitlan, lo que me ha llamado la atención han sido todas las ciudades que vi en mi camino. (Camina) El imperio es extenso, pero tanto territorio dividido y sin unificar puede ser algo peligroso.
\\
\par
MALINALLI
\\
\par
(Alarmada) ¿Peligroso? ¿Significa que algo malo te pasará?
\\
\par
\textsc{\underline{Tenépal }}
\\
\par
Nada me va a pasar, pequeña. El imperio es extenso y algún día será unificado. Tienes que prepárate para ello. Tendrás un papel importante cuando ese momento llegue.
\\
\par
MALINALLI
\\
\par
¿Cómo podre jugar un papel importante si soy solo una mujer? Madre dice que mi único papel será casarme y darle más hijos al imperio.
\\
\par
\textsc{\underline{Tenépal }}
\\
\par
Mi amada niña. Tu papel será más que eso. (Baja a Malinalli)
\\
\par
MALINALLI
\\
\par
¿Cómo puedes estar tan seguro de eso?
\\
\par
\textsc{\underline{Tenépal }}
\\
\par
Un padre siempre lo sabe. Malinalli, eres diferente de las demás chicas. (Se arrodilla ante su hija y la toma de las manos) Eres más inteligente y brillante de lo que crees. Prométeme que jamás permitirás que los demás subestimen los que estas destinada a ser. (Le pone el broche de flor en la cabeza a Malinalli) En tus manos está el poder de crear el ombligo de la Luna.
\\
\par
MALINALLI
\\
\par
Te lo prometo padre. 
\end{center}

\section{Cinematica 8. a las orillas del rio Apanohuacalhuia en el Itzcuintlan. Exterior /noche.}
 \textsc{Personajes}:
 \begin{itemize}
 	\item Malinalli Tenelpan (edad actual).
	\item Xólotl.
\textit{Malinalli está en el suelo a la orilla del lago, Xólotl está a su lado tratando de reanimarla con sus patas delanteras.}
 \end{itemize}
 
\begin{center}
MALINALLI
\\
\par
(Toce)
\\
\par
XÓLOTL
\\
\par
Despiertas. Por un momento creí que morirías. La parte positiva de si morías es que tu alma ya no tendría que cruzar el río y te habrías ahorrado un paso del viaje. Aunque morir por ahogamiento te habría llevado al cielo dominado por Tláloc.
\\
\par
MALINALLI
\\
\par
¿Qué ocurrió? (trata de levantarse)
\\
\par
XÓLOTL
\\
\par
Tómatelo con calma, casi te ahogas. Xochitónal te derribó. Por suerte evite que murieras. Antes de que te lo preguntes, lograste acabar con él. Nunca había visto a ningún humano tener tanta afinidad al tonalli de un Dios. 
\\
\par
MALINALLI
\\
\par
Si puedes transformarte en lo que sea ¿Por qué no te transformaste en algo que volara y evitábamos el río? fácilmente pudimos haber ido y atacar por aire a…
\\
\par
XÓLOTL
\\
\par
No es tan simple, la energía del Mictlán me impide transformarme en algo que no sea propio de cada nivel del Mictlán. ¿Haz visto alguna ave en esta zona? No, es por eso que no puedo hacerlo. Si tuviera más poder la energía del Mictlán no me afectaría. Suficiente charla. Tenemos que irnos. Destruir a Xochitónal alertará a cualquiera de los guardianes y Mictlantecuhtli enviará a alguien a investigar en cualquier momento.

\end{center}

 
 \section{Cinemática 9. Castillo de Mictlantecuhtli. Interior /noche.}
  \textsc{Personajes}:
  \begin{itemize}
  \item Mictlantecuhtli.
	\item Mictecacihuatl.
	\item Tepeyollotl.
	\item Mictlecayotl.
	\item Itzpapalotl.
	\item Itztlacoliuhqui.
	\item Nexoxcho.
	\item Tlazoltéolt.
  \end{itemize}
  
  \textit{La sala del trono en el castillo de Mictlantecuhtli tiene una iluminación de pocas velas. Al fondo y centrados a la pared se encuentran los tronos de los dioses Mictlantecuhtli y Mictecacihuatl. Ambos están sentados. Al lado de los tronos hay unos candelabros hechos de huesos. Frente a ellos se encuentras las figuras proyectadas de los demás guardianes del Mictlán. Las proyecciones de los dioses son de color verde y aproximan la figura de los mismos sin dar un detalle exacto de como son.}
  
\begin{center}
	MICTLANTECUHTLI
	\\
\par
Hay un intruso en el Mictlán. Uno peligroso al parecer ya que Xochitónal ha caído. Hace unos instantes sentí como su tonalli desparecía.  
\\
\par
TEPEYOLLOTL
\\
\par
No me sorprende su caída. Xochitónal siempre fue más palabras que acción.
\\
\par
MICTECACIHUATL
\\
\par
Puede que no fuera el más eficiente de los guardianes, pero esta situación demanda una atención prioritaria. Es un ataque directo al orden y no vamos a tolerar ninguna otra rebelión a la jerarquía divina.
\\
\par
ITZPAPALOTL
\\
\par
Majestad, ¿Cómo esta tan segura que el invasor pertenece a nuestra jerarquía? A lo que a mí respecta podría ser un Dios de otro pueblo, no sería la primera vez que estaríamos en guerra con Dioses extranjeros.
\\
\par
MICTLANTECUHTLI
\\
\par
Más motivo aun para actuar con cautela y terminar este asunto de golpe. Tepeyóllotl, los invasores se dirigen a tu morada. Termina el asunto rápido y eficientemente. Pueden retirarse.
\end{center}

\section{Cinemática 10. Sobre la cima de una montaña en el Tepeme Monamictlan. exterior /día.}
 \textsc{Personajes}:
 \begin{itemize}
 \item Xólotl (Forma xoloitzcuintle).
	\item Malinalli Tenelpan.
 \end{itemize}
\textit{El viento sopla, se observa un día brillante y despejado.  Las montañas se ven cubiertas de pasto.}

\begin{center}
XÓLOTL
\\
\par
Hermosa vista ¿No crees? (Malinalli lo mira desconcertada) ¿No iras a decirme que te da miedo la altura o sí? Porque si es así lamento decirte que la vas a pasar muy mal. Este lugar es el hogar de un pequeño y problemático gato. Cuando se instauro la nueva jerarquía divina luego de la creación del quinto sol, se asignaron diferentes dioses para salvaguardar el Mictlán. Aunque todos cumplen con su deber, pocos son los que están aquí de manera voluntaria y por gusto.
\\
\par
MALINALLI
\\
\par
No lo entiendo. Si el Mictlan crea su propia energía para evitar que alguien haga trampa en el viaje de purificación ¿Para que tener guardianes que vigilen?
\\
\par
XÓLOTL
\\
\par
Es bastante sencillo. ¿Te haz preguntado porque no fui directo a los trece cielos, conquistarlo y con mi nueva posición obligar a los habitantes del Mictlán a jurarme lealtad? La respuesta es sencilla, todos los habitantes de los trece cielos son leales a su líder y lucharían por él. Esa lealtad no siempre fue así ni es gratuita. Hace mucho tiempo, hubo diferentes rebeliones. Muchas deidades trataron de tomar el control de todo. Entonces los cuatro hermanos tomaron una medida para proteger su reinado: Sellaron la entrada del mundo de los mortales a los trece cielos. Así que solo se puede entrar por el Mictlán. Y para asegurarse de ninguna amenaza les llegara sin aviso, nombraron a diferentes divinidades los guardianes de Mictlán. Cada uno más poderoso que el anterior. 
\\
\par
MALINALLI
\\
\par
Eso significaría que cada uno de ellos sería un sacrificio para proteger el orden. ¿Cómo se garantiza de que todos sean leales sin son conscientes de que son desechables?
\\
\par
XÓLOTL
\\
\par
Haciendo que uno de los guardianes los vigile todo el tiempo con la promesa de que si los problemas tocan la puerta el podrá huir para informarle a Tezcatlipoca lo ocurrido. 
\\
\par
MALINALLI
\\
\par
¿Y si ese guardían fuera el traidor?
\\
\par
XÓLOTL
\\
\par
No lo sería. Si hay algo peor que un traidor es un traidor que destroza la confianza de Tezcatlipoca. El dios negro no es benevolente con nadie que le traicione. Eso no lleva a nuestro pequeño gato. Pase lo que pase, debemos de acabar con el antes de que pueda advertirle a Tezcatlipoca que he vuelto.
\end{center}

\section{cinemática 11. Sobre la cima de una montaña en el Tepeme Monamictlan. exterior /día.}
 \textsc{Personajes}:
 \begin{itemize}
 	\item Xólotl (Forma xoloitzcuintle).
	\item Malinalli Tenelpan.
	\item Tepeyollotl (armadura de piedra).
 \end{itemize}
 
 \textit{Malinalli y Xolotl llegan a la cima de la montaña más alta. De pronto Se proyecta una sombra en el suelo y cae Tepeyollotl.}
 
\begin{center}
TEPEYOLLOTL
\\
\par
Mira lo que nos ha traído el viento: Un cobarde traicionero. ¿Te cansaste de huir? ¿O te haz cansado ya de la eternidad y al ser un cobarde no tienes lo que se necesita para acabar con tu vida así que vienes a buscar la de cualquiera que sea más valiente? 
\\
\par
XÓLOTL
\\
\par
Tepeyollotl, ¡Cuánto tiempo! ¿Sigues ronroneando sobre el regazo de Tezcatlipoca?
\\
\par
TEPEYOLLOTL
\\
\par
¡Vaya! El perro no está solo. Que amable de tu parte traerme la merienda como pago por acabar contigo. La curiosidad sobre como acabaste con Xochitónal me mata, me temo que me quedare con la duda (adopta pose de combate y ruge). 
\end{center}

\section{Cinemática 12. Sobre la cima de una montaña en el Tepeme Monamictlan. exterior /día.}
\textsc{\underline{ }}
\begin{itemize}
\item Xólotl (Forma xoloitzcuintle).
\item Malinalli Tenelpan.
\item Tepeyollotl (armadura de piedra).
\end{itemize}
\textit{La armadura de Tepeyollotl se rompe y éste retrocede. Malinalli trata de atacarlo nuevamente al ver una apertura en su defensa, pero es contrarrestada por Tepeyollotl.}

\begin{center}
TEPEYOLLOTL
\\
\par
Parece que el perro ha tenido suerte. Disfruta el efímero sabor de la victoria. En nuestro siguiente enfrentamiento no seré tan benevolente.
\\
\par
XÓLOTL
\\
\par
No lo dejes escapar, Malinalli
\\
\par
MALINALLI
\\
\par
(Trata de alcanzarlo, pero Tepeyollotl es más rápido ya logra escapar).
\\
\par
XÓLOTL
\\
\par
Maldición. Bueno, si dijo que habría siguiente encuentro significa que aún no va a huir al regazo de su amo y señor.
\\
\par
MALINALLI
\\
\par
Lo lamento, no le deje huir.
\\
\par
XÓLOTL
\\
\par
No te disculpes. Tenemos que continuar.
\end{center}

\section{Cinemática 13. Guarida de Itzpápalot. interior/tarde.}
 \textsc{Personajes}:
 \begin{itemize}
 \item Itzpapalotl.
 \item Mictecacihuatl.
 \end{itemize}
 \textit{La guarida de Izpapalotl es iluminada por los cristales que hay. La zona está llena de mariposas que se posan sobre las paredes. Itzpapalotl está sentada en el centro sobre un montón de piedra. La penumbra no deja ver su figura con claridad, lo único visible es una mariposa que se posa sobre su mano.}
 \begin{center}
 ITZPAPALOTL
 \\
\par
Los invasores han hecho que cambiemos nuestro plan. La reina ha sido muy clara, no podemos abandonar nuestro nivel para para ir a otro. Mi afecto no lo mermará la distancia. (A la mariposa) Vuela y llévale mi mensaje, haz que mis palabras den calor a su corazón.
\\
\par
MICTECACIHUATL
\\
\par
(Aparece su figura proyectada por uno de los cristales más cercanos a Itzpapalotl) Con penosas noticias vengo, me temo (Iztpapalotl se arrodilla en señal de respeto). Tepetollotl, abandono su puesto para salvar su vida y ahora espera a los invasores en otro nivel del Mictlán. Los invasores se dirigen a tu sitio. Será mejor que te prepares. Termina este juego rápido.
\\
\par
ITZPAPALOTL
\\
\par
Como ordene, Majestad. 
 \end{center}
 
\section{Cinemática 14. Entrada al Itztépetl. Interior /noche.}
 \textsc{Personajes}:
 \begin{itemize}
 \item Xólotl (Forma xoloitzcuintle).
 \item Malinalli Tenelpan.

 \end{itemize}
 \textit{Ambos personajes están dentro de una cueva.  La cueva esta iluminada por la luz de cristales. Se puede escuchar de vez en cuando el sonido del agua cayendo. }

\begin{center}
MALINALLI
\\
\par
Me gustaría preguntarle algo. ¿Porqué los otros guardianes del inframundo lo llaman cobarde?
\\
\par
XÓLOTL
\\
\par
Digamos que cuando Quetzalcóatl creo el quinto Sol, exigió a los demás dioses sacrificarse para dar forma al nuevo mundo e instaurar un nuevo orden. Algunos se sacrificaron sin dudarlo, otros nos opusimos a morir. No me mires así. Morir en el fuego porque Quetzalcóatl y Tezcatlipoca destruyeron los cuatro soles anteriores por jugar a tener la razón no sonaba como algo justo para mí. Nadie quiso escuchar mis razones, solo me llamaron cobarde y traidor. Desde entonces estoy exiliado.
\\
\par
MALINALLI
\\
\par
Lamento oír eso.
\\
\par
XÓLOTL
\\
\par
No sientas pena por mí. Encontré mi verdadero destino en el exilio. Que no te engañen los cristales. Estamos en la entrada de un lugar peligroso. Si se puede, me gustaría tener a la guardiana de ese sitio como aliada. 
\end{center}

\section{Cinemática 15. guarida de itzpápalotl. interior/noche.}
 \textsc{Personajes}:
 \begin{itemize}
 \item Xólotl (Forma xoloitzcuintle).
 \item Malinalli Tenelpan.
 \item Itzpapalotl
 \end{itemize}
 \textit{La guarida de Itzpapalotl ya ha sido descrita en la cinemática 11. Malinalli entra a la guarida. Se ve maravillada por la cantidad de mariposas que hay. Intenta tocar una y entonces todas vuelan hacia el centro de la guarida, ahí vuelan en círculos formando un torbellino, después vuelan en todas direcciones dejando ver a Itzpapalotl.}
 \begin{center}
 XÓLOTL
 \\
\par
Itzpapalotl. Luces diferente desde la última vez que te vi. Me alegra verte.
\\
\par
ITZPAPALOTL
\\
\par
No puedo decir lo mismo que tú. Esperaba no volver a verte. Debiste haberte mantenido lejos de todo esto. Será rápido. No te preocupes por la mortal, la devolveré a donde pertenece, hare que Nexoxcho altere su memoria y piense que todo esto fue un sueño.
\\
\par
XÓLOTL
\\
\par
Muy amable de tu parte. No deseo enfrentarme a ti. Comprendo tus motivos por los que te refugiaste en el Mictlán. Por eso deseo una alianza. El orden que creare no necesitara más peleas.
\\
\par
ITZPAPALOTL
\\
\par
(Las mariposas vuelven a volar alrededor de Itzpapalotl) ¿Refugiarme? No comprendes nada en lo absoluto. Despídete de ella, en unos instantes serás la sombra de un mal sueño.  (Itzpapalotl extiende sus alas, invoca su arma y adopta una pose de combate).
 \end{center}
 
 \section{Cinemática 15. guarida de itzpápalotl. interior/noche.}
  \textsc{Personajes}:
  \begin{itemize}
   \item Xólotl (Forma xoloitzcuintle).
	\item Malinalli Tenelpan.
	\item Itzpapalotl
  \end{itemize}
\textit{Itzpapalotl ha sido derrotada. Cae al suelo, mientras varias mariposas velan en todas direcciones desvaneciéndose en el aire. Malinalli se acerca a Itzpapalotl para reclamar su tonalli. Las mariposas que aún siguen con vida vuelan hacia Itzpapalotl.}
\begin{center}
XÓLOTL
\\
\par
Lamento que esto haya tenido que terminar así.
\\
\par
ITZPAPALOTL
\\
\par 
(Se pone de pie y vuela, elevándose unos pocos metros sin intensiones de huir. Malinalli y Xólotl adoptan una pose defensiva).
\end{center}

\section{Cinemática 17. Zona desértica montañosa. exterior/tarde.}
\textsc{Personajes}:
\begin{itemize}
\item Itzpapalotl (Su piel es café al igual que sus ojos).
\item Dioses del norte.
\end{itemize}
\textit{Los dioses del norte rodean a Itzpapalotl. Ésta lucha contra ellos demostrando tener mucho poder.}
\begin{center}
ITZPAPALOTL
\\
\par
Solía creer que nadie podía derrotarme. ¿Cómo podrían? Era la diosa más poderosa de todas. El poder trajo consigo no solo una autoconfianza formidable, también vino con arrogancia. (La escena se oscurece. Itzpapalotl está en el suelo derrotada. Lo dioses del norte llevan antorchas en sus manos y las arrojan sobre ella). Y a veces tienes que poner los pies sobre la tierra de la forma más cruel (Itzpapalotl arde con el fuego, el fuego se disipa y de las cenizas se forma Itzpapalotl ahora con la piel totalmente blanca y los ojos negros). 
\end{center}

\section{Cinemática 18. Castillo de Mictlantecuhtli. Interior /noche.}
\textsc{Personajes}:
\begin{itemize}
	\item Itzpapalotl.
	\item Mictecacihuatl.
\end{itemize}

\textit{Mictecacihuatl está sentada en su trono. Itzpapalotl se arrodilla ante ella. Mictecacihuatl hace ademanes de tomar el juramento de Itzpapalotl como guardiana del Mictlán.}
\begin{center}
ITZPAPALOTL
\\
\par
Mi arrogancia nos llevó a perder una guerra contra los dioses del Norte. Era mi deber proteger las líneas del norte y falle. El Mictlán se convirtió en un nuevo comienzo, una oportunidad de enmendar mis errores.
\end{center}

\section{Cinemática 19. Guarida de Itzpápalotl. Interior /noche.}
\textsc{Personajes}:
\begin{itemize}
\item Itzpapalotl.
\item Itztlacoliuhqui. 
\end{itemize}
\textit{Itzpapalotl está sentada en su lugar de siempre, está perdida en sus pensamientos. Itztlacoliuhqui entra, la mira por unos instantes y sigue su camino.}
\begin{center}
ITZPAPALOTL
\\
\par
Así apareció él. En el momento menos esperado… de la manera menos esperada. Ambos veíamos al Mictlán como un modo de redención por nuestros errores del pasado. Y de pronto, la arrogancia que había en mí era solo una sombra del pasado. Había encontrado paz. Lo errores del pasado pueden perdonarse si trabajas duro para no repetirlos. (Se mostrarán diferentes interacciones entre Itzpapalotl e Itztlacoliuhqui: ambos sentados platicando, sentados mientras ella esta recargada en él tomándolo de la mano).
\end{center}

\section{Cinemática 20. Guarida de Itzpápalotl. Interior /noche.}
\textsc{Personajes}:
\begin{itemize}
\item Xólotl (Forma xoloitzcuintle).
\item Malinalli Tenelpan.
\item Itzpapalotl.
\end{itemize}
\textit{Las mariposas se detienen en las manos de Itzpapalotl formando una única mariposa. Itzpapalotl pone la mariposa cerca de su pecho.}
\begin{center}
ITZPAPALOTL
\\
\par
Mi afecto jamás será mermado por la distancia ni por el tiempo. Vuela. vuela y dile que mis palabras le den calor a su corazón (La mariposa vuela. Itzpapalotl se desvanece, en su lugar queda una esfera de luz. Malinalli toca la caracola y la esfera de luz entra en ella).
\end{center}

\section{Cinematica 21. Guarida de Itztlacoliuhqui. Interior /tarde.}
\textsc{Personajes}:
\begin{itemize}
\item Itztlacoliuhqui. 
\end{itemize}
\textit{El cielo es un perpetuo atardecer, las nubes estas teñidas de rojo. En la lejanía se observa como llueven flechas. Itztlacoliuhqui está dentro de un templo. La iluminación de sus aposentos se debe a cristales idénticos a los que hay en la guarida de Itzpapalotl. La habitación está decorada con diferentes objetos de oro y joyas preciosas pero lo que destaca son los objetos de obsidiana. Itztlacoliuhqui está en el centro de la habitación, en su mano lleva unas flechas. Una mariposa entra a la habitación.}
\begin{center}
ITZTLACOLIUHQUI
\\
\par
(Extiende su mano) Dos mensajes seguidos. Inusual (Escucha el mensaje, tira las flechas. La mariposa desaparece).  Itzpapalotl…
\end{center}

\section{Cinemática 22. entrada al Cehuelóyan. ext/día.}
\textsc{Personajes}:
\begin{itemize}
\item Xólotl
\item Malinalli Tenelpan.
\end{itemize}
\textit{El lugar está cubierto de nieve. El cielo está despejado, pero sopla el viento helado. De fondo se pueden ver montañas cubiertas totalmente por hielo.}
\begin{center}
XÓLOTL
\\
\par
Finalmente llegamos al lugar donde nacen los vientos del norte, casi es la mitad de nuestro viaje.
\\
\par
MALINALLI
\\
\par
Lo que paso con Itzpapalotl
\\
\par
XÓLOTL
\\
\par
Será mejor que no pienses en ello. Me gustaría pensar que la guardiana del Cehuelóyan podría ser una aliada. A diferencia de mí, ella fue directamente a tratar de matar a Tezcatlipoca.
\\
\par
MALINALLI
\\
\par
Suena como una divinidad muy temeraria.
\\
\par
XÓLOTL
\\
\par
Una poderosa aliada, de no ser que ella es muy devota a Quetzalcóatl. Difícilmente podría con fiar en mí.
\\
\par
MALINALLI
\\
\par
Ahora que lo mencionas. Quetzalcóatl es un dios benévolo. ¿Porqué no interfirió a tu favor cuando te rehusaste a ser sacrificado? 
\\
\par
XÓLOTL
\\
\par
Malinalli, no siempre los dioses son tan amables y buenos con otros dioses que como lo serían con los humanos. Quetzalcóatl no es ese ente benevolente y virtuoso que muchos creen. 
\end{center}

\section{Cinemática 23. Entrada a la guarida de Mictlecayotl. ext/día.}
 \textsc{Personajes}:
 \begin{itemize}
 \item Xólotl
 \item Malinalli Tenelpan.
 \item Mictlecayotl.
 \end{itemize}
\textit{Xólotl y Malinalli llegan a la entrada de la guarida de Mictlecayotl. Es una pequeña plaza con pedestales de piedra que tienen un débil fuego encendido. De fondo se ve la entrada al palacio de Mictlecayotl. Sopla una ventisca de nieve, de la ventisca sale Mictlecayotl. Mictlecayotl golpea a Xólotl, dejándolo semi inconsciente.}
\begin{center}
MICTLECAYOTL
\\
\par
¡Que decepción! Creí que como habías vencido a Itzapapalotl tendrías mejores reflejos. (Camina en dirección a Xolotl para acabar con él).
\\
\par
MALINALLI
\\
\par
(Ataca a Mictlecayotl con la caracola. Mictlecayotl bloquea su ataque) Él no es de lo único de lo que deberías preocuparte.
\\
\par
MICTLECAYOTL
\\
\par
¿Una mortal? (ríe) ¿Qué tan miserable te haz vuelto como para depender de una mortal? (Invoca su arma) De acuerdo pequeña, muéstrame de que estas hecha pero no esperes un trato amable.
\end{center}
\section{Cinemática 23. entrada al Cehuelóyan. ext/día.}
\textsc{Personajes}:
\begin{itemize}
\item Xólotl
\item Malinalli Tenelpan.
\end{itemize}
\textit{Mictlecayotl ha sido derrotada. Tira su arma y trata de retroceder para resguardarse. Se tambalea de un lado a otro.}
\begin{center}
MICTLECAYOTL
\\
\par
Que lastimosa imagen debe de ser esta. Derrotada por una mortal… Solía observar a tu gente. Siempre sentí que ustedes eran interesantes. El tiempo no tiene sentido para nosotros. Para los mortales la vida es solo un suspiro. Eso los lleva a luchar, actuar con cautela y aferrarse a sobrevivir. Evitan los errores porque saben que su vida es frágil. Para los dioses nada de eso funciona, podemos equivocarnos cuanto queramos. El tiempo nos permite vivir de mil modos. Pero incluso la eternidad llega a su fin (Se eleva en el aire) Disfruta tu victoria, mortal. Tuya es la gloria, no de él (Se transforma en una esfera de luz y es absorbida por la caracola).
\\
\par
 MALINALLI
 \\
\par
(Corre hacia Xólotl) ¿Está bien?
\\
\par
XÓLOTL
\\
\par
Estoy bien. No te preocupes. ¿Tu estas bien? (Malinalli asiente) Tenemos que seguir.
\end{center}

\section{Cinemática 25. Palacio de Oluta. int/día.}
 \textsc{Personajes}:
 \begin{itemize}
 \item Malinalli(niña).
 \item Madre de Malinalli.
\item Guardias.
\item Sacerdote.
\item Curandero.
\item Padrastro de Malinalli.
 \end{itemize}
 \textit{Malinalli está sentada en el suelo de la habitación. Lee diferentes códices. Entra un grupo de guardias cargando el cuerpo del padre de Malinalli. Esta herido y sangra. Malinalli se sobresalta y trata de acercarse al grupo de guardias. Entra un sacerdote y un curandero. Los guardias colocan el cuerpo sobre la mesa de la habitación. El curandero trata de atender el sangrado del padre de Malinalli.}
\begin{center}
MALINALLI
\\
\par
¡Padre! ¡Padre! (Grita tratando de acercarse a la mesa. Un guardia la intercepta y se dispone a sacarla de la habitación). 
\\
\par
PADRE DE MALINALLI
\\
\par
Malinalli... Mi amada niña (extiende su brazo como pidiendo que acerquen a su hija).
\\
\par
SACERDOTE
\\
\par
Saca a la niña de aquí, no tiene porque ver esto.
\\
\par
MALINALLI
\\
\par
(Llama a su padre mientras el soldado la saca de la habitación. La pantalla se oscurece. Malinalli ahora está con su madre. El curandero entra).
\\
\par
CURANDERO.
\\
\par
Mi señora, lamento informarle que su esposo ha muerto.
\\
\par
MADRE DE MALINALLI
\\
\par|
Agradezco sus servicios. Retírese por favor (El curandero sale. Malinalli llora abrazad de su madre. La escena se oscurece. Malinalli está en el suelo con la peineta que le dio su padre en sus manos. Su madre habla con el Padrastro de Malinalli)
\\
\par
PADRASTRO DE MALINALLI
\\
\par
Como nuevo señor de Oluta, necesitó una esposa de alta cuna y buen linaje. Nada me traída más dicha ni honraría más la memoria del antiguo señor que el hecho de que la tome por esposa.
\\
\par
MADRE DE MALINALLI.
\\
\par
Mi corazón está afligido por la pérdida de mi esposo y solo el matrimonio podría calmar la pena que siento. Acepto su propuesta, mi señor. Que de esta unión nazca la paz que Oluta necesita.
\end{center}

\section{Cinemática 26. Guarida de Tepeyóllotl en el Teyollocualoyan. int/día.}
 \textsc{Personajes}:
 \begin{itemize}
 \item Tepeyollotl (Sin coraza).
\item Mictecacihuatl
 \end{itemize}
\textit{Tepeyollotl está acostado en el centro de la sala sobre telas finas de algodón. La decoración del lugar es ostentosa con diferentes objetos de oro. Al lado de Tepeyollotl hay un espejo de jade. La figura de   Mictecacihuatl se proyecta del espejo.}
\begin{center}
MICTLANTECUHTLI
\\
\par
Itzpapalotl y Mictlecayotl han caído. El invasor sigue avanzando. Era tu deber detenerlo, en lugar de eso estas dormitando.
\\
\par
TEPEYOLLOTL
\\
\par
Dilo directamente, me culpas por el avance de Xólotl.
\\
\par
MICTLANTECUHTLI
\\
\par
Es el deber de un guardián detener a su enemigo y en caso de que se vea superado por éste, debe de morir llevándose a su enemigo consigo. Huir no está permitido. Hemos perdido tres de los nueve guardianes y cada guardián caído es equivalente a más poder para Xólotl.
\\
\par
TEPEYOLLOTL
\\
\par
No creo que Xólotl sea el problema.
\\
\par
MICTLANTECUHTLI
\\
\par
Explícate.
\\
\par
TEPEYOLLOTL
\\
\par
Xólotl es acompañado por un mortal, es ella la que se enfrenta a los Dioses. Xólotl no tiene la fuerza ni la valentía para hacerlo, pero si tiene la mente para planearlo. Me pregunto si la niña mortal es solo valentía y fuerza o, por el contrario.
\\
\par
MICTLANTECUHTLI
\\
\par
Una mortal dispuesta a asesinar dioses. Jamás había oído algo como eso. ¿Cómo puede existir algo así?
\\
\par
 TEPEYOLLOTL
\\
\par
Una esclava mortal. Xólotl debió de haberle ofrecido algo. Los humanos son criaturas simples, rara vez se embarcan en peligrosas empresas si no van a obtener algo que amerite el esfuerzo. Me enfrentare a la mortal una vez más. Piensa en lo que te dije, será de utilidad si fallamos.
\end{center}

\section{Cinemática 27.Entrada al Pancuetlacalóyan. int/día.}
 \textsc{Personajes}:
 \begin{itemize}
 \item Malinalli Tenelpan.
\item Xólotl.
 \end{itemize}
\textit{Frente a Xólotl y Malinalli hay un pórtico de piedra. El suelo muestra la transición entre el clima nevado y el pasto verde del verano. Malinalli se acerca al pórtico con intención de cruzarlo.}
\begin{center}
MALINALLI
\\
\par
Antes dijo que Quetzalcóatl no era un dios virtuoso. ¿Porque tiene esa impresión de él?
\\
\par
XÓLOTL
\\
\par
La última vez que lo vi, su orgullo y su necedad le trajeron la destrucción a un imperio entero. Le advertí sobre los planes que tenía Tezcatlipoca para derrocarlos, pero no quiso escucharme. No pude hacer mucho por la ciudad y sus habitantes.
\\
\par
MALINALLI
\\
\par
Al parecer no todo en los cielos es como lo pitan los códices (Se pone cerca del pórtico).
\\
\par
XOLÓTL
\\
\par
Yo sería más cuidadoso al cruzar si fuera tú. Lo que está al otro lado del pórtico, es no es nada parecido a lo que hemos pasado (Se transforma en un ave). Sube.
\\
\par
MALINALLI
\\
\par
Siempre me pregunte como sería volar.
\end{center}

\section{Cinemática 28. Guarida de Tlazoltéolt en el Pancuetlacalóyan. int/día.}
 \textsc{Personajes}:
 \begin{itemize}
 \item Malinalli Tenelpan.
\item Xólotl.
\item Tlazoltéolt.
 \end{itemize}
\textit{Xólotl y Malinalli sobrevuela evitando un par de piedras. Malinalli le indica a Xólotl moverse evitando así que los golpee Tlazoltéolt que iba cayendo. Tlazoltéolt emprende el vuelo para perseguir a Malinalli y Xólotl lanzándoles energía de corrupción. Tlazoltéolt los alcanza, subiéndose en Xólotl y ve de frente a Malinalli.}
\begin{center}
TLAZOLTÉOLT
\\
\par
He visto la suficiente locura y corrupción como para saber lo que aqueja tu corazón. A veces, te preguntas si por las noches ella escucha tus gritos mientras duerme. Me temo que no. Aunque la herida sea grande debes de continuar. (Malinalli observa molesta a Tlazoltéolt y trata de derribarla consiguiendo ser ella la que cae de Xólotl. Debido a la inexistente gravedad Malinalli se va hacia arriba. Xólotl consigue atraparla). 
\\
\par
XÓLOTL
\\
\par
Malinalli. Lo que sea que te haya dicho ¡No es momento para perder la cabeza! Tlazoltéolt puede meterse en tu mente y ver tus recuerdos. No le des el placer de romperte.
\\
\par
TLAZOLTÉOLT
\\
\par
Como guardiana del Pancuetlacalóyan, yo, Tlazoltéolt limpiadora de impurezas, detendré el avance de los invasores y restaurare a los guardianes caídos.
\end{center}

\section{Cinemática 29. Guarida de Tlazoltéolt en el Pancuetlacalóyan. int/día.}
 \textsc{Personajes}:
 \begin{itemize}
 \item Malinalli Tenelpan.
\item Xólotl.
\item Tlazoltéolt.
 \end{itemize}
\textit{Tlazoltéolt es derrotada, La gravedad comienza a afectar nuevamente al nivel del inframundo. Ella cae. Malinalli y Xólotl se lanzan en picada para obtener el tonalli de Tlazoltéolt.}
\begin{center}
TLAZOLTÉOLT
\\
\par
Xólotl. La mirada de esa niña mortal y el color de su corazón. Estas desatando sobre la tierra de los mortales algo que no vas a poder controlar, nadie podra… puedo verlo, tan claro que eriza mi piel… La caída del quinto sol. (Tlazoltéolt desaparece y solo queda su energía espiritual. Malinalli toma la tonalli de Tlazoltéolt) 
\\
\par
XÓLOTL
\\
\par
(Mirando a Malinalli) ¿Te encuentras bien? Hace un momento parecía que habías perdido el control de ti misma.
\\
\par
MALINALLI
\\
\par
Estoy bien. Será mejor que continuemos.
\end{center}

\section{Cinematica 30. Palacio de Olula. int/día. }
 \textsc{Personajes}:
 \begin{itemize}
 \item Malinalli Tenelpan.
\item Madre de Malinalli.
\item Padrastro de Malinalli.
\item Guardias.
 \end{itemize}
\textit{La madre de Malinalli está sentada en el centro de la sala. Carga en sus brazos un bebé. El padrastro de Malinalli entra a la sala. Besa a su esposa y mira con amor al bebé.}
\begin{center}
PADRASTRO DE MALINALLI
\\
\par
Mi amada esposa en compañía de mi primogénito. He oído rumores de lo más bajos en la ciudad. Un gobernante de las provincias lejanas desea casar a su hijo con Malinalli.
\\
\par
MADRE DE MALINALLI
\\
\par
¿No serían esas buenas noticias?
\\
\par
PADRASTRO DE MALINALLI
\\
\par
Tal parece que muchos desean usar el buen nombre de tu antiguo esposo para armar ejércitos que se alcen contra el imperio. Mientras tu hija exista, existirá peligro para el imperio y para el futuro de nuestro hijo.
\\
\par
MADRE DE MALINALLI
\\
\par
Entonces está claro lo que debes de hacer. No la asesines, o los partidarios de su padre la convertirán en una mártir. A ellos les interesa esposar una niña de cuna noble, no una esclava. 
(La escena se desvanece. Malinalli se encuentra en el suelo leyendo varios códices. Dos guardias entran y la toman de los brazos. Malinalli sin comprender que ocurre trata de resistirse, pero un guardia la abofetea. Malinalli grita. Su madre entra a la habitación. Malinalli rompe el agarre de los guardias y corre a refugiarse tras su madre. Los guardias la toma nuevamente)
\\
\par
MALINALLI
\\
\par
Suéltenme. Madre, diles que se detengan.
\\
\par
MADRE DE MALINALLI
\\
\par
Yo no parí esclavos. Llévensela. El Señor de Centla recibirá gustoso el pago por las provisiones.
\\
\par
MALINALLI
\\
\par
¡Madre! ¡Diles que se detengan! ¡Diles que ha sido un error! ¡Madre! (Los guardias se llevan a Malinalli).
\end{center}

\section{Cinemática 31. Entrada a Temiminaloyan. ext/tarde.}
 \textsc{Personajes}:
 \begin{itemize}
 \item Xólotl.
\item Malinalli.
 \end{itemize}
\textit{Temiminaloyan, se muestras como una zona boscosa. El cielo es un perpetuo atardecer. Su aire es seco. Malinalli camina. Xólotl se detiene.}
\begin{center}
XÓLOTL
\\
\par
¿A que se refería Tlazoltéolt cuando dijo que algo aquejaba tu corazón?
\\
\par
MALINALLI
\\
\par
Al igual que tú también fue traicionada por aquellos que alguna vez ame. Pero todo eso dejara de importar una vez mi padre esté devuelta. Él recuperara nuestro hogar. Seremos nuevamente él y yo.
\\
\par
XÓLOTL
\\
\par
Pero primero debemos derrotar a los guardianes del inframundo. Nuestro siguiente objetivo es Itztlacoliuhqui. Vencerlo no será tarea sencilla, en especial después de que tomamos el tonalli de su esposa.
\\
\par
MALINALLI
\\
\par
¿Su esposa?
\\
\par
XÓLOTL
\\
\par
Itzpapalotl.
\\
\par
MALINALLI
\\
\par
Entonces la mariposa que ella envió era para él. Entonces está claro. Bastara con hacer que se reúnan nuevamente una vez obtengamos el tonalli de Itztlacoliuhqui.
\\
\par
XÓLOTL
\\
\par
No dejes que las victorias pasadas se te suban a la cabeza.
\end{center}


\section{Cinemática 32. Guarida de Itztlacoliuhqui. int/tarde.}
 \textsc{Personajes}:
 \begin{itemize}
 \item Xólotl.
 \item Malinalli.
 \item Itztlacoliuhqui.
 \end{itemize}
\textit{Itztlacoliuhqui está sentado en el centro de la habitación, cuando ve a Malinalli y a Xólotl entrar mueve las manos y les lanza flechas. Malinalli bloquea las flechas con el poder de la caracola. Itztlacoliuhqui salta de su asiento y aterriza donde Xólotl y Malinalli se encuentran rompiendo la defensa de Malinalli, seguido de esto lanza a Malinalli hasta el otro lado de la habitación y toma a Xólotl del cuello.}
\begin{center}
ITZTLACOLIUHQUI
\\
\par
Dame un motivo para no romperte el cuello en estos momentos.
\\
\par
XÓLOTL
\\
\par
Después de todo lo que he hecho, estoy completamente seguro que Tezcatlipoca va a quererme vivo para castigarme él mismo. No querrás quitarle el gusto ¿O sí?  (Itztlacoliuhqui invoca más flechas) De acuerdo, mal motivo. Deseo una alianza. Venga, Itztlacoliuhqui. Tú y yo somos bastante parecidos. Ambos nos rebelamos contra el resto al inicio del quinto sol. Las cicatrices que te hizo Tonatiuh, puedes regresárselas sin más.
\\
\par
ITZTLACOLIUHQUI
\\
\par
No soy quien tu recuerdas. Aquello que buscas no me interesa. Solo hay algo que tienes que me importa y no necesito una alianza para recuperarla. (Malinalli se levanta y con la energía de la caracola ataca a Itztlacoliuhqui. Itztlacoliuhqui suelta a Xólotl. Xólotl se pone atrás de Malinalli).
\\
\par
MALINALLI
\\
\par
Él no va a escucharte. Solo hay dos maneras de que se reencuentre con su amada y no pienso caer después de estar tan cerca de recuperar todo.
\\
\par
ITZTLACOLIUHQUI
\\
\par
(Se reincorpora del ataque) ¿Cómo te atreves a usar su energía? He cambiado de opinión. Primero te destruiré a ti y después a Xólotl (Itztlacoliuhqui empieza a aumentar su tamaño hasta volverse un monstruo de grandes proporciones. Con su cambio de forma su guarida se destruye).
\end{center}


\section{Cinemática 33. Guarida de Itztlacoliuhqui. int/tarde.}
 \textsc{Personajes}:
 \begin{itemize}
 \item Xólotl.
 \item Malinalli.
 \item Itztlacoliuhqui.
 \end{itemize}
\textit{Itztlacoliuhqui es derrotado por Malinalli y retorna a su tamaño normal. Xólotl se acerca a él.}
\begin{center}
ITZTLACOLIUHQUI
\\
\par
Siempre creí que lo que sentía por ti era desprecio. Ahora me doy cuenta de que es lástima. Vives culpando a todos, sin poder aceptar tus miedos. Crees que el poder va a llenar el vacío que está en tu corazón, me temo que el camino que estas siguiendo terminara siendo todo menos satisfactorio.
\\
\par
XÓLOTL
\\
\par
No busco satisfacción. Solo deseo salvarlos a todos.
\\
\par
ITZTLACOLIUHQUI
\\
\par
(Ríe) ¿Exactamente de que nos vas a salvar? ¿Cómo esperas salvar a todos si no puedes salvarte a ti mismo?
\\
\par
XÓLOTL
\\
\par
Creí que podríamos entendernos. Es imposible que alguien que solo se guía por el honor y el deber pueda comprender mis ideales.
\\
\par
ITZTLACOLIUHQUI
\\
\par
¿Qué son el honor y el deber si no puedes proteger a aquellos que amas? (Malinalli mira fijamente a Itztlacoliuhqui mientras desaparece dejando solamente su tonalli. Malinalli reclama el tonalli).
\end{center}

\section{Cinemática 34. Palacio de Oluta. int/día. }
 \textsc{Personajes}:
 \begin{itemize}
 \item Malinalli Tenelpan (cinco años).
\item Padre de Malinalli.
\item Guardias.
 \end{itemize}
\textit{Malinalli esta con su padre. Él le muestra diversos códices. De entre los códices saca un mapa.}
\begin{center}
MALINALLI
\\
\par
Mi padre era un gobernante honorable gentil que protegía a su gente del abuso del imperio. Él veía las cosas con otros ojos. Donde la mayoría veía problemas, él veía una posibilidad de hacer las cosas mejores para su gente. Él era consciente de una realidad que el imperio prefería ignorar: El imperio era extenso, pero fuera de su capital no era amado. (Se muestra al padre de Malinalli hablando con diferentes hombres de alta cuna) Utilizando su poder político, Padre comenzó a unificar el sur del imperio: su misión era formar el legendario ombligo de la luna, aquella nación contada en canciones por los adivinos. La nación que heredaría toda la gloria de las culturas anteriores a ella. (El padre de Malinalli camina por los pasillos cuando lo abordan varios guardias y lo apuñalan) Sin embrago, un ideal grande genera una gran sombra. Traicionado por uno de sus hombres, Padre fue mandado a asesinar por el Tlatoani. Con padre muerto, sus seguidores abandonaron su causa por temor a compartir el mismo destino que él. Padre murió muy pronto, sin que el destino le permitiera cumplir su sueño. El deber y el honor no pudieron protegerlo ni a mí. Voy a cambiar eso, le daré una segunda oportunidad.
\end{center}

\section{Cinemática 35. Tepeyollocualoyan. ext/día. }
 \textsc{Personajes}:
 \begin{itemize}
 \item Malinalli Tenelpan.
\item Xólotl.
 \end{itemize}
\textit{Xólotl y Malinalli caminan por el sendero del Tepeyollocualoyan. Ambos se ven un poco cansados. }
\begin{center}
XÓLOTL
\\
\par
Finalmente llegamos al segundo nivel custodiado por Tepeyollotl. Espero hayamos llegado lo suficientemente prematuros como para evitar que se comunique con Tezcatlipoca.
\\
\par
MALINALLI
\\
\par
Lo considero poco probable, si ya supiera de la invasión al Mictlán ya habría armado una defensa para terminar con esta campaña.
\\
\par
XÓLOTL
\\
\par
Probablemente, a no ser que ya haya dado el Mictlán por perdido y prefiera centrarse en la defensa de lo que aún le pertenece totalmente. Solo nos queda concentrarnos en los últimos tres guardianes restantes. Recuerda no abandonar el sendero.
\end{center}

\section{Cinemática 36. guarida de Tepeyóllotl en el Tepeyollocualoyan. ext/día. }
 \textsc{Personajes}:
 \begin{itemize}
 \item Malinalli Tenelpan.
 \item Xólotl.
 \item Tepeyollotl (Sin coraza).
 \end{itemize}
\textit{Tepeyollotl está acostado sobre telas finas en el centro del salón principal de su guarida. Observa a Malinalli y a Xólotl cuando entran como si los hubiera estado esperando.}
\begin{center}
TEPEYOLLOTL
\\
\par
He oído lo que has hecho en cada uno de los niveles del Mictlán a los que haz ido. Estoy sorprendido. Jamás espere que llegarías tan lejos como para volvernos a encontrar. Henos aquí.
\\
\par
XÓLOTL
\\
\par
No fue sencillo debo de admitirlo.
\\
\par
TEPEYOLLOTL
\\
\par
No hablaba contigo. Es con ella con quien hablo. Malinalli, no sé que es lo que Xólotl te prometió a cambio de su ayuda. Lo que sea que hay sido, los guardianes restantes podemos dártelo. 
\\
\par
MALINALLI.
\\
\par
¿Ese es el modo en que actúan nuestros Dioses? ¿Si no pueden controlar algo simplemente lo destruyen o buscan sobornarlo? ¿Qué diferencia existe entre ustedes y los mortales? Xólotl tiene la razón, es necesario depurar el cielo, crear un nuevo orden.
\\
\par
TEPEYOLLOTL
\\
\par
Entonces tendrás que irte con las manos vacías.
\end{center}

\section{Cinemática 37. guarida de Tepeyóllotl en el Tepeyollocualoyan. ext/día. }
 \textsc{Personajes}:
 \begin{itemize}
 \item Malinalli Tenelpan.
 \item Xólotl.
 \item Tepeyollotl (Sin coraza).
 \end{itemize}
\textit{Tepeyollotl es derrotado. Xólotl observa la derrota de su enemigo con dicha. Malinalli se aproxima a él para tomar el tonalli. Tepeyollotl empieza a desvanecerse mostrando como fragmentos de luz se desprenden de él.}
\begin{center}
TEPEYOLLOTL
\\
\par
Xólotl, no podría despreciarte más. Tus acciones nos condenaran a todos (Se convierte en tinalli).
\\
\par
XÓLOTL
\\
\par
Uno menos, faltan dos. 
\end{center}
 
\section{Cinemática 38. Palacio de Mictlantecuhtli. int/día.}
 \textsc{Personajes}:
 \begin{itemize}
 \item Mictlantecuhtli.
 \item Mictecacihuatl.

 \end{itemize}
\textit{Mictlantecuhtli está sentado en su trono. Mictecacihuatl entra a toda velocidad, se acerca a Mictlantecuhtli. Él la observa, sabe lo que ella está a punto de decir.}
\begin{center}
MICTECACIHUATL
\\
\par
Tepeyolotl ha caído. Lo único que nos separa de los invasores e Nexoxcho. Tepeyolotl pensaba que podía negociar con la niña mortal.
\\
\par
MICTLANTECUHTLI
\\
\par
Por lo que desaparecio sin avisarle a Tezcatlipoca sobre la situación. 
\\
\par
MICTECACIHUATL
\\
\par
Exactamente.
\\
\par
MICTLANTECUHTLI
\\
\par
Se sobrepreocupa, mi señora. Desde que Nexoxcho fue asignado como guardián del octavo nivel, nadie ha sido capaz de pasar su guardia. Se requiere mucha valentía para tratar de enfrentarnos. Sin embargo, se requiere mayor valentía para enfrentar los conflictos del alma. Xólotl y su niña mortal no pasaran los dominios de Nexoxcho.
\\
\par
MICTECACIHUATL
\\
\par
He escuchado a siete de los nueve guardianes decir lo mismo respecto a si mismos. Todos han caído. ¿Qué garantiza que Nexoxcho será diferente?
\\
\par
MICTLANTECUHTLI
\\
\par
Si tanto le preocupa su seguridad. La invito a irse a los trece cielos a informarle a Tezcatlipoca lo que ha ocurrido. No voy a entregar mi reino sin dar batalla.
\\
\par
MICTECACIHUATL
\\
\par
Le tomare la palabra, mi señor. Que la próxima vez que nos veamos sean con noticias de victoria.
\end{center}

\section{Cinemática 39. A orillas del río del Apanohualoyan. ext/noche.}
 \textsc{Personajes}:
 \begin{itemize}
 \item Xólotl.
 \item Malinalli.
 \end{itemize}
\textit{Ambos personajes están de pie sobre las orillas del río. El agua del río es negra y calmada, no muestra ninguna perturbación, aunque la agites con la mano. El cielo esta estrellado y despejado, a lo lejos se observa un bosque.}
\begin{center}
MALINALLI
\\
\par
Nuevamente tendremos que nadar.
\\
\par
XÓLOTL
\\
\par
Esa es la idea. Al menos en gran medida. Esta vez no será como el río Apanohuacalhuia. Este tiene su pequeño truco. No esperes encontrar almas errantes aquí. Todo aquel que no lo logra cruzar ayuda a teñir el río de negro. Apanohualoyan, jamás debe de ser cruzado por la superficie o terminaras varado sin la posibilidad de volver a la orilla. Es en sus profundidades que se encuentra la guarida de Nexoxcho. Espero que estés lista para nadar, no te preocupes por la respiración; el tonalli de Mictlecayotl lo hara por ti. Mantén los ojos bien abiertos.
(Xólotl salta al agua. Malinalli lo sigue)
\end{center}


\section{Cinemática 40. Mercado de Tula.  ext/noche. }
 \textsc{Personajes}:
 \begin{itemize}
 \item Xólotl.
 \end{itemize}
\textit{Xolotl cae en una ciudad de compleja edificación. Hay muchos aldeanos. Se ven comerciantes de todos lados de Mesoamérica.}
\begin{center}
XÓLOTL
\\
\par
No esperaba menos de Nexoxcho. Tengo que encontrar a Malinalli.
\end{center}
 
\section{Cinemática 41. Templo mayor de Tula. int/noche.}
 \textsc{Personajes}:
 \begin{itemize}
 \item Xólotl.
 \item Quetzalcóatl.
 \item Malinalli (adulta).
 \end{itemize}
\textit{El templo mayor esta bellamente decorado con pinturas de los dioses y pedestales de oro. Quetzalcóatl está orando rodeado de diferentes sacerdotes. Xólotl entra al templo. Los sacerdotes se congelas y Quetzalcóatl se pone de pie.}
\begin{center}
QUETZALCÓATL
\\
\par
¿Te has cansado ya de correr y mentir? Cuando todos nos sacrificamos lo único en lo que pensabas era en ti mismo. Siempre buscas como justificarte. Te da tanto miedo el miedo que siempre buscaras alguien a quien responsabilizar por tus acciones.
\\
\par
XÓLOTL
\\
\par
Nexoxcho. Se que estás ahí. Deja de esconderte. Tus ilusiones no me afectan.
(El escenario se desmorona, se muestra la ciudad en llamas, Quetzalcóatl se desvanece).
¿Qué es lo que pretendes que te diga? ¿Qué no pude salvarlos? ¿Qué hice mal en tratar de negociar con Tezcatlipoca y Quetzalcóatl? El único miedo que siento es el tuyo.
(Aparece Malinalli)
\\
\par
MALINALLI
\\
\par
Hola, Xólotl.
\\
\par
XÓLOTL
\\
\par
¿Malinalli? Eres otra ilusión.
\\
\par
MALINALLI
\\
\par
¿En verdad lo soy o soy solo el reflejo de lo que estoy destinada a ser? (Malinalli extiende sus brazos el escenario cambia a la ciudad de Tenochtitlan, la ciudad es incendiada por el asalto de barcos) Hermoso ¿No lo crees? Estoy segura de que los viste cuando viajaste por todo el mundo buscando un compañero para tu travesía.
\\
\par
 XÓLOTL
\\
\par
¡Nexoxcho! ¿Qué pretendes? Muéstrate ya.
\\
\par
MALINALLI
\\
\par
¿Cuándo muere un dios verdaderamente? Cuando hieres su cuerpo de gravedad, el cuerpo desaparece y deja el tonalli. Si nadie reclama ese tonalli el Dios puede regresar al cabo de un tiempo. Por el contrario, si un dios reclama ese tonalli pero otro más derrota a ese dios y restaura el tonalli del primer dios caído, inevitablemente vuelve. ¿Cuándo desaparece un dios? Dímelo, Xólotl. Estoy segura de que tú lo sabes, viste varios desaparecer durante tus viajes.
\\
\par
XÓLOTL
\\
\par
Un dios desaparece cuando el pueblo que cree en él muere.
\\
\par
MALINALLI
\\
\par
(Se acerca a Xólotl y lo abraza) ¿No es exactamente lo que vez aquí? Pasaste tanto tiempo luchando contra tus hermanos que no pudiste prever el peor de los escenarios. Al final del camino, tú no ganas, Quetzalcóatl no gana ni Tezcatlipoca. Puede que Qutezalcóatl y Tezcatlipoca terminaran con ciudades. No importa, solo tú eres el que destrozara civilizaciones enteras. Felicidades.
\\
\par
XÓLOTL
\\
\par
Ha sido suficiente. ¡Nexoxcho! Tienes razón. Temo a morir y solo me importa lo que me pase a mí. Me importa un bledo la jerarquía divina y lo que le pase al resto de los dioses. Estoy cansado de vivir a la sombra de los demás como muchos otros. Le temo a la muerte tanto como ustedes le temen al cambio. Viven sus vidas gozando de su posición privilegiada que se han olvidado de algo muy importante: todo ciclo tiene un fin. (Xólotl usa un ataque de energía y rompe la ilusión, vuelve a estar en las profundidades del río). Debo de encontrar a Malinalli.  
\end{center}

\section{Cinemática 42. Palacio de Oluta. int/noche. }
 \textsc{Personajes}:
 \begin{itemize}
 \item Malinalli (15)
 \item Madre de Malinalli.
 \end{itemize}
\textit{Malinalli entra al comedor del palacio. Su madre está al otro lado del comedor. Ambas se miran sin decir palabra. }
\begin{center}
MADRE DE MALINALLI
\\
\par
Malinalli. Te he extrañado tanto. Nunca quise dejarte ir. Me forzaron a hacerlo. Sólo deseaba proteger a tu hermano (La madre de Malinalli extiende sus brazos. Malinalli corre a ella y la abraza llorando).
\\
\par
MALINALLI
\\
\par
Deseaba tanto poder volver. Temía que de verdad me odiaras. Temía que no me amaras.
\\
\par
MADRE DE MALINALLI.
\\
\par
Siempre desee poder deshacerme de ti. Cuando me case con tu padre lo único que pensaba era como mi posición iba a mejorar. Tenía un rol que cumplir como esposa y no me quedó más que tenerte. Trata de imaginar mi decepción cuando me dijeron que naciste mujer. 
\\
\par
MALINALLI
\\
\par
No lo entiendo. Madre…
\\
\par
MADRE DE MALINALLI
\\
\par
¿No lo entiendes? (Se ríe) Fui yo la que entrego a tu padre al Tlatoani. Deseaba liberarme de ustedes dos y tu padre me lo hizo muy fácil. Habríamos perdido todo por su patético sueño… (Malinalli ataca a su madre con la caracola. Su madre sale disparada contra la pared)
\\
\par
MALINALLI
\\
\par
No deseo escucharte, nunca más. Todo este tiempo creí que podría llegar a perdonarte. Me equivoque. Desde que me fui del palacio, no paso un día en el que no te odiara y deseara matarte. Imaginar la escena me daba fuerza cada vez que ellos me destrozaban por dentro. Nada me dio más fuerza que imaginar como mataba a tu horrendo vástago (Malinalli ataca a su madre hasta que desaparece).   
\end{center}


\section{Cinemática 43. Palacio de Oluta. int/noche. }
 \textsc{Personajes}:
 \begin{itemize}
 \item Malinalli (actual)
\item Nexoxcho (en la forma del Padre de Malinalli).
 \end{itemize}
\textit{Malinalli entra corriendo a antigua Sala de su padre. El padre de Malinalli esta de espaldas leyendo sus diferentes códices. Malinalli sonríe, corre en dirección a él y lo abraza. }
\begin{center}
NEXOXCHO
\\
\par 
He esperado mucho para verte. Tomaste mucho tiempo. ¿Porqué tomaste tanto tiempo?
\\
\par 
MALINALLI (ACTUAL)
\\
\par 
No importa, estamos juntos de nuevo.
\\
\par 
NEXOXCHO
\\
\par 
Es muy tarde (Se da la vuelta). Mi alma ya no existe. ¿Porqué dejaste que esto pasara? ¿Por qué no viniste a salvarme antes? (Intenta atacar a Malinalli ahorcándola con sus manos. Malinalli se resiste e invoca a la caracola). 
\end{center}


\section{Cinemática 44. Palacio de Oluta. int/noche. }
 \textsc{Personajes}:
 \begin{itemize}
 \item Malinalli (actual)
\item Nexoxcho (Malinalli niña)
\item Malinalli (adulta)
 \end{itemize}
\textit{Malinalli derrota a Nexoxcho. Este adopta la forma de Malinalli cuando niña para continuar su tortura psicológica.}
\begin{center}
MALINALLI (ACTUAL)
\\
\par 
Te equivocas. No hubo día en el que no deseara traerte de vuelta. 
\\
\par 
NEXOXCHO (MALINALLI NIÑA) 
\\
\par 
(Aparece a espaldas de Malinalli) Tú dejaste que esto pasara. Fuiste débil. Dejaste que nos hirieran una y otra vez. Mataste a Padre cuando olvidaste quien eras.
(Se escuchan diferentes voces culpando a Malinalli y diciéndole egoísta).
\\
\par 
MALINALLI (ACTUAL)
\\
\par 
¡Se equivocan! ¡Se equivocan!
(Brilla una luz. Aparece Malinalli adulta. Las figuras desaparecen)
\\
\par 
MALINALLI (ADULTA)
\\
\par 
No es tu culpa. No podías hacer nada. Ahora es diferente. Tienes un poder diferente. Puedes cambiar la historia. Nuestro padre no volverá jamás, no importa a cuantos dioses les arrebates su tonalli, hay cosas que no se pueden cambiar.
\\
\par 
MALINALLI (ACTUAL)
\\
\par 
Entonces… si nuestro padre no puede volver ¿Cuál es el sentido de continuar?
\\
\par 
MALINALLI (ADULTA)
\\
\par 
Terminar lo que él empezó. El imperio Mexica se cree indestructible porque goza de la protección de Huitzilopochtli, si él es otro pueblo más. Destruye a los dioses que protegen al imperio y crea el Ombligo de la luna.
(La ilusión se rompe) 
\end{center}

\section{Cinemática 45. Profundidades del río Apanohualoyan. ext/noche. }
 \textsc{Personajes}:
 \begin{itemize}
 \item Malinalli (actual)
\item Xólotl.

 \end{itemize}
\textit{Malinalli vuelve a las profundidades del rio. Frente a ella está Xólotl y el tonalli de Nexoxcho.}
\begin{center}
XÓLOTL
\\
\par
Iba dispuesto a ayudarte.  Y antes de que me diera cuenta ya habías destruido a Nexoxcho por tu cuenta. Bastante impresionante.
\\
\par
MALINALLI
\\
\par
¿Ese era Nexoxcho?
\\
\par
XÓLOTL
\\
\par
Ciertamente. Su mayor poder es tomar la forma de nuestros peores miedos y sumergirnos en una ilusión en donde los enfrentaremos a lo que sea que temamos.
\\
\par
MALINALLI
\\
\par
¿Ilusión? Parecía bastante real para mí. Pudiste habérmelo advertido.
\\
\par
XÓLOTL
\\
\par
Lo hice te dije que mantuvieras los ojos abiertos. Si te hubiera dicho directamente el efecto del agua en ti ahora sería agua negra. Hay quienes dicen que quienes logran escapar de la ilusión de Nexoxcho logran dar con una gran revelación. El mayor reto antes del descanso eterno no podría ser otro que confrontarte a ti mismo. Ultima parada el corazón del Mictlán: el palacio del rey.  
\end{center}

\section{Cinemática 46. Palacio de Mictlantecuhtli. int/noche.}
 \textsc{Personajes}:
 \begin{itemize}
 \item Malinalli (actual)
 \item Xólotl.
 \item Mictlantecuhtli.
 \end{itemize}
\textit{Malinalli y Xólotl llegan a la sala del trono. Mictlantecuhtli los espera sentado en el trono. }
\begin{center}
MICTLANTECUHTLI
\\
\par
No esperaba verte. Me ha tomado por sorpresa la caída de Nexoxcho. Aunque siendo tu segunda vez aquí debí haberlo previsto. Aquella vez tuve la impresión de que el éxito se debió a Quetzalcóatl únicamente. Me equivocaba.
\\
\par
XÓLOTL
\\
\par|
Nunca subestimes a tus enemigos,  Mictlantecuhtli. Puede que te lleves una muy desagradable sorpresa.
\\
\par
MICTLANTECUHTLI
\\
\par
Tomare tu consejo para la próxima.
\\
\par
XÓLOTL
\\
\par
No habrá una próxima.  
\end{center}

\section{Cinemática 4. Palacio de Mictlantecuhtli. int/noche.}
 \textsc{Personajes}:
 \begin{itemize}
 \item Malinalli (actual)
 \item Xólotl.
 \item Mictlantecuhtli.
 \end{itemize}
\textit{Mictlantecuhtli es derrotado.  La energía que libera cuando se transforma en tonalli hace que tiemble. El palacio comienza a derrumbarse. Malinalli usa la caracola para protegerse de la caída del castillo. Cuando el castillo ha terminado de derrumbarse solo queda el tonalli de Mictlantecuhtli. Malinalli lo toma. Cuando obtiene el Tonalli de Mictlantecuhtli, Malinalli comienza a brillar, cuando la luz se ha despejado sus ropas han cambiado: ahora porta una armadura de combate parecida a la de las Diosas guardianas del Mictlán. No hay rastro de Xólotl. Una segunda luz brilla. Xólotl sale de entre los escombros del palacio de Mictlantecuhtli. Su forma ya no es la de un xoloitzcuintle, Xólotl está en su forma humana. }
\begin{center}
MALINALLI
\\
\par 
¿Xólotl?
\\
\par 
XÓLOTL.
\\
\par 
Hola, Malinalli. Debe de ser impactante verme en esta forma. Solía ser mi forma verdadera hasta que Tezcatlipoca selló mis poderes hace mucho. Gracias a ti, he podido romper el sello. Nuestra aventura en el Mictlán ha acabado.
\\
\par 
MALINALLI
\\
\par 
¿Significa que volveré a casa?  
\\
\par 
XÓLOTL
\\
\par 
El alma de tu padre no está en el Mictlán. Al ser un guerrero, su lugar está en los trece cielos junto a Huitzilopochtli. En estos momentos no puedo traerlo de vuelta como lo prometí. Eso no significa que no vaya a cumplir lo que hemos acordado. Dos caminos se presentan ante ti, Malinalli. Puedes esperar a tu padre en el reino de los mortales o puedes ir conmigo a los trece cielos y ayudarme. 
\\
\par 
MALINALLI
\\
\par 
Si vuelvo al reino de los mortales ¿Mantendré el poder que he ganado aquí?
\\
\par 
XÓLOTL
\\
\par 
Me temo que no. El poder espiritual pertenece al reino espiritual. Una vez regreses al mundo de los mortales volverás a ser normal. El hecho de que tu atuendo haya cambiado al obtener el tonalli de Mictlantecuhtli demuestra que nuestra energía espiritual ha alcanzado un buen nivel de sincronización, encontrar un nuevo aliado tan bueno como tú me tomara tiempo.
\\
\par 
MALINALLI
\\
\par 
¿Me está pidiendo que le acompañe? (Xólotl asiente con la cabeza. Malinalli camina en dirección de Xólotl). ¿Qué estamos esperando? Los trece cielos nos esperan. 
\end{center}

\end{document}